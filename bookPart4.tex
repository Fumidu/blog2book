
\documentclass{book}
% Chargement d'extensions
\usepackage[francais]{babel} % Pour la langue française
\usepackage[utf8]{inputenc}
\usepackage[T1]{fontenc}
\usepackage{float}
\usepackage{eurosym}
\usepackage{textcomp}
\usepackage{wrapfig} % to wrap figure in text
%\usepackage{caption} % to remove fig #
\usepackage[font=it]{caption}
\usepackage{titlesec} % to modify chapter header
\usepackage[normalem]{ulem}
\usepackage{soul}

\titleformat{\chapter}[display]
  {\normalfont\bfseries}{}{0pt}{\Large}

\usepackage[
  paperwidth=15.557cm,
  paperheight=23.495cm,
  %showframe,
  margin=15mm
  % other options
]{geometry}

\hyphenation{CRRR-RA-A-A-A-AK-K-K-K-K-KR-R-R-RB-B-B-B-BRA-A-A-A-A-A-AOUM-M-M-M-M-MR-R-R-RM-M-M-ML-L-L-LLLLL}

%\usepackage{layout} %TODO remove

\usepackage{graphicx} % Pour les images
\graphicspath{ {Images/petit/} }
%\graphicspath{ {Images/} }

\title{Et demain on va où ?}
\author{Gérald Schmitt \and Marion Abrial}

\begin{document}
%\layout
\maketitle




\chapter{Un an déjà !}
Hé oui, comme le temps passe vite (première porte ouverte enfoncée) ! Mais quand on regarde derrière nous, que de chemin parcouru (hop, deuxième porte ouverte) !

Bref, ça fait un an, la date du retour approche, et le rythme de publication  ralenti. En même temps, ce n'est pas toujours facile de trouver du Wifi dans les montagnes au Népal. Mais ne vous inquiétez pas, quelques articles vont arriver bientôt.

En attendant voici une première petite photo du Népal, histoire de vous mettre en jambes pour ce qui arrive !


\begin{figure}[h]
\centering
\includegraphics[height=6cm,width=9cm,keepaspectratio]{p9139611.jpg}
\caption*{Il y a une faute sur le panneau : c'est 5160m. Vous allez dire que je chipote, mais 54m, ça compte mine de rien...}
\end{figure}

\chapter{Katmandou, premier contact avec le Népal}
Nous partîmes donc pour Katmandou. Après avoir franchi la frontière Inde-Népal, on prend un mini bus qui doit nous amener en 5 heures à destination. En fait, pas vraiment. Disons que 5 heures, c'est le temps de trajet donné par un GPS. Mais quand on rajoute :
\begin{itemize}
	\item la route défoncée
	\item des milliers de camions qui alimentent tous le Népal par le même chemin et qui ont tous des raisons d'être pressés et de vouloir dépasser les autres
	\item des passagers qui profitent de chaque pause pour se bourrer la gueule et ne veulent plus repartir
\end{itemize}
Eh bien on obtient un temps de trajet  de 12 heures. On est donc arrivé à 2h du matin au lieu de 7h du soir.


\begin{figure}[h]
\centering
\includegraphics[height=6cm,width=9cm,keepaspectratio]{p8297727.jpg}
\caption*{Technique de construction traditionnelle.}
\end{figure}

Malgré cette entrée en matière pas des plus reposante, Katmandou, après le choc de l'Inde, ça fait du bien. C'est plus calme, plus propre, et la circulation est moins stressante, même si tout autant chaotique. On en profite pour faire redescendre notre pression sanguine pendant quelques jours. Le seul point négatif, ce sont les gérants de l'hôtel qui nous demandent tout le temps ce qu'on va faire car ils veulent vraiment nous vendre des treks.


\begin{figure}[h]
\centering
\includegraphics[height=6cm,width=9cm,keepaspectratio]{p8287367.jpg}
\caption*{Katmandou c'est un peu moche, alors voici les ailes d'un papillon en échange.}
\end{figure}

Puis Marion m'abandonne 10 jours pour une retraite de méditation vipassana. C'est une forme extrême de méditation : interdiction totale de distraction. Pas de lecture, pas de téléphone, pas de conversation, pas de contact visuel. Les deux repas quotidiens se font face à un mur ! Ce fut dur, ce fut long, elle fut forte et elle revint changée, mais elle vous en parlera mieux que moi. De mon coté, après de longues hésitations, plutôt que de l'accompagner, je suis resté glander en ville. Enfin, quand je dis glander, j'ai surtout fait du blog ! C'est que ça ne s'écrit pas tout seul ces petits articles ! Je trouve aussi une salle d'escalade, et en profite pour me faire une tendinite dès la première séance... Je me contenterai donc ensuite de lever le coude avec mon partenaire de grimpe temporaire.


\begin{figure}[h]
\centering
\includegraphics[height=6cm,width=9cm,keepaspectratio]{img_20160825_072215.jpg}
\caption*{Le planning de la méditation.}
\end{figure}

Une fois réunis, on a commencé à visiter la vallée de Katmandou. Pour ça, le scooter est idéal, à condition d'être TRÈS prudent : c'est pas l'Inde, mais pas loin, alors on laisse la priorité à tout le monde (surtout aux plus gros), on klaxonne sans cesse (les gens ne regardent pas autour d'eux) et on ne dépasse jamais les 30km à l'heure. Notre première sortie nous amène à un petit parc national un peu au Nord de Katmandou. C'est une petite colline à gravir pour avoir une vue sur la ville, idéal pour une balade à la journée. Deux kilomètres avant l'entrée du parc, ça monte et notre scooter cale. On termine donc à pied et on arrive au guichet.


\begin{figure}[h]
\centering
\includegraphics[height=6cm,width=9cm,keepaspectratio]{p8267097.jpg}
\caption*{En se baladant, on tombe parfois sur d'étranges poulets ! Vous ne trouvez pas qu'il a un air à Donald Trump ?}
\end{figure}

C'est un parc national, donc il faut payer l'entrée. Et là, on nous explique qu'on doit aussi embaucher un guide. La loi date d'il y a quelques semaines, et c'est obligatoire. C'est pour notre "sécurité", et notre "information". Nous, on sortait de Katmandou pour être un peu tranquilles, et fuir les sollicitations constantes de gens qui veulent nous vendre des flûtes faites main, du baume du tigre, du haschisch, des informations sur les treks, des taxis, des rickshaws... Du coup, pas trop envie d'avoir une nounou sur le dos pour surveiller qu'on ne sort pas du sentier. Comme la plupart des autres visiteurs, on est allés se balader ailleurs...


\begin{figure}[h]
\centering
\includegraphics[height=6cm,width=9cm,keepaspectratio]{p8287387.jpg}
\caption*{Comme cette araignée, on a pris nos grandes pattes et on est allés voir ailleurs !}
\end{figure}

Notre deuxième sortie nous amène un peu plus loin dans la vallée, à Nagarkot. C'est une toute petite route en lacet qui nous y amène, quasiment vide de circulation. On arrive dans un petit village au sommet d'une colline qui surplombe la vallée. La vue est magnifique, et des petits sentiers de randonnée nous permettent de nous remettre un peu en forme. La vallée du Katmandou, c'est bien, mais on a du gros trek qui se profile à l'horizon...


\begin{figure}[h]
\centering
\includegraphics[height=6cm,width=9cm,keepaspectratio]{p8297732.jpg}
\caption*{Rizières dans la vallée de Katmandou.}
\end{figure}



\chapter{Oh Manaslu, mon beau Manaslu, où es-tu ?}
Nous partîmes donc pour une petite balade autour du Manaslu. Une petite balade de 22 jours quand même, autour d'un sommet qui culmine à plus de 8000m, et sans porteur ! Juste nous deux et un guide : Dipak, qui nous avait été chaudement recommandé quelques mois auparavant, par un couple qui nous avait donné envie en nous parlant justement de ce trek.


\begin{figure}[h]
\centering
\includegraphics[height=6cm,width=9cm,keepaspectratio]{p9139585.jpg}
\caption*{Est-ce le Manaslu ? Eh bien non !}
\end{figure}

Pourquoi ce trek, plutôt que le classique et renommé "Tour des Annapurnas" ? C'est simple, depuis quelques années, une route carrossable fait quasiment tout le tour des Annapurnas, alors que le Manaslu est uniquement accessible à pied, et de fait bien plus préservé de ces saloperies de touristes... C'est en bus local qu'on fait la première partie du voyage. Nous quittons rapidement l'autoroute (autoroute dans le sens népalais, c'est une route assez large pour que, la plupart du temps, deux véhicules puissent se croiser sans trop freiner) et nous attaquons les petites routes. A partir de là, on n'a plus l'impression de s'approcher de notre but, mais plutôt d'y "tendre" : plus on s'approche, plus on ralentit... La mousson tardive et les camions transforment les routes en champs labourés. Nous devons régulièrement descendre du bus pour le laisser faire des manœuvres incroyables destinées à traverser un champ de boue en montée. Mais toute la bonne volonté du chauffeur, ses coups de volants et d'accélérateur n'y pourront rien : le bus fini embourbé comme tout le monde et nous nous retrouvons piétons plus tôt que prévu. Heureusement, notre guide trouve un autre bus peu après, mais nous devons voyager sur le toit. Commencent alors 4 très, très longues heures. Le toit est blindé de gens, nous sommes assis sur une grille où il est impossible de trouver un position confortable, et les cahots de la route sont tels qu'on se cramponne à la grille au point d'en avoir mal aux mains, de peur d'être éjecté au prochain nid de poule.


\begin{figure}[h]
\centering
\includegraphics[height=6cm,width=9cm,keepaspectratio]{p9017749.jpg}
\caption*{Vingt kilomètres dans ces conditions, ça prend du temps.}
\end{figure}

Contre toute attente, nous finissons par arriver à bon port, avec à peine 4h de retard. Dernière ville, dernière douche chaude, et surtout, dernière connexion à internet, avant de pénétrer dans la montagne. La première journée nous met bien dans le rythme : plus de 20km, avec du dénivelé et des sacs. On est contents de se poser le soir. Contrairement à tout ce qu'on imaginait sur les treks au Népal, point de paysages minéraux ni de neige ni de vue incroyable sur des sommets enneigés : on marche dans la jungle. La végétation foisonne. Il fait chaud et humide. Et les nuages nous bloquent la vue même quand il ne pleut pas.


\begin{figure}[h]
\centering
\includegraphics[height=6cm,width=9cm,keepaspectratio]{p9047795.jpg}
\caption*{C'est beau, mais brumeux !}
\end{figure}

La pluie ne fait pas que retarder les randonneurs : elle façonne la montagne ! Depuis le terrible tremblement de terre de 2015, la montagne n'est toujours pas stabilisée. En ajoutant le poids et la lubrification de la pluie, on obtient des glissements de terrain quotidiens, pour ne pas dire horaires. Ces grondements sourds qui nous intriguaient régulièrement n'avaient rien à voir avec de l'orage ! Nous sommes passés plusieurs fois sur des glissements tout frais qui avaient emporté quelques dizaines de mètres de sentier, à peine quelques minutes avant notre arrivée. Nous avons vu un petit pan de forêt glisser subitement dans le torrent de la taille d'un fleuve, élargissant la vallée sous nos yeux, les temps géologiques se superposant à l'histoire humaine le temps de quelques secondes.


\begin{figure}[h]
\centering
\includegraphics[height=6cm,width=9cm,keepaspectratio]{p9179931.jpg}
\caption*{Marion traversant un glissement de terrain tout frais.}
\end{figure}


\begin{figure}[h]
\centering
\includegraphics[height=6cm,width=9cm,keepaspectratio]{p9098985.jpg}
\caption*{Ceci est un épouvantail... un épouvantail à singes !}
\end{figure}

En passant sur un pierrier, notre guide nous dit soudain qu'il y avait un village ici. On ne voit plus rien. Il a été enterré sous des milliers, des millions de tonnes de pierres. Pas de plaque, pas de stèle, pas de message, aucune trace de ce village si ce n'est dans les mémoires. L'Himalaya est un massif jeune, en pleine croissance, qui bouge tous les jours. Dans ces conditions, toute construction est forcément temporaire. Dipak nous précise que 20 personnes et 50 mules sont mortes. Évoquer en même temps les vies humaines et les mules me semble incongru sur le moment, mais ces mules sont le sang de ces vallées. En l'absence de routes praticables, ce sont des convois de mules qui transportent quasiment tout. Les choses fragiles comme les vitres sont transportées à dos d'hommes, et les plus riches se permettent hélicoptère, mais les récoltes, les bouteilles de gaz, le ciment, les packs de bière et les snickers à destination des touristes, c'est pour les mules !


\begin{figure}[h]
\centering
\includegraphics[height=6cm,width=9cm,keepaspectratio]{p9099252.jpg}
\caption*{Des mules au travail.}
\end{figure}


\begin{figure}[h]
\centering
\includegraphics[height=6cm,width=9cm,keepaspectratio]{p9078968.jpg}
\caption*{Ça pousse comme de la mauvaise herbe... alors que c'est de la bonne !}
\end{figure}

Quelques jours plus tard, notre guide a une mauvaise nouvelle : un glissement de terrain a rendu très dangereux l'accès à la vallée de Tsum. C'est une vallée qu'on peut visiter quand on fait le tour du Manaslu. Des exilés tibétains s'y sont installés, et on avait prévu d'y passer 5 jours, le temps de faire l'aller retour jusqu'au bout de la vallée. On décide de quand même d'aller voir a quoi ressemble le passage... Ok, donc c'est de l'escalade, sans corde bien évidemment. On a déjà grimpé du beaucoup plus dur, mais assuré. Là, si on tombe, c'est 30m de chute, directement dans le torrent en colère. Et ça fait longtemps qu'on n'a pas grimpé. Et on a des sacs. Et il pleut. Et on a quelques heures de marche dans les pattes. Et on n'est pas venu pour ça, alors tant pis, on fait demi-tour.


\begin{figure}[h]
\centering
\includegraphics[height=6cm,width=9cm,keepaspectratio]{p9098994.jpg}
\caption*{Petit village et rizières.}
\end{figure}

La pluie, ça n'entraine pas que des catastrophes ici. Ça permet aussi à tout un tas d'animaux mignons de sortir prendre l'air. Vous connaissez tous le proverbe "Il pleut, il mouille : c'est la fête à la .....  sangsue d'mes c**illes !". La premières fois, c'est comme ça qu'on les remarque : on enlève les chaussures et chaussettes à la fin de la journée, et on se demande d'où vient tout ce sang ! Elles ne se contentent pas d'en prélever quelques gouttes, elles laissent aussi une plaie pleine d'anticoagulant qui continue de saigner encore une à deux heures après le décès (brutal) de ladite sangsue. Mais comme une image valant 1000 injures, en voici deux :


\begin{figure}[h]
\centering
\includegraphics[height=6cm,width=9cm,keepaspectratio]{p9068470.jpg}
\caption*{Les petits points rouges, ce sont des morsures de sangsues.}
\end{figure}


\begin{figure}[h]
\centering
\includegraphics[height=6cm,width=9cm,keepaspectratio]{p9068473.jpg}
\caption*{Bien grasse et bien dodue... Hmmm !!!}
\end{figure}

Le plus surprenant, c'est avec quelle facilité elles arrivent à s'accrocher à la chaussure, remonter sous le pantalon, redescendre dans la chaussure, puis traverser la chaussette. On a tenté plusieurs stratégies, mais la seule qui marche un tant soit peu, c'est une pause nettoyage des sangsues toutes les 20 minutes. Malgré tout, on finira chacun avec une dizaine de cicatrices...


\begin{figure}[h]
\centering
\includegraphics[height=6cm,width=9cm,keepaspectratio]{p9098992.jpg}
\caption*{Les yeux de Bouddha surveillent la vallée.}
\end{figure}

Tout ceci ne nous empêche pas d'avancer. Nous partons tôt tous les matins : à 7h, nous sommes généralement déjà en train de marcher, et le plus souvent, on est arrivés avant 14h à destination. Toutes les guesthouses se ressemblent, et tous les menus sont identiques. C'est répétitif mais nous avons beaucoup plus de confort que ce à quoi on s'attendait. Nous avons à chaque fois notre chambre individuelle, et nous mangeons beaucoup de pâtes et de riz, parfois même des roestis ! Le plat traditionnel, c'est normalement le Dal Bhat : une grosse plâtrée de riz, avec une soupe de lentilles et un curry de légumes. Les Népalais en mangent tous les jours, et ce n'est pas une image : notre guide nous a dit, que les rares fois où il mange des pâtes ou une pizza, il mange aussi du Dal Bhat une heure plus tard. Il est toujours servi à volonté, et très rapidement car il est préparé à l'avance. Mais nous n'en mangeons presque pas car à notre grande surprise, c'est toujours le plat le plus cher ! Étonnant, alors que c'est le seul plat qui ne leur demande aucun boulot supplémentaire étant donné qu'il y en a toujours qui mijote dans la cuisine...


\begin{figure}[h]
\centering
\includegraphics[height=6cm,width=9cm,keepaspectratio]{p9099269.jpg}
\caption*{Au chaud dans la cuisine !}
\end{figure}

Ah... la cuisine... Alors que tous les jours nous montons un peu plus haut dans la vallée, la température baisse petit à petit, et il commence à faire froid. La cuisine étant la seule pièce chauffée, toute la famille s'y retrouve, ainsi que les guides et les porteurs de passage. Mais pas les touristes, relégués dans la grande salle à manger toute froide. Dès qu'on entre dans la cuisine, on est poliment invités à passer dans la pièce à coté. Seule une famille fera exception pendant tout le trek. Autre chose surprenante : notre guide ne mange jamais avec nous ! Il reste parfois debout à coté de la table pendant que nous mangeons, refuse de s'asseoir quand on l'invite, nous regarde, et parfois fouille dans nos affaires. Oui, vous avez bien lu. C'est qu'ils ont une autre conception de l'intimité par ici...


\begin{figure}[h]
\centering
\includegraphics[height=6cm,width=9cm,keepaspectratio]{p9078961.jpg}
\caption*{Confiance !}
\end{figure}

Nous traversons régulièrement des rivières sur des ponts suspendus. Le temps des ponts en lianes est révolu, ce sont des câbles en acier à présent. Je ne peux m'empêcher de remarquer que mêmes les ponts tout neufs présentent déjà des problèmes d'entretien : des écrous manquent un peu partout, et ceux qui restent semblent sérieusement déserrés. Renseignement pris : il semble que personne n'entretient ces ponts ! On trouve du budget pour les construire, il y a des petites stèles qui explique qui a financé, qui a réalisé quoi, et... plus rien ensuite. Dommage.


\begin{figure}[h]
\centering
\includegraphics[height=6cm,width=9cm,keepaspectratio]{p9078962.jpg}
\caption*{Pas confiance...}
\end{figure}

A partir de 3500m d'altitude, on commence enfin à sortir de la jungle, et ça commence à ressembler un peu plus à ce qu'on s'imaginait du Népal. Nous arrivons ainsi à Samagon, un des premiers endroit où on peut voir le sommet, si les nuages coopèrent, bien entendu. Et ils ont coopéré, un matin entre 5h45 et 6h. Oui, 15 minutes sur toute la journée, faut pas les louper ! Samagon est aussi le village de départ pour les expéditions vers le sommet. En fait, la quasi totalité des rares touristes que nous rencontrons font partie d'une expédition, et vont passer quelques semaines à naviguer entre 5000m et 8000m d'altitude et des poussières. A coté d'eux, on a vraiment l'impression d'être des promeneurs du dimanche...


\begin{figure}[h]
\centering
\includegraphics[height=6cm,width=9cm,keepaspectratio]{p9119323.jpg}
\caption*{Et en plus, il sourit pour la photo. Trop facile...}
\end{figure}


\begin{figure}[h]
\centering
\includegraphics[height=6cm,width=9cm,keepaspectratio]{p9098999.jpg}
\caption*{Les femmes aussi portent des trucs incroyables, comme des troncs d'arbre !}
\end{figure}

On voit aussi le travail incroyable des porteurs : les mules ne peuvent pas aller si haut, et il faut une quantité incroyable de bordel pour installer le camp de base et les différents camps temporaires des expéditions. On assiste à un ballet incessant d'énormes sacs, de bouteilles de gaz, de chaises, de casiers de poulets vivants, j'en passe et des meilleurs. Et si on regarde bien, on remarque, sous le gros tas de sac qui avance dans la montagne, oui, en effet, il y a un népalais en tong ! Même si la législation interdit normalement des charges supérieures à 25kg, il se dit que certains peuvent monter leur propre poids, à une altitude où un homme normalement constitué trouve déjà pénible de soulever simplement ses propres fesses. On a assisté à une scène improbable, où deux hommes étaient nécessaires pour mettre une charge sur le dos d'un troisième. Ce sont des surhommes.


\begin{figure}[h]
\centering
\includegraphics[height=6cm,width=9cm,keepaspectratio]{p9109284.jpg}
\caption*{Les gamins sont trognons.}
\end{figure}


\begin{figure}[h]
\centering
\includegraphics[height=5cm,width=9cm,keepaspectratio]{p9119362.jpg}
\caption*{Messe du dimanche dans un monastère. Il y avait une de ces ambiance... c'est simple, on se serait cru dans un monastère perdu au milieu de l'Himalaya !}
\end{figure}

Après une journée d'acclimatation à Samagon, occupée à ne pas aller au camp de base pour cause de brouillard, on attaque le point culminant du trek : le col de Larke. On commence par rejoindre la dernière guesthouse avant le col : une cahute en pierre et en terre, à 4400m d'altitude.


\begin{figure}[h]
\centering
\includegraphics[height=5cm,width=9cm,keepaspectratio]{p9129575.jpg}
\caption*{Je crois qu'on est encore un peu hors saison...}
\end{figure}


\begin{figure}[h]
\centering
\includegraphics[height=5cm,width=9cm,keepaspectratio]{p9129574.jpg}
\caption*{On se paie le luxe d'une chambre triple !}
\end{figure}

Le lendemain matin, on part à 6h pour la dernière ascension, avec comme objectif le col à 5160m d'altitude. La matinée commence bien : il fait beau, on voit quelques sommets, et plein de petites souris bourrues qui courent se réfugier sous des pierres à notre approche. On passe devant un lac tellement bleu qu'on se demande s'il n'a pas été photoshoppé.


Le silence n'est brisé que par les bruits de cailloux qui dévalent les pentes, et par notre respiration qui se fait courte. On commence à sentir les effets de l'altitude. A 5000m, il n'y a plus que la moitié de l'air, et donc de l'oxygène, qu'on trouve au niveau de la mer. Résultat : on a l'impression de se trainer comme des vieux pépés. On va mettre 4 heures pour faire 7km et et les derniers 800m de dénivelé. Deux heures avant d'arriver au col, il se met à neiger, et à venter. Nous sommes maintenant dans un désert de cailloux, où plus rien d'autre que du lichen ne pousse, et nous faisons des pauses toutes les 10 minutes. C'est une sensation étrange : ce n'est pas douloureux ni désagréable, mais chaque pas est épuisant, et on se demande ce qui nous arrive, à être mous comme ça...


\begin{figure}[h]
\centering
\includegraphics[height=9cm,width=12cm,keepaspectratio]{p91396111.jpg}
\caption*{Ça valait le coup de monter, non ?}
\end{figure}

Le col, c'est d'abord une tâche colorée au loin. Il y a des milliers de drapeaux, et un panneau qui nous félicite d'être arrivé là. Dommage, le panneau a une faute de frappe sur l'altitude. Il indique 5106m alors que c'est 5160m. On est très contents d'être arrivés, mais avec le brouillard, le vent et la neige, on ne voit rien et il fait froid, donc inutile de s'attarder, et on attaque les 1700m de descente dans les cailloux et la boue. La neige s'est transformée en pluie, le chemin est une pataugeoire, et nos goretex ont cessé de remplir leur rôle. Affamés, trempés et heureux, nous arrivons enfin au village, après 4 heures de descente !


\begin{figure}[h]
\centering
\includegraphics[height=9cm,width=12cm,keepaspectratio]{p9139578.jpg}
\caption*{Faut pas louper le créneau pour la photo.}
\end{figure}

Le lendemain matin, comme pour se faire pardonner du temps pourri de la veille, la montagne se dévoile enfin, et nous avons droit au lever de soleil sur le Manaslu entièrement dégagé. On ne sera pas venus pour rien !







\chapter{L'ABC des Alsaciens}
Quelle ne fut pas ma surprise, quand, au détour d'un mail, mon père m'annonce qu'il nous rejoindrait bien quelque part dans notre voyage ! Et moi qui croyait qu'avec son tout nouvel emploi du temps de retraité, ce serait impossible... On se met rapidement d'accord sur le Népal, et je lui suggère d'embarquer Kévin dans l'affaire. Une chose en entrainant une autre, la moitié de l'Alsace (environ) décide de se joindre à l'aventure.


\begin{figure}[h]
\centering
\includegraphics[height=5.5cm,width=9cm,keepaspectratio]{p9290699.jpg}
\caption*{La moitié de l'Alsace (environ) et un chien.}
\end{figure}

Par ordre alphabétique, Aurélien, Camille, Jean-Luc, Kévin et Pierre partirent donc pour Katmandou où nous les récupérâmes. Et là, tout s'enchaîne : on fait les permis de trek, on rentre à l'hôtel , et on boit une bière bien méritée, accompagnée d'un peu de saucisson et de fromage. Quel pied ! Le lendemain, une journée de bus nous amène à Pokhara, et le surlendemain, un taxi nous amène au début de la randonnée menant au camp de base des Annapurnas, en anglais : Annapurnas Base Camp, soit ABC.


\begin{figure}[h]
\centering
\includegraphics[height=5.5cm,width=9cm,keepaspectratio]{p9280330.jpg}
\caption*{C'est dur les escaliers. Même Pierre a l'air fatigué, c'est dire !}
\end{figure}


\begin{figure}[h]
\centering
\includegraphics[height=5.5cm,width=9cm,keepaspectratio]{p9280333.jpg}
\caption*{La vue une petite heure après le départ.}
\end{figure}

Le chemin commence par un escalier interminable. L'entrée en matière est rude ! Mais toute l'équipe s'est préparée physiquement, et tout le monde a bien optimisé le poids du sac, et ça passe sans problème. Il faut dire qu'on a fait le choix de faire ce trek sans guide ni porteur, juste entre nous :-)


\begin{figure}[h]
\centering
\includegraphics[height=5.5cm,width=9cm,keepaspectratio]{p9280344.jpg}
\caption*{Kévin "la gazelle de l'Himalaya", franchi d'un bond gracieux le petit ruisseau.}
\end{figure}

Les premiers jours, nous sommes quasiment seuls sur le chemin, mais rapidement, on se retrouve à croiser plus de Chinois que n'importe quelle autre nationalité. Les frontières sont ouvertes pour ce pays depuis peu, et quand un milliard et des poussières (poussières grandes comme l'Europe...) de Chinois rattrapent 50 ans de tourisme, ici, ça se remarque !


\begin{figure}[h]
\centering
\includegraphics[height=5.5cm,width=9cm,keepaspectratio]{p9280339.jpg}
\caption*{Il y a des ponts...}
\end{figure}


\begin{figure}[h]
\centering
\includegraphics[height=5.5cm,width=9cm,keepaspectratio]{p9290569.jpg}
\caption*{... et des rivières...}
\end{figure}


\begin{figure}[h]
\centering
\includegraphics[height=5.5cm,width=9cm,keepaspectratio]{p9290584.jpg}
\caption*{... et encore des ponts...}
\end{figure}


\begin{figure}[h]
\centering
\includegraphics[height=5.5cm,width=9cm,keepaspectratio]{p9290577.jpg}
\caption*{... et des troupeaux de chèvre, guidés par Kévin, le berger des Annapurnas.}
\end{figure}

Comme pour le Manaslu, il pleut tous les jours, vers 15h en général. Nous sommes donc ob-bli-gés de nous arrêter, et passons l'après-midi à boire de la bière (car l'hydratation c'est important), à manger du saucisson et du fromage (il faut alléger les sacs), et à jouer au tarot (non, là je n'ai pas de justification).


\begin{figure}[h]
\centering
\includegraphics[height=5.5cm,width=9cm,keepaspectratio]{p9280349.jpg}
\caption*{On boit aussi du rhum (merci Pierre !).}
\end{figure}


\begin{figure}[h]
\centering
\includegraphics[height=5.5cm,width=9cm,keepaspectratio]{p9280351.jpg}
\caption*{On mange même du lard fumé ! Rien que de l'écrire, j'en salive... Merci Papa !}
\end{figure}


\begin{figure}[h]
\centering
\includegraphics[height=5.5cm,width=9cm,keepaspectratio]{p9280356.jpg}
\caption*{On agrémente l'apéritif grâce à Kévin, le boucher des sommets.}
\end{figure}

La mousson a désormais 6 semaines de retard, mais on sent qu'on approche de la fin : les nuages sont moins présents, et on a souvent du soleil le matin, ce qui nous permet de voir de temps en temps quelques sommets, comme le fameux Machapuchare, qui culmine à quasiment 7000m. Sa forme particulière lui a valu son nom : ça veut dire \emph{queue de poisson} en népalais. C'est aussi un des rares sommet à n'avoir jamais été conquis. Il est interdit d'ascension depuis les années cinquante pour raison religieuse, et les quelques renégats irrespectueux de cette supposée demeure de Shiva ont tous échoué.


\begin{figure}[h]
\centering
\includegraphics[height=5.5cm,width=9cm,keepaspectratio]{pa0416381.jpg}
\caption*{La "queue de poisson".}
\end{figure}


\begin{figure}[h]
\centering
\includegraphics[height=5.5cm,width=9cm,keepaspectratio]{pa031601.jpg}
\caption*{L'Annapurna sud, lors d'une des rares nuits sans nuages. La neige du sommet est éclairée par un orage plus bas dans la vallée !}
\end{figure}

C'est arrivé au pied de ce sommet maudit (ou sacré, chacun son point de vue) qu'on sort enfin de la jungle. Les nuages continuent de s'accrocher aux cimes, mais quelle ambiance ! Nous sommes à 4000m d'altitude, et pourtant entourés de montagne presque encore deux fois plus hautes ! Encore une petite heure de montée, et nous sommes au ABC, au bord d'un reste de glacier. Le paysage est magnifique, mais le sommet continue de faire son timide, et reste caché sous un voile pudique.


\begin{figure}[h]
\centering
\includegraphics[height=5.5cm,width=9cm,keepaspectratio]{pa020944.jpg}
\caption*{OK, on ne voit pas le sommet, mais quand même, c'est beau !}
\end{figure}


\begin{figure}[h]
\centering
\includegraphics[height=5.5cm,width=9cm,keepaspectratio]{pa020966.jpg}
\caption*{Victoire !}
\end{figure}


\begin{figure}[h]
\centering
\includegraphics[height=5.5cm,width=9cm,keepaspectratio]{pa021000.jpg}
\caption*{Deux frères contents de voyager ensemble !}
\end{figure}


\begin{figure}[h]
\centering
\includegraphics[height=5.5cm,width=9cm,keepaspectratio]{pa021010.jpg}
\caption*{Deux frères qui... euh... ben qui voyagent ensemble !}
\end{figure}


\begin{figure}[h]
\centering
\includegraphics[height=5.5cm,width=9cm,keepaspectratio]{pa020978.jpg}
\caption*{Papa bien entouré !}
\end{figure}


\begin{figure}[h]
\centering
\includegraphics[height=5.5cm,width=9cm,keepaspectratio]{pa020982.jpg}
\caption*{Papa, sérieusement encadré !}
\end{figure}


\begin{figure}[h]
\centering
\includegraphics[height=5.5cm,width=9cm,keepaspectratio]{pa020954.jpg}
\caption*{Câlin ou casse-croûte ?}
\end{figure}

Pendant la redescente, certains souffrent, en particulier des genoux. La pudeur m'oblige à laisser leur identité cachée, la NSA n'en saura rien ! Par contre, c'est vraiment très pratique d'avoir un kiné dans l'équipe, et là, il ne m'en voudra pas de lui faire de la pub : Aurélien ralenti la chute des genoux, appliquez-en une couche généreuse deux fois par jour pour des résultats optimaux. Les marches n'ont qu'à bien se tenir !


\begin{figure}[h]
\centering
\includegraphics[height=5.5cm,width=9cm,keepaspectratio]{pa010734.jpg}
\caption*{Si on veut un sac léger ET sentir bon, faut laver ses affaires tous les jours !}
\end{figure}


\begin{figure}[h]
\centering
\includegraphics[height=5.5cm,width=9cm,keepaspectratio]{pa072275.jpg}
\caption*{J'aime quand les gens synchronisent leurs vêtements !}
\end{figure}

Le retour à Pokhara deux jours plus tôt que prévu nous laisse un peu de temps pour aller visiter la pagode de la paix. Ça commence de manière fourbe par un petit tour en bateau, et ensuite le piège : des marches qui font mal aux genoux ! Non, je déconne, tout s'est bien passé, tout le monde est monté, et tous les 10 genoux sont rentrés en état de marche en Alsace.


\begin{figure}[h]
\centering
\includegraphics[height=5.5cm,width=9cm,keepaspectratio]{pa072279.jpg}
\caption*{Kévin, le rebelle aux pieds nus, manifeste son mécontentement.}
\end{figure}


\begin{figure}[h]
\centering
\includegraphics[height=5.5cm,width=9cm,keepaspectratio]{p9200108.jpg}
\caption*{Le lac de Pokhara.}
\end{figure}

A la moitié de l'Alsace : Nous avons été ravis de tous vous voir le temps de ce trek, c'était une excellente idée de tous vous incruster avec Papa, et on espère que vous avez aimé ce bout de voyage autant que nous. Puisse cette expérience être la première d'une longue série !


\begin{figure}[h]
\centering
\includegraphics[height=5.5cm,width=9cm,keepaspectratio]{pa062231.jpg}
\caption*{La bière du retour !}
\end{figure}


\begin{figure}[h]
\centering
\includegraphics[height=5.5cm,width=9cm,keepaspectratio]{pa082283.jpg}
\caption*{Ça fatigue les voyages !}
\end{figure}



\chapter{Un dernier tour au Népal, et puis s'en vont...}
Les Alsaciens repartis, on avait encore quelques jours à passer à Katmandou avant de repartir nous-même. On n'a pas fait grand chose de vraiment digne d'être raconté, mais comme on a pris de jolies photos, voici quand même un petit article.


\begin{figure}[h]
\centering
\includegraphics[height=6cm,width=9cm,keepaspectratio]{pa122555.jpg}
\caption*{Je vooooooooole !}
\end{figure}

C'est ainsi que l'envie d'aller visiter un peu Katmandou nous a pris par surprise. Nous avons donc affronté encore un peu de circulation anarchique, et nous sommes arrivés au Durbar Square, la plus célèbre et principale concentration de temples de la ville. Et c'est aussi un endroit qui a beaucoup souffert du dernier tremblement de terre.


\begin{figure}[h]
\centering
\includegraphics[height=6cm,width=9cm,keepaspectratio]{pa092305.jpg}
\caption*{Je crois que j'ai trouvé le temple des pigeons, et son roi.}
\end{figure}


\begin{figure}[h]
\centering
\includegraphics[height=6cm,width=9cm,keepaspectratio]{pa092309.jpg}
\caption*{Un chien faisant honneur à ses ancêtres loups.}
\end{figure}

De nombreux temples ont été sérieusement endommagés, au point de n'être plus du tout visitables, et certains ont été totalement détruits. On trouve ainsi des enclos autour de l'espace vide ou se trouvait un temple, illustré par une photo montrant le temple avant le tremblement de terre, et une photo des gravats... Même si on a vu à de nombreux endroits les traces du tremblement de terre - de nombreux bâtiments sont encore en réparation, si ce n'est abandonnés - c'est ici qu'on a ressenti le plus fort l'étendu des dégâts.


\begin{figure}[h]
\centering
\includegraphics[height=6cm,width=9cm,keepaspectratio]{pa092292.jpg}
\caption*{Dégâts à Durbar Square.}
\end{figure}

C'était aussi une période de fête pour le Népal : quasiment 15 jours de vacances pour que tout le monde puisse fêter convenablement Dashain. Les temples sont pris d'assaut, les transports sont bondés car c'est le moment de rentrer dans sa famille, et la plupart des commerces réduisent leurs horaires voire ferment complètement. On voit aussi apparaître un peu partout des balançoires en bambou !


\begin{figure}[h]
\centering
\includegraphics[height=6cm,width=9cm,keepaspectratio]{pa122547.jpg}
\caption*{A deux, c'est mieux !}
\end{figure}


\begin{figure}[h]
\centering
\includegraphics[height=6cm,width=9cm,keepaspectratio]{pa092329.jpg}
\caption*{Un temple illuminé pour les fêtes.}
\end{figure}

Elles sont immenses, occupées du matin au soir, et les gamins vont haut, très haut, et les balançoires plient, jusqu'à ce que le ou les balanceurs atteignent ce point de rupture, où la corde est à l'horizontale et donc n'est plus tendue par la gravité : pendant une fraction de seconde, on flotte en l'air, sans appui, avant d'être sèchement appuyé à nouveau sur le siège par la gravité et la corde qui se tend. Ceux qui ont déjà fait cette expérience savent exactement ce que j'essaie de décrire... les autres, ben trouvez une balançoire ! A chaque fois, on les voit monter, monter, jusqu'à ce moment où le cœur semble louper un battement... et ils se laissent aller et laisse leur place au suivant.


\begin{figure}[h]
\centering
\includegraphics[height=6cm,width=9cm,keepaspectratio]{pa132717.jpg}
\caption*{Des étoiles en haut, et en bas !}
\end{figure}

C'est dans la vallée de Katmandou que nous passons nos derniers jours au Népal, à faire du scooter entre Dhulikel et Nagarkot, loin de la circulation anarchique de la ville. Nous avons eu la chambre avec probablement la plus belle vue de tout notre voyage : dans un angle avec deux grandes fenêtres donnant sur une vallée magnifique avec au loin la chaîne de l'Himalaya.


\begin{figure}[h]
\centering
\includegraphics[height=6cm,width=9cm,keepaspectratio]{pa132873.jpg}
\caption*{La vue depuis notre chambre à Dhulikel.}
\end{figure}

Je crois que c'est à ce moment là qu'on s'est le plus rendu compte de la taille de ces montagnes. On sent instinctivement où devrait être la ligne d'horizon, mais les montagnes sont plusieurs kilomètres au-dessus ! C'est très étrange d'en prendre conscience, et malheureusement, cet effet ne prend pas avec une simple photo : allez-y si vous le pouvez, et priez pour que les nuages vous laissent un peu de vue !


\chapter{Astana, la ville champignon dans le désert.}


\begin{wrapfigure}{l}{0.55\textwidth}
\centering
\includegraphics[width=0.5\textwidth]{pa243328.jpg}
\caption*{Et en plus, il neige !}
\end{wrapfigure}


Nous arrivâmes donc à Astana. En avion, malheureusement. Ben oui, si on voulait traverser la Chine depuis le Népal, nous aurions du le faire via un voyage organisé hors de prix, et ça, pas question ! C'est donc après une escale de 19h dans un aéroport trop froid malgré les 35\textdegree C à l'extérieur que nous sommes arrivés dans la nouvelle capitale du Kazakhstan.


Nous commençons par attendre 2h à la douane le consul pour qu'il nous fasse nos visas. En tant que français, nous avons droit à 15 jours sans aucune formalité préalable ! Le consul arrive, et nous dit, tout étonné : "Vous êtes français, vous n'avez pas besoin de visa, vous passez la frontière, on vous met un tampon, et voilà !". Nous : "... euh ... merci !"
Eh oui, deux heures d'attente pour rien !


\begin{figure}[h]
\centering
\includegraphics[height=6cm,width=9cm,keepaspectratio]{pa243302.jpg}
\caption*{Cet immeuble ressemble à un briquet : c'est le siège du géant du gaz kazakh.}
\end{figure}

Et nous sortîmes de l'aéroport pour découvrir ce pays, et quel bonheur ! Il y a des trottoirs ! Pas de moto sur les trottoirs, ils sont réservés aux piétons ! Il n'y a pas de vaches sur la route ! Il n'y a pas de nids de poules ! Il y a des horaires dans les arrêts de bus ! Les gens sont gentils et nous aident naturellement sans essayer de nous vendre des souvenirs/hashish/hôtel ou autre ! Les voitures s'arrêtent pour nous laisser traverser ! Il y a une cuisine dans l'auberge de jeunesse ! Et surtout, et enfin, depuis le temps qu'on l'attendait : il fait FROID ! Oui, je sais, ça a l'air con dit comme ça, mais de pouvoir nous balader tranquillement dans la rue, emmitouflés dans nos manteaux, à regarder la neige tomber, et à se réjouir d'avance pour le thé chaud qu'on va boire en rentrant, ça nous a rendu ivres de bonheur.


\begin{figure}[h]
\centering
\includegraphics[height=6cm,width=9cm,keepaspectratio]{pa243347.jpg}
\caption*{Encore de la neige !}
\end{figure}


\begin{figure}[h]
\centering
\includegraphics[height=9cm,width=12cm,keepaspectratio]{pa233261.jpg}
\caption*{C'est le symbole de la ville.}
\end{figure}

La ville ne nous a pas semblé très vivante. Peut-être que le contraste avec les klaxons et la circulation de Katmandou a joué un rôle dans notre perception, mais quel plaisir, quel calme ! C'est une ville nouvelle, capitale récente, et couverte de nouveaux bâtiments modernes et audacieux qui sont la principale attraction de la ville. Donc après avoir visité tout ça, on s'est dit qu'on ferait bien un peu de cuisine.


\begin{figure}[h]
\centering
\includegraphics[height=9cm,width=12cm,keepaspectratio]{pa243326.jpg}
\caption*{Tim, qui a voulu nous faire gouter une spécialité kazakh : le lait de chamelle fermenté. Je vous laisse imaginer...}
\end{figure}


\begin{figure}[h]
\centering
\includegraphics[height=9cm,width=12cm,keepaspectratio]{pa243293.jpg}
\caption*{La prochaine exposition universelle aura lieu ici. Alors on en profite pour soutenir la France !}
\end{figure}

Mais petit dilemme : dans la guesthouse, la mémé fait de la cuisine pour tout le monde pour une somme dérisoires, c'est très bon, ça fait une grande tablée sympathique et on ne voulait pas faire bande à part, donc nous avons fait un dessert pour tout le monde ! La mémé était inquiète au début de nous voir cuisiner, car elle avait déjà prévu notre repas. On a donc tenté de lui faire comprendre qu'on ne faisait qu'un dessert, et on a appris au passage comment dire "cuisson au bain marie" en russe, et c'est plus facile que ce qu'on pense ! C'est ainsi qu'une bande de Kazakhs ont découvert le flan aux œufs !


\begin{figure}[h]
\centering
\includegraphics[height=9cm,width=12cm,keepaspectratio]{pa243309.jpg}
\caption*{Une yourte ? Non : un centre commercial !}
\end{figure}

Oui, c'est déjà la fin de l'article, mais mieux vaut ça que rien, non ? Plus de photos bientôt !

\chapter{La traversée du Kazakhstan et de la mer Caspienne.}
Nous prîmes ensuite le train pour Almaty. C'est avec beaucoup de plaisir que nous avons retrouvé ces fameux trains couchette qui nous avaient tant plus sur la traversée de la Russie. Une nuit plus tard, nous voilà dans l'ancienne capitale. L'ambiance est totalement différente : il y a plein de vieux bâtiments ! On dirait que c'est une ville qui a un peu plus d'histoire.


\begin{figure}[h]
\centering
\includegraphics[height=6cm,width=9cm,keepaspectratio]{pa283588.jpg}
\caption*{Mangez des pommes !}
\end{figure}

Cela dit, on n'a pas trouvé le centre ville passionnant. Une petite allée piétonne, un marché, une petite cathédrale, et voilà, on a fait le tour. Non, en vrai, ce qui est intéressant à Almaty, ce sont les montagnes aux alentours, donc on a pris nos chaussures de rando (oui bon, ce sont nos seules chaussures en même temps) et on est partis en direction du lac, un peu au hasard. Il faisait beau, on a marché dans la neige, et on n'a croisé qu'une seule personne. Hmmm...


\begin{figure}[h]
\centering
\includegraphics[height=6cm,width=9cm,keepaspectratio]{pa273564.jpg}
\caption*{Geronimooooooooooo !!!}
\end{figure}


\begin{figure}[h]
\centering
\includegraphics[height=6cm,width=9cm,keepaspectratio]{pa263376.jpg}
\caption*{Vous ne trouvez pas que ce pigeon a un air à John Snow ?}
\end{figure}

Non, en vrai, ce qui est intéressant à Almaty, ce sont les restaurants : conseillé par notre guesthouse, nous voilà dans un grand restaurant (grand = beaucoup de places ici) ambiance mardi soir de novembre à La Bourboule : les chaises sont recouvertes de moumoute, une table sur 10 est occupée, et il y a de la musique live pas trop forte ! Et on a mangé des super brochettes, notre premier gros repas de viande depuis ... depuis... oulà, bien tout ça oui ! Mouais...


\begin{figure}[h]
\centering
\includegraphics[height=6cm,width=9cm,keepaspectratio]{pa283581.jpg}
\caption*{Non, cette photo n'a aucun rapport avec le paragraphe précédent.}
\end{figure}

Non, en vrai, ce qui intéressant à Almaty, c'est, euh... attendez, je cherche... ah oui : les pommes ! Non, je vous jure, la légende dit que l'humanité a trouvé les pommes ici, et d'ailleurs le nom de la ville signifie "ville des pommes". C'est bien mis en valeur à l'entrée d'un grand parc avec quelques œuvres d'art. C'est vendeur, non ?


\begin{figure}[h]
\centering
\includegraphics[height=6cm,width=9cm,keepaspectratio]{pa283591.jpg}
\caption*{Le jour ou Apple entend parler de cette ville, je ne vous raconte pas le procès...}
\end{figure}

OK, une dernière tentative : non, en vrai, ce qui est intéressant à Almaty, c'est quand on prend le train pour s'en aller ! Et on n'a pas fait les choses à moitié : 61h, 3 nuits pour traverser tout le pays, et arriver au bord de la mer Caspienne à Aktau. On a donc bien eu le temps de papoter avec nos voisins, même si la barrière de la langue a rendu l'échange un peu laborieux. On a quand même compris que l'argent les intéressait vraiment : combien coute ci en France, et combien coute ça, et un appartement, et combien tu gagnes etc. On a un peu raconté notre voyage, et quand on a parlé de la Chine, notre voisin, un vrai kazakh, a rigolé et mimé un chinois en se bridant les yeux. On n'a pas su comment réagir, car il était déjà lui-même bridé naturellement... Comme quoi, tout est relatif !


\begin{figure}[h]
\centering
\includegraphics[height=6cm,width=9cm,keepaspectratio]{pa303622.jpg}
\caption*{Coucher de soleil depuis le train.}
\end{figure}

En arrivant à Aktau au petit matin, on file directement au bureau des ferrys pour acheter un billet pour traverser la mer Caspienne. On doit mettre toutes les chances de notre coté, car les ferrys sont rares et imprévisibles : certains voyageurs ont attendu plus d'une semaine, tout en étant prévenus à peine 2h à l'avance du départ. Mais pour nous, double coup de bol : on nous dit de revenir le soir même car il y aura probablement un ferry, et en plus, on rencontre un couple de Français qui font le même trajet !


\begin{figure}[h]
\centering
\includegraphics[height=6cm,width=9cm,keepaspectratio]{pa313634.jpg}
\caption*{On n'a pas réussi à se mettre d'accord sur la position des bras.}
\end{figure}

On passe donc la journée à papoter en bonne compagnie (de toute façon, il n'y a rien à faire à Aktau), et le soir, on se dirige vers le port. Il faut d'abord trouver un taxi. Ici, on agite le bras au bord de la route, et n'importe quelle voiture peut s'arrêter et s'improviser taxi. La première voiture à s'arrêter nous demande le double du prix normal, et on n'arrive pas à négocier. Ça attire un passant, qui nous demande si il peut nous aider, et après une courte explication, il se met au bord de la route, arrête une autre voiture, négocie, et... paie pour nous ! Il n'a rien voulu savoir quand on a voulu payer, et nous a souhaité un bon voyage avec un grand sourire !


\begin{figure}[h]
\centering
\includegraphics[height=6cm,width=9cm,keepaspectratio]{pb013639.jpg}
\caption*{La mer Caspienne.}
\end{figure}

Évidemment, une fois au port, il faut encore attendre. A 22h, on nous dit qu'on pourra probablement embarquer vers 1h du matin. On continue de papoter avec nos compagnons de voyage, et on parle du visa pour l’Azerbaïdjan, notre prochaine destination. Eux ont un visa, pas nous. Ils sont surpris quand on leur dit qu'il n'y a pas besoin de visa si on reste moins de trois jours. Ça nous met un doute, et après vérification, il y a un silence gêné. Hmmm. On n'a donc pas de visa. Et... il faut bien un visa... C'est ballot. On faisait les fiers avec nos un an de vagabondage en Asie, mais là, on a l'air bien con quand on ramasse nos affaires pour rentrer en ville au milieu de la nuit.


\begin{figure}[h]
\centering
\includegraphics[height=6cm,width=9cm,keepaspectratio]{pb023643.jpg}
\caption*{Allégorie de notre capacité à entrer en Azerbaïdjan à ce moment là.}
\end{figure}

Du coup, pas le choix, si on veut prendre le ferry, il faut faire le visa. Ça nous prend trois jours, et re-coup de bol : alors qu'on est en route pour le consulat azéri, le bureau des ferry nous appelle pour nous dire qu'un ferry part le soir-même ! On passe un très bon voyage. Marion étant la seule passagère, on bénéficie d'un traitement de luxe en se voyant attribuer la seule cabine privée avec salle de bain ! Hormis deux autres touristes, les passagers sont tous des chauffeurs de poids-lourds qui font des trajets du type "Almaty-Toulon". La mer est calme, la nourriture est bonne, et le seul évènement notable de la traversée, c'est la découverte d'une nuée d'oiseau morts sur le pont. On n'a toujours pas d'explication...


\begin{figure}[h]
\centering
\includegraphics[height=6cm,width=9cm,keepaspectratio]{pa313631.jpg}
\caption*{Allégorie de notre capacité à aller en Azerbaïdjan quelques jours plus tard.}
\end{figure}


\begin{figure}[h]
\centering
\includegraphics[height=6cm,width=9cm,keepaspectratio]{pb064278.jpg}
\caption*{Lui et ses cinquante potes sont aussi arrivés en Azerbaïdjan, mais bon, comment dire...}
\end{figure}

Enfin, on débarque à Baku. Nos visas sont acceptés, et nous voilà sur le sol azéri. L'Europe se rapproche !

.



\chapter{L'Azerbaïdjan, le pays des volcans de boue et des flammes de verre.}
Nous débarquâmes donc à Baku. Après quelques formalités (mais oui, on a bien nos visas, cf épisode précédent), nous voici dans le métro, à essayer de comprendre comment ça marche, quelle carte acheter etc. Les gens nous voient galérer, et plutôt que de nous aider à acheter la bonne carte, ils nous ont payé l'entrée, tout simplement ! Voilà des gens qui savent vendre leur pays !


\begin{figure}[h]
\centering
\includegraphics[height=6cm,width=9cm,keepaspectratio]{pb074849.jpg}
\caption*{Le centre de Baku.}
\end{figure}

Et nous voilà dans le centre de la plus grande ville du pays. Mais ce n'est pas la capitale, d'ailleurs, je crois que personne ne connait la capitale de ce pays. Par contre, c'est la capitale économique, et on sent bien l'argent apporté par les champs de pétrole et de gaz de la mer Caspienne : le centre a l'air tout neuf, tout est propre, même la magnifique vieille ville est toute neuve, à mi-chemin entre la Suisse et un parc d'attraction.


\begin{figure}[h]
\centering
\includegraphics[height=6cm,width=9cm,keepaspectratio]{pb074640.jpg}
\caption*{Le fameux "Carpet Museum": Subtile métaphore architecturale !}
\end{figure}

La ville est dominée par trois tours immanquables en forme de flamme. Évidemment, elle sont illuminées. Évidemment, les illuminations imitent des flammes. C'est classe !


\begin{figure}[h]
\centering
\includegraphics[height=6cm,width=9cm,keepaspectratio]{pb094983.jpg}
\caption*{C'est classe de loin !}
\end{figure}


\begin{figure}[h]
\centering
\includegraphics[height=6cm,width=9cm,keepaspectratio]{pb074829.jpg}
\caption*{De près aussi c'est pas mal !}
\end{figure}

Grâce à notre visa de transit, nous n'avons que 5 jours à passer dans le pays, ce qui nous laisse tout juste le temps de visiter la ville, mais aussi les volcans de boue situés à une heure de bus et quelques minutes de taxi hors de la ville. La partie bus n'a pas posé trop de problèmes, mais la partie taxi en revanche... Un taxi nous attendait à la descente du bus, probablement prévenu par le chauffeur de bus par ailleurs très sympathique. On savait à l'avance que le prix normal pour la balade était d'environ 30 Manats, et c'est armés de cette certitude que l'on s'approche du chauffeur souriant pour entamer les négociations.


\begin{figure}[h]
\centering
\includegraphics[height=6cm,width=9cm,keepaspectratio]{pb084896.jpg}
\caption*{Impressionnant, non ?}
\end{figure}

Alors que le chauffeur nous presse de monter dans son bolide, on lui demande "Combien ?". Il dit 40. C'est trop. On répond "Non, 30". Il nous dit Ok. Super, c'est ce qu'on voulait, et la négociation fut facile, à croire qu'on prend le coup de main ! Le chauffeur est sympa, et malgré son faible niveau d'anglais, l'ambiance dans la voiture est bonne. Il nous offre même des mandarines. Mais plus on avance sur le trajet, plus il insiste sur le fait que le trajet fait 15km (le GPS dit 5km, mais ne chipotons pas), et que la route est mauvaise (c'est vrai, c'est un chemin de terre plein de boue) et plus il fait des mimiques inquiètes à chaque nid de poule en nous faisant comprendre que sa voiture est vieille. On voit bien là où il veut en venir, mais ça ne prend pas. Puis il appelle quelqu'un et nous passe le téléphone : c'est sa fille (selon elle), et elle en revanche parle très bien anglais ! Et elle commence à nous dire que son père ne sait pas parler anglais, qu'il y a eu une incompréhension avec lui, que la voiture est vieille, que les cailloux sont trop durs, que la boue elle est trop molle, que la vie elle est pas facile et que le prix c'est 50 manats. Mouais. Bien sûr. Ça ne sent pas du tout la pièce de théâtre jouée tous les jours avec tous les touristes. On ne cède rien, on tient bon, on se réjouit de n'avoir rien payé en avance et on arrive aux volcans.


\begin{figure}[h]
\centering
\includegraphics[height=6cm,width=9cm,keepaspectratio]{pb084959.jpg}
\caption*{Heureusement, la boue, ça part au lavage.}
\end{figure}

C'est très rigolo. On dirait des projets de collégiens : ils font 30cm à 2m de haut maximum, et crachent une belle boue si lisse et douce qu'on a envie de rouler dedans. Étonnamment, la boue est froide ! Bref, c'est marrant, on prend des photos, on se fait éclabousser par les imprévisibles éruptions et on repart pour la suite. Il y a des pétroglyphes dans le coin, et ça fait partie de tour classique avec les volcans. On remonte dans la voiture, et là, re-coup de fil : maintenant, la fille joue la colère, les pleurs tant et si bien que d'épuisement, on finit par craquer et on augmente le tarif à 35 manats, mais ce n'est pas suffisant pour calmer ni la fille qui est toujours en colère, ni le père qui continue de geindre. D'ailleurs, il a désormais totalement oublié le peu d'anglais qu'il connaissait quelques minutes plus tôt, pour mieux coller à son rôle de victime incomprise...


\begin{figure}[h]
\centering
\includegraphics[height=6cm,width=9cm,keepaspectratio]{pb084905.jpg}
\caption*{:-)}
\end{figure}

Les pétroglyphes, c'est sympa aussi ! Il y a un musée tout neuf plutôt intéressant, qui explique l'histoire du coin. Ou plutôt la préhistoire : Les pétroglyphes, comme leur nom l'indique, sont des gravures sur des rochers, qui datent d'époques diverses allant de l'âge de la pierre aux tout récents Romains : un centurion arrivé dans le coin il y a quelque 2000 ans (à pied donc...), a vu ces gravures, et, tel un lycéen visitant Notre Dame, a gravé dans un coin "Caïus était là". Aucun respect ces envahisseurs...


\begin{figure}[h]
\centering
\includegraphics[height=6cm,width=9cm,keepaspectratio]{pb084980.jpg}
\caption*{Tout ces pétroglyphes, moi, ça me donne surtout envie de grimper !}
\end{figure}

Puis c'est le retour dans le taxi. On paie les 35 manats et personne n'est content : le chauffeur geint, et nous avons l'impression de nous être fait entuber de 5 manats. Un peu plus tard, en revoyant Flavie et Julien, nos presque compagnons de ferry, on compare nos expériences, et on se rend compte qu'on a eu le même chauffeur de taxi, qui a eu exactement la même stratégie. Eux en revanche n'ont payé que 23 manats.

Je crois que c'est définitif : je hais les taxis...



\chapter{La Géorgie, l'autre pays du vin et du fromage.}
Nous partîmes donc pour Tibliss... Tibil... Tsibil... Tbilissi ! Oui, nous aussi on a du s'entrainer à le dire, ne vous inquiétez pas. Et inquiétez-vous d'autant moins qu'une fois sur place, ça n'a plus vraiment d'importance : les locaux parlent tous de "Tifliss". Allez comprendre... Faut dire que niveau langue, le Géorgien, ça en impose : non seulement la langue n'a aucun lien avec aucune des langues des pays voisins, mais en plus l'alphabet est complètement différent. Pour nous, ça ressemblait au Thaïlandais, c'est dire ! Mais mis à part ces menus problèmes de communication, la Géorgie, ça a vraiment été une des bonnes surprises du voyage pour moi.


\begin{figure}[h]
\centering
\includegraphics[height=6cm,width=9cm,keepaspectratio]{pb145173.jpg}
\caption*{Un monastère sur une colline.}
\end{figure}

Déjà, la cuisine géorgienne est excellente ! Ils ont du bon pain, ils mettent du fromage partout, et ils ont une tradition vinicole vieille de près de 9000 ans. Oui, j'ai bien dit neuf mille ans ! Du pain, du vin, du fromage : ces gens savent vivre. Et je n'ai même pas encore parlé du fameux Katchpuri ! Le nom peu vendeur ne nous a pas freinés. C'est simple, on a fait un score parfait : on en a mangé tous les jours sans se lasser, en particulier le fameux parmi les fameux : Katchapuri Adjaruli. C'est un pain fourré au fromage, plié en forme de barque pour faire un contenant, qu'on rempli de fromage, avec un oeuf à cheval, et des morceaux de beurre par dessus. Voilà. Est-ce que j'ai vraiment besoin d'en rajouter ? Si vous avez pris 3kg juste en lisant ce passage, c'est une réaction normale !


\begin{figure}[h]
\centering
\includegraphics[height=6cm,width=9cm,keepaspectratio]{pb104999.jpg}
\caption*{Trouver une légende pour cette photo me fait saliver...}
\end{figure}

On trouve aussi des bains chauds dans un quartier bien précis de la ville, où les cinq établissements, pas un de plus, pas un de moins, sont collés les uns contre les autres. Ils sont vieux comme la route de la soie, et une citation d'Alexandre Pouchkine faisant la fierté de la ville est affichée partout : "Je n'ai jamais rien vu d'aussi beau que les bains de Tbilissi". On peut en déduire que ce pauvre monsieur n'a pas du beaucoup voyager dans sa vie... Faut être réaliste : ça reste des bêtes bâtiments en brique, avec un peu de mosaïque. Pas de quoi en écrire des romans ! Je pense plutôt que ce brave monsieur a un peu trop abusé de l'hospitalité et de la tradition vinicole des géorgiens, et, bourré comme un coing, a dit des trucs qui ne sont pas tombés dans l'oreille d'un sourd. Cela dit, c'est très agréable : pour une somme modique, on dispose d'un bain privé : une première salle pour se changer, ouvrant sur la salle ... euh, ben la salle de bain du coup, contenant une grande baignoire d'eau chaude alimentée en permanence par l'eau du sous sol. Et pour 4\euro de plus, on peut avoir un massage !


\begin{figure}[h]
\centering
\includegraphics[height=6cm,width=9cm,keepaspectratio]{pb1250311.jpg}
\caption*{Marion n'a pas voulu faire le modèle dans le bain. Je la remplace donc.}
\end{figure}

C'est aussi le pays où mon pote Pierre nous a rejoint. Comme il n'a eu que quelques jours à passer en notre compagnie, le rythme s'est sensiblement accéléré : plus question de glander par ci par là, il faut optimiser notre temps, profiter au maximum du pays !




\begin{figure}[h]
\centering
\includegraphics[height=6cm,width=9cm,keepaspectratio]{pb1250421.jpg}
\caption*{Et on a même bu du Cabernet Sauvignon !}
\end{figure}


\begin{figure}[h]
\centering
\includegraphics[height=6cm,width=9cm,keepaspectratio]{pb155638.jpg}
\caption*{En plein centre ville, un architecte inspiré s'est lâché... et le bâtiment semble abandonné !}
\end{figure}

On commence par une escapade à la frontière russe, à Stéphantsminda. Le nom ne vous dit probablement rien (moi, c'est à peine si je savais placer la Géorgie sur une carte...) mais vous avez peut-être déjà vu une photo de ce monastère au sommet d'une montagne.




\begin{figure}[h]
\centering
\includegraphics[height=6cm,width=9cm,keepaspectratio]{pb135078.jpg}
\caption*{C'est le même monastère que la première photo de l'article.}
\end{figure}

Le panorama est exceptionnel. Le mont Kazbek domine toute la vallée, du haut de ses 5000 et quelques mètres. Il fait la frontière avec la Russie. La météo étant clémente, nous sommes montés plus haut que le monastère, pour voir le glacier au pied du mont. Comme la plupart des glaciers, il a fondu, et on ne voit plus que des cailloux, mais ce n'est pas (trop) grave (enfin si, c'est très grave, mais là n'est pas le sujet), la vue est quand même magnifique !




\begin{figure}[h]
\centering
\includegraphics[height=6cm,width=9cm,keepaspectratio]{pb145183.jpg}
\caption*{Quelle vue !}
\end{figure}


\begin{figure}[h]
\centering
\includegraphics[height=6cm,width=9cm,keepaspectratio]{pb145615.jpg}
\caption*{Petit monument trouvé par hasard au bord de la route.}
\end{figure}

Notre seconde escapade nous amène à Signaghi, où nous retrouvons aussi Flavie et Julien. Mais oui, on avais failli faire la traversée de la mer Caspienne avec eux avant de se rendre compte que... hmmm, bref, passons. Ce village est connu pour son vin, et aussi pour le fait qu'il a presque été converti en un grand centre de vacances soviétique.




\begin{figure}[h]
\centering
\includegraphics[height=6cm,width=9cm,keepaspectratio]{pb155649.jpg}
\caption*{Détail de la statue "Mère de la Géorgie".}
\end{figure}


\begin{figure}[h]
\centering
\includegraphics[height=6cm,width=9cm,keepaspectratio]{pb105003.jpg}
\caption*{C'est une spécialité locale : des fruits secs enrobés d'une sorte de pâte de marc de raisin. C'est ... typique !}
\end{figure}

Nous logeons chez M. Shota : Un papy trapu gentil comme un nounours, qui après quelques minutes dans sa maison, nous amène dans son musée personnel : il n'est pas seulement trapu, mais aussi accessoirement champion d'haltérophilie ! Champion du monde même : 340kg en développé couché ! C'est là qu'on prend aussi conscience que dans nounours, il y a ours ! Son musée est rempli de coupes, de médailles et d'articles en tout genre ! Sûrement inquiet de nous voir si maigres, il nous offre sans cesse à manger et à boire : du raisin, des kakis, du vin. Tout vient de sa production personnelle. Le petit déjeuner fut un festin. En arrivant, la table était remplie : fromage, pain, muffin, khinkalis (j'y reviendrai), saucisses, œufs... et à peine une assiette se vidait, qu'il en apportait une nouvelle. A ma grande honte, on n'a pas pu finir. Même pas la moitié. Il nous a obligé à emporter le reste avec nous. Et on ne négocie pas avec un champion du monde d'haltérophilie. On dit oui monsieur, merci monsieur, le petit déjeuner était très bon monsieur !




\begin{figure}[h]
\centering
\includegraphics[height=6cm,width=9cm,keepaspectratio]{pb155663.jpg}
\caption*{Il sait accueillir !}
\end{figure}


\begin{figure}[h]
\centering
\includegraphics[height=6cm,width=9cm,keepaspectratio]{pb155661.jpg}
\caption*{On ne voit qu'un petit bout de son musée !}
\end{figure}

Près de Signaghi, sur la frontière azéri, se trouve un monastère troglodyte incontournable, que nous avons donc visité, si on peut vraiment appeler ça une visite, étant donné que l'on était quasiment privés de vue tellement le brouillard était épais. Le paysage est incroyable en théorie, les grottes ont une vue imprenable sur une grande vallée. Nous, on avait surtout du mal à voir nos pieds...




\begin{figure}[h]
\centering
\includegraphics[height=6cm,width=9cm,keepaspectratio]{pb165742.jpg}
\caption*{Comme il n'y a pas de paysage, on compense comme on peut...}
\end{figure}

Heureusement, le lendemain, nous sommes allés nous perdre dans d'autres grottes troglodytes à Uplistsikhe. Et là, la météo s'est rattrapée au niveau du brouillard ! Beau ciel bleu, quelques nuages dans le ciel, et surtout, un vent à décorner les bœufs ! On a du mal à tenir debout, mais ça fait des timelapses impressionnants.




\begin{figure}[h]
\centering
\includegraphics[height=6cm,width=9cm,keepaspectratio]{pb206333.jpg}
\caption*{Le chat fait tout pour avoir l'air mignon !}
\end{figure}

Juste à coté se trouve la ville de Gori. Ce nom éveillera peut-être de vieux souvenirs de cours d'histoire : c'est la ville natale de ce bon vieux Joseph Vissarionovitch Djougachvili, dit Staline ! On peut donc y visiter sa maison natale, le train dans lequel il cachait sa peur de l'avion, et un grand musée à sa gloire. Je ne vous cache pas qu'il y a quelques trous dans son hagiographie, et qu'on le voit surtout en chef de guerre victorieux et prenant la pose avec des enfants bien nourris.




\begin{figure}[h]
\centering
\includegraphics[height=6cm,width=9cm,keepaspectratio]{martinpecheur.jpg}
\caption*{Rassurez-vous, la plupart des Géorgiens ne sont pas des dictateurs, bien au contraire. Vous remarquerez aussi que j'ai été un peu lent pour la dernière photo...}
\end{figure}

Puis nous sommes allés à Mtskheta (aucun chat n'a marché sur le clavier, j'ai revérifié, c'est la bonne orthographe). Et nous avons visité, je vous le donne en mille : des monastères sur des collines, ainsi qu'une grande église au centre. La spécialité locale, c'est le \emph{lobio}, une soupe de haricot rouge. En vrai, les restaurants en ont rarement. J'imagine que ça fait moins rêver que le Katchapuri, plein de beurre et de fromage ?




\begin{figure}[h]
\centering
\includegraphics[height=6cm,width=9cm,keepaspectratio]{pb186278.jpg}
\caption*{En même temps, c'est beau les églises !}
\end{figure}


\begin{figure}[h]
\centering
\includegraphics[height=6cm,width=9cm,keepaspectratio]{pb186290.jpg}
\caption*{Avec Pierre sous la main, Marion a pu souffler un peu pour les photos.}
\end{figure}

Et puisque je parle de bouffe, revenons au khinkali : c'est l'autre grande spécialité de la Géorgie. De gros raviolis fourrés à un peu ce qu'on veut. Dans les restaurants, on en commande de grands plats à partager. Ensuite, attention à l'étiquette : il faut les saisir fermement par la queue, mais pas trop non plus, sinon on les casse, pour ensuite faire une petite ouverture délicate à la base, et tout de suite aspirer le jus de viande ! Peu importe de faire du bruit, mais il est primordial de ne pas laisser tomber une goutte dans l'assiette. Ensuite, on peut grignoter petit à petit le khinkali, ou bien le dévorer en une bouchée, mais seul un malandrin sans éducation le mangera jusqu'au bout : c'est faire son radin que de manger la queue, il faut la laisser sur le bord de l'assiette avec ses petits camarades, ce qui permet de compter les victimes, et une chose en entrainant une autre, ça amène à faire des concours où personne ne gagne, si ce n'est le restaurateur.




\begin{figure}[h]
\centering
\includegraphics[height=6cm,width=9cm,keepaspectratio]{pb135087.jpg}
\caption*{Tout au bout de la table, il y a une assiette de Khinkalis. Désolé, je n'ai pas mieux...}
\end{figure}

Comme à chaque fois, l'heure du départ arriva, et Pierre du retourner en France. Nous avons alors filé plein ouest, dans le train le plus confortable de tout notre voyage : tout neuf, avec de grands sièges, et surtout, du wifi très rapide et des prises pour l'ordi ! Dans ces conditions, le temps passe encore plus vite... Une petite nuit à Batumi, une ville pleine de casinos sur le bord de la Mer Noire, et nous voilà prêts à rentrer en Turquie !




\begin{figure}[h]
\centering
\includegraphics[height=6cm,width=9cm,keepaspectratio]{pb206746.jpg}
\caption*{Une œuvre d'art à Batumi. Et en plus, ça bouge !}
\end{figure}


\begin{figure}[h]
\centering
\includegraphics[height=6cm,width=9cm,keepaspectratio]{pb206333.jpg}
\caption*{Le chat fait tout pour avoir l'air mignon !}
\end{figure}





\chapter{La Turquie : on arrive en Europe !}
Nous arrivâmes donc en Turquie. Le passage de la frontière se fait sans encombre. Seul évènement notable : après un passage dans le poste de la douane, on me rend mon passeport avec une tache de nourriture. Mais pas n'importe quelle tache : une tache juste sur mon visage, autour de mon œil gauche, comme pour me rappeler un certain sauna russe... Ce passeport, c'est désormais mon portrait de Dorian Gray !


\begin{figure}[h]
\centering
\includegraphics[height=6cm,width=9cm,keepaspectratio]{pb236791.jpg}
\caption*{La mosquée bleue (bleue surtout à l'intérieur nous a-t-on dit).}
\end{figure}

Après un bref passage à Trabzon, sur la côte de la Mer Noire, nous poursuivons directement notre voyage en direction d'Istanbul. Après une nuit dans un bus très confortable et des autoroutes sans nids de poules, nous voilà en Europe et à deux pas de la mosquée bleue !


\begin{figure}[h]
\centering
\includegraphics[height=6cm,width=9cm,keepaspectratio]{pb236799-panorama.jpg}
\caption*{L'intérieur de la mosquée bleue. Oui, je vois ce que vous voulez dire, mais ce n'est pas moi qui l'ai baptisée (d'autant plus que je ne suis pas sûr que ça se fasse de baptiser une mosquée)}
\end{figure}

Cette dernière fait face à la cathédrale mosquée Sainte Sophie, et ensemble elles dominent complètement le quartier de Sultanamet. C'est ici la vieille ville, et donc c'est ici qu'étaient successivement Byzance puis Constantinople ! L'histoire est présente à chaque coin de rue.


\begin{figure}[h]
\centering
\includegraphics[height=6cm,width=9cm,keepaspectratio]{pb246853.jpg}
\caption*{Une jolie pièce dans le palais de Topkapi.}
\end{figure}

Nous visitons quelques mosquées, dont la mosquée bleue, mais Hagia Sophia (Sainte Sophie) est vraiment trop chère, nous l'admirons donc de l'extérieur. Par contre, la Citerne Basilique vaut vraiment l'entrée : il s'agit d'un ancien et énorme réservoir d'eau datant de Byzance ! Il a été oublié pendant des siècles, avant qu'un archéologue se demande d'où les habitants d'un quartier tiraient leur eau. En descendant dans un des puits, il découvre une salle énorme pleine de piliers, façon mines de la Moria inondée. Évidemment, la salle était presque entièrement envasée, mais après le nettoyage de quelques centaines de tonnes de boue, ça ressemblait à ça.


\begin{figure}[h]
\centering
\includegraphics[height=6cm,width=9cm,keepaspectratio]{pb256877.jpg}
\caption*{Ok, peut-être que l'éclairage n'est pas d'origine, mais quand même !}
\end{figure}

En prime, ils ont trouvé des têtes de méduse à la base de deux piliers ! Imaginez un peu l'émotion des découvreurs : trouver un bâtiment aussi beau et aussi grand, en plein centre d'une grande ville !


\begin{figure}[h]
\centering
\includegraphics[height=6cm,width=9cm,keepaspectratio]{pb256872.jpg}
\caption*{Une des deux têtes de méduse.}
\end{figure}

Nous visitons aussi le palais de Topkapi. Notre partie préférée fut la cuisine ! Alors que les autres pièces sont surtout des belles pièces dans lesquelles il ne se passe rien, la cuisine est un lieu toujours en ébullition ! Elle nourrissait entre dix et quinze mille personnes, alors je n'ose imaginer la corvée épluchage de patates. Les explications sont fournies sur l'importance de cet endroit, l'organisation des repas, comment les plats ont évolué au contact avec le monde occidental, avec par exemple l'apparition de la fourchette !


\begin{figure}[h]
\centering
\includegraphics[height=6cm,width=9cm,keepaspectratio]{pb246832.jpg}
\caption*{Casseroles pour une armée !}
\end{figure}


\begin{figure}[h]
\centering
\includegraphics[height=6cm,width=9cm,keepaspectratio]{pb246840.jpg}
\caption*{Un arbre ? Plus ? Vous voyez quelqu'un ? Regardez mieux !}
\end{figure}

Et la nourriture n'est pas importante que dans les musée ! De notre coté, on mange beaucoup de Kebab, qui bien entendu n'ont pas grand chose à voir avec ceux qu'on trouve en France. On mange des baclavas qui existent dans d'innombrables variations, certains salons de thé ont des menus de 10 ou 15 pages uniquement pour ça ! Il ne faut pas oublier non plus les loukoums. Au début, ce fut l'échec : les premiers achetés au hasard dans un attrape touriste se sont avérés tellement immangeables qu'on a jeté la boîte et même - c'est dire - recraché le demi loukoum. Celui à la menthe donnait l'impression de manger un gros bloc de dentifrice... Après quelques recherches, on trouve enfin des loukoums dignes de ce nom dans une boutique centenaire : fondants, goutus, et surtout... nourrissants !


\begin{figure}[h]
\centering
\includegraphics[height=6cm,width=9cm,keepaspectratio]{pb256883.jpg}
\caption*{Hmmmm !!!!}
\end{figure}

Nous retrouvons aussi Berfu, une Turque que nous avions rencontrée en Inde ! Elle et son amie nous ont fait découvrir les quartiers plébiscités de la jeunesse stambouliote (oui, ça fait bizarre, mais istamboulaise c'est pire...). Nous avons eu droit à un excellent café turc (à ne surtout pas remuer avant de boire), et ce qui est bien avec le café turc, c'est quand il est fini, car on peut attaquer la divination !


\begin{figure}[h]
\centering
\includegraphics[height=6cm,width=9cm,keepaspectratio]{pb256905.jpg}
\caption*{Je vois... je vois.... une girafe !}
\end{figure}

Elles nous ont aussi raconté comment la Turquie était en train de changer en ce moment : c'est devenu officiellement une dictature. Critiquer le gouvernement sur Internet, ou trop proches d'oreille indiscrètes peut vous amener directement en prison. Même un truc aussi anodin que courir dans la rue amène des policiers à vous arrêter car c'est comportement louche. En même temps, la montée de l'islamisme fait que de plus en plus, une fille est mal vue si elle se promène seule etc. Bref, le futur s’assombrit, alors qu'il n'y a pas si longtemps, la Turquie considérait de peut-être se rapprocher de l'UE. Quel changement de trajectoire !


\begin{figure}[h]
\centering
\includegraphics[height=6cm,width=9cm,keepaspectratio]{pb266919.jpg}
\caption*{En Turquie, mollesse interdite !}
\end{figure}

Notre trajectoire à nous, nous amène ensuite quelques jours à Budapest (quelle transition incroyablement souple, n'est-ce pas ?). Nous avons dormi dans une petite auberge de jeunesse sans jamais rencontrer le propriétaire : la porte s'ouvre par un code, la chambre est déjà prête et la clé est sur la porte, et quand on part, on laisse le paiement dans une boite au lettre rouge près de la porte ! Pourquoi s'embêter ? Nous avons mangé du goulasch servi dans une miche de pain, visité les marchés de Noël, et cuisiné dans l'auberge de jeunesse où nous étions seuls la plupart du temps. On ralentit un peu le rythme car quelques jours plus tard, ça va accélérer...


\begin{figure}[h]
\centering
\includegraphics[height=6cm,width=9cm,keepaspectratio]{pb256917.jpg}
\caption*{La mosquée bleue de nuit. Eh oui, encore loupé !}
\end{figure}





\chapter{Prague, la dernière pièce du puzzle}
Nous arrivâmes donc à Prague, après une traversée express de l'Europe ! Mais pourquoi tant de précipitation ? La réponse est simple : il manquait la dernière pièce du puzzle. Faites le compte (et relisez le blog), pendant le voyage, Marion avait reçu la visite de son ... et vous voyez maintenant là où je veux en venir ! C'est donc à Prague que ma mère nous a rejoint pour un dernier petit bout de voyage en notre compagnie, accompagnée par mon oncle et ma marraine !


\begin{figure}[h]
\centering
\includegraphics[height=6cm,width=9cm,keepaspectratio]{pc027679.jpg}
\caption*{C'est Noël !}
\end{figure}

Pour ceux qui ne le savaient pas, Prague est une des plus belle ville de l'Europe. Et même si l'exotisme de l'Asie est maintenant loin derrière nous, nous avons été vraiment surpris. On nous avait dit, et nous avions lu que cette ville était vraiment magnifique, mais malgré nos attentes forcément un plus élevées, on a trouvé que c'était encore plus beau que ce qu'on imaginait.


\begin{figure}[h]
\centering
\includegraphics[height=6cm,width=9cm,keepaspectratio]{pc037001.jpg}
\caption*{Ma maman !}
\end{figure}

Le fait d'être en décembre rend la ville encore plus intéressante, avec les décorations et marchés de Noël, et à part le week-end, les rues ne sont pas trop envahies de touristes. Mais un peu quand même. Bon, OK, j'avoue, c'est blindé de touristes, et on se croirait parfois pendant la fête des lumières à Lyon, mais par moment on peut se balader dans les rues sans slalomer entre les (autres) touristes.


\begin{figure}[h]
\centering
\includegraphics[height=6cm,width=9cm,keepaspectratio]{pc026960.jpg}
\caption*{Du cristal de Bohème.}
\end{figure}

Gérard était très curieux de visiter la ville, car c'est la deuxième fois qu'il vient. La première fois, c'était avant la chute du mur... Autant dire que la ville était méconnaissable. Aucun magasin, presque personne dans les rues, tous les murs couleur poussière de charbon... Et par chance, on a retrouvé l'hôtel (les trois autruches) dans lequel il avait dormi, juste à coté du fameux pont Charles !


\begin{figure}[h]
\centering
\includegraphics[height=6cm,width=9cm,keepaspectratio]{pc067174.jpg}
\caption*{Gérard (mon oncle) et les trois autruches.}
\end{figure}

Ce pont Charles est un très vieux pont, couvert de statues posées par les catholiques pour essayer de convertir les protestants qui traversaient le pont tous les jours. A présent, elles servent surtout de perchoirs à mouette et agrémentent les photos de coucher de soleil.




\begin{figure}[h]
\centering
\includegraphics[height=6cm,width=9cm,keepaspectratio]{pc087210.jpg}
\caption*{Aucun respect !}
\end{figure}


\begin{figure}[h]
\centering
\includegraphics[height=6cm,width=9cm,keepaspectratio]{pc067183.jpg}
\caption*{Coucher de soleil sur le pont.}
\end{figure}


\begin{figure}[h]
\centering
\includegraphics[height=6cm,width=9cm,keepaspectratio]{pc037028.jpg}
\caption*{Le pont Charles de nuit}
\end{figure}

Cela dit, on sent que la religion a été (est encore ?) très importante ici, ne serait-ce que par la profusion d'églises magnifiques ! Il y a du gothique comme on peut en voir pas mal en France, et surtout beaucoup de baroque : le gothique en met plein la vue avec des colonnes, des voutes et des arcs-boutant, et c'est aussi impressionnant à l'intérieur qu'à l'extérieur.


\begin{figure}[h]
\centering
\includegraphics[height=6cm,width=9cm,keepaspectratio]{pc057086.jpg}
\caption*{La cathédrale Saint Guy : c'est beau dehors ! (avec Gérard, Adrienne ma marraine, maman, et moi)}
\end{figure}


\begin{figure}[h]
\centering
\includegraphics[height=6cm,width=9cm,keepaspectratio]{pc057103.jpg}
\caption*{La cathédrale Saint Guy : c'est beau dedans !}
\end{figure}


\begin{figure}[h]
\centering
\includegraphics[height=6cm,width=9cm,keepaspectratio]{pc057100-panorama.jpg}
\caption*{Et il y a des magnifiques vitraux d'Alfons  Mucha, un des fers de lance de l'art nouveau.}
\end{figure}

Le baroque, de son coté, c'est surtout à l'intérieur que ça se joue. La plupart des façades ne laissent rien présager de l'intérieur, et sans un guide pour donner l'information, on passerait devant sans même tourner la tête. Mais quand on sait, et qu'on entre, on en prend plein les mirettes ! Après, faut aimer le style, car le moins qu'on puisse dire, c'est que c'est chargé ! Il y a des angelots et des dorures à foison, et plus c'est alambiqué, mieux c'est ! Les grandes plaques de marbre qui couvrent les colonnes de l'église Saint Nicolas nous ont vraiment impressionné, jusqu'à ce qu'on comprenne que c'est de l'imitation en stuc.


\begin{figure}[h]
\centering
\includegraphics[height=6cm,width=9cm,keepaspectratio]{pc067148-panorama.jpg}
\caption*{C'est chargé, non ?}
\end{figure}


\begin{figure}[h]
\centering
\includegraphics[height=6cm,width=9cm,keepaspectratio]{pc067162.jpg}
\caption*{Ce stuc est tellement réputé que l'artiste qui l'a réalisé a un plaque à son nom dans l'église Saint Nicolas.}
\end{figure}


\begin{figure}[h]
\centering
\includegraphics[height=6cm,width=9cm,keepaspectratio]{pc067165.jpg}
\caption*{Un visiteur dans l'église Saint Nicolas.}
\end{figure}

Et il n'y a pas que le marbre : les bougies aussi sont désormais des imitations. Il faut mettre une pièce dans une machine pour illuminer des bougies électriques pendant 15mn. Je comprends l'intérêt au niveau de la suie, des dangers d'incendie, et même des émissions de CO2, mais au fond de moi, mon petit cœur de pyromane est triste.


\begin{figure}[h]
\centering
\includegraphics[height=6cm,width=9cm,keepaspectratio]{pc057110.jpg}
\caption*{Ils auraient au moins pu mettre un peu de variété dans la forme des bougies.}
\end{figure}

Heureusement les trdelniks (ça se prononce "trdelniks", facile !) sont là pour nous remonter le moral. Ce sont des sortes de brioches cuites au barbecue : on peut donc se réchauffer de l'extérieur pendant la cuisson avant de se réchauffer aussi de l'intérieur en les mangeant !


\begin{figure}[h]
\centering
\includegraphics[height=6cm,width=9cm,keepaspectratio]{pc026955.jpg}
\caption*{Les trdelniks en pleine cuisson !}
\end{figure}

Évidemment, il n'y a pas que de la pâtisserie à manger : la cuisine tchèque est riche en choux, patate et viande ! C'est bien pour nous, car le choc du retour en Alsace sera moins rude :-) On mange donc de la choucroute, du canard, de la charcuterie, et aussi des travers de porc mémorables. On n'avait pas lu les petites lignes sur le menu qui disaient "1kg". Je me suis donc retrouvé avec 1kg de barbaque devant moi. Et bien à ma grande honte, et, si mes souvenirs sont justes, pour la première fois du voyage, je n'ai pas réussi à finir mon assiette... Pour vous achever, le prix de ce plat : 7\euro !


\begin{figure}[h]
\centering
\includegraphics[height=6cm,width=9cm,keepaspectratio]{pc036981.jpg}
\caption*{Bon appétit, et bon courage !}
\end{figure}

Nous avons aussi profité de la ville pour ses concerts de musique classique. On a commencé en douceur par un petit concert dans une jolie église pas chauffée, et ensuite, conseillé par un guide qui nous a fait découvrir la ville, nous avons trouvé le bureau qui vend les billets de concerts que les habitants de Prague vont voir. La qualité monte d'un cran : nous avons vu l'orchestre symphonique de Prague, dirigé par José Cura : une centaine de musicien dans la plus grande salle de spectacle de la ville ! En arrivant à la salle de concert le soir, on a quand même eu un moment de doute en voyant les autres spectateurs en tenue de gala, alors que nous étions en tenue de chaussure de marche et polaire portée tous les jours depuis plus d'un an... Le concert fut génial, et le chef d'orchestre s'est même permis une petite fantaisie en laissant un percussionniste diriger le Boléro de Ravel à la caisse claire, tandis que lui-même allait s'asseoir dans le public. Il nous a confié ensuite qu'on lui avait conseillé de ne pas faire ça, car on saurait alors qu'il ne servait à rien. Mais il s'en moque, il a trente ans de carrière et plus rien à prouver ! En tout cas, le percussionniste, dont c'était le dernier jour avant la retraite, était fier comme un paon !


\begin{figure}[h]
\centering
\includegraphics[height=6cm,width=9cm,keepaspectratio]{pc047080.jpg}
\caption*{Le premier concert.}
\end{figure}
\begin{figure}[h]
\centering
\includegraphics[height=6cm,width=9cm,keepaspectratio]{pc036986.jpg}
\caption*{PC036986.jpg}
\end{figure}La maison dansante de Frank Gehry


\begin{figure}[h]
\centering
\includegraphics[height=6cm,width=9cm,keepaspectratio]{pc037019.jpg}
\caption*{Encore une petite statue du pont Charles pour la route.}
\end{figure}


\begin{figure}[h]
\centering
\includegraphics[height=6cm,width=9cm,keepaspectratio]{pc047069.jpg}
\caption*{Le cimetière juif, où des milliers de personnes sont enterrées sur plusieurs couches.}
\end{figure}


\begin{figure}[h]
\centering
\includegraphics[height=6cm,width=9cm,keepaspectratio]{pc087269.jpg}
\caption*{Dédicace à Pierre !}
\end{figure}

Après ces quelques jours à Prague, nos visiteurs s'en retournent chez eux, et il ne nous reste alors plus que quelques jours à passer seuls tous les deux. Nous visitons Cesky Krumlov, un petit village médiéval dans le sud de la République Tchèque. On s'octroie ainsi nos derniers jours de repos, avant de prochainement rentrer pour de bon en France et d'enchainer les Noëls en famille. (Oui, pour ceux qui ne l'ont toujours pas compris, ce blog est en retard...)







\chapter{Retour}
\emph{Petit préambule : oui, j'ai traîné pour ce dernier article, mais votre patience va enfin être récompensée...}

Nous partîmes donc en direction... de la France ! Oui, pour de vrai ! Un direct Prague Mulhouse en bus. Nous sommes arrivés au milieu de la nuit, et mon frère Kévin nous attendait. C'était déjà lui, qui, un an deux mois et six jours plus tôt nous avait déjà amenés au milieu de la nuit à ce même arrêt de bus : la boucle est définitivement bouclée !


\begin{figure}[h]
\centering
\includegraphics[height=6cm,width=9cm,keepaspectratio]{pc157697.jpg}
\caption*{Dernière bière avant de rentrer en France !}
\end{figure}

On ne se sait pas trop ce qu'on ressent, tout se mélange... Nous sommes un peu tristes que ce soit fini, et en même temps contents d'être rentrés, surpris de se sentir chez soit aussi vite. Tout nous semble tellement normal et habituel, que nous avons presque l'impression de n'avoir jamais quitté la France. Et pourtant nous avons la tête tellement pleine de souvenirs que le cerveau déborde.

Nous avons lu que le retour après un grand voyage pouvait être difficile. Avoir vécu tellement de choses, avoir changé au point d'avoir l'impression d'être devenu quelqu'un d'autre, pourrait, ou plutôt aurait pu, rendre difficile la confrontation avec une France restée la même. Mais nous, pour l'instant, ça va.

Le tout premier matin, nous partons à la chasse au pain au chocolat. Ce qui nous frappe, c'est cette quantité de gens qui parlent français ! On comprend tout ! Notre cerveau qui a été à l’affût, voire en manque de notre langue maternelle, qui s'est montré capable d'identifier la moindre syllabe de français dans une foule, est soudain saturé. Nous prenons conscience de toutes les conversations de la rue avec une acuité jamais ressentie auparavant. Nous sommes déconcentrés au point d'avoir du mal à suivre notre propre conversation : Imaginez Doug, le chien de Là-haut (Pixar), soudainement plongé dans un élevage d'écureuils.


\begin{figure}[h]
\centering
\includegraphics[height=6cm,width=9cm,keepaspectratio]{doug.jpg}
\caption*{Pain au chocolat !}
\end{figure}

Petit moment de flottement au moment de commander les fameux pains au chocolat dans la boulangerie : "I would like...euh... Je voudrais deux pains au chocolats s'il vous plait". Je me sens handicapé dans ma propre langue. Je ne me moquerai plus jamais de Jean-Claude Van Damme et de ses anglicismes de snob/mec qui a passé trop de temps à l'étranger. Il nous faudra encore quelques jours pour arrêter de dire "sorry", plutôt que "pardon" aux gens dans la rue.

Nous passons quelques jours dans ma famille en Alsace. Nous fêtons Noël 3 ou 4 fois tout en tentant de limiter les excès... J'ai bien dit "tenté", pas "réussi". On parle quand même de NOËL en ALSACE, après 14 mois de riz et de nouilles ! Personnellement, j'ai eu le niveau de volonté d'un enfant de 3 ans dans un magasin de bonbons.


\begin{figure}[h]
\centering
\includegraphics[height=6cm,width=9cm,keepaspectratio]{pc277732.jpg}
\caption*{Ça, ça nous avait manqué...}
\end{figure}

Puis il est temps d'aller enfin retrouver la famille de Marion ! La première étape sera chez mamie Josette, la grand mère de Marion. Histoire de faire une petite surprise, on arrive un jour plus tôt que ce qu'on lui avait dit... Cette petite surprise de dernière minute amplifie les émotions, mais on survit aux embrassades, avec tout de même les yeux un peu humides...

Et c'est reparti pour Noël ! Et oui, la famille de Marion aussi nous attendait ! Là encore, on a tenté de limiter les dégâts, mais.... NOËL en ARDÈCHE ! Bref, aucune illustration n'est nécessaire à mon avis.

Nous avons aussi eu la joie/l'angoisse de retrouver toutes nos affaires. En se retrouvant au milieu de tous ces cartons, je me sens dépassé par les événements. Il va nous falloir des sacrés sacs dos ! Mais ça fait du bien d'enfiler autre chose que les deux mêmes sempiternels T-shirts. Quoiqu'avoir trop de choix fait parfois perdre plus que temps de nécessaire au moment de s'habiller : c'est ce qui s'appelle avoir des problèmes de riche !


\begin{figure}[h]
\centering
\includegraphics[height=6cm,width=9cm,keepaspectratio]{p1077966.jpg}
\caption*{Marion : Mon chapeau !!!!}
\end{figure}


\begin{figure}[h]
\centering
\includegraphics[height=6cm,width=9cm,keepaspectratio]{p1077968.jpg}
\caption*{Gérald : Ça ne rentrera jamais dans mon sac !}
\end{figure}

Bon, c'est pas tout de se faire gâter, après toutes ces retrouvailles, il est temps de recommencer une vie normale : \textbf{et demain, on se pose où ?}


\begin{figure}[h]
\centering
\includegraphics[height=6cm,width=9cm,keepaspectratio]{p2268335.jpg}
\caption*{Ceci est un indice pour nous trouver à présent.}
\end{figure}

\tableofcontents
\end{document}
