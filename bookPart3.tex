
\documentclass{book}
% Chargement d'extensions
\usepackage[francais]{babel} % Pour la langue française
\usepackage[utf8]{inputenc}
\usepackage[T1]{fontenc}
\usepackage{float}
\usepackage{eurosym}
\usepackage{textcomp}
\usepackage{wrapfig} % to wrap figure in text
%\usepackage{caption} % to remove fig #
\usepackage[font=it]{caption}
\usepackage{titlesec} % to modify chapter header
\usepackage[normalem]{ulem}
\usepackage{soul}

\usepackage{subcaption} % to use subfigure

\titleformat{\chapter}[display]
  {\normalfont\bfseries}{}{0pt}{\Large}

\usepackage[
  paperwidth=15.557cm,
  paperheight=23.495cm,
  %showframe,
  margin=15mm
  % other options
]{geometry}

\hyphenation{CRRR-RA-A-A-A-AK-K-K-K-K-KR-R-R-RB-B-B-B-BRA-A-A-A-A-A-AOUM-M-M-M-M-MR-R-R-RM-M-M-ML-L-L-LLLLL}

%\usepackage{layout} %TODO remove

\usepackage{graphicx} % Pour les images
\graphicspath{ {Images/petit/} }
%\graphicspath{ {Images/} }

\title{Et demain on va où ?}
\author{Gérald Schmitt \and Marion Abrial}

\begin{document}
%\layout
\maketitle




\chapter{Arrivée... et traversée de la Thaïlande, direction la plongée}
Nous quittâmes donc le Laos. D'abord parce que notre visa arrive à expiration, et ensuite parce que... parce qu'on envie de faire autre chose ! On aurait bien voulu partir avant, mais pendant la fête de l'eau, impossible de circuler. Après 4 jours de bataille d'eau et de musique à fond, les pistolets à eau sont rangés, et les bus ressortis, et nous voilà enfin en route vers la Thaïlande.


\begin{figure}[h]
\centering
\includegraphics[height=6cm,width=9cm,keepaspectratio]{p1060405.jpg}
\caption*{Alors comme on n'a pris aucune photo du trajet ou de Bangkok, on va tout illustrer avec des photos sous-marines (ce sont des vers "sapin de noël").}
\end{figure}

Le passage de la frontière est facile, même s'ils ont des taxes supplémentaires pour les jours fériés (non non ce n'est pas de la corruption, c'est obligatoire).
Notre arrivée en Thaïlande a commencé par un petit autocollant que l'on nous a collé sur le t-shirt (il ne faudrait pas que l'on se perde tout de même) et puis une gentille dame nous a conduit jusqu'au bus. C'est bien organisé la Thaïlande, ils ont l'habitude des touristes un peu hagards. Même si la moitié de notre petit groupe a disparu entre la frontière et le bus...?


\begin{figure}[h]
\centering
\includegraphics[height=6cm,width=9cm,keepaspectratio]{p1060410.jpg}
\caption*{Attention : trop de snorkelling peut faire pousser la moustache !}
\end{figure}

Petite déception quand même en entrant dans le bus : point de couchette, juste des sièges inclinables. Quelques heures de mauvais sommeil plus tard, nous voilà donc au petit matin, mal réveillés, dans la banlieue d'une ville immense. Après quelques mois loin des grandes agglomérations, et bien ça fait étrangement du bien. On a été très surpris, mais le fait de retrouver un métro, des centres commerciaux, ça nous a fait du bien les premières heures. Ensuite, les embouteillages, le bruits, les odeurs se rappellent aussi à notre bon souvenir... Nous sommes repartis le soir même en direction des îles du sud, sans même passer une seule nuit en ville. On a donc enchainé deux bus de nuit, c'est vous dire la motivation des troupes !


\begin{figure}[h]
\centering
\includegraphics[height=6cm,width=9cm,keepaspectratio]{p1060426.jpg}
\caption*{Ça vaut le coup de se dépêcher, non ?}
\end{figure}

Oui, nous avons fait complètement l'impasse sur le nord de la Thaïlande. Laissez moi vous expliquer pourquoi : ça faisait 4 mois en gros qu'on visitait des petits villages, des temples, des grottes et des cascades dans la forêt. Et on trouve dans le nord de la Thaïlande... (suspense à deux balles) des petits villages, des temples, des grottes et des cascades dans la forêt. Bref, après un plongeon profond dans le regard l'un de l'autre, nous nous sommes rendu compte à temps qu'on avait tous les deux envie de plage, et nous avons changé nos plan pour foncer vers la mer à toute vitesse.


\begin{figure}[h]
\centering
\includegraphics[height=6cm,width=9cm,keepaspectratio]{p4284283-modifier.jpg}
\caption*{De nuit aussi c'est pas mal.}
\end{figure}

Nous arrivâmes donc à Koh Tao, une petite île située dans le golfe de Thaïlande. C'est l'île de la plongée par excellence ! Il y a plus de 70 clubs de plongée, et la concurrence tire les prix vers le bas, aussi bien pour les fun-dive (plongée pour le plaisir) que pour les formations. Après une petite piqure de rappel (brancher ça, là, et ça, ici, mettre le truc dans le machin et ensuite vérifier que quand on appuie sur ce truc, ça fait des bulles !), et quelques fun dives, on attaque la formation advanced ! A l'issue de cette formation, on ne sera plus juste des poussins qui suivent sagement la maman canard jusqu'à 18m de profondeur : non non non, on sera désormais des poussins qui suivent sagement la maman canard jusqu'à 30m de profondeur !


\begin{figure}[h]
\centering
\includegraphics[height=6cm,width=9cm,keepaspectratio]{g0181052.jpg}
\caption*{C'est nous avec Jérôme, notre maman canard !}
\end{figure}

Autour de Koh Tao, il y a déjà des plongées très sympathiques à faire : nous qui n'avions plongé qu'en Méditerranée, nous avons découvert un monde de corail et de poissons colorés. Mais... pour être tout à fait honnête, on en attendait un peu plus. Les émissions de Cousteau ont peut-être mis la barre un peu haut dans nos esprits ? C'est pourquoi nous sommes allés à Sail Rock : un rocher de deux mètres perdu au milieu de l'océan, à deux heures de bateau des côtes. Et là, on a vu du poisson. Du gros, et plein. Au tout début de la plongée, Marion, qui fermait la marche, enfin, la palme, nous la tire (la palme donc) pour attirer notre attention sur un mérou. On venait de passer un mètre au dessus sans le voir. Un mérou... LE mérou ! Deux mètres de long ! La taille d'un gros veau ! Ensuite, ce fut le festival : des bancs de poissons argentés énormes, des bat-fishs, des barracudas... Là, on a commencé à se croire vraiment dans un Cousteau !


\begin{figure}[h]
\centering
\includegraphics[height=6cm,width=9cm,keepaspectratio]{gopr1076.jpg}
\caption*{Un petit mérou.}
\end{figure}

On peut aussi faire des plongées de nuit ! Alors déjà que la plongée, ce n'est pas un truc de claustrophobique, je vous laisse imaginer de nuit... Et comme je ne suis pas sûr que vous imaginiez comme il faut, je vais aussi raconter cette plongée : on part à la tombée de la nuit, et on se met à l'eau quand il commence à faire sombre. Une fois dans l'eau, la lampe est indispensable sinon, on ne voit rien. Si on tourne la tête, sans tourner la lampe, et qu'il n'y a aucun autre plongeur dans le petit champ de vision que nous laisse le masque, on se retrouve dans un monde noir d'encre (et pas que de seiche), on ne sait plus où sont le haut et le bas, le manque de repères visuels donne le vertige. Pour ne rien simplifier, ce soir là, il y avait pas mal de courant et une visibilité un peu moins bonne. Bref, on s'accroche fort à la lampe, et on ne quitte notre guide des yeux que le temps de regarder un poisson, et encore, pas trop longtemps. Mais alors, pourquoi donc plonger de nuit ? Parce que les poissons ne sont pas les mêmes : on en voit certains dormir, d'autres chasser, et on a vu un énorme bernard-l'hermite, qui trainait une coquille de presque 30cm ! Et on plonge aussi de nuit à cause de l'ambiance. L'impression d'être un étranger est encore plus forte que de jour. Des ombres, des reflets se manifestent en permanence dans le coin de l’œil, mais le temps de tourner la lampe, plus rien...


\begin{figure}[h]
\centering
\includegraphics[height=6cm,width=9cm,keepaspectratio]{p1060623.jpg}
\caption*{Le fameux poisson chirurgien !}
\end{figure}

Il y a aussi le snorkelling (masque/tuba). Plusieurs avantages à la plongée : c'est gratuit, quantité d'air inépuisable, et on est silencieux, donc on fait moins peur à certains poissons, en particulier aux requins. On a ainsi pu voir de nombreux petits requins à pointe noire à quelques mètres à peine du rivage ! Parfois, jusqu'à 5 ou 6 d'affilée. On a beau savoir qu'ils sont inoffensifs, on a tendance à se sentir tout nu quand ils se montrent un peu curieux.


\begin{figure}[h]
\centering
\includegraphics[height=6cm,width=9cm,keepaspectratio]{p1060394.jpg}
\caption*{Ça fait bizarre de nager à coté d'un requin, même un petit...}
\end{figure}

Ah, et j'ai failli oublier de vous parler de la température : il fait chaud. Et la mer aussi est chaude : 30\textdegree C de manière uniforme, même à 20m sous l'eau ! En fin de journée, dans les baies peu profondes, l'eau chauffée toute la journée en devient même inconfortable en montant à 36 ou 37\textdegree C, c'est impossible de se rafraichir en se baignant... On n'a vraiment pas une vie facile !


\begin{figure}[h]
\centering
\includegraphics[height=5cm,width=9cm,keepaspectratio]{p1060458.jpg}
\caption*{Le nom de ce coquillage est "boring clam", qu'on peut traduire par "moule ennuyeuse". Ça ne s'invente pas !}
\end{figure}


\begin{figure}[h]
\centering
\includegraphics[height=5cm,width=9cm,keepaspectratio]{p1060483.jpg}
\caption*{Le poisson perroquet, qui passe sa journée à ronger du corail et à chier du sable : jusqu'à 3kg par jour et par individu. Sans eux, pas de plage de sable blanc !}
\end{figure}


\begin{figure}[h]
\centering
\includegraphics[height=5cm,width=9cm,keepaspectratio]{p1060549.jpg}
\caption*{Le poisson demoiselle, qui est tellement territorial qu'il attaque tout ce qui passe, même un humain environ mille fois lourd !}
\end{figure}




\chapter{Koh Chang en famille}
Nous partîmes finalement de Koh Tao, un peu tristes de quitter cette île très agréable, mais tout excités à l'idée de retrouver la maman de Marion ! Nous les retrouvons dans leur hôtel (un vrai hôtel, pas une guesthouse !), et c'est parti pour une petite semaine en leur compagnie !


\begin{figure}[h]
\centering
\includegraphics[height=6cm,width=9cm,keepaspectratio]{p5087096.jpg}
\caption*{Jongleurs de feu sur la plage.}
\end{figure}

Le lendemain, direction l'île de Koh Chang, en voiture privée, s'il vous plait ! On arrive ensuite dans un vrai centre de vacances : deux grandes piscines, une chambre immense avec tout le confort moderne, un accès direct à la plage, et la cerise sur le gâteau (en tout cas pour moi) : le buffet petit déjeuner ! 20m de long, avec du vrai pain, des céréales, des œufs sous toutes leurs formes, des nouilles, du bacon, des dim-sums, du muesli, de la salade de fruits... J'avais l'impression de ne jamais avoir assez faim ! Chaque soir, je m'endormais en pensant déjà avec quelle stratégie j'allais attaquer le buffet du lendemain. Est-ce qu'il valait mieux attaquer par le lourd : œuf/bacon, et finir sur du muesli, où bien attaquer en douceur par une salade de fruit, et monter en régime progressivement ? Que de choix, que de goûts, que de combinaisons possibles !


Malgré un poids sur l'estomac, en tout cas pour certains d'entre nous, on essaie de faire un peu de snorkelling, mais impossible de retrouver la qualité et l'ambiance qu'on avait à Koh Tao. Tout est gris, l'eau est trouble, les poissons se cachent, les coraux sont morts. Dominique se fait même attaquer lâchement par un oursin, ce qui lui fera un tatouage au bas du mollet. Tout ceci nous convaincs de faire une sortie snorkelling digne de ce nom. On se retrouve tous les quatre dans un grand bateau, en compagnie d'environ 200 personnes, et nous sommes les seuls touristes occidentaux !


\begin{figure}[h]
\centering
\includegraphics[height=5cm,width=9cm,keepaspectratio]{p1060910.jpg}
\caption*{Et nous ne sommes pas le seul bateau !}
\end{figure}

On apprend un peu plus tard que ce sont les vacances thaïlandaises, voilà pourquoi le bateau est plein de touristes thaïs. On se dit que ça va être surpeuplé une fois dans l'eau, mais non, on s'en sort plutôt bien : la plupart des thaïs se savent pas nager, ils ont donc un gilet de sauvetage et ne bougent pas trop une fois à l'eau. Certains embauchent même un guide empalmé qui va remorquer une bouée à laquelle les gens s'accrochent, afin qu'ils puissent se balader un peu. Tout ceci fait qu'on peut facilement s'écarter de la foule et trouver des coins tranquilles. Et enfin, on retrouve des poissons, des coraux multicolores, et on a même la chance de voir des seiches et des gros crabes !


\begin{figure}[h]
\centering
\includegraphics[height=5cm,width=9cm,keepaspectratio]{p1070018.jpg}
\caption*{Il y a des touristes qui ont du pain dans les poches. Ça augmente fortement la concentration locale de poissons !}
\end{figure}


\begin{figure}[h]
\centering
\includegraphics[height=5cm,width=9cm,keepaspectratio]{p1060918.jpg}
\caption*{Agile comme un mérou !}
\end{figure}


\begin{figure}[h]
\centering
\includegraphics[height=5cm,width=9cm,keepaspectratio]{p1070073.jpg}
\caption*{Un gros crabe !}
\end{figure}

Le débarcadère qui nous mène jusqu'aux bateaux est une attraction à lui tout seul : il doit faire dans les 400m de long, mais on ne voit pas la mer : une file ininterrompue de boutiques bordent le chemin. On entend la mer, on sent la mer, et c'est tout !


\begin{figure}[h]
\centering
\includegraphics[height=5cm,width=9cm,keepaspectratio]{p5034973.jpg}
\caption*{Quand on prend de l'essence, on a le choix entre menthe, orange et grenadine.}
\end{figure}

La mer, c'est bien, mais cette île est montagneuse, ce serait dommage de ne pas profiter de la vue ! On essaie plusieurs fois d'aller se balader tout seul, mais c'est compliqué ici aussi : aucune carte disponible, aucun panneau, tout est fait pour encourager l'embauche de guides locaux ! On apprend même qu'il y avait des sentiers de randonnée bien équipés, signalés, mais tous les panneaux ont été arrachés il y a quelques années déjà.


\begin{figure}[h]
\centering
\includegraphics[height=5cm,width=9cm,keepaspectratio]{p5055495.jpg}
\caption*{Le lézard, nettement plus facile à photographier que les singes !}
\end{figure}

En se baladant au hasard, on voit tout de même de belles choses : des forêts de bambou, des termitières, et même des singes sauvages ! Par contre, dans certains endroits de la jungle, c'est délicats de trouver un endroit ou s'arrêter tellement il y a de fourmis et de termites qui nous grimpent dessus à la moindre pause. Par endroits, nous nous sommes même retrouvés à devoir courir sur quelques mètres pour passer par-dessus des 2x8 voies de fourmis affamées.


\begin{figure}[h]
\centering
\includegraphics[height=5cm,width=9cm,keepaspectratio]{p5055484.jpg}
\caption*{La rando c'est bien, mais ça donne chaud...}
\end{figure}

Le rythme n'est pas le même que quand on est juste Marion et moi. Nous, on se la coule douce, et on ne va certainement pas faire du kayak une heure avant de partir pour l'aéroport ! Mais toutes les bonnes choses ont une fin (sauf le saucisson bien entendu), et nous prenons la route de l'aéroport. Les adieux sont tristes, mais nous repartons tous regonflés du plaisir d'avoir passé un peu de temps avec la famille.


\begin{figure}[h]
\centering
\includegraphics[height=5cm,width=9cm,keepaspectratio]{p5055444.jpg}
\caption*{On a aussi vu des bébés éléphants !}
\end{figure}


\begin{figure}[h]
\centering
\includegraphics[height=6cm,width=9cm,keepaspectratio]{p5045125.jpg}
\caption*{Et on a vu des jolis coucher de soleil !}
\end{figure}


\begin{figure}[h]
\centering
\includegraphics[height=5cm,width=9cm,keepaspectratio]{p5045437.jpg}
\caption*{L'anniversaire qui fait du bien aux papilles !}
\end{figure}



\chapter{Plongée dans les îles Similan}
Nous partîmes donc pour les îles Similan. Pour une croisière liveaboard. Oui, vous avez bien lu ! C'est le gros craquage budgétaire du voyage. C'est pour ce genre de choses qu'après chaque pays, on faisait les comptes pour savoir si on avait dépensé plus ou moins que prévu : la différence allait dans le budget plongée en Thaïlande. Mais quel rapport, allez-vous me dire, entre une croisière et la plongée ? Réponse : dans une croisière "liveaboard" on ne fait que de la plongée :-)


\begin{figure}[h]
\centering
\includegraphics[height=5cm,width=9cm,keepaspectratio]{similan-2016-433.jpg}
\caption*{Le poisson juste en dessous de la raie fait presque un mètre de long...}
\end{figure}

Vous vous en doutez, autant faire ce genre d'activité dans un endroit qui vaut un peu le coup : les îles Similan sont parmi les meilleures destinations de plongée du monde. Bref, on est super excités. Depuis qu'on a réservé cette croisière, tous les gens à qui on en parle rêvent soit d'y aller, soit d'y retourner, et des étoiles s'allument dans leurs yeux.


\begin{figure}[h]
\centering
\includegraphics[height=5cm,width=9cm,keepaspectratio]{similan-2016-483.jpg}
\caption*{Dive . Eat . Sleep . Repeat}
\end{figure}

On se retrouve à 30 plongeurs sur un bateau de 35 mètres, avec un programme en 4 mots : \emph{dive, eat, sleep, repeat} (plonge, mange, dors, recommence). Quinze plongées en quatre jours et quatre nuits : il s'agirait de ne pas chômer ! Toute une équipe est là pour nous faciliter la vie. La journée commence par le réveil en fanfare, on a alors 30mn pour manger un morceau et se préparer, avant le briefing. Quelques explications sur le site de plongée plus tard, et on descend s'équiper. On a chacun sa place qui ne bougera pas de toute la croisière. Les membres d'équipage se chargent de remplir les bouteilles au fur et à mesure, et nous aident même à enfiler les palmes !


\begin{figure}[h]
\centering
\includegraphics[height=5cm,width=9cm,keepaspectratio]{similan-2016-569.jpg}
\caption*{C'est nous, juste sous les lettres T et A.}
\end{figure}

Une fois en position sur la plateforme, on attend la sonnerie du capitaine qui nous autorise à sauter, et là, c'est comme des parachutistes qui sautent d'un avion : les uns derrières les autres, quatre par quatre, sous les encouragements de l'équipe, on saute dans le bleu.


\begin{figure}[h]
\centering
\includegraphics[height=5cm,width=9cm,keepaspectratio]{similan-2016-563.jpg}
\caption*{Le programme de la troisième journée.}
\end{figure}

Et on en prend plein les mirettes. Pour illustrer mon propos, rien ne vaut quelques images (n'étant pas équipé pour la photo à cette profondeur, cet article est illustré avec les photos de Kenneth, un photographe Hong-Kongais qui a fait la croisière avec nous).


\begin{figure}[h]
\centering
\includegraphics[height=5cm,width=9cm,keepaspectratio]{similan-2016-118.jpg}
\caption*{Un poisson lion.}
\end{figure}


\begin{figure}[h]
\centering
\includegraphics[height=5cm,width=9cm,keepaspectratio]{similan-2016-591.jpg}
\caption*{Un hippocampe !}
\end{figure}


\begin{figure}[h]
\centering
\includegraphics[height=5cm,width=9cm,keepaspectratio]{similan-2016-606.jpg}
\caption*{On a retrouvé Némo :-)}
\end{figure}


\begin{figure}[h]
\centering
\includegraphics[height=5cm,width=9cm,keepaspectratio]{similan-2016-603.jpg}
\caption*{Des petites crevettes !}
\end{figure}



\begin{figure}[h]
\centering
\includegraphics[height=5cm,width=9cm,keepaspectratio]{similan-2016-731.jpg}
\caption*{Ce truc violet aux bouts oranges, c'est une sorte de limace de mer.}
\end{figure}


Environ une heure sous l'eau plus tard, on remonte, le bateau vient nous chercher et on remonte pour un petit-déjeuner/déjeuner/goûter/diner (rayer la mention inutile en fonction du nombre de plongées déjà effectuées). Une petite sieste, ou une grosse nuit plus tard, on recommence ! Les repas sont supers bons et variés, on passe la journée en maillot de bain quand on n'est pas au milieu des poissons.



\begin{figure}[h]
\centering
\includegraphics[height=5cm,width=9cm,keepaspectratio]{similan-2016-546.jpg}
\caption*{On se sent tout petit quand un engin comme ça nous passe au-dessus.}
\end{figure}

Dès le deuxième jour, nous avons la chance de voir des Raies Manta ! C'est énorme, impressionnant, majestueux, lent et léger à la fois. L'émotion est si forte que le souvenir de ce moment est gravé dans mon esprit. En imaginant ces raies manta, toutes les sensations du moment revivent : je respire à travers un détendeur, je sens le gout du sel, j'entends les bulle, et j'ai l'impression d'avoir la poitrine qui me serre et de ne plus savoir comment respirer ! Nous sommes tentés de les suivre après leur passage, mais nos guides sous-marins nous font signe que non, on attend au même endroit. Quelques instants plus tard, les mantas reviennent puis repartent : elles tournent en rond ! Toutes les palanquées sont aux aguets, le premier qui en voit une tape sur sa bouteille pour attirer l'attention des autres. Ce qui n'empêche pas certains plongeurs d'être à coté de la plaque par moment : Roger par exemple, a mis quelques instants à comprendre pourquoi tous les autres plongeurs le montrait subitement. C'est quand il s'est retourné et a vu la manta deux mètres au dessus de lui qu'il a compris. Découvrir un poisson de 6m juste dans son dos, ça doit faire un choc !


\begin{figure}[h]
\centering
\includegraphics[height=7cm,width=11cm,keepaspectratio]{p1070166.jpg}
\caption*{Petite pause sur une île paradisiaque.}
\end{figure}

Un des sites les plus attendu, c'est le rocher de Richelieu. Nous avons de la chance car nous sommes le seul bateau de plongée sur le site, alors qu'il y en a d'habitude une dizaine. Il se trouve que nous sommes sur la dernière croisière de la saison. Les îles Similan sont dans un parc national qui ferme 6 mois par ans, à la fois pour permettre à la nature de se régénérer, et aussi parce que c'est la saison de la mousson, donc pas génial pour la plongée. Ce n'est pas parce qu'on aime pas être mouillé, mais parce que le vent, la pluie et les vagues remuent la mer, et diminuent la visibilité sous l'eau. Nous avons même droit à une petite dérogation : le parc est officiellement fermé, mais les autorités nous laissent terminer la croisière. Donc nous somme seuls ! Cet endroit grouille de vie, et on voit pour la première fois autant de petites bêtes partout, comme la fameuse crevette mante, capable de donner des coups de patte supersoniques. La vitesse est telle que le mouvement crée des bulles de vides qui en s'effondrant font des petits flashs de lumière ! Il est même compliqué de la garder en aquarium : souvent, le verre ne lui résiste pas. On voit aussi des hippocampes, tout petits et immobiles, camouflés dans leur environnement, presque invisibles sans l’œil aiguisé de Nick notre guide russe.


Nous sommes le seul bateau de plongée sur le site, mais il y a aussi, malheureusement, des bateaux de pêche. On nous explique que malgré le statut de parc national, les gardiens se font facilement corrompre avec une caisse de bière. Résultat, pendant la fermeture du parc, les poissons sont pêchés, et c'est pendant la saison touristique qu'ils sont le plus tranquilles...


\begin{figure}[h]
\centering
\includegraphics[height=9cm,width=9cm,keepaspectratio]{p5177112.jpg}
\caption*{Ils étaient à deux mètres de nous : on avait l'impression qu'on pouvait les toucher en tendant un peu plus le bras (et en tombant du bateau accessoirement).}
\end{figure}

Sur le chemin du retour, petit bonus inattendu, des dauphins font leur apparition et viennent jouer dans les vagues à la proue du bateau. Là, on s'est vraiment crus dans un épisode de Cousteau !



\chapter{Où on fait un tour sur les autres îles de Thaïlande}
Nous arrivâmes donc à Phuket, les pieds sur terre, mais la tête encore au milieu des poissons. Nous avons fait le trajet avec Edwige et Antoine, un couple de Français rencontré pendant la croisière. Ils voyagent depuis bien plus longtemps que nous : 3 ans (et là, je pense qu'avec cette phrase, je fais peur aux mamans...). Donc nous voilà à Phuket, temple du tourisme de masse en Thaïlande.


\begin{figure}[h]
\centering
\includegraphics[height=5cm,width=9cm,keepaspectratio]{p5238341.jpg}
\caption*{Avec un voilier, ce serait du William Turner.}
\end{figure}

Une grande plage de sable blanc sans le moindre rocher ni poisson attire les touristes la journée. L'activité phare est le parachute ascensionnel. C'est un parachute tiré par un hors-bord, un peu comme un enfant qui fait voler un cerf-volant en courant, sauf qu'un touriste a payé une place dans le cerf-volant. C'est pratiqué bien entendu en suivant toutes les règles de sécurité et de bon sens : par exemple, les hors-bord ont un accès réservé à la plage. Ce serait quand même super dangereux de les faire passer au milieu des baigneurs, non ? De la même manière, le pilote du parachute - car le touriste ne sait pas comment atterrir en douceur sur la plage - est correctement attaché. Ce serait, situation prise au hasard, complètement con et suicidaire qu'il se contente de s'agripper à la main aux suspentes, ne pensez-vous pas ?


\begin{figure}[h]
\centering
\includegraphics[height=5cm,width=9cm,keepaspectratio]{p5197179.jpg}
\caption*{Il y a aussi de jolis couchers de soleil.}
\end{figure}

Derrière la plage, la ville pleine de rues, des rues pleines de bars, des bars pleins d'Australiens, des Australiens plein de bière. On prend quand même le temps de visiter un peu les lieux, plus par curiosité que par envie, après tout, c'est aussi représentatif de la Thaïlande ! Dans la rue la plus active, on remarque un décalage certain entre l'ambiance festive et l'expression désabusée des femmes qui essayent d'attirer le chaland vers un ping-pong show.


\begin{figure}[h]
\centering
\includegraphics[height=5cm,width=9cm,keepaspectratio]{p5187166.jpg}
\caption*{"Swamp eel show" ??!!!? Le show anguille des marais ? Quelqu'un sait comment éteindre l'imagination ? Si ça se trouve, c'est une anguille qui joue au ping-pong...}
\end{figure}


\begin{figure}[h]
\centering
\includegraphics[height=5cm,width=9cm,keepaspectratio]{p5187175.jpg}
\caption*{Le mec, il regarde la télé. Genre, il est venu là pour voir du foot !}
\end{figure}

Après un jour à Phuket, ce qui est bien suffisant pour nous, direction Krabi, autre lieu touristique s'il en est ! Les plages sont un peu plus jolies, surtout grâce aux karsts qui agrémentent l'horizon. Mais il n'y a toujours pas de poissons à voir, alors on va plutôt visiter un temple en haut d'une montagne ! Il fait très chaud, et l'escalier est tellement raide qu'on voit de nombreux touristes se sentir mal dans la montée, certains au point d'en vomir. Et comme dans les tous les temples dans la montagne, il est colonisé par les singes, ainsi que par les vendeurs de nourriture pour singe, qui ont l'avantage, par rapport aux singes, d'éviter de monter sur ton sac à dos pour te piquer des trucs de valeur comme un sachet en plastique vide. A par ça, Krabi c'est assez semblable à Phuket, donc on passe aussi rapidement à la prochaine étape : Railay !


\begin{figure}[h]
\centering
\includegraphics[height=5cm,width=9cm,keepaspectratio]{p5207903.jpg}
\caption*{Vue après quelques centaines de marches.}
\end{figure}




\begin{figure}[h]
\centering
\includegraphics[height=5cm,width=9cm,keepaspectratio]{p5207646.jpg}
\caption*{En plus de sentir bon, ça éloigne les moustiques.}
\end{figure}


\begin{figure}[h]
\centering
\includegraphics[height=5cm,width=9cm,keepaspectratio]{p5207668.jpg}
\caption*{Une photo de singe, ça ne fait pas de mal !}
\end{figure}


\begin{figure}[h]
\centering
\includegraphics[height=5cm,width=9cm,keepaspectratio]{p5207672.jpg}
\caption*{Une photo de s... oh, salut Antoine ! Je ne t'avais pas reconnu ! (on ne dirait pas comme ça, mais je l'aime bien en vrai, et en plus on gagne à la coinche ensemble)}
\end{figure}


\begin{figure}[h]
\centering
\includegraphics[height=5cm,width=9cm,keepaspectratio]{p5197184.jpg}
\caption*{Bière, burger, coinche ! Le bonheur !}
\end{figure}

Railay, c'est une petite crique juste à coté de Krabi. Elle a l'insigne avantage d'être uniquement accessible en bateau, et c'est aussi et surtout le paradis des grimpeurs. Imaginez un peu : une crique entourée de falaises de calcaire, une plage, et des bungalows pleins de hippies/grimpeurs ! Malheureusement pour nous, il a plu tout notre séjour. Il restait bien quelques voies au sec, mais que du dévers. Nous avons donc fait l'impasse sur l'escalade, et plutôt fait du kayak, pour aller tourner autour de quelques îles en face de la crique. Nous avons pu admirer de loin quelques voies de "deep water soloing" (il s'agit de grimper sans corde au dessus de l'eau) qui ont fait la réputation de Railay.


\begin{figure}[h]
\centering
\includegraphics[height=5cm,width=9cm,keepaspectratio]{p5218288.jpg}
\caption*{Le moteur du bateau. Confiance !}
\end{figure}


\begin{figure}[h]
\centering
\includegraphics[height=5cm,width=9cm,keepaspectratio]{p5218290.jpg}
\caption*{Le capitaine du bateau. Confiant.}
\end{figure}


\begin{figure}[h]
\centering
\includegraphics[height=5cm,width=12cm,keepaspectratio]{p1070230.jpg}
\caption*{C'est tout petit un kayak !}
\end{figure}

Après quelques jours pluvieux à baver devant des voies inaccessibles, on passe à l'île suivante : Koh Lanta ! Autant le dire tout de suite : rien à voir avec l'émission. Il y a tout un chapelet d'îles qui portent ce nom, et nous logeons sur la plus grande, plutôt bien équipée pour la survie des aventuriers. Nous regrettons rapidement le climat pluvieux de Railay, car ici, c'est plutôt la tempête ! Le début de la mousson a chassé la plupart des touristes, beaucoup de commerces sont fermés, bref, c'est l'échec. Le seul avantage est que la température redevient acceptable, désormais, si on est dehors sans bouger, on ne transpire plus ! Une petite accalmie nous permet de faire un tour sur l'île, mais rien de bien folichon : il pleut toujours, nous repartons directement vers Koh Tao.


\begin{figure}[h]
\centering
\includegraphics[height=6cm,width=12cm,keepaspectratio]{p5238330.jpg}
\caption*{Un petit crabe sur la plage de Koh Lanta.}
\end{figure}


\begin{figure}[h]
\centering
\includegraphics[height=6cm,width=12cm,keepaspectratio]{p5248359.jpg}
\caption*{Pêcheurs à Koh Lanta.}
\end{figure}

Il faut savoir que depuis la fin de la croisière, nous avions envie de retourner à Koh Tao, mais on culpabilisait un peu de ne pas visiter d'autres îles. Et si nous allions louper quelque chose d'incroyable ? Si tout le monde va à Krabi/Phuket/Koh Lanta, il doit bien y avoir une raison ? Mais après une semaine à courir d'île en île, c'est sûr : Koh Tao, nous revoilà :-)



\chapter{Koh Tao, bis repetita placent}
Nous arrivâmes donc à Koh Tao. "Encore ?" allez-vous me dire. "Oui" vais-je vous répondre. Mais plutôt que de faire un copier-coller de l'article précédent, je vais vous parler d'un de nos poisson préféré : le trigger-fish, ou baliste pour les francophones.


\begin{figure}[h]
\centering
\includegraphics[height=5cm,width=9cm,keepaspectratio]{p5318583.jpg}
\caption*{Sombres présages !}
\end{figure}

\textbf{Note pour alléger ma conscience :} je me rends compte que, de plus en plus, mon propos est émaillé de mots anglais qui pourraient très bien être traduits en français. Snorkeling, guesthouse, fruitshake, trigger-fish... Mais d'une part la traduction est parfois malaisée (palme/masque/tuba = snorkelling), et d'autre part, tout le monde utilise ces mots par ici, et on se met, même entre nous, à parler un franglais de voyage peu esthétique mais plus intuitif. Les mots nous sortent comme ça. J'ai horreur de ça, je trouve que ça fait snob et/ou mec qui se la pète genre \emph{"han, je suis complètement trop jet-laggé, faut que je call mon boss asap pour postponer le meeting"}. Je sais, personne ne s'en était plaint, mais moi, je m'énerve tout seul quand je fais ça, et ça me fait du bien d'en parler !


\begin{figure}[h]
\centering
\includegraphics[height=5cm,width=9cm,keepaspectratio]{p6018591.jpg}
\caption*{Le volley, c'est pour ceux qui ne savent pas sauter.}
\end{figure}

Trêve de jérémiades : le trigger-fish, pourquoi est-il cool ? Déjà, quand un plongeur indique un trigger-fish, il a l'air d'un gangster en mimant un pistolet avec sa main. Top méga cool ! Ensuite, c'est un des seuls poissons (un peu) dangereux qu'on peut croiser sous l'eau. Il est plutôt territorial, et si on passe à proximité, et en particulier au-dessus de son nid, il peut s'énerver. Un signe qui ne trompe pas, c'est sa nageoire dorsale qui se dresse, comme une gâchette (trigger) avant une attaque. A ce moment, il faut se mettre en position de pouvoir donner des coups de palmes au poisson en cas de besoin, car il va tenter de nous foncer dessus à toute vitesse pour nous donner des coups de dents ! Bon, il faut relativiser, il fait environ 50cm, quelques kilos à tout péter, donc il ne va tuer personne. Mais il peut sonner un plongeur, ou lui casser son masque, et sous l'eau, ça peut être gênant. Bref, on a eu droit à un petit briefing spécialement sur ce poisson, et ça fait un petit quelque chose quand on le voit la première fois.


\begin{figure}[h]
\centering
\includegraphics[height=5cm,width=9cm,keepaspectratio]{p1070351.jpg}
\caption*{C'est notre meilleure photo du poisson en question.}
\end{figure}

Quand il ne charge pas des plongeurs, on dirait qu'il se contente de ronger du corail avec ses gros chicots. Mais quand on y regarde de plus près, (ce que j'ai fait pendant un petit bout de temps en snorkelling) on s'aperçoit qu'il creuse le corail pour y déloger de délicieux coquillages qu'il va directement consommer en sashimi. C'est un poisson raffiné. Quand je pense qu'en tant que plongeurs on n'a même pas le droit de toucher au corail, il y a vraiment deux poids deux mesures sous l'eau !


\begin{figure}[h]
\centering
\includegraphics[height=5cm,width=9cm,keepaspectratio]{p5298515.jpg}
\caption*{De gauche à droite : trigger-fish, tortue, requin léopard, poulpe. (je parle des signes de main bien sûr, pas des gens)}
\end{figure}

Sinon, on a aussi fait beaucoup de coinche, grâce à Edwige et Antoine qu'on a encore une fois retrouvé ici. C'est con hein, mais taper le carton en buvant des bières avec des potes jusqu'à point d'heure, mais qu'est-ce que ça fait du bien !


\begin{figure}[h]
\centering
\includegraphics[height=5cm,width=9cm,keepaspectratio]{p5288481.jpg}
\caption*{Jongleurs de feu tous les soirs sur la plage.}
\end{figure}


\begin{figure}[h]
\centering
\includegraphics[height=5cm,width=9cm,keepaspectratio]{p5288409.jpg}
\caption*{Photo réussie complètement par accident. Ce n'était pas du tout ça que j'essayais de faire !}
\end{figure}

J'ai aussi envie de vous raconter un plat improbable : Salade de porc à l'ail et au citron. Ce jour là, j'avais envie d'un truc un peu plus léger, alors pourquoi pas une salade. Quand le plat est arrivé, j'ai cherché la verdure : il n'y en avait pas... C'était un tas de morceaux de porc grillé, baignant dans une sauce au citron, accompagné, sans mentir, d'au moins 20 gousses d'ails émincées, soit l'équivalent d'une tête. C'était excellent et surprenant. Chaque bouchée me faisait pleurer, à chaque fois je me disais que c'était la dernière, mais quelques secondes plus tard le piquant de l'ail s'estompait, et je repartais pour la bouchée suivante. J'ai senti l'ail pendant deux jours...


\begin{figure}[h]
\centering
\includegraphics[height=6cm,width=9cm,keepaspectratio]{p5318564.jpg}
\caption*{C'est vraiment l'enseigne d'un restaurant. Je... je n'ai pas d'explications...}
\end{figure}

Cet article est complètement décousu, je passe du coq(uillage) à l'ail sans arriver à trouver de liaison fluide entre tous les sujets dont au sujet duquel que j'ai envie que je m'exprime. Et les photos n'ont rien à voir. Ma foi, tant pis, je continue : Bien que Koh Tao soit réputée être une île plus calme que ses voisines Koh Phangan et surtout Koh Samui, on voit quand même des trucs assez rigolos/tristes. Le grand classique : le tatouage bourré de 2h du matin ! J'imagine le retour : "Tu veux vraiment garder ton col roulé ?". On voit les gens qui arrivent à peine à marcher, mais qui se disent, plein d'espoir, que ça ira mieux une fois sur le scooter. Ils enlèvent la béquille, et paf, la gravité l'emporte. Il faut savoir que sur Koh Tao, les loueurs de scooter ont des règles très strictes, et peut-être justement dues aux mecs bourrés : la moindre rayure implique de payer la pièce de rechange neuve, et de nombreux touristes se retrouvent à payer des fortunes pour des petites rayures qu'ils ne sont même pas sûrs d'avoir faites. De notre coté, vu que l'île est toute petite, on a choisi de tout faire à pied. Ainsi, on culpabilise moins de glander à la plage si on a marché une heure pour y arriver. Et de toute façon, nos plages préférées ne sont même pas accessibles en scooter. Tranquillité assurée :-)


Sans transition : fin de l'article !



\chapter{Hpa An, entrée au Myanmar.}
Nous partîmes donc pour le Myanmar. Par la frontière terrestre s'il vous plait ! Eh oui, depuis quelques années, le Myanmar s'ouvre de plus en plus au tourisme, et il est maintenant possible d'y rentrer à pied, par le pont de l'amitié. L'entrée dans le pays est une simple formalité. Le douanier prend même le temps de nous parler un peu de la coupe d'Europe qu'il suit depuis son boulot. On est accueilli ensuite par un jeune qui parle plutôt bien anglais, qui nous guide vers un DAB qui me file des chocs électriques plutôt que du cash. Tant pis, on paiera le taxi en dollars !


\begin{figure}[h]
\centering
\includegraphics[height=6cm,width=9cm,keepaspectratio]{p6160437.jpg}
\caption*{Environ 0,3\% du jardin des Bouddhas.}
\end{figure}

Ah, le taxi. Quelle belle entrée en matière ! Il restait deux places dans le taxi partagé. Pile pour nous. En fait, on manquait d'imagination : il restait trois places ! A quatre à l'arrière plus les bagages, et si l'on rajoute, on plutôt, si on soustrait les suspensions, on comprend d'un coup la relativité : le temps s'étire, et plus on s'approche de notre destination, plus les secondes durent ! Heureusement, une crevaison nous a permis de faire une petite pause.


\begin{figure}[h]
\centering
\includegraphics[height=6cm,width=9cm,keepaspectratio]{p6159125.jpg}
\caption*{Le mauvais type de pneu clouté.}
\end{figure}


\begin{figure}[h]
\centering
\includegraphics[height=6cm,width=9cm,keepaspectratio]{p6159117.jpg}
\caption*{Ça marche moins bien comme ça.}
\end{figure}

Nous arrivons à Hpa An (prononcer "pah aaane"), et nous visitons d'un seul coup 50\% des guesthouses de la ville. Fini le choix de la Thaïlande. Ici, on fait avec ce qu'on trouve. Pour le petit déjeuner, ça se passe dans un "tea shop". Pas de carte : on s’assoit, on nous demande si on veut du thé. "Heu, oui ?" C'est bizarre comme question, car il y a déjà une thermos de thé noir en libre service sur toutes les tables... Le thé arrive : il est au lait concentré et accompagné par des assiettes de petites pâtisseries frites. On n'avait pas commandé ça, mais bon, tout le monde à l'air d'avoir les mêmes assiettes... C'est excellent, et la première bouchée est à chaque fois une surprise : noix de coco, banane, poulet, légumes, curry, il y a tous les gouts ! Au moment de l'addition, facile : ils comptent ce qui manque des assiettes et voilà !


\begin{figure}[h]
\centering
\includegraphics[height=6cm,width=9cm,keepaspectratio]{p6169502.jpg}
\caption*{C'est bon, mais (donc ?) c'est gras.}
\end{figure}

Puis on prend les vélos et on va se balader dans la campagne. L'accueil des gens est fabuleux : tout le monde nous hèle à grand coups de "hello" ou de "bye bye" (certains sont encore en train d'apprendre les bases de l'anglais). On leur répond des "Mingalaba" qui les font marrer. Ceux qui savent parler un peu plus anglais nous accueillent avec "Where do you come from?". En entendant "From France", on a invariablement droit à "Oooooohhh, Fraaaance ! Boujourcomançavaaaa ?". On nous avait dit que la Thaïlande était le pays du sourire. Nous avons trouvé leur maître :-)


\begin{figure}[h]
\centering
\includegraphics[height=6cm,width=9cm,keepaspectratio]{p6169510.jpg}
\caption*{Aucun rapport : des termites et une chenille !}
\end{figure}

Les paysages sont magnifiques. La mousson a commencé, et c'est plutôt une bonne chose : la température est supportable, les paysages sont verts, et le ciel est toujours plein de gros cumulus bien joufflus. Idéal pour les photos, il n'y a rien de plus ennuyeux qu'un ciel bleu uniforme...


\begin{figure}[h]
\centering
\includegraphics[height=6cm,width=9cm,keepaspectratio]{p6160445.jpg}
\caption*{On a désespérément attendu une prise...}
\end{figure}

Comme partout dans le monde, les hommes construisent des trucs pas possibles dans des endroits pas croyables. Ici, ça ne loupe pas : A quelques kilomètres de la ville, on trouve un parc rempli de statues de Bouddha. On ne les a pas comptées, mais de source sûre, il y en aurait 1000 !


\begin{figure}[h]
\centering
\includegraphics[height=6cm,width=9cm,keepaspectratio]{p6169508.jpg}
\caption*{1, 2, 3, ... 1000 !}
\end{figure}

Puis ça continue avec un monastère au sommet d'un piton karstique de 700m de haut. Après quelques heures de montée sans croiser un chat, on arrive au point de vue. Imaginez un peu : on est au bord d'une falaise qui tombe à pic jusqu'à la plaine inondée en contrebas, et il nous semble qu'il nous suffirait de lever le bras pour toucher les nuages. On en voit d'autres au loin qui se laissent aller sur les champs, tout en avançant doucement. Juste à coté de nous, un moine et une fille étendent du linge en rigolant. On a mal aux mollets, mais qu'est-ce qu'on est bien ici !


\begin{figure}[h]
\centering
\includegraphics[height=6cm,width=9cm,keepaspectratio]{p6169768.jpg}
\caption*{Dire qu'il y a des gens qui vivent ici toute l'année !}
\end{figure}


\begin{figure}[h]
\centering
\includegraphics[height=6cm,width=9cm,keepaspectratio]{p6160407.jpg}
\caption*{Le moine (non, les moines ne sourient pas sur les photos).}
\end{figure}


\begin{figure}[h]
\centering
\includegraphics[height=6cm,width=9cm,keepaspectratio]{p6160409.jpg}
\caption*{Mais les filles peuvent sourire !}
\end{figure}


\begin{figure}[h]
\centering
\includegraphics[height=9cm,width=12cm,keepaspectratio]{p6160410.jpg}
\caption*{Un crabe trouvé en haut de la montagne !}
\end{figure}



\chapter{Bagan, le temple des pagodes}
Nous partîmes donc pour Bagan. 24h de bus d'affilée ! On aurait pu s'arrêter à Yangon (Rangoun), mais ... mais non, pas envie. On se retrouve donc dans ce bus surclimatisé sans pouvoir rien faire quant à la température glaciale : on voyait tous les birmans autour de nous tenter de s'emmitoufler, un moine s'est fait rembarrer par le chauffeur après avoir visiblement montré l'aération. Des vis ont même été rajoutées pour bloquer les aérations en position grande ouverte. Peut-être que la climatisation, c'est un truc hyper classe dans un pays où il fait toujours chaud : "Ouah, c'était hyper classe ce trajet, tellement classe que mon thé a gelé !". Il y a aussi les vidéos de karaoké amateur toute la nuit à fond. Bref, c'était long...


\begin{figure}[h]
\centering
\includegraphics[height=6cm,width=9cm,keepaspectratio]{p6191364_hdr.jpg}
\caption*{Ça vaut le coup d'endurer un peu de mauvais karaoké...}
\end{figure}

A la sortie du bus à notre correspondance, nous sommes accueillis comme il se doit par une armée de rabatteurs. Mais... ce ne sont pas les mêmes rabatteurs que dans les autres pays. Ceux-ci nous ont vraiment aidé en nous amenant à la banque, puis à la bonne agence de bus, alors qu'ils nous proposaient des taxis à la base. Ils sont partis sans qu'on ait eu le temps de les remercier. Ils ont sûrement réussi à chopper une commission quelque part, mais honnêtement, je n'en suis même pas si sûr, tellement ils étaient gentils et avaient l'air content de nous aider...


\begin{figure}[h]
\centering
\includegraphics[height=6cm,width=9cm,keepaspectratio]{p6180451.jpg}
\caption*{Les nonnes aussi sont rasées.}
\end{figure}

On arrive à Bagan à 3h du matin. C'est l'endroit le plus connu de tout le Myanmar, et ici, ils connaissent les touristes : On repousse un taxi qui voulait 10 dollars pour faire 5km, et on part à pied vers notre hôtel pendant que le taxi nous hurle qu'il y a des serpents sur la route...


\begin{figure}[h]
\centering
\includegraphics[height=6cm,width=9cm,keepaspectratio]{p6191342.jpg}
\caption*{Il y a encore quelques années, le cheval était le principal moyen de transport.}
\end{figure}

Le soleil se lève pendant qu'on marche dans la campagne, et on devine au loin les premières des 4000 pagodes qui ont été construites pendant une période de frénésie immobilière et spirituelle il y a quelques siècles. En arrivant dans le village au petit matin, on voit les nonnes, tout de rose vêtues, parcourir les rues à la recherche d'offrandes. Contrairement aux moines qui reçoivent de la nourriture prête à être consommée, les nonnes reçoivent surtout du riz cru. La raison est qu'un homme, tout moine qu'il soit, ne peut pas faire la cuisine, c'est un boulot de femme. Soupir.


\begin{figure}[h]
\centering
\includegraphics[height=6cm,width=9cm,keepaspectratio]{p6191374.jpg}
\caption*{Des grands temples.}
\end{figure}


\begin{figure}[h]
\centering
\includegraphics[height=6cm,width=9cm,keepaspectratio]{p6201593.jpg}
\caption*{Mais aussi des tout petits !}
\end{figure}

Et si nous arrêtions de tourner autour du pot : les temples ! Ils sont pleins, ils sont partout, ils de toutes les tailles, et ils sont surtout en brique rouge, ce qui s'accorde parfaitement au vert des arbres et au bleu du ciel. Ils sont remplis de statues Bouddha et de peintres sur sable. Le ciel est magnifique avec tous ces petit nuages. On passe quelques jours géniaux à se balader à vélo dans des petits chemins plein de poussière, au milieu de tous ces temples. Quand vient le coucher de soleil, on grimpe au sommet d'un temple, on voit alors les ombres des pagodes et des arbres qui s'allongent et on essaie de penser à garder la bouche fermée. C'est dur.


\begin{figure}[h]
\centering
\includegraphics[height=6cm,width=9cm,keepaspectratio]{p6180682.jpg}
\caption*{Oiseau en bonus !}
\end{figure}


\begin{figure}[h]
\centering
\includegraphics[height=6cm,width=9cm,keepaspectratio]{p6180900.jpg}
\caption*{Les temples ont toujours été en activité, et continuent aujourd'hui à accueillir des croyants.}
\end{figure}


\begin{figure}[h]
\centering
\includegraphics[height=6cm,width=9cm,keepaspectratio]{p6180910.jpg}
\caption*{Une des rares pagode peinte.}
\end{figure}


\begin{figure}[h]
\centering
\includegraphics[height=5.5cm,width=9cm,keepaspectratio]{p6181249.jpg}
\caption*{Arc en ciel et pluie en bonus.}
\end{figure}

\textbf{Petite anecdote inutile :} on s'est dit qu'on a passé trop de temps dans les grottes et les temples abandonnés quand on s'est rendu compte qu'on était capables d'identifier l'odeur des crottes de chauve souris avant même de rentrer dans les endroits qui leurs servent d'habitat. Ouais, on peut carrément faire ça ! Et ça ne sert à rien !


\begin{figure}[h]
\centering
\includegraphics[height=5.5cm,width=9cm,keepaspectratio]{p6191277.jpg}
\caption*{Un caillou, un temple.}
\end{figure}

A quelques kilomètres de Bagan, il y a aussi le fameux mont Popa. Encore un temple construit au sommet d'un caillou. Mais à la différence de celui de Hpa An, on peut atteindre le sommet en 20 minutes par un escalier protégé des intempéries. Il y a donc beaucoup plus de monde, et de singes ! Et comme tout le caillou est considéré comme un temple, on doit y aller pieds nus. Vous croyez que ça porte chance la crotte de singe ?


\begin{figure}[h]
\centering
\includegraphics[height=5.5cm,width=9cm,keepaspectratio]{p6191324.jpg}
\caption*{Les singes volent des graines.}
\end{figure}


\begin{figure}[h]
\centering
\includegraphics[height=9cm,width=12cm,keepaspectratio]{p6191315.jpg}
\caption*{Donc ils se font descendre.}
\end{figure}


\begin{figure}[h]
\centering
\includegraphics[height=9cm,width=12cm,keepaspectratio]{p6191313.jpg}
\caption*{Et ils finissent en prison !}
\end{figure}



\chapter{Kalaw, trois jours de voyage dans le temps}
Nous arrivâmes donc à Kalaw, ville surtout connue pour être le départ d'un trek de trois jours en direction du lac Inle. Et ... nous partîmes donc pour un trek de trois jours en direction du lac Inle, accompagnés d'un guide et de 8 autres randonneurs !


\begin{figure}[h]
\centering
\includegraphics[height=5cm,width=9cm,keepaspectratio]{p6242126.jpg}
\caption*{Notre hôtesse le temps d'une nuit.}
\end{figure}

Notre guide, bien que sympathique, ne parlait pas beaucoup. Son mutisme fut toutefois largement compensé par la taille du groupe : Nous étions un peu déçus au début d'être dans un groupe si inhabituellement grand, mais s'il n'y a que des gens sympathiques, c'est un plaisir !




\begin{figure}[h]
\centering
\includegraphics[height=5cm,width=9cm,keepaspectratio]{p6242138.jpg}
\caption*{La bande au complet. Le guide est celui qui a des moustaches sur le ventre.}
\end{figure}

Les paysages étaient magnifiques tout le long du trajet, et changeaient en permanence. Nous avons alterné entre sentier un peu montagneux, petits champs cultivés, et enfin forêt et terre rouge. Si l'on rajoute de gros nuages, et des paysans qui font tout à la main, on obtient ce genre de paysages.




\begin{figure}[h]
\centering
\includegraphics[height=5cm,width=9cm,keepaspectratio]{p6242156.jpg}
\caption*{La mémé n'approuve pas qu'on sourie aux touristes.}
\end{figure}


\begin{figure}[h]
\centering
\includegraphics[height=5cm,width=9cm,keepaspectratio]{p6242142.jpg}
\caption*{Il est beau mon buffle !}
\end{figure}


\begin{figure}[h]
\centering
\includegraphics[height=5cm,width=9cm,keepaspectratio]{p6231726.jpg}
\caption*{C'est inquiétant tout ce gris...}
\end{figure}


\begin{figure}[h]
\centering
\includegraphics[height=5cm,width=9cm,keepaspectratio]{p6242167.jpg}
\caption*{Vous vous imaginez aussi ce qui passe par la tête du taureau à ce moment ? (Indice, c'est une charrue au premier plan)}
\end{figure}

Voir ces paysans cultiver leur terre nous fait voyager dans le temps : pas de tracteur, aucune machine, et énormément de main d’œuvre pour désherber en permanence. Les champs sont labourés avec des buffles et des charrues en bois ! Seule concession visible à la modernité : quelques sacs d'engrais qui trainent par ci par là. Et malgré ce retour dans le temps (juste pour nous, pas pour eux), tous ces gens sont hyper souriants, et ont l'air de passer, pour la plupart, des super journées entre potes : ils bossent en petits groupes, papotent et rigolent toute la journée. (A moins que ce ne soit juste les touristes qui les fassent rire, si ça se trouve, ils font la gueule le reste du temps... éternel problème de l'observateur qui influence l'expérience). Cette ambiance a par moments un goût de paradis perdu : et si ces gens étaient les plus heureux du monde ? Doit-on les envier ? Voudraient-ils changer de vie s'ils le pouvaient ? Nous envient-ils ? Je n'ai pas de certitudes, mais il me semble qu'il y a un couac dans ce tableau : le choix. Leur mode de vie est subi, que feraient-ils s'ils avaient le choix ?




\begin{figure}[h]
\centering
\includegraphics[height=5cm,width=9cm,keepaspectratio]{p6242160.jpg}
\caption*{Ils posent même pour les photos.}
\end{figure}


\begin{figure}[h]
\centering
\includegraphics[height=5.5cm,width=9cm,keepaspectratio]{p6242163.jpg}
\caption*{On dirait qu'un des gamins est en train de... disons fertiliser le champ !}
\end{figure}


\begin{figure}[h]
\centering
\includegraphics[height=5.5cm,width=9cm,keepaspectratio]{p6252333.jpg}
\caption*{Ouh les belles grimaces !}
\end{figure}


\begin{figure}[h]
\centering
\includegraphics[height=5.5cm,width=9cm,keepaspectratio]{p6231746.jpg}
\caption*{Coucher de soleil.}
\end{figure}

La première nuit se passe dans une guesthouse : à l'étage, un grand dortoir, et une petite pièce pour la famille. Les repas sont excellents, on découvre des nouvelles saveurs, comme des plants de potirons ou de moutarde (pas le fruit hein, toute la plante donc !). La salle de bain est dehors : trois vagues panneaux de bambou tressés autour d'un bac d'eau, dans la boue, ça fera bien l'affaire non ? Certains ne sont pas à l'aise en s'y lavant, mais on se rend compte que les locaux sont très respectueux, ou s'en foutent, et ne regardent pas.




\begin{figure}[h]
\centering
\includegraphics[height=5.5cm,width=9cm,keepaspectratio]{p6242122.jpg}
\caption*{Voici la salle de bain. Intimité au top !}
\end{figure}

Petite anecdote du prof : alors que nous mangions tous des bananes, tout le groupe s'est mis à parler des multiples qualités de ce fruit courbé, et j'ai balancé au hasard qu'il était plutôt radioactif. Ça a intrigué tout le monde, et ils se sont mis à me poser plein de questions sur les bananes et la radioactivité. Ceux qui me connaissent doivent imaginer la scène... Bref, ils ont décrété que je serais le prof du groupe, et m'ont demandé une autre leçon sur le sujet de mon choix lors du repas du soir ! Si si, pour de vrai ! Ils ont eu le droit à l'histoire de l'Univers.




\begin{figure}[h]
\centering
\includegraphics[height=5.5cm,width=9cm,keepaspectratio]{p6242214.jpg}
\caption*{Quoi de mieux qu'une photo de la voie lactée pour illustrer le cours ?}
\end{figure}


\begin{figure}[h]
\centering
\includegraphics[height=5.5cm,width=9cm,keepaspectratio]{p6242208.jpg}
\caption*{Le diner aux chandelles.}
\end{figure}

Nous passons la deuxième nuit dans un monastère. Il est rempli de gamins qui courent dans tous les sens, et nous demandent de jouer au foot avec eux. J'ai donc gagné un point hippie en tapant le ballon en tong dans le gravier avec un petit moine qui portait sa toge orange sur l'épaule. Tellement cliché qu'on aurait pu tourner une pub pour Benetton. Ces mêmes petits moines (ce sont en réalité des novices, mais appelons les quand même des moines) se sont levés le lendemain à 5 heures pour la prière. Comme nous dormions dans la salle de prière, juste séparés de l'autel par un drap tendu, nous avons pu admirer ces petits crânes rasés psalmodier en cœur leur prières.




\begin{figure}[h]
\centering
\includegraphics[height=5.5cm,width=9cm,keepaspectratio]{p6242201.jpg}
\caption*{Notre dortoir.}
\end{figure}

Le plus étonnant dans ce monastère, c'est que les adultes sont quasiment invisibles. Il nous a semblé que ces gamins de 5 à 10 ans étaient complètement autonomes, s'occupant tout seuls de la prière, des repas, et des douches. Ah oui, les douches, parlons-en : les moines se lavent autour d'un grand réservoir dans la cour, en gardant leur toge. Comme les touristes n'ont pas les mêmes habitudes ou vêtements, pour notre plus grand confort ils ont ajouté 4 murets à l'autre bout de la cour. On peut donc se laver à l'abri des regards en utilisant parcimonieusement l'eau des seaux qu'on a apporté.




\begin{figure}[h]
\centering
\includegraphics[height=5.5cm,width=9cm,keepaspectratio]{p6242206.jpg}
\caption*{En route pour la douche.}
\end{figure}


\begin{figure}[h]
\centering
\includegraphics[height=5.5cm,width=9cm,keepaspectratio]{p6242202.jpg}
\caption*{La salle de bain.}
\end{figure}

Enfin, nous arrivâmes en vue du lac. Il ne ressemble pas du tout à un lac comme chez nous, avec, par exemple, une berge. Celui-là n'en a pas : il est bordé de marécages, plein d'herbes hautes, et on ne peut naviguer en pleine eau qu'après avoir zigzagué dans des canots pendant une dizaine de minutes au milieu des villages flottants et de leurs champs ... flottants eux aussi ! Ils cultivent toutes sortes de légumes qui ne disposent pas de traduction officielle en français, mais on a quand même reconnu des tomates ! Après avoir passé trois jours dans cette campagne magnifique, voir ces gens vivre sur le lac, toujours sur un bateau ou des pilotis, avec en fond des nuages énormes qui se préparaient, je ne sais pourquoi, mais je me suis senti super ému à l'idée que ce mode de vie était peut-être entrain de disparaitre...






\begin{figure}[h]
\centering
\includegraphics[height=5.5cm,width=9cm,keepaspectratio]{p6252392.jpg}
\caption*{Village sur le lac.}
\end{figure}


\begin{figure}[h]
\centering
\includegraphics[height=5.5cm,width=9cm,keepaspectratio]{p6252410.jpg}
\caption*{Transport public sur le lac.}
\end{figure}


\begin{figure}[h]
\centering
\includegraphics[height=5.5cm,width=9cm,keepaspectratio]{p6262423.jpg}
\caption*{Petite dégustation de vin à l'arrivée au lac.}
\end{figure}




\chapter{Hsipaw, trek, et rebelles dans la montagne.}
Nous arrivâmes donc à Hsipaw. A 3 heures du matin, c'est tellement plus sympa. Mais le tuk-tuk du Red Dragon attendait déjà, et réussi à attirer en son antre les quelques backpackers encore hagards, grâce à la promesse d'une portion de nuit gratuite.


\begin{figure}[h]
\centering
\includegraphics[height=6cm,width=9cm,keepaspectratio]{p7013352.jpg}
\caption*{C'est la mousson !}
\end{figure}

Après une demi-nuit, à moitié reposés, on commence à organiser le trek suivant, car c'est pour ça que les gens viennent à Hsipaw. Comme on a lu qu'il y avait une sorte de mafia du trek dans le village, on essaie de la contourner. En effet, contrairement à d'autres villages où les rabatteurs se jettent sur nous comme des canards sur du pain sec, ici, rien. C'est simple : Les treks sont tous organisés par Mr. Charles. Tous ? Non : Un petit village... euh non... une petite agence résiste encore et toujours à l'hégémonie de l'empire du mal... euh... de l'empire de Charles. Le nom de l'agence ne s'invente pas : "Ma Boat Boat". Sans savoir où elle se trouve, on la cherche et on tombe dessus en 30 secondes. Trop facile ! Il s'avère que Ma Boat Boat est le nom de la gironde propriétaire de l'agence, et elle nous organise rapidement un trek de trois jours, sous réserve de - détail rassurant - l'absence de combats dans le coin.


\begin{figure}[h]
\centering
\includegraphics[height=6cm,width=9cm,keepaspectratio]{p6293189.jpg}
\caption*{Le triage du riz.}
\end{figure}

Alors que nous discutions organisation de trek, une bande de jeunes nonnes se sont retrouvées coincées par la pluie avec nous sous l'auvent. Je dois avouer que quand on est entourés de petites filles tout de rose vêtues, qui piaillent et nous observent du coin de l’œil en se poussant du coude, c'est difficile de ne pas sourire bêtement.


\begin{figure}[h]
\centering
\includegraphics[height=6cm,width=9cm,keepaspectratio]{p7023570.jpg}
\caption*{Elle est pas TROP MIMI ?}
\end{figure}


\begin{figure}[h]
\centering
\includegraphics[height=6cm,width=9cm,keepaspectratio]{p7033572.jpg}
\caption*{Il y en a des tas ! Partout !}
\end{figure}

On ne part pas tout de suite pour le trek : on se garde quand même une journée entière pour visiter la ville. Ce sera notre "journée échec". D'abord, on commence par se lever hyper tôt pour aller voir le fameux marché à la bougie : A 5h du mat, on est en route pour le marché, et... il fait déjà jour ! Bref, on a vu une bougie, et on est retourné se coucher. En fait, quand on est arrivés avec notre bus de nuit, ça aurait été la bonne heure. Mais franchement, quelle idée de faire des marchés à 3h du matin ?


\begin{figure}[h]
\centering
\includegraphics[height=6cm,width=9cm,keepaspectratio]{p6303192.jpg}
\caption*{La seule bougie du marché.}
\end{figure}

L'échec se poursuit avec notre tentative ratée de marcher jusqu'aux fameuses sources chaudes. Notre GPS indiquait bien un sentier, mais après avoir mis 30mn pour faire 500m au milieu des rizières inondées, à ne pas savoir si on marche sur un chemin ou dans un ruisseau, on a fait demi-tour. Et comme souvent, c'est quand on décide d'arrêter d'essayer de faire des trucs que ça commence à s'arranger : on tombe d'abord sur "little bagan". Ce sont 4 petites pagodes en ruines. C'est joli et paisible, même si le nom est un peu usurpé (ce serait comme si on baptisait "Little Paris" un pylône de haute tension), mais surtout, nos pas nous amènent directement vers un restaurant tout aussi joli et paisible dans lequel on a pu manger des schniztels avec de la purée ! RRRRRAAAAAaaaaaaahhhhh enfin des gens qui savent cuisiner la viande et les patates !!!! Si vous allez dans le coin, ça s'appelle "popcorn garden".


\begin{figure}[h]
\centering
\includegraphics[height=6cm,width=9cm,keepaspectratio]{p6303200.jpg}
\caption*{Voilà ce qui arrive quand on ne désherbe pas.}
\end{figure}

Le lendemain, nous partons pour le trek. Nous avons deux guides pour nous : Pioupiou, le guide officiel et Danny, son pote en formation, 34 ans à eux deux. Oui, 34 ans, c'est aussi mon âge à moi tout seul (Aïe). On se balade de petits villages en petits villages, et les langues changent tous les quelques kilomètres : birman, puis palaung, puis shan. On voit des champs d'ananas, et aussi des champs de thé qui ici, ne sont cultivés que par des Palaung, et on devine en voyant les maisons que le thé est plus rentable que les ananas. Qui l'eut cru ?


\begin{figure}[h]
\centering
\includegraphics[height=6cm,width=9cm,keepaspectratio]{p7013358.jpg}
\caption*{Une mémé dans le village.}
\end{figure}


\begin{figure}[h]
\centering
\includegraphics[height=6cm,width=9cm,keepaspectratio]{p7013526.jpg}
\caption*{Notre hôtesse et sa mère.}
\end{figure}

En arrivant au village où nous devons dormir, on apprend deux choses : Premièrement, la maison dans laquelle on dort appartient à Mr Charles. Deuxièmement, il y a des combattants plus loin dans la vallée, donc hors de question d'aller plus loin, on fait demi-tour demain matin. Dommage... Mais cela ne nous empêche pas d'adorer ce petit village et même de se faire inviter pour le thé chez la maman de notre hôte !


\begin{figure}[h]
\centering
\includegraphics[height=6cm,width=9cm,keepaspectratio]{p7013529.jpg}
\caption*{Le village en question.}
\end{figure}

Le lendemain, ce sera donc retour anticipé, heureusement par un autre chemin. Le point d'orgue de la journée (d'après notre guide) : une cascade ! On a essayé d'être enthousiastes, et de lui dire que oui oui, elle était très belle, et très haute, et très... humide. Je ne sais pas si on a été très convaincants.


\begin{figure}[h]
\centering
\includegraphics[height=6cm,width=9cm,keepaspectratio]{p7043577.jpg}
\caption*{Sur le quai de la gare.}
\end{figure}



\chapter{Kyaukme, un village occupé, un moine nominé}
Nous arrivâmes donc à Kyaukme (prononcez "Chie au met"), après un petit voyage en train sur la voie la moins bien entretenue du monde ! Dès qu'on dépasse les trente à l'heure, le train se met à tanguer de manière impressionnante, et les abords des voies sont si mal entretenus (entendre, pas du tout), que c'est le train lui-même qui élague approximativement les branches et les herbes hautes à chaque passage. Marion l'apprendra à ses dépends, il est dangereux de se tenir près d'une fenêtre ouverte : c'est un coup à se prendre une branche dans la margoulette !


\begin{figure}[h]
\centering
\includegraphics[height=6cm,width=9cm,keepaspectratio]{p7053598.jpg}
\caption*{C'est un forgeron. Vous comprendrez plus tard.}
\end{figure}

Kyaukme s'ouvre doucement au tourisme : il n'y a que deux guesthouses autorisées pour les touristes. Nous faisons une petite balade dans la ville, qui nous amène dans un temple au sommet d'une petite colline. Le jardinier nous accueille chaleureusement et s'improvise immédiatement guide. Moment notable de la visite : au moment de redescendre de l'autre coté de la colline, on se fait presque encercler par une bande de chiens errants et - une fois n'est pas coutume - agressifs. Demi-tour.


\begin{figure}[h]
\centering
\includegraphics[height=5cm,width=9cm,keepaspectratio]{p7053631.jpg}
\caption*{La seule qui ne s'arrête pas de bosser pour la photo, c'est la femme enceinte. Si, je vous jure.}
\end{figure}


\begin{figure}[h]
\centering
\includegraphics[height=5cm,width=9cm,keepaspectratio]{p7053634.jpg}
\caption*{Là, vous me croyez ?}
\end{figure}

La ville est surtout connue pour être le point de départ des treks organisés par Thura. Nous le rencontrons le soir même, ainsi qu'un autre couple de Français, et décidons de partir tous les cinq le lendemain ! Une condition : je vais devoir apprendre, sur le tas, à conduire une moto semi-automatique, car le début du trek est loin du centre, et c'est le seul moyen de transport possible.


\begin{figure}[h]
\centering
\includegraphics[height=5cm,width=9cm,keepaspectratio]{p7053645.jpg}
\caption*{Ici, on prépare...}
\end{figure}


\begin{figure}[h]
\centering
\includegraphics[height=5cm,width=9cm,keepaspectratio]{p7053647.jpg}
\caption*{... des "nouilles shan"...}
\end{figure}


\begin{figure}[h]
\centering
\includegraphics[height=5cm,width=9cm,keepaspectratio]{p7053646.jpg}
\caption*{...et de la salade de thé fermenté !}
\end{figure}

Après quelques aller-retour dans une rue déserte, me voici avec deux packs d'eau entre les jambes, tout fier au guidon de ma moto, en train de rouler sur les petites routes birmanes ! Étant le pilote débutant, je suis le seul des trois motos à ne pas avoir de passagère dans le dos.



Le rythme est cool, et nous faisons de nombreuses pauses. Par exemple pour tester la noix de bétel à chiquer : on a eu droit à une version pour touristes, avec plein d'épices et peu de noix. Malgré ces artifices, j'ai envie de dire que "ça reste une boisson d'homme". Ce n'est pas très bon, j'ai surtout l'impression de mâcher du bois vert, et avec le peu de noix, difficile de sentir un quelconque effet...


\begin{figure}[h]
\centering
\includegraphics[height=5cm,width=9cm,keepaspectratio]{p7053621.jpg}
\caption*{L'usine de papier de bambou.}
\end{figure}

Petite pause également chez un forgeron local, chez qui on fera une partie de billard indien, avant d'aller visiter deux usines de papier de bambou : une manuelle et une automatique.


\begin{figure}[h]
\centering
\includegraphics[height=5cm,width=9cm,keepaspectratio]{p7053613.jpg}
\caption*{Le billard indien.}
\end{figure}

Après quelques kilomètres sur les petits chemins dans la montagne, on comprend mieux l'intérêt de la moto : route défoncée, mares de boue, cailloux... on est tout le temps en train de slalomer, et plus on s'approche de notre destination, pire c'est. Pour le dernier kilomètre, les passagères deviennent des piétons et partent devant, pendant qu'on galère à 2 km/h sur le sentier. Et enfin, nous arrivons : des maisons en bambou, des rues pleines de gamins qui courent et de vieux qui errent, nous voici dans un village palaung.


\begin{figure}[h]
\centering
\includegraphics[height=5cm,width=9cm,keepaspectratio]{p7053676.jpg}
\caption*{Quatre touristes devant le nez, et pas une qui le lève de ses leçons !}
\end{figure}


\begin{figure}[h]
\centering
\includegraphics[height=5cm,width=9cm,keepaspectratio]{p7053686.jpg}
\caption*{Un petit garçon qui a tout compris à la composition !}
\end{figure}

Après une petite visite de l'école, très studieuse, on va rendre visite ... roulement de tambours ... aux militaires ! (Je crois que j'ai rarement utilisé des roulements de tambours de manière autant appropriée) Hé oui, le village est occupé par l'armée Shan. Pour faire simple, et parce que je ne suis pas sûr d'avoir compris beaucoup plus, les Shans forment la minorité la plus importante du Myanmar, et certains voudraient l'indépendance de l'état Shan, et considèrent que les Palaungs sont sur leur territoire.


\begin{figure}[h]
\centering
\includegraphics[height=5cm,width=9cm,keepaspectratio]{p7053739.jpg}
\caption*{Au milieu, c'est le chef !}
\end{figure}

On fait la rencontre d'une bande de jeunes qui squattent au sommet du village, et ont l'air ravi de rencontrer des occidentaux ! Sérieusement, cette rencontre a été surréaliste : on a pris des selfies avec des rebelles armés ! Et ils étaient super contents ! Ils sont même allés chercher leur pétoire pour poser avec, avant de nous les prêter ! On apprend qu'ils n'ont pas vraiment le choix d'être ici : de temps en temps, l'armée Shan débarque dans un village, et demande tant de volontaires. Le village a intérêt à les fournir.


\begin{figure}[h]
\centering
\includegraphics[height=5cm,width=9cm,keepaspectratio]{p7053727.jpg}
\caption*{Les journées sont longues. L'un d'entre eux nous a avoué que ça faisait deux ans qu'il n'était pas rentré chez lui, alors que c'est à moins d'une journée de marche.}
\end{figure}

Ces "volontaires" sont ensuite catapultés dans un village à occuper. Sans solde, sans nourriture... Du coup, les volontaires catapultés se retrouvent à devoir demander au village de les nourrir. Difficile de promouvoir l'amitié des peuples dans ces conditions, mais malgré tout j'ai eu l'impression que la plupart des gens (villageois et militaires) étaient conscients qu'ils étaient tous plus des victimes qu'autre chose. On a vu par exemple les militaires descendre de leur colline le soir pour aller aider les villageois à rouler le thé fermenté...


\begin{figure}[h]
\centering
\includegraphics[height=5cm,width=9cm,keepaspectratio]{p7053755.jpg}
\caption*{Ici, le thé se boit un peu, mais se mange surtout fermenté. Première étape : le cuire à la vapeur.}
\end{figure}


\begin{figure}[h]
\centering
\includegraphics[height=5cm,width=9cm,keepaspectratio]{p70638211.jpg}
\caption*{Puis il est roulé/froissé avec ces machines.}
\end{figure}


\begin{figure}[h]
\centering
\includegraphics[height=5cm,width=9cm,keepaspectratio]{p7053764.jpg}
\caption*{C'est convivial, on peut s'y mettre à quatre.}
\end{figure}


\begin{figure}[h]
\centering
\includegraphics[height=5cm,width=9cm,keepaspectratio]{p7053689.jpg}
\caption*{Et on obtient ça ! C'est un peu comme des épinards, mais amer. C'est leur spécialité locale, ils en sont très fiers, alors je ne vais rien dire.}
\end{figure}

Nous commençons à marcher seulement le lendemain, en direction d'un autre village. Ce ne sera pas très intense : 8km en 4 heures, ce n'est pas trop violent. Thura connait tout le monde, et chaque petite cabane est l'occasion d'une pause : on s'arrête manger du jack-fruit tout juste cueilli, on rencontre plusieurs de cueilleurs de thé, et on fait une pause chez un pépé qui a un fusil ! Mais quel fusil ! Pas de cartouche, c'est à l'ancienne : un bout de coton, de la poudre et des morceaux de plomb taillés au canif, 10mn en tout pour recharger. A ce qu'il parait, il chasse le sanglier avec !


\begin{figure}[h]
\centering
\includegraphics[height=5cm,width=9cm,keepaspectratio]{p7063826.jpg}
\caption*{Rien à voir avec le durian. Le jack-fruit, c'est bon ! Le seul problème : c'est hyper collant. Si on le touche une fois, les mains collent pour la journée. Le dépiauter à la cuillère est devenu un art pour certains !}
\end{figure}


\begin{figure}[h]
\centering
\includegraphics[height=5cm,width=9cm,keepaspectratio]{p7063834.jpg}
\caption*{Et là, je hurle :"ATTENTION, UN OURS" ! Haha, non, je déconne...}
\end{figure}

Ce deuxième village ressemble beaucoup au premier, si ce n'est qu'il est libre de militaires. Les gens sont très curieux et très accueillants : on se fait tellement inviter à boire le thé qu'on ne fait plus que ça de la journée. Le moment est bien choisi : on nous explique que ce soir c'est la fête, car le lendemain matin, un novice sera intronisé moine, et prendra la direction du monastère qui était orphelin depuis le départ du moine précédent. C'est très rare, et le village est en effervescence, et tout le monde est en train de se préparer (entendre, les femmes cuisinent et les hommes boivent et fument). On passe ainsi la journée à boire du thé, et à prendre des photos avec tout le monde. Alex et moi sommes même pris à part un moment, et amené dans une petite pièce avec que des hommes, où on nous a discrètement offert de l'alcool de riz et de la viande fermentée. On avait l'impression d'être des ados en train de boire en cachette des parents !


\begin{figure}[h]
\centering
\includegraphics[height=5cm,width=9cm,keepaspectratio]{p7063871.jpg}
\caption*{Oui, le chef du village me caresse le genou... "Marion, fait quelque chose !"}
\end{figure}


\begin{figure}[h]
\centering
\includegraphics[height=5cm,width=9cm,keepaspectratio]{p7063842.jpg}
\caption*{Une petite fille très photogénique.}
\end{figure}

Le soir arrive, et la fête commence au son des tambours et des cymbales. "Tsoum tsoum tsoum BAM" en boucle. Pendant des heures. C'était tellement simple que j'ai réussi à en jouer du premier coup. C'était très monotone, mais ça les éclatait ! La petite particularité marrante, c'est qu'ils étalent du riz cuit sur la peau du tambour. Le son serait meilleur. On a surtout la main qui colle à la fin de la session.


\begin{figure}
\centering
  \begin{subfigure}[t]{0.4\textwidth}
    \includegraphics[width=\textwidth]{p7064401.jpg}
    \caption*{Ça vaut 10 point hippies d'un coup !}
  \end{subfigure}
  %
  \begin{subfigure}[t]{0.4\textwidth}
    \includegraphics[width=\textwidth]{p7064388.jpg}
    \caption*{Peu de gens le savent : c'est comme ça qu'on fait des galettes de riz soufflé !}
  \end{subfigure}
\end{figure}

Nous attendions surtout la danse, alors on a été un peu déçu quand la messe a commencé... C'était long ! Déjà que c'est long quand on comprend la langue, mais quand on ne comprend ni la langue, ni la religion... Au bout d'une heure dans le brouillard (il était 22h), on est rentré faire une sieste en se disant qu'on ressortirait quand on entendrait la musique. Et contre toute attente, nous l'avons vraiment fait ! A 23h, après deux heures de messe, les tambours sont chauds, et nous aussi, et on assiste à la fameuse danse. Une file de filles tourne en rond autour d'un poteau. Puis une file de garçons tourne autour des filles. Voilà. 23h30 nous étions couchés à nouveau.


\begin{figure}[h]
\centering
\includegraphics[height=5cm,width=9cm,keepaspectratio]{p7064416.jpg}
\caption*{La danse tant attendue !}
\end{figure}

La procession nous a réveillé à l'aube. Tambour, cymbales et la moitié du village juste sous sa fenêtre, c'est plutôt efficace ! Ils ont fait quelques tours autour du monastère, en montrant à tous les offrandes, et ils ont planté un mât en l'honneur du moine avant d'attaquer une nouvelle messe. Ce coup-ci, on n'a pas attendu et on s'est éclipsés discrètement, pour retourner à la civilisation.


\begin{figure}[h]
\centering
\includegraphics[height=5cm,width=9cm,keepaspectratio]{p7074418.jpg}
\caption*{La procession matinale.}
\end{figure}


\begin{figure}[h]
\centering
\includegraphics[height=5cm,width=9cm,keepaspectratio]{p7074433.jpg}
\caption*{C'est la fête !}
\end{figure}


\begin{figure}[h]
\centering
\includegraphics[height=5cm,width=9cm,keepaspectratio]{p7074445.jpg}
\caption*{"Je peux vous prendre en photo ?" Les avis semblent partagés...}
\end{figure}


\begin{figure}[h]
\centering
\includegraphics[height=5cm,width=9cm,keepaspectratio]{p7074453.jpg}
\caption*{Elles sont vraiment habillées comme ça tous les jours. La tradition, ça veut dire quelque chose ici !}
\end{figure}

\chapter{Le choc de l'Inde}
Nous arrivâmes donc à Delhi. Par avion. L'échec. Je sais, c'est triste. Mais pour notre défense, on n'avait pas le choix si on voulait visiter l'Inde cette année : les règles d'obtention du visa indien ayant changé au début de notre voyage - je vous passe les détails - la seule possibilité était le visa électronique, facile à obtenir, mais nous imposant de rentrer dans le pays par un aéroport et nous limitant à 30 jours dans le pays. On est arrivé en fin d'après midi, et Sophie, la copine de Marion, arrivait au même aéroport 5h plus tard.


\begin{figure}[h]
\centering
\includegraphics[height=6cm,width=9cm,keepaspectratio]{p7214486.jpg}
\caption*{C'est quand même pratique l'avion !}
\end{figure}

On est sorti faire un tour, histoire de se dégourdir les jambes, et face à la chaleur ambiante, on se précipite pour rentrer et attendre les 5h à venir au frais. Non. Comment ça non ? Ben non, on ne rentre pas dans l'aéroport sans billet d'avion. Ah, c'est ballot. Bon, ben c'est pas grave, on va se poser sur un morceau de trottoir brûlant en attendant Sophie...


\begin{figure}[h]
\centering
\includegraphics[height=6cm,width=9cm,keepaspectratio]{p7214492.jpg}
\caption*{Complètement blasés de prendre l'avion...}
\end{figure}

Le premier contact avec l'Inde est rude : déjà, la circulation ! Je sais, j'ai déjà dit beaucoup de mal de la circulation de beaucoup de pays. Mais. Et je pèse mes mots. C'est. L'Inde. Le. Pire. Tout le monde force le passage en permanence, les motos, les tuk-tuks, les piétons, les voitures, les chariots, les vélos et les camions slaloment entre les vaches qui s'en foutent et les cratères de la route défoncée. Les gens se bloquent mutuellement en refusant de céder un centimètre, et foncent dès qu'ils ont deux mètres de libre devant eux, le tout en utilisant leurs putains de bordel de klaxons modifiés de merde qui défoncent les oreilles ! Autre exemple : je suis dans une rue large comme un couloir, une moto arrive par derrière en ronflant, je me pousse, mais il klaxonne quand même ! Je HAIS les klaxons, merde !!!! (pfiou, désolé que vous ayez du endurer tout ça, mais ça fait un bien fou).


\begin{figure}[h]
\centering
\includegraphics[height=6cm,width=9cm,keepaspectratio]{p7295285.jpg}
\caption*{Normal.}
\end{figure}

Et puis les arnaques. Le chauffeur a tenté tout ce qu'on avait lu sur le web : heureusement qu'on était préparés. Le coup classique, c'est le touriste qui débarque à l'aéroport avec sa réservation d'hôtel. Le chauffeur, sympa, offre de passer un coup de fil pour prévenir de l'arrivée du voyageur. Et là, pas de bol, au téléphone, un mec explique au touriste qu'il y a eu un problème, blablabla, bref, l'hôtel est complet. Mais heureusement, le taxi a un pote... et le touriste se retrouve dans un autre hôtel. Évidemment, c'était un coup monté, et c'est à un complice, et non à l'hôtel que le touriste a parlé par le téléphone.


\begin{figure}[h]
\centering
\includegraphics[height=6cm,width=9cm,keepaspectratio]{p7224496.jpg}
\caption*{Quelques friandises apportées par Sophie pour adoucir le choc.}
\end{figure}

Il y a aussi, "le centre ville est fermé la nuit", "tous les hôtels de Delhi sont complets, mais je peux te conduire au Taj-Mahal", "c'est la révolution", "les extra-terrestres ont débarqué, mais tu peux dormir chez ma belle sœur" ou encore, l'hôtel porte le bon nom, mais c'est juste une mauvaise copie d'un hôtel qui marche mieux la rue d'à coté. Il faut toujours refuser les propositions de taxis, rester ferme sur la destination, et si possible, ostensiblement lui montrer qu'on sait où on va avec le GPS du téléphone. Et par défaut, quand un mec qui bosse dans le tourisme s'adresse à un touriste, c'est une bonne pratique de tout mettre en doute.


\begin{figure}[h]
\centering
\includegraphics[height=6cm,width=9cm,keepaspectratio]{p8056623.jpg}
\caption*{Quand on voit la tête de certains conducteurs, faut pas s'étonner du bordel dans les rues...}
\end{figure}

On fini par arriver sain et sauf à notre hôtel, après avoir cru mourir trois fois sur la route. Là, ils n'ont pas notre chambre, qu'on avait pourtant réservé. Il est 2h du matin, on en a marre, on se retrouve donc entassés dans la pire chambre de tout notre voyage. Le lendemain, on attaque l'organisation du reste du voyage, car il y a du boulot. On comptait aller dans les montagnes pour échapper à la mousson, mais la mousson en avait décidé autrement : inondations, glissements de terrain, ce n'est pas le bon moment... Direction le Taj-Mahal, et pour ça, il nous faut des billets de train. C'est là, qu'on a commencé à visiter des agences de voyage. La première était conseillée  par notre hôtel. Quand on en est sortis, un homme qui passait par là, nous a dit que cette agence était une arnaque, qu'ils mentent sur le fait de travailler avec le gouvernement ! Ah ben oui, c'est aussi ce qu'il nous semblait. Et ce mec nous propose de nous amener à la vraie agence. Tu parles. On en a fait trois comme ça, avant de comprendre qu'il fallait arrêter d'essayer d'être poli avec les gens qui faisaient semblant de vouloir nous aider. On a pris une carte, marché en ignorant les gens qui nous parlaient, et fini par arriver dans une agence. Ils nous ont dit que les trains pour le Taj Mahal étaient complets pour les 2 prochains jours. Et ils ont commencé à dérouler leur speech, et peu après, on ressortait de l'agence en compagnie de Karma, notre chauffeur pour les 14 prochains jours :-)


\begin{figure}[h]
\centering
\includegraphics[height=6cm,width=9cm,keepaspectratio]{p7244836.jpg}
\caption*{Sophie, Marion, et Karma, notre chauffeur.}
\end{figure}

Au programme : le Rajasthan ! Normalement, cet état aurait du être inondé à cause de la mousson, mais pas cette année. Ben oui, les nuages sont allé directement dans les montagnes, là où justement on ne peut plus aller. C'est vraiment le monde à l'envers... Mais avant, un petit tour par Agra pour visiter le Taj Mahal, et le départ c'est tout de suite ! Et rassurez-vous, même si ce premier article sur l'Inde semble un peu négatif, il n'y a pas que des arnaques et des klaxons en Inde...


\begin{figure}[h]
\centering
\includegraphics[height=6cm,width=9cm,keepaspectratio]{p8056621.jpg}
\caption*{Non, pas que des arnaques et des klaxons, il y a aussi des déchets partout... mais rassurez-vous, il n'y a pas que ça !}
\end{figure}

Et non, pas de photos pour le moment. Je n'ai pas pris une seule photo de Delhi, faudra attendre les articles suivants !

\chapter{Le Taj-Mahal, et comment j'ai noyé mon guide.}
Nous arrivâmes donc à Agra, berceau du Taj-Mahal. C'était le soir, et Karma, notre chauffeur, nous a tout de suite conduit à un endroit un peu secret pour admirer le coucher du soleil sur le "Taj" - comme on dit par ici - mais sans payer l'entrée du parc. C'est facile, on longe le parc jusqu'à la rivière, on contourne le champs de fouille archéologique, et on tombe là-dessus :


\begin{figure}[h]
\centering
\includegraphics[height=6cm,width=9cm,keepaspectratio]{p7224493.jpg}
\caption*{La photo de rend pas très bien compte de la taille. C'est grand !}
\end{figure}

Karma nous conduit ensuite à un hôtel de son choix. Il nous avait dit qu'il connaissait tous les hôtels, et qu'on avait qu'à donner notre budget. On s'est laissé tenté. L'hôtel est très bien, on a même du, pour la première fois, donner un pourboire au mec qui nous a monté nos bagages dans la chambre afin qu'il accepte de nous laisser tranquille ! Le seul truc qui nous chiffonne, c'est que le prix de l'hôtel est, à la roupie près, le budget maximum qu'on avait donné à Karma. On demandera moins cher la prochaine fois. Malgré ce petit détail, c'est très agréable d'avoir un chauffeur qui s'occupe de tout. Pas besoin de renseigner sur les transports, pas d'horaire à respecter, et en prime, on a même droit à quelques bons plans. Il se charge aussi de nous prévenir des différentes arnaques à touriste ayant cours ici et là. Bref, c'est les vacances !


\begin{figure}[h]
\centering
\includegraphics[height=6cm,width=9cm,keepaspectratio]{p7234504.jpg}
\caption*{L'entrée du Taj.}
\end{figure}


\begin{figure}[h]
\centering
\includegraphics[height=6cm,width=9cm,keepaspectratio]{p7234514.jpg}
\caption*{:-)}
\end{figure}

Levé à l'aube le lendemain, on part en direction du Taj-Mahal. Au lever de soleil, c'est triplement mieux : déjà, ben c'est le lever de soleil donc c'est joli, c'est déjà pas mal, ensuite, il fait frais, et ça, c'est appréciable, et enfin, la plupart des touristes sont des grosses flemmes qui ont du mal à se lever le matin (j'en sait quelque chose...) et donc il y a moins de monde, et ça c'est cool. Bref, c'est beau, il y a un petit parc, plein de marbre, et chose étonnante : pour la première fois depuis notre arrivée en Inde, aucun déchet en vue !




\begin{figure}[h]
\centering
\includegraphics[height=6cm,width=9cm,keepaspectratio]{p7234528.jpg}
\caption*{Sophie tente de méditer.}
\end{figure}


\begin{figure}[h]
\centering
\includegraphics[height=6cm,width=9cm,keepaspectratio]{p7234747.jpg}
\caption*{Vue depuis le Taj.}
\end{figure}


\begin{figure}[h]
\centering
\includegraphics[height=6cm,width=9cm,keepaspectratio]{p7234745.jpg}
\caption*{Il y a foule !}
\end{figure}

Deuxième chose étonnante : l'audio-guide est très bon ! Il y a des anecdotes, de la mise en scène, de l'histoire, c'est raconté par de vrais acteurs. Mais... mais comment raconter ça. Le plus simple, c'est encore de faire la scène au ralenti, étape par étape :
\begin{itemize}
	\item \textbf{7:31:22,03} : Alors que j'étais immobile, à profiter de la vue en écoutant l'histoire du Taj, le velcro de l'étui qui contient le boitier de l'audio guide se défait. Le boitier commence à glisser vers le bas.
	\item \textbf{7:31:22,16} : Le câble audio tente de ralentir la descente du boitier. Sans succès. Le câble se détache. Le son se coupe. Le boitier tombe.
	\item \textbf{7:31:23,20} : La coupure du son m'étonne. Je suis presque sûr que le fronton n'est pas couvert de \emph{"gra"}. Je baisse le regard vers l'étui vide.
	\item \textbf{7:31:24,89} : J'amorce un hoquet de surprise. Le boitier heurte le chemin en pierre.
	\item \textbf{7:31:24,94} : Sous la violence du choc, le boitier s'ouvre. Un pigeon s'envole. A cause du bruit bien entendu. C'est une très mauvaise idée d'essayer de faire tenir un pigeon dans un audio guide.
	\item \textbf{7:31:25,56} : Le boitier ouvert rebondi. Je vois une carte électronique scintiller au soleil. C'est beau. L'eau scintille également. L'eau ? ... Oh putain, l'eau !
	\item \textbf{7:31:26,56} : Le boitier ouvert tombe dans 10cm d'eau.
	\item \textbf{7:31:28:02} : Les dernières bulles s'échappent du boitier, emportant mes derniers espoirs. Ma main atteint le boitier.
	\item \textbf{7:31:33} : Je vide et sèche tant bien que mal le boitier.
	\item \textbf{7:32:03} : Le boitier marche !
	\item \textbf{7:32:06} : Le boitier fume ?
	\item \textbf{7:32:07} : Bon, ben tant pis alors...
	\item \textbf{9:14:30} : On rend nos trois audio-guides en gardant un air innocent tout à fait convaincant, l'employé, teste le premier boitier. OK. Il teste le deuxième. OK. Il... range le troisième sans le tester.
	\item \textbf{9:14:45} : Marion me dit de recommencer à respirer. On fait demi-tour en marchant aussi vite qu'il est possible de marcher sans donner l'impression de fuir.
\end{itemize}



\begin{figure}[h]
\centering
\includegraphics[height=6cm,width=9cm,keepaspectratio]{p7234521.jpg}
\caption*{Le lieu du crime.}
\end{figure}


\begin{figure}[h]
\centering
\includegraphics[height=6cm,width=9cm,keepaspectratio]{p7234738.jpg}
\caption*{Il est interdit de prendre des photos à l'intérieur du Taj, alors gardez cette photo pour vous, et n'oubliez pas d'admirer la finesse des sculptures !}
\end{figure}

Sinon, le Taj-Mahal est un grand tombeau, symbole de l'amour indéfectible pour sa defunte épouse, d'un mec très riche, qui n'a pas hésité une seule seconde à ruiner son pays, se mettre ses héritiers à dos, déclencher une guerre de succession et plus ou moins détruire sa dynastie afin que nous, pauvres mortels puissions nous dire quelques siècles plus tard que, quand même, c'est beau l'amour.


\begin{figure}[h]
\centering
\includegraphics[height=6cm,width=9cm,keepaspectratio]{p7234759.jpg}
\caption*{Dernier coup d’œil en partant.}
\end{figure}





\chapter{Jaipur, la patrie des éléphants partiellement roses.}
Nous partîmes donc pour Jaipur, toujours avec Karma notre chauffeur. On fait une première pause à l'entrée de la ville au fameux temple des singes. Dans la religion hindou, Hanuman, le dieu singe, est un des dieux principaux, et est très aimé, d'où la présence de nombreux temples qui lui sont dédiés. Ce temple s'avère être une colline couverte de bâtiments et de singes. L'entrée est gratuite, alors on fait un geste en achetant un petit peu de nourriture auprès du "maitre des singes" comme il s'est présenté, très fier. On vient d'entrer, et tout de suite, un jeune nous fait un petit signe amical et nous invite à le suivre dans son temple.


\begin{figure}[h]
\centering
\includegraphics[height=6cm,width=9cm,keepaspectratio]{p7234786.jpg}
\caption*{Un singe et sa mangue.}
\end{figure}

Il nous présente rapidement son temple et commence à nous bénir en nous attachant un bracelet en ficelle (hop, 1 point hippie) et un coup de peinture rouge sur le front. C'est sympa, c'est typique, même si, n'étant pas croyant, je trouve ça un peu, voire carrément hypocrite. Comme il se doit, on se penche pour faire une petite offrande de 10 roupies. Et là, il nous arrête : "ah non, pour les étrangers, c'est 100 roupies". Bon, mon petit gars, c'est pas des manière de procéder, ce n'est pas ça le principe d'une donation. Mais bon, on n'est pas chez nous, on vient d'arriver, alors on va dire que c'est le prix de la leçon. Je donne 100 roupies. On fait mine de repartir, mais il nous arrête encore : "C'est 100 roupies : par personne". Ah. Eh bien ! Rien de tel qu'une bénédiction de bon matin pour se mettre en joie ! On le regarde dans les yeux, et on rajoute deux billets de 10 roupies. Ce sera notre dernière bénédiction en Inde.


\begin{figure}[h]
\centering
\includegraphics[height=6cm,width=9cm,keepaspectratio]{p7234782.jpg}
\caption*{Le prêtre scellant notre destin !}
\end{figure}

On continue quand même notre visite, tout en ignorant les appels émanant des nombreux temples qui émaillent le chemin grimpant la colline. On donne prudemment des cacahouètes une par une aux singes, qui hardis, préfèrent essayer de piquer tout le paquet d'un coup en venant par derrière ce qui a le don de nous rendre super nerveux.


\begin{figure}[h]
\centering
\includegraphics[height=6cm,width=9cm,keepaspectratio]{p72348191.jpg}
\caption*{Les singes font les cons dans l'eau.}
\end{figure}

Des locaux qui parlent un peu anglais commencent à faire le chemin avec nous en papotant de tout et de rien. Et là, on est dans un situation délicate : on sait que certaines personnes font ça juste pour pouvoir demander de l'argent en tant que guide un peu plus tard. Il y en a aussi qui sympathisent pour augmenter leurs chances de vente de bracelets/bénédictions/service de guide etc. Alors comment se comporter avec ces gens qui semblent super accueillants ? Un peu refroidi par la fausse bénédiction, on reste sur nos gardes, distants, en répondant de manière évasives aux questions. Au moment de partir, le plus "collant" prend son courage à deux mains, et demande si on peut prendre une photo tous ensemble. Ce n'était donc que ça ? Évidemment qu'on va prendre une photo ! Deux même ! Et là, je me sens un peu con/mal/frustré : on s'ouvre, on se fait arnaquer, et quelques instant plus tard, on est distants et méfiants avec des gens qui voulaient sincèrement faire notre connaissance...


\begin{figure}[h]
\centering
\includegraphics[height=6cm,width=9cm,keepaspectratio]{p7234789.jpg}
\caption*{Ambiance Indiana Jones/Tomb Raider (à choisir en fonction de votre génération).}
\end{figure}

Enfin c'est le moment de découvrir l'hôtel que Karma nous a sélectionné. On a bien précisé qu'on voulait moins cher ce coup-ci. Et on a ... 100 roupies de moins ! Youpie ! Bref, il voit qu'on est pas content, et nous conduit dans un autre hôtel, où, après d'âpres négociations, on obtient le même prix qu'au premier hôtel. Super. Le problème que l'on commence à entrevoir, est que Karma touche une commission quasiment partout où il nous conduit : restaurants, hôtels, boutiques, ce qui rend les prix plus difficiles à négocier. On va donc changer de stratégie à l'avenir, mais aujourd'hui, on est fatigués.


\begin{figure}[h]
\centering
\includegraphics[height=6cm,width=9cm,keepaspectratio]{p7244835.jpg}
\caption*{Un vendeur de \emph{massala tea} : un thé épicé excellent qu'on trouve partout.}
\end{figure}


\begin{figure}[h]
\centering
\includegraphics[height=6cm,width=9cm,keepaspectratio]{p7244851.jpg}
\caption*{Sur la route, priorité aux éléphants.}
\end{figure}

Tout ceci ne nous empêche pas de profiter du magnifique Fort d'Amber. C'est un grand fort construit sur une colline. Sa particularité est qu'on peut y monter à dos d'éléphant ! Une file continue de pachydermes multicolores chargés de touristes radieux jusqu'aux oreilles alimente l'immense porte d'entrée du fort. Il y a même des embouteillages ! Oui madame, des embouteillages d'éléphants roses (et bleus, et verts) ! Et maintenant, juste pour rire, imaginez Lyon, le tunnel de Fourvière, un week-end de départ en vacances. Ne pensez pas à des éléphants roses.


\begin{figure}[h]
\centering
\includegraphics[height=6cm,width=9cm,keepaspectratio]{p7244882.jpg}
\caption*{Ne vous inquiétez pas, les gens à jeun voient probablement la même chose que vous}
\end{figure}


\begin{figure}[h]
\centering
\includegraphics[height=6cm,width=9cm,keepaspectratio]{p7244875.jpg}
\caption*{Heureusement, les éléphants ont un rayon de braquage pas trop mauvais.}
\end{figure}


\begin{figure}[h]
\centering
\includegraphics[height=6cm,width=9cm,keepaspectratio]{p7244870.jpg}
\caption*{Coucou !}
\end{figure}


\begin{figure}[h]
\centering
\includegraphics[height=6cm,width=9cm,keepaspectratio]{p7244898.jpg}
\caption*{Petite pause au calme.}
\end{figure}

Et comme les palais, ça commence à bien faire, plutôt que d'aller ensuite visiter le classique palais de Jaipur, on opte pour l'observatoire, le Jantar Mantar, qui est très bien selon notre guide, et qui montre une collection impressionnante d'instruments de mesure. Et ça tombe bien, il commence à pleuvoir, alors on se précipite acheter nos billets, et on rentre à l'intéri... comment ça pas d'intérieur ? On découvre alors que les instruments de mesure sont grands comme des maisons et sont dans un parc à l'extérieur, et il pleut un peu comme si c'était la mousson en Inde.


\begin{figure}[h]
\centering
\includegraphics[height=6cm,width=9cm,keepaspectratio]{p72448991.jpg}
\caption*{C'est bien beau, mais sans soleil, c'est dommage...}
\end{figure}

J'avoue, j'exagère un peu afin d'augmenter l'intensité dramatique, mais on finit quand même par nous indiquer au bout d'un moment une petite salle qui projette un film explicatif. La salle est pleine comme un tuk-tuk, mais nous permet d'attendre au sec. On découvre ensuite une collection de super sextants et de gnomons géants. La pièce la plus impressionnante est un cadran solaire de 27 mètres de haut, précis à deux secondes près. J'apprécie d'autant plus la compacité de ma montre, qui a aussi l'autre avantage de marcher même à l'ombre. Parce que là, même si la pluie a cessé, le soleil est toujours caché. Dommage, alors qu'on vient de passer toute la matinée à cuire au soleil en visitant le Fort...


\begin{figure}[h]
\centering
\includegraphics[height=6cm,width=9cm,keepaspectratio]{p7254918.jpg}
\caption*{L'inévitable palais des vents, construit pour garder un harem au frais, tout en gardant ses occupantes invisibles, mais diverties par le spectacle de la rue.}
\end{figure}





\chapter{Pushkar, ses collines et son lac sacré.}
Nous arrivâmes donc à Pushkar. On sent bien que notre chauffeur est un peu inquiet : on a réservé nous-même notre hôtel via internet. Pas de commission pour Karma, mais on a un meilleur hôtel pour deux fois moins cher que les précédents choisis par Karma, ce qui nous amène à nous poser des questions sur le montant des commissions ! Mais que cela ne nous empêche pas de profiter de la visite : Pushkar est une petite ville sainte construite autour d'un lac. Ce dernier et sacré, et de nombreux Indiens y viennent en pèlerinage.


\begin{figure}[h]
\centering
\includegraphics[height=6cm,width=9cm,keepaspectratio]{p7265002.jpg}
\caption*{Le temple Sikh.}
\end{figure}

Là encore, il faut naviguer entre les arnaques. La plus connue : un passant sympathique te donne quelques pétales de fleurs à jeter dans le lac, car ça porte chance. Au bord du lac, les touristes sont aussitôt repérés par de faux prêtres qui vont faire pression jusqu'à obtenir une sérieuse donation. Bref, c'est de l'extorsion. Donc on envoie balader sans ménagement tous les gens qui nous offrent des fleurs, et on remarque bien qu'ils ne ciblent que les touristes ! Ce lac, parlons-en : on pensait voir un joli petit lac, et on découvre un grand bassin bétonné entouré d'escaliers et d'hôtels. Il n'y a quasiment pas de végétation, c'est un peu triste...


\begin{figure}[h]
\centering
\includegraphics[height=6cm,width=9cm,keepaspectratio]{p7264994.jpg}
\caption*{Dans les rues de Pushkar.}
\end{figure}

Karma nous fait découvrir le temple Sikh. Son temple en quelque sorte, étant donné qu'il est lui-même Sikh. Le temple est tout neuf, construit il y a à peine quelques années, le marbre est immaculé, bref, c'est magnifique. On y entre pieds nus et tête couverte. On observe d'abord un prêtre récitant des prières au micro avant de descendre d'un étage, rejoindre la cantine : tous les temples Sikhs font office de soupe populaire, et fournissent un repas simple à qui le demande, sans conditions. Là au moins, on sait pourquoi on fait une donation !


\begin{figure}[h]
\centering
\includegraphics[height=6cm,width=9cm,keepaspectratio]{p7254935.jpg}
\caption*{Le prêtre qui s'occupe de la distribution des repas.}
\end{figure}

On se lève tôt le matin pour aller observer le lever de soleil depuis une petite colline proche. On marche, on s'installe, et on observe. Le lever de soleil est naze, mais le spectacle est fourni par les paons en contrebas qui font la court à grand coups de "LEON" sonores. Le suspens est à son comble quand un mâle faisant la roue, entouré de 5 ou 6 femelles, se retrouve face un compétiteur... Mais nous sommes dérangés par un prêtre qui sort soudain du tout petit temple un peu plus loin. Ben oui, une colline : un temple ! Sympa, il nous invite à voir la cérémonie du matin. On refuse gentiment, parce qu'on n'a jamais de combat de paon, alors que des temples... Un peu plus tard, on amorce la descente, et là, le prêtre ressort, vraiment en colère, et commence à nous insulter, à nous dire qu'il faut venir voir le temple, qu'il faut faire une donation, que l'Inde ce n'est pas gratuit, et qu'on n'a pas intérêt à revenir ! C'est sûr qu'ils vont attirer des gens comme ça...


\begin{figure}[h]
\centering
\includegraphics[height=6cm,width=9cm,keepaspectratio]{p7264959.jpg}
\caption*{Vous en comptez combien ?}
\end{figure}

Pour Marion, cette journée fut un peu éprouvante : déjà, on la fait lever aux aurores, alors qu'elle avait été malade toute la journée précédente. Ensuite, elle se fait engueuler par le prêtre. Et ce n'est pas fini : un peu plus tard, alors que nous étions entrain de regarder nos parantas cuire, et de papoter avec le cuistot, une vache a soudain jugé opportun de passer une de ses cornes dans une anse du sac de Marion, et de continuer son chemin comme si de rien était, sans se rendre compte qu'elle trainait une Marion dans son sillage... Et elle a fini la journée avec une Sophie sur le dos.


\begin{figure}[h]
\centering
\includegraphics[height=6cm,width=9cm,keepaspectratio]{img_paranta.jpg}
\caption*{Quelques instants avant l'encornage.}
\end{figure}


\begin{figure}[h]
\centering
\includegraphics[height=6cm,width=9cm,keepaspectratio]{p7265013.jpg}
\caption*{Je ne sais pas laquelle est la plus contente...}
\end{figure}

Le soir, coucher de soleil depuis la colline de l'autre coté du village. C'est encore un échec car il pleut, mais on peut admirer le plus petit téléphérique du monde : à vue de pif, il remplace 10mn de marche et 50m de dénivelé. Et ce coup-ci, pas de prêtre pour nous agresser, mais des gens qui ne comprennent pas pourquoi on tient tant à marcher alors qu'il y a un téléphérique tout neuf.


\begin{figure}[h]
\centering
\includegraphics[height=6cm,width=9cm,keepaspectratio]{p7264975.jpg}
\caption*{Même les arbres en béton ont perdu leurs feuilles...}
\end{figure}


\begin{figure}[h]
\centering
\includegraphics[height=6cm,width=9cm,keepaspectratio]{p7264964.jpg}
\caption*{Couleurs à vendre. Mais on n'a pas compris pourquoi en faire des tours.}
\end{figure}



\chapter{Bundi, Udaipur et Jodhpur.}
Nous arrivâmes donc à Bundi. C'est une petite ville connue pour son joli centre ville, son Fort, et le fait que Rudyard Kipling a vécu ici quelques mois le temps d'écrire Kim. Soyons honnête un instant : qui a lu ce livre ? Personnellement, je n'en avais jamais entendu parler avant d'arriver dans cette ville. En revanche, tout le monde connait le livre de la jungle, non ? Ben voilà, vous êtes à Bundi.


\begin{figure}[h]
\centering
\includegraphics[height=6cm,width=9cm,keepaspectratio]{p7275032.jpg}
\caption*{Les indiens adorent qu'on les prennent en photo !}
\end{figure}

Non, bon OK, il n'y a pas de boa ni de tigres dans le centre ville. Mais allez visiter le fort de Bundi ! Il est à moitié envahi par la végétation, et complètement envahi par les singes. Il ne faut pas beaucoup d'imagination pour sentir la présence du roi Louis ou de Sherkhan.


\begin{figure}[h]
\centering
\includegraphics[height=6cm,width=9cm,keepaspectratio]{p7285226.jpg}
\caption*{D'intrépides exploratrices.}
\end{figure}


\begin{figure}[h]
\centering
\includegraphics[height=6cm,width=9cm,keepaspectratio]{p7285224.jpg}
\caption*{Un ancien réservoir.}
\end{figure}


\begin{figure}[h]
\centering
\includegraphics[height=6cm,width=9cm,keepaspectratio]{p7285239.jpg}
\caption*{Vue depuis le sommet du fort.}
\end{figure}

On avait rencontré un peu avant d'autres touristes qui nous avait mis en garde contre les singes du fort qui étaient "agressifs" et avaient de "grandes dents", au contraire des autres petits singes sympas du centre ville. Ils nous conseillaient donc de prendre un guide qui sait comment chasser les singes. On imaginait des babouins... Bon, en fait, ces singes sont des simples macaques, les mêmes que partout ailleurs, il suffit de jeter un cailloux ou d'agiter un bâton pour qu'ils se barrent, du coup, on s'est retrouvé avec un guide un peu inutile sur les bras, qui a chassé un pauvre singe et nous a raconté à chaque nouvelle ruine que qu'il fallait imaginer ici des femmes en sari qui chantaient et jouaient de la musique.




\begin{figure}[h]
\centering
\includegraphics[height=6cm,width=9cm,keepaspectratio]{p7285250.jpg}
\caption*{Rappel d'un ancien commentaire : ça souri une chauve-souris ?}
\end{figure}


\begin{figure}[h]
\centering
\includegraphics[height=6cm,width=9cm,keepaspectratio]{p7275040.jpg}
\caption*{Une famille sympathique.}
\end{figure}


\begin{figure}[h]
\centering
\includegraphics[height=6cm,width=9cm,keepaspectratio]{p7285045.jpg}
\caption*{C'est une ville très colorée !}
\end{figure}


\begin{figure}[h]
\centering
\includegraphics[height=6cm,width=9cm,keepaspectratio]{p7285242.jpg}
\caption*{La palais de Bundi est lui aussi plutôt impressionnant.}
\end{figure}

Nous arrivâmes ensuite à Udaipur. Appelée aussi la "Venise du Rajasthan". Et c'est effectivement une très jolie ville : Il y a un palais sur le lac qui a servit de décors pour un vieux James Bond : Octopussy. Depuis cette époque glorieuse, la plupart des hôtels passent le film tous les soirs !


\begin{figure}[h]
\centering
\includegraphics[height=6cm,width=9cm,keepaspectratio]{p7295448.jpg}
\caption*{A croire que ces gens posaient pour moi !}
\end{figure}


\begin{figure}[h]
\centering
\includegraphics[height=6cm,width=9cm,keepaspectratio]{p7295455.jpg}
\caption*{:-)}
\end{figure}


\begin{figure}[h]
\centering
\includegraphics[height=6cm,width=9cm,keepaspectratio]{p7295466.jpg}
\caption*{Coucher de soleil sur le lac.}
\end{figure}

Plutôt qu'un film, nous avons vu un magnifique spectacle : De la danse avec des robes qui brillent, des marionnettes agiles et rigolotes, des instruments de musique improbable. Chose plutôt rare : les danseuses étaient de tous les âges, et on a pu constater que l'expérience, ça compte. En particulier quand elles se sont mises à jouer des sortes de clochettes : imaginez un petit disque en métal attaché au bout d'une ficelle. Le but est maintenant de frapper avec ce disque différentes clochettes attachées un peu partout sur le corps, sur le bout des pieds, au dos des mains, ou sur une épée tenue entre les dents !


\begin{figure}[h]
\centering
\includegraphics[height=6cm,width=9cm,keepaspectratio]{p7295716.jpg}
\caption*{Combien d'heures de pratiques pour en arriver là ?}
\end{figure}


\begin{figure}[h]
\centering
\includegraphics[height=6cm,width=9cm,keepaspectratio]{p7295675.jpg}
\caption*{Les danseuses se préparent.}
\end{figure}


\begin{figure}[h]
\centering
\includegraphics[height=6cm,width=9cm,keepaspectratio]{p7295687.jpg}
\caption*{Concentrée...}
\end{figure}


\begin{figure}[h]
\centering
\includegraphics[height=6cm,width=9cm,keepaspectratio]{p7295708.jpg}
\caption*{Le mec à 4 pattes incarne un tigre. Il faut le savoir.}
\end{figure}


\begin{figure}[h]
\centering
\includegraphics[height=6cm,width=9cm,keepaspectratio]{p7295738.jpg}
\caption*{C'est beau, non ?}
\end{figure}


\begin{figure}[h]
\centering
\includegraphics[height=6cm,width=9cm,keepaspectratio]{p7295754.jpg}
\caption*{...}
\end{figure}

Nous arrivâmes enfin à Jodhpur. Là encore : un fort. Mais quel fort ! Il écrase la ville par sa présence ! Et ce fort est en plus très bien entretenu, ce qui ne gâche rien. Il n'a jamais été conquis, malgré les nombreux sièges. C'est étonnant de voir les astuces de défenses anti-éléphants (haha, des défenses ... contre les éléphants, c'est ironique !) : des couloirs à angle droit pour les empêcher de prendre de la vitesse, des gros piques à hauteur de leur tête.


\begin{figure}[h]
\centering
\includegraphics[height=6cm,width=9cm,keepaspectratio]{p8016273.jpg}
\caption*{Le fort de Jodhpur.}
\end{figure}


\begin{figure}[h]
\centering
\includegraphics[height=6cm,width=9cm,keepaspectratio]{p8016277.jpg}
\caption*{Le fort vu de l'intérieur.}
\end{figure}

Jodhpur est aussi la capitale des épices. Alors on a fait le stock ! A notre retour, ceux qui nous rendrons visite auront droit a un peu de cuisine s'ils ont de la chance. La cuisine indienne est franchement excellente, et majoritairement végétarienne. Nous n'avons quasiment pas mangé de viande pendant tout notre séjour sans que ça nous manque. Le restaurant végétarien est la norme, et même quand il y a de la viande, c'est poulet ou mouton. Les vaches sont sacrées, et ça veut dire quelque chose là-bas ! On a aussi vu de nombreux porc se balader dans les villes, mais personne ne les mange. En même temps, les Hindous, les Jaïnistes et les Sikhs sont végétariens, et il ne reste que les Musulmans qui mangent un peu de viande. Bref, on a l'impression que la seule utilité des porcs est de manger des déchets et de nettoyer les caniveaux. Revenons à la nourriture : c'est TRÈS épicé. Quand on demande "pas du tout épicé" dans un restaurant, on obtient habituellement quelque chose qui en France est considéré comme le maximum supportable. A force d'en manger tous les jours, on commence à s'habituer, et j'ai même mangé quelques fois un plat épicé normalement. Disons qu'on le sent bien passer. Deux fois. Heureusement, le lassi, une sorte de yaourt à boire aide à faire passer toutes ces brûlures, et on en fait une consommation intensive.


\begin{figure}[h]
\centering
\includegraphics[height=6cm,width=9cm,keepaspectratio]{p7275018.jpg}
\caption*{Un petit porc dans la rue.}
\end{figure}

Histoire de pousser l'expérience de la cuisine indienne jusqu'au bout, on s'inscrit à un cours. C'est une famille très sympathique qui s'en occupe. Et ce fut honnêtement le meilleur repas de toute l'Inde : du lassi au safran, du fromage frit, du curry de chou-fleur, des boules de coco, ... 9 plats au total, décorés avec des feuilles d'argent, et surtout carrément délicieux. On est impatient d'avoir à nouveau une cuisine rien que pour nous ! L'histoire de cette famille est aussi très intéressante : c'est un mariage d'amour, chose très rare en Inde ! Ils ne viennent pas de la même caste, et leurs deux familles (oui, même la famille de la caste la plus basse) se sont farouchement opposé au mariage. Ils ont du fuir leur village, et ont galéré quelques années avant de remonter la pente, d'économiser assez pour acheter un tuk-tuk, et ensuite de lancer ces cours de cuisine qui marchent très très bien. Ils ont un garçon, et une fille qui veut devenir scientifique, et à présent, certains membres de leur famille commencent à leur parler à nouveau...


\begin{figure}[h]
\centering
\includegraphics[height=6cm,width=9cm,keepaspectratio]{img_20160801_212403.jpg}
\caption*{Après le cours, le repas !}
\end{figure}


\begin{figure}[h]
\centering
\includegraphics[height=6cm,width=9cm,keepaspectratio]{p8036353.jpg}
\caption*{Un haveli (maison de riche marchand) dans Mandawa.}
\end{figure}







 Oui, je sais je n'ai rien raconté sur Mandawa, mais en même temps, on n'a rien fait à part se balader dans les rues. C'est la première ville un peu calme qu'on a visité en Inde.

\chapter{Varanasi, la ville sacrée au bord du Gange.}
Nous partîmes donc pour Varanasi. Enfin... nous tentâmes de partir pour Varanasi ! On est arrivé à la gare deux heures avant le départ, histoire d'être sûr, mais le train est parti avec 6h de retard, et est arrivé à 16h de l'après-midi, au lieu de 6h45 du matin... C'est officiellement notre record de retard de train. Varanasi est aussi une ville sainte. Elle est au bord du fleuve sacré qu'est le Gange. Tout le long de la rive droite est couvert de Gaths, des temples/escaliers au bord de l'eau, qui permettent aux fidèles de venir se purifier.


\begin{figure}[h]
\centering
\includegraphics[height=6cm,width=9cm,keepaspectratio]{p8066934.jpg}
\caption*{Faites un voeu !}
\end{figure}

Le gath le plus en amont est très particulier, car il est dédié à la crémation. Les hindous brûlent leurs morts ici, car le lieu est tellement saint, que l'âme monte directement au paradis et interrompt ainsi le cycle des réincarnations. C'est garanti. Alors forcément, il y a de la demande ! Et visiblement, la caste qui gère la crémation en profite et fait payer cher tout le monde, en fonction de leurs revenus. Les cendres sont ensuite répandues dans le fleuve.


\begin{figure}[h]
\centering
\includegraphics[height=6cm,width=9cm,keepaspectratio]{p8066669.jpg}
\caption*{Crémation en arrière plan.}
\end{figure}


\begin{figure}[h]
\centering
\includegraphics[height=6cm,width=9cm,keepaspectratio]{p8066671.jpg}
\caption*{Il faut compter environ 200kg de bois pour un corps.}
\end{figure}

Il y a 5 exceptions à la crémation, pour 5 cas où les Hindous estiment que la personne est morte en étant pure. Les trois premiers sont : Les prêtres, les enfants, les femmes enceintes, ça semble assez logique. Les deux derniers cas sont plus originaux à mon avis : les gens morts d'une morsure de serpent. C'est lié à la nature divine de Krishna, qui a résisté à une morsure de cobra (mais ça l'a rendu tout bleu). Et le dernier, ce sont les lépreux ! Ben oui, ils perdent leur peau, et avec, toutes leurs impuretés. Tous ces gens n'ont pas besoin d'être purifiés par le feu après leur mort. Ils sont donc simplement jetés dans Gange, tel quel. Quand on sait ça, et qu'on voit un truc tout gonflé qui descend le courant, on espère très fort que c'est juste le cadavre d'une vache...


\begin{figure}[h]
\centering
\includegraphics[height=6cm,width=9cm,keepaspectratio]{p8066696.jpg}
\caption*{Purification matinale}
\end{figure}


\begin{figure}[h]
\centering
\includegraphics[height=6cm,width=9cm,keepaspectratio]{p8066700.jpg}
\caption*{Presque toutes les marches sont inondées.}
\end{figure}


\begin{figure}[h]
\centering
\includegraphics[height=6cm,width=9cm,keepaspectratio]{p8066628.jpg}
\caption*{Il faut parfois se faire sa propre baignoire dans les plantes.}
\end{figure}

Continuons cette charmante découverte des traditions du Gange : il existe une tribu, un ordre de moines, qui, à des fins de quête spirituelle et de recherche d'illumination, collectent les cadavres, et les mangent. Ce sont les Aghori. Évidemment, ils vivent en marge de la société, et ils dégoutent tous les autres Indiens. Mais quand même, c'est un pays assez fou !


\begin{figure}[h]
\centering
\includegraphics[height=6cm,width=9cm,keepaspectratio]{p8066635.jpg}
\caption*{La cérémonie du lever du soleil.}
\end{figure}

Pour découvrir la vie du fleuve, il faut se lever tôt : la cérémonie du lever du soleil est pratiquée tous les jours à 5 heures du matin. Ça consiste globalement à agiter à bout de bras des gros fumoirs d'encens. Et quand c'est une rangée de prêtre en costume qui le fait de manière synchronisée, ça rend plutôt bien ! Ce matin nous avons eu de la chance, car la cérémonie était suivie d'un cours de yoga, et apparemment, le professeur qu'on a eu est une célébrité. On a fait quelques exercices, les plus notables étant sans conteste l'imitation du cri du tigre, puis l'éclat de rire forcé qui dégénère en hilarité générale. Le Yoga, c'est plus fun que ce que je pensais ! Ensuite nous avons enchainé sur un tour en bateau, afin de voir l'activité du matin sur les différents ghats. Comme c'est la mousson, le niveau du Gange empêche de se promener autrement qu'en bateau le long du fleuve, et seul le haut des escaliers est accessible. On voit les pèlerins qui viennent faire leur toilette, dans une des eaux les plus polluées du monde. Il y en a même qui se gargarisent, c'est sûrement pour renforcer le système immunitaire. Et avant que vous ne me posiez la question : oui, soyez rassurés, la crémation a bien lieu en aval de tous les gaths.


\begin{figure}[h]
\centering
\includegraphics[height=6cm,width=9cm,keepaspectratio]{p8066718.jpg}
\caption*{Une vendeur de bougies.}
\end{figure}

Pendant le tour en bateau, nous sympathisons avec notre guide, et il nous fait rencontrer un de ses amis astrologue, et nous avons ainsi droit à un thème astral gratuit ! Le soir, c'est le moment d'admirer le cérémonie du coucher du soleil. Même principe que le matin, mais avec des gros bougeoirs, ce qui est d'autant plus beau qu'il fait nuit. A la fin de la cérémonie, les fidèles, ainsi que les touristes vont mettre à l'eau une petite bougie.


\begin{figure}[h]
\centering
\includegraphics[height=6cm,width=9cm,keepaspectratio]{p8066973.jpg}
\caption*{Pyrotechnie, te voici !}
\end{figure}


\begin{figure}[h]
\centering
\includegraphics[height=6cm,width=9cm,keepaspectratio]{p8066988.jpg}
\caption*{Trois bougies, trois vœux !}
\end{figure}

Nous avons aussi visité la ville à pied, mais quelle galère ! Nous somme tombé le jour d'un festival, et les rues étaient bondées. Cela dit, c'était très intéressant de voir tous ces gens habillés en orange aller remplir les temples ! Nous ne sommes pas aller voir les crémations (sauf depuis le bateau). Même si c'est une attraction touristique, ça ne nous semble pas vraiment approprié. Je ne peux m'empêcher d'imaginer mes sentiments si une bande de touristes venaient visiter l'enterrement d'un de mes proches...


\begin{figure}[h]
\centering
\includegraphics[height=6cm,width=9cm,keepaspectratio]{p8077041.jpg}
\caption*{Les rues en plein festival.}
\end{figure}

Le temps passe, et c'est déjà l'heure du retour pour Sophie. Ce coup ci, pas de retard, le train arrive à l'heure, tout comme l'avion.

\emph{De Marion :} Merci ma Sophie d'être venue nous voir, d'avoir apporté avec toi de la bonne humeur, des mains magiques qui, je crois, ont réparé mon genou, et de l'énergie sans fin (sauf à l'heure de la sieste). Vivement les prochaines aventures ensemble !


\begin{figure}[h]
\centering
\includegraphics[height=6cm,width=9cm,keepaspectratio]{p8077043.jpg}
\caption*{La gare sous la pluie. Les quais sont immenses : jusqu'à 1km de long !}
\end{figure}


\begin{figure}[h]
\centering
\includegraphics[height=6cm,width=9cm,keepaspectratio]{p8097075.jpg}
\caption*{Déjà l'heure du départ...}
\end{figure}




\tableofcontents
\end{document}
