
\documentclass{book}
% Chargement d'extensions
\usepackage[francais]{babel} % Pour la langue française
\usepackage[utf8]{inputenc}
\usepackage[T1]{fontenc}
\usepackage{float}
\usepackage{eurosym}
\usepackage{textcomp}
\usepackage{wrapfig} % to wrap figure in text
%\usepackage{caption} % to remove fig #
\usepackage[font=it]{caption}
\usepackage{titlesec} % to modify chapter header
\usepackage[normalem]{ulem}
\usepackage{soul}

\titleformat{\chapter}[display]
  {\normalfont\bfseries}{}{0pt}{\Large}

\usepackage[
  paperwidth=15.557cm,
  paperheight=23.495cm,
  %showframe,
  margin=15mm
  % other options
]{geometry}

\hyphenation{CRRR-RA-A-A-A-AK-K-K-K-K-KR-R-R-RB-B-B-B-BRA-A-A-A-A-A-AOUM-M-M-M-M-MR-R-R-RM-M-M-ML-L-L-LLLLL}

%\usepackage{layout} %TODO remove

\usepackage{graphicx} % Pour les images
\graphicspath{ {Images/petit/} }
%\graphicspath{ {Images/} }

\title{Et demain on va où ?}
\author{Gérald Schmitt \and Marion Abrial}

\begin{document}
%\layout
\maketitle

\chapter{Lancement du blog !}
Voilà, c'est officiellement notre premier article de blog consacré à notre voyage à venir !

Et c'est même mon premier article de blog tout court. Je ne sais pas encore très bien comment ça marche, ni si je vais réussir à raconter des trucs intéressants... mais c'est un début !

On a envoyé le préavis pour rendre l'appartement il y a quelques jours. Ça rend d'un coup le voyage très proche et très réel ! Ça fait envie (surtout à Marion) et aussi un peu peur (surtout à Gérald). Mais à deux, ça va l'faire ! :-)

Bon, et avant de se demander où on va demain, la prochaine question sera plutôt, et demain, on fait quoi pour préparer tout ça ?

\chapter{Le symbole de la liberté !}
Voici mon porte-clé :


\begin{figure}[h]
\centering
\includegraphics[height=9cm,width=12cm,keepaspectratio]{p9280216.jpg}
\caption*{Mon porte clé}
\end{figure}

Eh oui, il est passé par une sérieuse cure d'amaigrissement cette semaine. Plus d'appartement, plus de boulot, plus de voiture.
Reste ce réflexe :

Je tâte ma poche gauche. Téléphone présent.

Je tâte ma poche droite... vide ! Mon coeur s'arrête une fraction de seconde. Je suis coincé dehors !

Soupir de soulagement. Gérald, t'es con. Pour ne rien arranger, c'est la troisième fois en dix minutes que tu paniques...
Et oui, nous sommes bien coincés dehors. Et le monde entier est potentiellement notre maison.

\chapter{Du bordel plein le dos}
Voilà le bordel qu'on va avoir sur le dos pour l'année à venir :


\begin{figure}[h]
\centering
\includegraphics[height=9cm,width=12cm,keepaspectratio]{p9300220.jpg}
\caption*{ Un an de bordel}
\end{figure}

Un minuscule extrait de l'amoncellement de bordel qu'on a réussi à accumuler dans 70m\textsuperscript{2} à Lyon. Et c'est là dessus qu'on va compter pour tout :
\begin{itemize}
	\item Manger
	\item Dormir
	\item Se laver
	\item Se soigner
	\item S'habiller
	\item Cuisiner
	\item Communiquer
	\item Photographier
	\item Glander
\end{itemize}
On est parti d'un appartement plein, qu'on a vidé en quasi totalité une semaine avant de le rendre. Mais on a quand même gardé un peu plus que nos affaires du tour d'Asie. Ensuite, on est allés squatter chez les parents, mais on n'a pas encore renoncé totalement à tout ce qui ne va pas partir en Asie...
C'est comme si on repoussait l'inéluctable, ce moment ou on n'aura plus que notre sac à dos, et vraiment plus rien d'autre.
Je pense à ce sketch de Roland Magdane sur le merdier... Si je trouve un lien, je le mettrai dans l'article, car ça correspond assez bien à ce qu'on vit en ce moment.

Tout ça pour en arriver à un petit sac de bordel de 12kg environ. On est plutôt content ! On visait les 10kg, mais il aurait fallu faire l'impasse sur la tente ou l'appareil photo.

Et maintenant, on se demande dans quel sens le poids du sac va évoluer au cours du voyage.

Vous faites quel pronostic ?

\chapter{Le sac est prêt !}
Juste une petite photo de nous avec le sac à dos complet. On est prêt à prendre le bus.
\begin{figure}[h]
\centering
\includegraphics[height=9cm,width=12cm,keepaspectratio]{pa100286.jpg}
\caption*{}
\end{figure}

Au programme : Mulhouse - Berlin - Varsovie - Brest (en Biélorussie) - Saint Pertersburg. Arrivée prévue à Saint-Petersburg jeudi 15 octobre.

On vous tient au courant dès qu'on a un peu de temps. A bientôt :-)

\chapter{Le bus de Babel}
Donc on a pris le premier bus de notre voyage. On part de France (Mulhouse) pour aller en Allemagne (Berlin). Le chauffeur du bus ne parle qu'une seule langue : quoi de plus logique que l'espagnol ? Le bus est rempli en majorité de... Polonais !

Et là, on nous met un film dans le bus. Un film anglais (avec Hugues Grant mais aucun rapport). Doublé en Russe. Ben oui ! Vous savez, ce fameux doublage russe d'une qualité comment dire... disons d'une qualité remarquable. Un gars lit son texte sans aucune intonation. Et le même gars fait toutes les voix ! Et la bande son du doubleur est juste superposée à la bande originale du film. Je ne sais pas si les Russophones comprennent quelque chose au film, mais pour les Anglophones, aucune chance !

Et tout ça, alors qu'on était même pas encore vraiment sorti d'Alsace. Tu parles d'un départ sur les chapeaux de roues :-)


\begin{figure}[h]
\centering
\includegraphics[height=8cm,width=8cm,keepaspectratio]{pa120299.jpg}
\caption*{On est à Berlin}
\end{figure}

On a quand même finit par arriver à Berlin.

\chapter{Des conditions aux bornes}
Donc nous voilà parti de Berlin. Après l'expérience du bus Mulhouse-Berlin, on ne savait pas trop à quoi s'attendre pour le Berlin-Varsovie.
Eh bien on a été pour le moins surpris. Vous saviez, vous, qu'il existait des bus avec prise 220V à chaque siège, petit écran de divertissement individuel avec plein de films récents comme dans les avions, un petit bouton pour appeler l’hôtesse et lui demander un thé, et, oh douce surprise : du wifi ! Du wifi qui marche pour de vrai ! Dans un bus en mouvement entre l'Allemagne et la Pologne !
Et l’hôtesse parlait même anglais.

On arrive à Varsovie à l'heure prévue, à 6h du matin. Notre bus suivant pour Brest ne part qu'à 9h. Donc on attend. Et on cherche des toilettes. Et oui, il y en a. elles sont payantes. Et en zloti. Et pas moyen de négocier avec la dame pipi, se sont des zloti, ou bien rien. Dans notre cas, ce sera rien...

Puis, Varsovie-Brest. On arrive à la frontière vers 13h30 heure locale. Notre destination est 4km derrière la frontière. On se dit qu'on va arriver bien avant les 15h00 prévues. Eh bien non ! Le passage de la frontière dure, et dure, et dure encore. La sortie de Pologne est assez rapide. Mais l'entrée en Biélorussie... On remplit le questionnaire d'immigration. On attend. Un garde passe prendre nos passeports. On attend. Longtemps. Personne dans le bus n'ose rien dire. On a l'impression qu'au premier éternuement, c'est direction le goulag. Des gens sortent nos sacs du bus. On attend. On sort, on prends nos sacs, on rentre dans les bâtiments de la douane pour une fouille, et on peut enfin reprendre le bus.
Il est 15h45...

A 16h, le bus nous dépose dans le centre de Brest, et on se met en route pour notre auberge de jeunesse. On la trouve au sous-sol d'un grand ensemble d'immeubles. Nous sommes les seuls voyageurs de toute l'auberge. Tout est neuf. On pourrait croire que nous sommes les premiers clients tout court ! Après deux jours à dormir dans des bus. Pas envie de partir crapahuter dans Brest. Donc on fait des courses dans un supermarché tout ce qu'il y a de plus classique, sauf qu'on ne comprend rien aux étiquettes. On a acheté un peu par hasard des petits gâteaux qui ressemblait à des pains d'épices. En fait, ils étaient à la menthe. C'est marrant, on a vraiment l'impression de manger du dentifrice !

\begin{figure}[h]
\centering
\includegraphics[height=9cm,width=12cm,keepaspectratio]{pa130322.jpg}
\caption*{Eglise dans Brest}
\end{figure}
Et le lendemain, enfin on prend le train pour Saint-Petersburg.
Dans le train, on commence à réfléchir au passage de frontière et aux visas. Et on prend conscience d'une chose terrible : on a bien un visa Biélorusse jusqu'au 14 octobre, et un visa Russe à partir du 15 octobre, et nous sommes dans la nuit du 14 au 15 octobre, mais nous ne sommes pas dans un avion, bien dans un train. Si on passe la frontière avant minuit, peut-être que les Russes ne nous laisseront pas rentrer ? Si on passe la frontière après minuit, peut-être que les Biélorusses vont nous dire que notre visa a expiré et nous mettre une prune ? On essaie de se rassurer en se disant que quand même, on ne fait rien de mal, que les douaniers ne sont pas si méchants etc. Mais quand même, on suit anxieusement l'approche de la frontière sur le gps.


\begin{figure}[h]
\centering
\includegraphics[height=8cm,width=11cm,keepaspectratio]{pa140330.jpg}
\caption*{Notre train vers la Russie}
\end{figure}
Heureusement, nous sympathisons avec nos voisins de compartiment. Entre l'allemand et un russe balbutiant, nous arrivons à comprendre qu'ils sont un couple d'ingénieurs russes à la retraite, de retour d'un sanatorium en Biélorussie. Il faut savoir qu'il n'y a rien d'autre à faire en Biélorussie que la chasse, cueillir des champignons, et globalement se promener dans la nature. C'est un pays complètement plat, dont le point culminant ne dépasse pas 400m.
Et ce couple sympathique s’évertue à nous parler de la Russie, à nous montrer des photos, nous dire quoi faire absolument à Saint Petersburg. Et bien entendu, impossible de boire un thé sans qu'ils tentent de nous l'améliorer avec de la gnôle à base de plantes :-)


\begin{figure}[h]
\centering
\includegraphics[height=10cm,width=9cm,keepaspectratio]{pa140331.jpg}
\caption*{Notre compartiment}
\end{figure}
Et on continue de s'approcher de la frontière. Vers 22h, tout le monde monte dans sa couchette, et la frontière approche. Je me demande si on va réveiller les gens. Bref, c'est étrange. Un peu avant minuit, on passe la frontière, le train s'arrête et (suspens insoutenable)... ben il repart, et c'est tout. En fait, et on l'a appris ensuite : il n'y a pas de frontière. C'est un espace économique commun comme l'Europe.
Du coup, on arrive sans encombre à Saint Petersburg.

\chapter{Quelques jours à Saint-Petersbourg}
Alors on est arrivé à Saint Petersbourg mercredi 15 octobre à 7h00 du matin, encore tout surpris de n'avoir pas franchi de douane. Et à cette heure, la ville est encore assez calme.
On a trouvé assez facilement notre auberge de jeunesse, on a laissé nos bagages, et on est partis en ville en suivant la fameuse Nevky Prospekt. Disons que c'est l'équivalent des Champs-Elysées, mais en Russe. Tous les bâtiments sont incroyables. A chaque coin de rue, on est obligé de s'arrêter devant une nouvelle cathédrale, une statue monumentale, ou un bâtiment couvert de moulures dorées.


\begin{figure}[h]
\centering
\includegraphics[height=6cm,width=9cm,keepaspectratio]{pa150337.jpg}
\caption*{Cathédrale du sang versé}
\end{figure}
On ne fait qu'une seule visite ce jour-là. Mais alors quelle visite ! La cathédrale du sang versé et du Christ ressuscité. Pour nous qui sommes habitués aux cathédrales gothiques, voir tant de couleurs et de dorures est stupéfiant dans une église, que ce soit à l'intérieur ou à l'extérieur.


\begin{figure}[h]
\centering
\includegraphics[height=6cm,width=9cm,keepaspectratio]{pa150351.jpg}
\caption*{ L'intérieur de la cathédrale est entièrement recouvert de mosaïques.}
\end{figure}

Nous avons aussi fait le tour de notre dame de Kazan, qui tient plus du temple romain que de l'église catholique pour ce qui est des colonnes.


\begin{figure}[h]
\centering
\includegraphics[height=6cm,width=9cm,keepaspectratio]{pa150333.jpg}
\caption*{ Oui, c'est une église, en effet !}
\end{figure}

Et on s'est aussi arrêté dans un restaurant un peu au hasard. Un endroit qui avait l'air un peu moins attrape touriste que les autres, mais bon, ne nous faisons pas trop d'illusions, il y avait quand même des menus en anglais. Et à peine installés, on entend d'autres français dans le restau un peu plus loin. Mais ils changent de table et viennent s'installer sur la table voisine. Ils sont trois, un couple et une amie. La conversation commence :
L'amie : "Alors vous faites quoi ?"
Le couple : "On s'est pris un an pour faire le tour de l'Asie, et on a commencé la semaine dernière par la Russie".
Ben oui, ce genre de coïncidence arrive vraiment. 5mn plus tard, on avait fait table commune et on comparait nos trajets ! (Coucou à Aurélie, Sophie et Gauthier si vous nous lisez).

Le lendemain, nous avons passé une journée entière consacrée à l'Ermitage, le musée emblématique de Saint Petersbourg. Une journée, ce n'est pas de trop. Les livres nous encourageaient à prendre les billets en avance tellement il y a de monde. Et c'est ça qu'il y a de bien en voyageant hors saison : pas de souci dans le genre, on peut tout organiser à la dernière minute, il y a de la place !


\begin{figure}[h]
\centering
\includegraphics[height=6cm,width=9cm,keepaspectratio]{pa150370.jpg}
\caption*{ Et encore, en n'en voit pas le quart sur cette photo.}
\end{figure}

On s'attendait à visiter un musée classique, genre une enfilade de pièces remplies de tableaux et d’œuvres diverses, mais que nenni ! Un des intérêts de l'Ermitage, c'est que c'était un palais. Et un palais possédé par des gens qui voulaient en mettre plein la vue des visiteurs. Je ne vais pas essayer de décrire ces pièces avec des mots. Je n'ai ni le talent, ni le temps. Et même les photos ne rendent pas justice. Il faut y aller et le voir de ses propres yeux. Je dirais juste que l'audioguide a qualifié de "petit" un lustre grand comme notre salle de bain. Je vous laisse imaginer la taille du grand juste à coté.
Et dans toutes les pièces, une mama russe en uniforme qui surveille son coin de salle depuis sa petite chaise. Il y a celle qui fusille du regard si on se penche trop vers un tableau, celle qui ne dors pas-si-si-je-vous-jure, celle qui tient absolument à te donner une double page d'un texte dense en Russe sur l'art Iranien, et celle qui passe des trèèèès longues journées à surveiller une salle vide.


\begin{figure}[h]
\centering
\includegraphics[height=6cm,width=9cm,keepaspectratio]{pa160417.jpg}
\caption*{ Pour une idée de la taille de la salle, cherchez la chaise.}
\end{figure}

On nous a beaucoup vanté les fontaines de Peterhof, donc nous y sommes allés. C'était presque une expédition : on a pris le métro puis le bus en vérifiant toutes les 30 secondes qu'on avait bien pris le bon bus, et on est arrivé à Peterhof. Et alors ce qu'il y a de bien quand on voyage hors saison, c'est que parfois c'est gratuit parce que les fontaines sont fermées... C'est con, on s'était dépêché pour être à l'heure pour l'ouverture des robinets dont on nous avait vanté la cérémonie, et... ben rien. Pas d'eau. Il fait déjà trop froid ! On a donc passé la journée dans un grand parc plein de fontaines éteintes. Et on a jeté une pièce de 10 kopecks dans la botte d'une statue. Les Russes passent leur temps à jeter des pièces dans des endroits improbables, à tel point qu'on a vu des panneaux l'interdisant. En même temps, quand on sait ce que ça vaut 10 kopecks (1,4 cents pour les curieux, pas sûr que ça vaille son poids en métal).
Mais bon, c'était quand même un très joli parc.


\begin{figure}[h]
\centering
\includegraphics[height=6cm,width=9cm,keepaspectratio]{pa170473.jpg}
\caption*{ Les jardins de Peterhof.}
\end{figure}


\begin{wrapfigure}{l}{0.45\textwidth}
\centering
\includegraphics[width=0.4\textwidth]{pa180535.jpg}
\caption*{ La légende dit que 200kg d'or ont été nécessaires pour toutes les dorures.}
\end{wrapfigure}


Enfin, on ne pouvait pas partir de Saint Petersbourg sans visiter le palais de Catherine dans la ville de Pouchkine. Mais alors, quelle expédition ! Car on a décidé d'y aller en train local. Et on a réussi à acheter des billets de train en parlant seulement Russe ! Là, je peux vous dire qu'on était fiers comme des paons. On avait vraiment l'impression d'être des aventuriers.


Et ce palais, comment dire... Non, vraiment, je n'ai pas les mots. Et je n'ai pas toutes les photos non plus. Une enfilade de salles toutes plus magnifiques les unes que les autres, avec la salle d'ambre en point d'orgue (photos interdites, désolés, faudra y aller vous même).
C'est ... pffff... non mais allez-y, vraiment. Ça vaut le voyage !



\chapter{Moscou}
On a fait le trajet Saint-Pétersbourg Moscou dans le SAPSAN, une sorte de TGV russe. En plus lent. Mais mieux sur tous les autres points ! 200km/h de vitesse max, pas de quoi défriser un mouton, mais quel confort ! La deuxième classe est largement aussi confortable que la première en France. Pour vous donner une idée de la largeur : les toilettes sont accessibles en fauteuil roulant. Et ils passent même des films, écouteurs fournis.


\begin{wrapfigure}{l}{0.55\textwidth}
\centering
\includegraphics[width=0.5\textwidth]{pa200641.jpg}
\caption*{ La cathédrale Saint Basile, construite en l'honneur d'un fou qui se baladait à poil.}
\end{wrapfigure}

Quand on rentre sur la Place Rouge la première fois, on a vraiment l'impression de rentrer dans l'histoire. C'est grand, c'est solennel, il fait froid. La cathédrale de Saint Basil est plus petite que ce qu'on imaginait, et le Kremlin plus grand.
Et de part et d'autre de la place se font face :
A droite : Le mausolée de Lénine, ainsi qu'un cimetière des grands communistes, dont Staline.
A gauche : Le GUM. Un immense et magnifique centre commercial rempli de boutiques de luxe. Toutes les grandes marques sont présentes.
Vous la sentez aussi, la grosse ironie de l'Histoire ?


\begin{figure}[h]
\centering
\includegraphics[height=6cm,width=9cm,keepaspectratio]{pa200640.jpg}
\caption*{ L'intérieur du GUM}
\end{figure}

Un peu plus loin, au bord de la Moskva, on a eu le souffle coupé en rentrant dans le monastère du Christ Ressuscité. Encore une fois, pas mots, pas de photos. C'est immense, doré, rempli de bougies, de marbre et de mosaïques. Et vraiment, c'est immense. Allez-y !


\begin{figure}[h]
\centering
\includegraphics[height=6cm,width=9cm,keepaspectratio]{pa220685.jpg}
\caption*{ Oui, on fait aussi des photos de "touristes".}
\end{figure}

Le lendemain, le Kremlin : inévitable. Donc le Kremlin (qui à la base veut fortifications, donc il y a des kremlins un peu partout en Russie) c'est une grande muraille de quelques kilomètres qui entoure des bâtiments officiels du gouvernement et des cathédrales ainsi qu'un musée. Bien entendu, il y a un militaire à chaque coin de rue qui surveille ces monceaux de touristes circulant le nez en l'air sans remarquer qu'ils sont entrain de traverser hors des clous. Un coup de sifflet, voir deux ou trois, et tout rentre dans l'ordre.


\begin{figure}[h]
\centering
\includegraphics[height=6cm,width=9cm,keepaspectratio]{pa210673.jpg}
\caption*{ La tour d'Yvan, dans le Kremlin.}
\end{figure}

Et on se fait fouiller à l'entrée bien entendu, et on a du se séparer de nos couteaux suisses. On s'est rendu compte à cette occasion que le saucisson, ça peut aussi se découper à coups de dents (merci Nico Villedieu, c'est ton saucisson qu'on a apporté).
Dans l'Armurerie, un musée comportant des armes, mais pas que, on a vu assez de richesses pour lancer un programme d'exploration spatiale. Des bibles recouvertes de diamants, tous les objets de la vie courante, mais en or, et toutes les décorations les plus kitsch possibles aussi en or. Cela dit, difficile de qualifier tout cela de kitsch une fois qu'on sait que la moindre babiole vaut plus que le PIB d'un petit état africain (estimation à la louche par moi).
Beaucoup d'objets sont des cadeaux offerts par des diplomates en visite, et on sent la conccurence : ce sera à celui qui apportera le cadeau le plus doré et le plus extravagant. J'ai noté un brule-parfum en or en forme de montagne surmontée de châteaux, dont les fenêtres laissent échapper la fumée.
Les ancêtres des Kinder Surprises sont aussi exposés, les fameux œufs de Fabergé. Dont un fantastique qui contient une locomotive et des wagons, en platine et en or cela va sans dire, et dont la locomotive peut être remontée avec une clé ! Un petit cadeau pour fêter l'aboutissement du transsibérien.
On a terminé la journée avec nos nouveaux amis Belges (cf la rencontre improbable à Saint-Pétersbourg) dans un restaurant Géorgien ou on a bu notre première bière depuis notre arrivée en Russie !


\begin{figure}[h]
\centering
\includegraphics[height=6cm,width=9cm,keepaspectratio]{pa210678.jpg}
\caption*{ Nous avec Gauthier et Sophie.}
\end{figure}

Suite au désistement d'un couchsurfeur qui devait nous héberger à Moscou jusqu'à dimanche, on décide de partir vendredi à Ekaterinbourg. Et comme on est jeudi, il faut acheter les billets de trains. On l'a déjà fait en ligne, les billets électroniques, ça marche très bien. Mais là, non, rien ne marche, le site mouline. On a perdu toute la matinée avant de réussir à les acheter en utilisant la version Russe du site grâce à la réceptionniste de l'auberge de jeunesse. A croire que certains jours, seuls ceux qui parlent russe peuvent acheter des billets de train en ligne.
Ensuite, on s'est dit qu'on irait bien visiter le mausolée de Lénine. On regarde les horaires : ca ferme à 13h, et il est 12h15. Oups. Bon, ben demain alors ? Demain, c'est fermé. Ah. Il y a eu un blanc de deux secondes, on s'est regardé, puis on s'est mis à courir. Dans tous les sens dans un premier temps, puis en direction de la place rouge dans un deuxième temps. Et on a vu le corps de Lénine. Mais vraiment le corps de Lénine, momifié et entretenu depuis sa mort il y a plus de 90 ans ! C'est silencieux, en sous-sol, tout de marbre noir recouvert, et on voit d'autres touristes s'arrêter le temps d'une révérence. On en oublie notre course effrénée à travers la ville pour arriver à l'heure.


\begin{figure}[h]
\centering
\includegraphics[height=12cm,width=12cm,keepaspectratio]{pa220694.jpg}
\caption*{ Moi, et Pierre le Grand.}
\end{figure}

Puis le monument dédié à Pierre le grand : une énorme statue d'une pile de bateaux empilés à travers une gerbe d'écume, surmontée d'un navire piloté par un Pierre plus grand encore que le navire qu'il pilote !
Et on fini par manger dans un restaurant sur pilotis avec Frédéric. Il nous a raconté qu'en hiver, l'étang gèle autour du restaurant et se transforme en patinoire. Les gens bien emmitouflés s'installent sur la terrasse, chauffée par des braseros, et regardent les patineurs. A imaginer cette scène, on se croyait dans un conte russe...

\chapter{Ekaterinbourg}
Enfin le jour du départ pour Ekaterinbourg.
C'est la première "vraie" étape du transsibérien pour nous. Départ à 13h de Moscou, arrivée à Ekaterinbourg le lendemain à 16h heure de Moscou, mais 18h, heure locale. Eh oui, pour plus de "simplicité", tous les trains, toutes les gares, tous les billets sont à l'heure de Moscou. Il faut toujours avoir en tête le décalage horaire par rapport à Moscou quand on prend le train.


\begin{wrapfigure}{l}{0.55\textwidth}
\centering
\includegraphics[width=0.5\textwidth]{pa240716.jpg}
\caption*{ Grosse ambiance dans le train.}
\end{wrapfigure}

Dans notre compartiment, on fait la connaissance d'un  médecin russe parti seul faire de la randonnée en Crimée. Notre langue commune : l'allemand, encore une fois. On apprend qu'il est père de 5 enfants, et qu'il adore la nature. Et il insiste pour parler politique, à défendre avec beaucoup de nuances les décisions du gouvernement. Et il nous fait aussi regretter de ne pas nous être arrêtés à Kazan, sa ville natale.
A Kazan, il part, et est remplacé par un autre gars, qui ne parle que Russe. On arrive à comprendre qu'il est dans le train pour 3 jours entiers. On comprend la taille de son stock de provisions ! Provisions avec lesquelles il est très généreux, et il insiste pour qu'on goute à tout. On partage à notre tour, mais sans succès, à chaque fois, il dit non en montrant son énorme sac de provision. Mais à regarder nos voisins, on a vraiment l'impression que le train est un endroit très convivial, et beaucoup d'échanges ont lieu !
Globalement, le wagon est très calme, et nous sommes surpris de ne voir personne boire. Autant pour les préjugés !


\begin{figure}[h]
\centering
\includegraphics[height=6cm,width=9cm,keepaspectratio]{pa240717.jpg}
\caption*{ Notre voisin très généreux avec sa nourriture.}
\end{figure}

En fait, on l'aura appris un peu plus tard : il est interdit de consommer de l'alcool dans le train (sauf celui du wagon restaurant), donc ceux qui se mettent des mines essaient de le faire discrètement.

A Ekaterinbourg, on dort dans une auberge de jeunesse tenue par Vladimir. Il parle anglais, est très sympa et fait tout pour nous faciliter la vie. Il nous trouve un chauffeur pas trop cher pour aller visiter les deux attractions majeures du coin, accompagné de Daniel, un chef cuisiner Italien qui loge également dans l'auberge :
D'abord, la frontière Europe-Asie un peu à l'extérieur de la ville. On s'y arrête quelques minutes le temps de faire les même photos que tout le monde, et voir les rubans laissés sur les arbres par les jeunes mariés. C'est le long de l'autoroute, sur une ligne de séparation des eaux. Le nom du blog commence enfin à avoir du sens !


\begin{figure}[h]
\centering
\includegraphics[height=6cm,width=9cm,keepaspectratio]{pa250726.jpg}
\caption*{ Un pied en Asie !}
\end{figure}

Ensuite Ganina Yama : un monastère fondé à l'endroit où les corps de la famille Romanov ont été retrouvés. Afin de respecter les coutumes locales, les visiteuses se doivent de porter un foulard ET une jupe, le tout gracieusement fourni à l'entrée par un moine barbu.


\begin{figure}[h]
\centering
\includegraphics[height=6cm,width=9cm,keepaspectratio]{pa250744.jpg}
\caption*{ Jupe obligatoire !}
\end{figure}

C'est tout joli et tout bucolique, plein de petites et grandes chapelles réparties au milieu de la forêt de bouleaux (note : en Russie, deux arbres sur trois sont des bouleaux, au bas mot). Mais bon, quand on ne parle pas couramment russe, on ne comprend pas grand chose...


\begin{figure}[h]
\centering
\includegraphics[height=6cm,width=9cm,keepaspectratio]{pa250740.jpg}
\caption*{ Ganina Yama}
\end{figure}

Et voilà en gros pour Ekaterinbourg. On a aussi fait le tour du centre ville, qui est équipé d'une ligne rouge peinte sur le trottoir pour que les touristes sachent ou aller. On va dire que ça occupe...


\begin{figure}[h]
\centering
\includegraphics[height=6cm,width=9cm,keepaspectratio]{pa250751.jpg}
\caption*{ La ligne rouge (et du boulot de fainéasse).}
\end{figure}

Et la boue. On ne vous pas encore parlé de la boue ? Alors il faut imaginer que les petites chutes de neiges fondent à cette époque de l'année, et la neige fondue se mélange avec la poussière, la pollution et forme une couche uniforme d'une belle boue bien lisse qui sert à repeindre tous les véhicules en marron.


\begin{figure}[h]
\centering
\includegraphics[height=6cm,width=9cm,keepaspectratio]{pa270782.jpg}
\caption*{ Jaune devant, marron derrière ! Une photo de Lénine au premier qui chope la référence.}
\end{figure}

Et les chaussures aussi. A moins d'imiter le pas des locaux, qui arrivent à marcher là-dedans assez précautionneusement pour ne pas éclabousser.


\begin{figure}[h]
\centering
\includegraphics[height=9cm,width=12cm,keepaspectratio]{pa270784.jpg}
\caption*{ La boue !}
\end{figure}

On est aussi sortis un peu du centre, pour aller marcher une journée sur une colline voisine. Première sortie nature pour nous depuis deux semaines, on en avait besoin !


\begin{figure}[h]
\centering
\includegraphics[height=9cm,width=9cm,keepaspectratio]{pa260773.jpg}
\caption*{ Oh ! Là-bas, un bouleau !}
\end{figure}

Et la dernière journée, rien de prévu, sauf prendre le train le soir, à 17h30 heure de Moscou. Donc 19h30 heure locale. Alors on glande, on traine, on va faire des courses, et à 17h, on récupère nos bagages. Marion vérifie son billet, puis demande d'une petite voix : "On a bien 4h de décalage avec Moscou ?"
"Non, deux heures."
"Tu es vraiment sûr ?"
"Oui."
"Alors notre train part dans 30mn."
"..."
Oui, notre train était à 17h30 heure locale, pas heure de Moscou. Il était 17h, et on était dans le centre ville. On ne sait toujours pas comment on a fait pour se planter comme ça, mais devant le fait accompli, on ne prend pas le temps de chercher un coupable.
Plein d'espoir, on est parti en courant, des sacs plein le dos. Évidement, c'est pile ce jour là que la police juge judicieux de nous arrêter pour passer nos bagages aux rayons X en entrant dans le métro...
Eh bien, croyez le ou non, mais on a eu notre train, à 1mn. On a presque eu le temps de s'asseoir avant que le train ne parte. Rien de tel qu'une bonne suée avant deux jours de train sans douche !



\chapter{Krasnoyarsk}
Nous partîmes donc pour 39 heures de train, entre Ekaterinbourg et Krasnoyarsk. Deux nuits et une journée entière dans le train. Et ce coup-ci, on a deux couchettes l'une au dessus de l'autre, au lieu d'avoir deux couchettes hautes comme les autres fois. Au moins on peut décider nous même quand on dort et quand on mange. Eh oui, quand on est sur la couchette du haut, on a un accès à la couchette du bas uniquement si son occupant est d'accord. Si l'occupant du bas veut dormir, celui du haut n'a pas le choix, il doit s'allonger lui aussi, et non, on ne tient pas assis en haut !

\begin{wrapfigure}{l}{0.55\textwidth}
\centering
\includegraphics[width=0.5\textwidth]{pa280790.jpg}
\caption*{ Tout le wagon profite de l'odeur !}
\end{wrapfigure}


Et plus on va vers l'est, plus les gens sont curieux. Saint-Pétersbourg est blindé de touristes, même hors saison. Mais dans ce train, tous les regards convergent vers nous. Du moins, dès qu'ils nous entendent parler. Et ils sont tous très sympas et très accueillants, et veulent savoir d'où on vient, où on va, et nous donnent à manger : On a ainsi reçu un demi poisson fumé de la part de nos voisins de compartiment, et ils ont vraiment insisté ! Et ce poisson était vraiment très bon, mais contribuait pour une fraction non négligeable au fumet du wagon. On l'a mangé très vite. Ce coup-ci, personne ne parle anglais, ni même allemand. La conversation est plus difficile, mais on sent qu'on progresse en russe et en langue des signes.


On arrive à Krasnoyarsk le matin, et on se balade au hasard dans la ville pour la découvrir. Un pêcheur nous alpague en anglais, et nous raconte qu'il adore Paris, et qu'il faut qu'il pêche un poisson pour ses deux chats. Une Russe nous dit un petit bonjour timide en français, et s'étonne de voir des touristes en Sibérie en hiver. Et on mange dans un restaurant allemand ! Bon, disons germano-russe. Mais quand même, ils avaient des vraies bières allemandes, et même des belges !


\begin{figure}[h]
\centering
\includegraphics[height=6cm,width=9cm,keepaspectratio]{pa290795.jpg}
\caption*{ Valentin, qui pêche pour ses chats.}
\end{figure}

Bien que le propriétaire de notre auberge propose des tours organisés des montagnes environnantes, on décide d'y aller par nous même, accompagné de Cathy, une allemande voisine de chambrée.


\begin{figure}[h]
\centering
\includegraphics[height=6cm,width=9cm,keepaspectratio]{pa300852.jpg}
\caption*{ Ça ne vous donne pas envie de grimper ?}
\end{figure}

Ces montagnes s'appellent "Stolby", qui veut dire "piliers", car il y a des éperons rocheux au sommet de certaines buttes. Montagnes, c'est un bien grand mot pour une colline de 700m de haut. Mais quand on vient de faire 4000km en train dans une grande plaine, ça fait du bien ! Il y a un peu de neige, la balade est sympa, et de temps en temps, on croise un "stolb", qui s'avère être un bête caillou de quelques mètres de haut, et sur lequel on ne peut plus grimper car il est couvert de glace. Mais c'est joli quand même. Les écureuils et les oiseaux ont quasiment été dressés à sauter dans les bras des promeneurs. Un groupe de jeune s'évertue à nous soutirer des informations du genre "est-ce qu'il fait beau chez vous" à grand renforts de dessins dans la neige. C'est comme jouer au pictionnary, mais sans savoir si on a gagné.


\begin{figure}[h]
\centering
\includegraphics[height=6cm,width=9cm,keepaspectratio]{pa300829.jpg}
\caption*{ Nos compagnons de pictionnary.}
\end{figure}

On y refait même un tour le lendemain, ce coup-ci en montant par le télésiège de la station de ski locale. Et on remarque un couple de vieux venus eux aussi marcher un peu. On les dépasse. On s'arrête pour voir un truc. Ils nous rattrapent. On fait ça encore une fois, et ce coup ci, erreur tragique, on s'arrête devant un panneau. Et là, ils commencent à nous expliquer les chemins, combien de temps, quels cailloux, et pleins d'autres trucs qui ont l'air très importants, à voir l’énergie que ce vieux met dans ses explications. Du coup, on part dans un chemin de traverse voir encore un caillou. Et au retour, ils nous avaient encore suivis... (oui bon, ce n'est pas facile de rendre passionnant une balade dans la forêt, alors on fait avec ce qu'on a).


\begin{figure}[h]
\centering
\includegraphics[height=6cm,width=9cm,keepaspectratio]{pa310880-panorama.jpg}
\caption*{ Panorama de Stolby.}
\end{figure}

Et on apprend quelques trucs sur les populations autochtone dans le musée régional, qui ressemble de l'extérieur à un temple égyptien. Le musée est pas mal, mais on n'a pas vu le rapport avec les égyptiens...

Et le soir, on a fait un excellent restaurant Ukrainien. On était fatigué, et on cherchait un restaurant pas loin de l'auberge. On sort, on part dans la mauvaise direction, on fait le tour du pâté de maison, et on fini par trouver l'entrée... juste en face de l'auberge, il n'y avait qu'à traverser la rue ! Et l'Ukraine, gastronomiquement parlant, c'est quand même assez proche de l'Alsace ! Ils ont même du vrai lard. Pas du lard pollué par de la viande, non, juste le gras, coupé en fines en tranches. Tout le monde n'est pas fan...


\begin{figure}[h]
\centering
\includegraphics[height=6cm,width=9cm,keepaspectratio]{pa310921.jpg}
\caption*{ Une assiette de charcuterie (à peine entamée).}
\end{figure}

Le lendemain... rien. Enfin si, on a rencontré des français qui avaient déjà entendu parlé de nous, via Cathy. Faut dire qu'à cette époque, les touristes sont rares. Et on a réussi à acheter nos billets pour la Mongolie. Et ça a pris une heure à la guichetière, littéralement. Une heure à taper des trucs sur l'ordinateur, à imprimer des bidules, à remplir d'autres bidules à la main, à nous faire signer des papiers en russe. J'espère juste que je ne me suis pas vendu corps et âme à l'administration russe en signant ces papiers !



\chapter{Irkoutsk}
Direction Irkoutsk. Petit trajet en train. 19h, une nuit. On a à peine eu le temps de s'installer qu'on était déjà arrivés. C'est qu'on y prendrait goût à voyager dans ces conditions !
Irkoutsk donc, une des plus grande ville de Sibérie. Connue pour sa proximité avec le lac Baïkal, mais c'est une proximité toute relative : 70km. Même si ici, c'est une paille, difficile d'y aller à pied. On ira donc un peu plus tard.


\begin{figure}[h]
\centering
\includegraphics[height=6cm,width=9cm,keepaspectratio]{pb041101.jpg}
\caption*{ De la neige !}
\end{figure}

Durant notre première nuit sur place a eu lieu la première vraie chute de neige : 20cm d'un coup. On se balade donc toute la journée dans la neige et la ville. Je ne sais pas vous, mais nous, lors des premières neiges de l'année, on est un peu comme des gamins. Et là, on a pleins de nouveaux trucs à voir ! Comment, les gens roulent (comme s'il n'y avait pas de neige), comment les gens s'habillent (ben, comme s'il n'y avait pas de neige, on dirait que les femmes russes ont des chaussures à talons crantés), comment ils font pour déblayer la neige (ben... avec des pelles et des chasse-neige, on espérait des tanks et des lance-flammes, on est un peu déçus).


\begin{figure}[h]
\centering
\includegraphics[height=6cm,width=9cm,keepaspectratio]{pb030948.jpg}
\caption*{ Ok, les chasse-neige sont classiques, mais ils se baladent en troupeaux !}
\end{figure}

Et ce coup-ci, ça ne fond pas ! On dirait que le froid s'est installé pour l'hiver, les prévisions de températures pour les jours à venir oscillent entre -1\textdegree C et -18\textdegree C. On a fait péter les doudounes, les collants, les gros bonnets... et on se demande comment on fera avec les potentiels -30\textdegree C voire -40\textdegree C de la Mongolie.
On passe la journée, puis la soirée, avec Evgéniya, une couch-surfeuse photographe native d'Irkoutsk. Elle nous apprend quelques trucs sur Irkoutsk, mais aussi et surtout sur l'Inde, dont elle est fan !


\begin{figure}[h]
\centering
\includegraphics[height=6cm,width=9cm,keepaspectratio]{pb030987.jpg}
\caption*{ Enfin une vraie bière !}
\end{figure}

Elle nous avait donné rendez-vous devant le "Babr". Au début, j'ai cru à une faute de frappe en lisant babr. Mais non, c'est bien ça, le Babr, animal mythique de Sibérie, ressemblant à un gros tigre, assoiffé de sang. C'est le symbole de la ville, sur le blason de laquelle il apparait. Et non, je n'invente pas le "assoiffé de sang", ça fait partie de la description officielle héraldique.


\begin{figure}[h]
\centering
\includegraphics[height=6cm,width=9cm,keepaspectratio]{pb030961.jpg}
\caption*{ Promis, sous la neige, il y a vraiment un Babr.}
\end{figure}

Les prix baissent avec l'éloignement de l'Europe, donc on mange de plus en plus dehors. On a quand même réussi à manger dans un petit restaurant Bouriate dont la serveuse, Bouriate elle aussi, avait fait des études à Strasbourg !
Alors comme j'en entends au fond dire "C'est où la Bouriatie ?", voici quelques précisions : Les Bouriates sont une ethnie de la Sibérie, vivants principalement autour du lac Baïkal, et qui font de très bons raviolis à la viande.

Plutôt que de faire un énième musée, on opte pour un aquarium spécialisé dans les phoques de Nerpa. Se sont des phoques d'eau douce qu'on ne trouve que dans le lac Baïkal. Et on se retrouve en fait devant un véritable numéro de cirque ! Avec des phoques, mignons, mais alors mignons ! Des grands yeux noirs humides, une petite tête, bien gras, vifs comme tout ! Et les voilà à chanter, enfin disons renifler, peindre des tableaux, jouer de la trompette, faire du foot et du basket, se faire la cour l'un à l'autre, j'en passe et des meilleures.
Bon, ce sont des animaux sauvages maintenus en captivité et dressés. Clairement, on peut douter du bien-être de ces animaux. Mais là, honnêtement, on a envie d'y croire. Je ne sais pas si leurs grands yeux y sont pour quelque chose, mais on a l'impression qu'ils sont contents de nous amuser.


\begin{figure}[h]
\centering
\includegraphics[height=6cm,width=9cm,keepaspectratio]{pb030982.jpg}
\caption*{ Si ça se trouve, c'est même le deuxième étage...}
\end{figure}

En se baladant dans la ville, on voit aussi de nombreuses vieilles maisons en bois. Ces maisons datent du 19ième siècle, à l'époque où la ville s'est fortement développée sous l'action conjointe d'une ruée vers l'or et d'une série d'exil d'opposants au tsar. Et une chose marrante se passe avec ces maisons : le niveau des rues monte, mais pas les maisons, qui se retrouvent petit à petit noyées par les couches successives de gravats, pavés et autres enrobés.

Ensuite, petite variation sur le thème "Je visite la plus belle église de la ville". Ce coup-ci, on la visite ... pendant la messe ! Bon, honnêtement, en entrant, on ne savait pas encore, il n'y a pas de lumière rouge au dessus de la porte avec un panneau "Attention, messe en cours". Donc on entre, et il y a plein de gens debout. Hé oui, ici, la messe, faut la mériter, voire, l'endurer... Bref, on se fait petit, on se met dans un coin au fond et on profite des chœurs qui étaient déjà à l’œuvre. La messe a un coté hypnotisant ici, le prêtre psalmodie quelque chose en rythme, puis la foule des fidèles psalmodie une réponse elle aussi en rythme. Et tout hypnotisés qu'on était, on se rend à peine compte que le prêtre est entrain de faire le tour de la salle en balançant son encensoir dans tous les sens. Et on dirait bien qu'il compte aussi enfumer le fond de l'église, ou nous étions pourtant bien cachés. Et il nous a effectivement enfumés en nous regardant bien dans les yeux. Je sais pas vous, mais moi, un gros barbu qui balance un truc brûlant à 20cm de mon nez tout en me regardant très... sérieusement, ben ça a tendance à ne pas me mettre à l'aise.


\begin{figure}[h]
\centering
\includegraphics[height=6cm,width=9cm,keepaspectratio]{pb030978.jpg}
\caption*{ On n'a pas osé prendre de photos de la messe, donc voici juste l'extérieur, imaginez le reste.}
\end{figure}

Et comme la ville, ça commence à bien faire, on décide d'aller visiter sérieusement le lac Baïkal en allant passer quelques jours sur l'ile d'Olkhon. Promis, on vous raconte tout ça très vite !



\chapter{Lac Baïkal - Île d'Olkhon}
Nous partîmes donc un matin vers l'île d'Olkhon (prononcer "olrrrone", avec un "r" espagnol, et non pas "oh l'con"). Le trajet dure environ 7h en minibus, 300km de piste en terre. Deux choses notables durant le trajet :
Lors de la première pause, il ne faisait "que" -14\textdegree C, et ça nous a semblé vraiment, mais alors vraiment très très froid sur le moment.


\begin{figure}[h]
\centering
\includegraphics[height=6cm,width=9cm,keepaspectratio]{pb051128.jpg}
\caption*{ Le ferry pour l'île.}
\end{figure}

Ensuite, on a pris un petit ferry pour rejoindre l'île. Et ce sont les dernières semaines pendant lesquels ce ferry circule. Quand le lac est gelé, le minibus passe directement sur la glace. Et pendant quelques semaines à l'intersaison, quand les glaces sont en formation ou entrain de fondre, l'île est isolée du reste du monde, à moins de pouvoir se payer un petit coucou. Et pour vous donner une meilleure idée de l'isolement de l'île, sachez que l'électricité n'y est arrivée qu'en 2006, soit 37 ans après le premier homme sur la lune...


\begin{figure}[h]
\centering
\includegraphics[height=6cm,width=9cm,keepaspectratio]{pb102254.jpg}
\caption*{ Isolement -> pas de pollution lumineuse -> photo de la voie lactée :-)}
\end{figure}

On arrive dans une auberge ouverte par un ancien champion de tennis de table soviétique : Nikita homestead. C'est sous l'impulsion de ce type que le tourisme s'est développé sur l'île. Il n'y a que 1500 habitants permanent sur cette île de 70 sur 15km, mais on nous a dit que la population peut passer à 10000 en haute saison.


\begin{figure}[h]
\centering
\includegraphics[height=6cm,width=9cm,keepaspectratio]{pb112291.jpg}
\caption*{ J'aime comme ils recyclent les chaudières.}
\end{figure}

En novembre, je ne vous cache pas qu'on est loin de la haute saison. Nous sommes tranquilles pour profiter de la magnifique auberge, toute en bois sculpté, ainsi que de l'île bien entendu (et si vous avez compris que c'était une île toute en bois sculpté, relisez doucement la phrase).

Le lac n'a pas encore commencé a geler. Il est tellement grand qu'il adoucit la région un peu comme la mer le ferait. Du coup, l'hiver est retardé de quelques semaines, et il fait 5 à 10\textdegree C de plus qu'à Irkoutsk. Il n'empêche que les plages sont complètement gelées, le sable est dur comme du béton, et localement très glissant.


\begin{figure}[h]
\centering
\includegraphics[height=6cm,width=9cm,keepaspectratio]{pb071591.jpg}
\caption*{ Ça commence à geler !}
\end{figure}

La nourriture dans cette auberge était vraiment excellente : menu unique pour tout le monde, et c'était comme manger à la maison, voire, dans le cas de certain plats, comme chez ma grand-mère. Il y a beaucoup de soupes, de ragouts, toujours des crudités, et tous les jours, c'est viande ET poisson. Et le petit-dej... Leur spécialité, ce sont les "blinis", qui sont en fait des crêpes tout ce qu'il a de plus classiques, donc blinis à tous les petits dej, et les Russes les mangent avec du lait concentré sucré. On va apporter cette pratique en France ! Et comme les crêpes ne suffisent pas, on a du porridge tous les matins. Et on a eu un porridge différent pour tous les 7 petits déjeuners pris sur place.


\begin{figure}[h]
\centering
\includegraphics[height=6cm,width=9cm,keepaspectratio]{pb122293.jpg}
\caption*{ Le cuistot et le buffet du petit-déjeuner.}
\end{figure}

On est allé visité le Nord de l'île dans une sorte de vieux mini-bus 4x4 soviétique très commun dans le coin. Ça tombe tout le temps en panne, mais c'est assez simple à réparer, et visiblement, ça résiste mieux au froid que le reste.


\begin{figure}[h]
\centering
\includegraphics[height=6cm,width=9cm,keepaspectratio]{pb071541.jpg}
\caption*{ Le 4x4 !}
\end{figure}

Ça nous prend 7h pour faire 70km dans la neige. Parfois, on a l'impression que Sergueï, le chauffeur, se balade vraiment au hasard, et ne tient absolument pas compte des routes, ou du moins des traces, déjà existantes. On se demande si la voiture va tenir jusqu'au bout tellement elle grince à chaque cahot, et des cahots, il y en a ! Mais le seul truc qui arrêtera la voiture sera une congère qui nous obligera à pelleter. Enfin, nous, surtout Sergueï, vu qu'il n'y avait qu'une seule pelle...


\begin{figure}[h]
\centering
\includegraphics[height=6cm,width=9cm,keepaspectratio]{pb071625.jpg}
\caption*{ Au boulot !}
\end{figure}

Depuis la pointe Nord de l'île, on peut enfin apprécier, ou tenter d'apprécier les dimensions du lac. Si face à l'auberge, on voit la rive opposée à 15 km de distance, là, c'est plutôt 300km. Comme l'air est très pur et très sec, on voit bien qu'on n'en voit pas bout. C'est majestueux. Nous restons là ce qui semble une éternité à apprécier le paysage.


\begin{figure}[h]
\centering
\includegraphics[height=6cm,width=9cm,keepaspectratio]{pb071544.jpg}
\caption*{ On se sent tout petit face à ce lac.}
\end{figure}

Et nous ne sommes probablement pas les premiers à sentir cette ambiance particulière : De nombreux arbres sont couverts de rubans multicolores. Mais pourquoi donc ? La réponse est dans le chamanisme ! Pour les shamans bouriates (cf Irkoutsk), ce sont des endroits sacrés, donc il convient d'accrocher des rubans aux branches des arbres en faisant un vœux. Quand la branche en grossissant déchire le ruban, le vœux est réalisé. Facile ! Du coup, là où on comprend moins, c'est quand on voit des arbres morts ou des poteaux plein de rubans tout neufs. Enfin bon, on nous a aussi dit que la plupart de gens ne savent plus pourquoi ils font ça.


\begin{figure}[h]
\centering
\includegraphics[height=6cm,width=9cm,keepaspectratio]{pb061510.jpg}
\caption*{ Tout le monde ne sait peut-être pas à quoi ça sert, mais en tous cas, c'est beau.}
\end{figure}

A propos de ne pas savoir quoi faire, on a aussi été bien embêté quand une shamane nous a donné un verre de lait un jour qu'on se baladait devant le site le plus sacré d'Olkhon : quand on est arrivé, une shamane était entrain d'asperger le sol de lait avec une grande cuillère. Très intrigués, nous nous arrêtons pour observer cet étrange rituel. Et là, à notre grande surprise, une autre shamane, toute jeune, vient nous parler quelques instants dans un anglais impeccable. On apprend qu'elle vient d'Oulan-Bator, et qu'on assiste à une offrande aux esprits. Et avant qu'on ait le temps de poser plus de questions, elle nous file un verre de lait dans un gobelet en plastique, et s'enfuit en nous criant qu'elle va prendre des photos comme n'importe quel touriste en voyage. Et nous voilà seuls avec nos questions : on fait partie des esprits auxquels on fait une offrande, où bien sommes nous supposés faire nous aussi une offrande aux esprits, mais sans la cuillère ? Dans le doute, et ne voulant prendre aucun risque, on a bu la moitié, et jeté le reste devant les poteaux sacrés. Vous pensez qu'on a bien fait ?


\begin{figure}[h]
\centering
\includegraphics[height=6cm,width=9cm,keepaspectratio]{pb061344.jpg}
\caption*{ Bon appétit, les esprits !}
\end{figure}

Une nuit, on a été réveillés par des grignotements. Hé oui, les maisons tout en bois, ça offre plein de passages pour tout un tas de rongeurs attirés par notre réserve de nourriture. Si au moins la souris était silencieuse. Mais vous avez déjà vu quelqu'un réussir à tripoter, déchirer, ou marcher dans un sachet plastique silencieusement ? Bref, on change plusieurs fois de suite le sachet de bouffe d'endroit, mais à chaque fois la souris trouve un chemin pour y accéder et nous réveille... Le matin, on signale donc le problème à l'auberge. Leur réponse... comment dire, était à la fois tellement surprenante, et pourtant d'une logique implacable : "Vous êtes allergiques aux chats ?". On ne savait pas trop de quelle façon prendre cette réponse. Et la bonne façon était : sérieusement. Quand on est rentré le soir même, il y avait un petit chat blanc tout paniqué enfermé dans la chambre ! Évidemment, les chats de nos jours, ce n'est plus ce que c'était. Ils passent leur journée à fumer des pétards et à jouer au baby-foot, et la souris était encore là pour nous réveiller. Pour avoir la paix, on s'est résolu à laisser un bout de fromage facilement accessible et rongeable par une souris sans bruit. Avec le recul, j'ai vraiment l'impression qu'on s'est fait manipuler par la souris...


\begin{figure}[h]
\centering
\includegraphics[height=6cm,width=9cm,keepaspectratio]{pb081666.jpg}
\caption*{ Je n'ai pas de photo de souris, alors voici un peu de glace.}
\end{figure}

Une des principales activité à faire sur place reste quand même la balade. Mais cette année, on a eu droit à des avertissement inhabituels pour le coin : attention aux ours ! Normalement, l'île est trop petite pour nourrir un ours. Mais des incendies de forêt juste en face de l'île ont privé les ours de leur nourriture. Du coup, ils ont fait 15km à la nage pour venir sur l'île, et les locaux n'étaient pas encore sûrs que les ours avaient tous commencé à hiberner. Donc il ne fallait pas trop s'aventurer dans la forêt. Je savais déjà que les ours couraient plus vite et grimpaient mieux aux arbres que nous, et maintenant, ils nagent bien mieux que nous...


\begin{figure}[h]
\centering
\includegraphics[height=6cm,width=9cm,keepaspectratio]{pb071523.jpg}
\caption*{ Oui, c'est ça que les ours ont traversé...}
\end{figure}

On a aussi profité du froid pour faire du patin à glace au bord de la plage : de grandes flaques d'eau sur l'herbe ont gelé, et on a été les premiers à faire du patin dessus :-)


\begin{figure}[h]
\centering
\includegraphics[height=6cm,width=9cm,keepaspectratio]{pb081672.jpg}
\caption*{ Sympa comme patinoire.}
\end{figure}

C'est sûr, ce n'est pas pareil que de faire du patin sur le lac Baïkal lui-même, mais nous on avait un chien errant pour nous suivre ! Il a commencé à nous suivre dans le village, et ne nous a pas lâché jusqu'à qu'on soit de retour, et il a même tenté de nous suivre sur la glace. Et s'il y a bien un truc qui fait rire à chaque fois, c'est un animal innocent qui se pète la gueule tout seul !

Enfin, le dernier jour sur place, alors qu'on en parlait depuis notre arrivée, on a trouvé le courage de mettre notre plan à exécution : on a trouvé une plage isolée et... on s'est baigné ! On avait lu quelque part que quiconque se baignerait dans le lac se verrait accorder une bonne santé toute sa vie. Personne sur l'île n'avait l'air au courant de ça, mais ça ne nous a pas dissuadé pour autant !


\begin{figure}[h]
\centering
\includegraphics[height=6cm,width=9cm,keepaspectratio]{pb112279.jpg}
\caption*{ Le rocher du shaman.}
\end{figure}

C'est avec regret, et après avoir prolongé notre séjour le plus possible, que nous avons quitté cette île et la Russie, notre visa arrivant à expiration. 30 jours, ce n'est pas tant que ça en fin de compte...


\begin{figure}[h]
\centering
\includegraphics[height=6cm,width=9cm,keepaspectratio]{pb112288.jpg}
\caption*{ Dernier coucher de soleil sur le lac.}
\end{figure}



\chapter{La curiosité est un vilain défaut, parfois.}
Avertissement: ceci n'est pas pour les enfants ou personnes sensibles.
Mais tout se termine bien.

Après une journée froide, rien de mieux qu'un peu de temps au chaud dans un banya. Particulièrement après que tous ceux que l'on a rencontré nous l'aient conseillé.
C'est une sorte de sauna à la Russe : une petite cabane dont une pièce est chauffée au feu de bois, avec une cuve d'eau froide et un robinet d'eau chaude ainsi que des casseroles pour se rincer.
Tout se passait bien lorsque un petit événement inattendu arriva:
J'étais en train de me rincer lorsque j'entends Gérald (qui se trouvait dans le sauna) crier des noms d'oiseaux.
"Mince il s'est fait mal."
Puis plus un bruit.
"Bon ben ça va c'était pas grave."
Et là, Gérald sort en trombe du sauna et plonge sa tête dans la cuve d'eau froide. Lorsqu'il la ressort... Il avait une partie de la peau du visage qui c'était volatilisée et une autre qui n'était plus tout à fait accrochée comme ça aurait du l'être.
"Oh p***** , oh p***** !!!" (j'étais légèrement paniquée) Je lui dis de vite remettre la tête dans l'eau froide !

Il est resté la tête dans la cuve en attendant que je l'aide à se rhabiller, et grâce à une casserole d'eau froide on a pu retourner à notre chambre.
Après une heure dans l'eau, constatation des dégâts : le nez et la joue gauche n'ont plus d'épiderme, et l’œil gauche est enflé mais pas endommagé.

Pour que vous ayez quelques explications sur ce qu'il c'est passé: et ben Gérald a fait son Gérald. La curiosité le pousse à savoir comment les choses marchent même si cela implique de soulever le couvercle d'un récipient d'eau bouillante et de braises fumantes, alors qu'il est nu... Et pas de bol le couvercle est lourd et lui échappe des mains.

Je lui tartine le visage de Flamazine, lui donne un doliprane pour la douleur, et cours chercher un antalgique un peu plus fort (dont il n'aura même pas l'utilité, quel homme!)

On commence à avoir le ventre qui gargouille : si la faim domine la douleur c'est que ce n'est pas si grave, non ? Dans le restaurant Gérald attire les regards, la crème s'écaille lui donnant un aspect de peau de crocodile à l’œil rouge.

Mes craintes étaient infondées, il dort très bien, mais le lendemain repos et passage à la pharmacie. Ils nous on donné de la bépanthène russe, c'est à dire que ce n'était pas de la crème mais de la mousse ! Un peu surprenant et pénible à étaler, mais efficace.

Dans les jours qui suivirent, la peau autour des brûlures s'est mise à peler, et de la belle peau rose est apparue. Deux semaines plus tard on ne voit presque rien.

Marie-Christine, Jean-Luc, tout va bien. Votre fils est toujours aussi beau !


\begin{figure}[h]
\centering
\includegraphics[height=9cm,width=12cm,keepaspectratio]{pb091702.jpg}
\caption*{ Le lendemain de l'accident.}
\end{figure}


\begin{figure}[h]
\centering
\includegraphics[height=9cm,width=12cm,keepaspectratio]{pb232683.jpg}
\caption*{ Quelques jours plus tard, au frais.}
\end{figure}

\chapter{Mongolie}
Nous partîmes donc pour la Mongolie. En train vu qu'on y a pris goût. Et ce coup ci, pas le choix, 2e classe obligatoire, donc on se retrouve dans des petits compartiments de 4 personnes. C'est moins facile de rencontrer des gens, mais c'est un peu plus confortable que la 3ème classe : on peut enfin s'allonger de tout son long. Nous n'avons qu'une seule voisine de compartiment, une Mongole qui parle aussi Russe. On en profite donc et on prend un petit cours de mongol, et c'est très compliqué à prononcer. Deux semaines plus tard, on ne saura toujours pas exactement comment on prononce merci !
La frontière prend environ 10h à passer, a faire des aller retour en train, remplir des papiers, faire tamponner les passeports, assister à la fouille du train... Et enfin, on entre en Mongolie !


\begin{figure}[h]
\centering
\includegraphics[height=6cm,width=9cm,keepaspectratio]{pb142303.jpg}
\caption*{ Et derrière les buildings, on voit les montagnes !}
\end{figure}

On arrive à 6h du matin dans une ville déserte. On trouve un café avec du wifi, histoire de dire à notre hôte couchsurfeur qu'on est arrivés. Et on se reconnecte aussi à l'actualité, qui n'est pas très glorieuse en ce vendredi 13 novembre (les attentats de Paris). On encaisse le coup. Ça fait bizarre de se sentir plus en sécurité à 10000 km de chez soi...

A 9h, on est chez Naraa et Tunga pour le petit-déjeuner. Et on se sent tout de suite à la maison. Ils viennent d'avoir un bébé, et vivent aussi avec la grand-mère de Tunga. Tunga est une excellente cuisinière qui nous apprendra à cuisiner une spécialité mongole : des beignets au mouton. On lui apprendra à faire de la mousse au chocolat, après avoir bien galéré pour trouver une bête tablette de chocolat.


\begin{figure}[h]
\centering
\includegraphics[height=6cm,width=9cm,keepaspectratio]{pb142312.jpg}
\caption*{ C'est nous qui l'avons fait, enfin, surtout Tunga.}
\end{figure}

Comme à chaque fois qu'on arrive dans une grande ville, on commence par se balader à pied. Il fait froid, mais beau... mais froid quand même ! Le centre-ville est plein de gens et de voitures. Il y a des bouchons partout, ce qui est assez ironique quand on sait que la Mongolie est le pays avec la plus faible densité de population du monde : 1,4 habitants au km\textsuperscript{2}. Mais la moitié de la population du pays est concentrée sur une toute petite surface, alors forcément, ça pose des problèmes...

Tout le pays voue un culte autour de Gengis Khan (ou Chinggis Khan). Sa statue en souverain trône devant le bâtiment du gouvernement, sur la place qui porte son nom, il a sa tête sur tous les gros billets (et les petits billets, ce sont des chevaux), et on lui a construit une autre statue monumentale de lui à cheval, dans un grand parc touristique en devenir en banlieue d'Oulan-Bator. Je ne sais pas vous, mais ce que j'ai retenu de mes cours d'histoire sur Gengis Khan, c'est surtout le coté envahisseur sans pitié. Pas le coté "J'ai conquis et unifié le plus grand royaume de l'Histoire, mis en place des routes commerciales à travers tout le continent, garanti la liberté religieuse, unifié le système de taxe et mis en place une zone de libre échange, et inventé l'immunité diplomatique à une époque où il était plutôt de coutume de torturer sur la place publique les messagers de l'ennemi". La vérité est probablement quelque part entre les deux...


\begin{figure}[h]
\centering
\includegraphics[height=6cm,width=9cm,keepaspectratio]{pb142304.jpg}
\caption*{ Le bâtiment du gouvernement.}
\end{figure}


\begin{figure}[h]
\centering
\includegraphics[height=6cm,width=9cm,keepaspectratio]{pb152387.jpg}
\caption*{ Devinez qui c'est !}
\end{figure}

Naraa et Tunga nous emmènent une journée dans le parc national de Terelj, pour une première petite rando. Mais finalement, Tunga reste dans la voiture avec le bébé, car il ne fait que -14\textdegree C au milieu de l'après midi. C'est beau, nous sommes seuls, et il y a des chevaux sauvages-mais-pas-tant-que-ça partout sur le trajet. Naraa nous explique que les nomades laissent leurs chevaux se débrouiller seuls pendant l'hiver. Les chevaux restent naturellement en troupeaux, et les nomades se contentent de les surveiller de loin, histoire de savoir vaguement dans quels coins ils sont. C'est au printemps qu'ils doivent de nouveau les ré-apprivoiser.


\begin{figure}[h]
\centering
\includegraphics[height=6cm,width=9cm,keepaspectratio]{pb152366.jpg}
\caption*{ Un cheval à moitié sauvage.}
\end{figure}

On passe aussi une journée à faire de la paperasserie pour obtenir le visa chinois. Ce jour là, alors qu'on se baladait entre cyber-cafés, photomatons manuels, ambassades et agences de voyage qui comprennent les besoins en faux billets (d'avions) des voyageurs, un truc incroyable nous est arrivé : soudain, une jeune fille mongole nous rattrape, l'air paniquée, et me tend ma liseuse électronique qui aurait vraiment du se trouver alors dans mon sac. Elle nous explique qu'on doit faire attention, qu'il y a des pickpockets partout : Nos deux sacs à dos étaient grand ouverts, ils avaient pris la liseuse dans le mien. Elle les avait alors visiblement agressés en retour pour la reprendre, non sans se prendre des menaces de morts au passage. Et on n'avait rien remarqué ! Entre l'empressement de la jeune fille à aller quelque part (sûrement un job de super héro ou un truc dans le genre) et notre propre incrédulité quant aux 5 dernières minutes, on n'a même pas eu le temps de la remercier convenablement. Je le fais donc ici : Jeune fille héroïque, nous te sommes éternellement reconnaissant. Le monde a besoin de gens comme toi !


Et à peine quelques heures plus tard, c'est au tour de Marion d'engueuler un mec qui avait lui aussi réussi à ouvrir son sac ! Je commence à croire que les filles sont le cauchemar des pickpockets mongols.

\begin{wrapfigure}{l}{0.55\textwidth}
\centering
\includegraphics[width=0.5\textwidth]{pb152319.jpg}
\caption*{ Un vautour pas du tout sauvage.}
\end{wrapfigure}


Au cas où, voici leur tactique : un groupe de deux ou trois grand gars font une masse compacte autour d'un gringalet collé aux basques de sa victime. Protégé des regards, le gringalet ouvre petit à petit le sac et se sert. En cas de pépin, la petite troupe se disperse aussitôt. De plus, il y a ce que j'espère être des légendes urbaines sur la supposée violence des pickpockets : ils se baladeraient avec des petites lames avec lesquelles ils tranchent les yeux des témoins qui l'ouvrent un peu trop, tout en leur disant "Comme tu as des grands yeux !" (S'il vous plait, dites moi que les Mongols ont juste mal compris le petit chaperon rouge).

On a quand même eu beaucoup de chance. On n'a rien perdu, mais au contraire gagné en expérience !

Tout ceci ne nous empêche pas de perdre de vue notre objectif immédiat : organiser avec Naraa un road-trip d'une semaine dans la campagne. Il compte se lancer dans le tourisme à terme, et nous l'engageons pour sa première mission : nous faire découvrir la Mongolie en vivant et voyageant comme un mongol. Coup de bol, on arrive même à trouver un troisième compagnon en la personne d’Élodie. On a alors besoin d'une plus grosse voiture, ce qui rajoute Mandul et son 4x4 Toyota à l'aventure.


\begin{figure}[h]
\centering
\includegraphics[height=6cm,width=9cm,keepaspectratio]{pb182412.jpg}
\caption*{ Une autoroute en bon état.}
\end{figure}

Alors, la Mongolie, c'est grand. Et leurs autoroutes sont plus ou moins l'équivalent chez nous d'une route communale du bout du village, vous savez bien, les derniers 100m où il n'y a plus de maisons, juste avant d'arriver à la forêt. Non, je n'exagère même pas, il y a vraiment des nids de poules sur leurs autoroutes. On y roule entre 40 et 80km/h selon l'état. Et c'est payant, car c'est tout de même beaucoup plus rapide que les pistes. Et parfois, en particulier quand on cherche une yourte d'un nomade, la route c'est simplement : je vais tout droit vers cette montagne en évitant les plus gros cailloux.


\begin{figure}[h]
\centering
\includegraphics[height=6cm,width=9cm,keepaspectratio]{pb182427.jpg}
\caption*{ Oui, c'est une route.}
\end{figure}

Ainsi, après une journée entière de voiture, nous arrivons à la tombée de la nuit au campement d'un cousin éloigné de Naraa.


\begin{figure}[h]
\centering
\includegraphics[height=6cm,width=9cm,keepaspectratio]{pb182436.jpg}
\caption*{ La yourte sous les étoiles.}
\end{figure}

Traditionnellement, l'hospitalité en Mongolie n'est pas tant une faveur qu'une condition de survie dans un pays où le prochain campement est "quelque part" 10 ou 20km plus loin. Mais il est quand même bien vu de faire des cadeaux à l'hôte : pain, bonbons, vodka, pâtes sont toujours les bienvenus (c'est bien entendu différent dans un camp de yourtes pour touristes). Bref, nous voilà assis dans la yourte avec la femme du cousin, à attendre le retour dudit cousin en sirotant un thé. Enfin, je dis thé parce que c'est comme ça qu'on nous l'a présenté, mais c'est véritablement un bouillon de viande salé, mélangé à du lait entier, infusé au thé. Les amateurs de thé seront surpris, mais pour une soupe légère, c'est plutôt pas mal.


\begin{wrapfigure}{l}{0.5\textwidth}
\centering
\includegraphics[width=0.45\textwidth]{pb182450.jpg}
\caption*{ Belles bottes, non ?}
\end{wrapfigure}


Enfin arrive le cousin. Quel moment ! La porte s'ouvre, et une forme bleue entre par la toute petite porte, et se déplie en un robuste gaillard souriant qui prend le temps d'observer tout ce petit monde. Il n'y a pas à dire, on sent que c'est chez lui ! Il porte les habits traditionnels : un long manteau qui descend sous les genoux, un chapeau de fourrure, et ses énormes bottes sont une œuvre d'art.

Arrive alors l'heure du repas. On nous présente un grand plat d'os bouillis. Mais quand on regarde bien, on remarque qu'il reste de la viande sur les os. Alors, avec les couteaux fournis, on commence à attaquer les os. Il y a aussi ce qui semble être de l'estomac (ou tablier de sapeur pour les amateurs) et une sorte de saucisse, mais probablement faite avec le colon, et remplie de gros morceaux. Mais... malgré toute la bonne volonté du monde, c'est dur. Je parle bien de la viande qui est très dure. On coupe des tous petits bouts, et on mâche, on mâche, et on remâche encore. Et pas question de recracher, ce serait impoli devant nos hôtes qui nous logent et nous nourrissent si généreusement. Et ce serait d'autant plus gênant que nous sommes les seuls à manger, et que le chef de famille, sa femme, et leurs 4 autres invités nous regardent avec attention !


Donc on avale tout rond. Devant notre inexpérience manifeste, notre hôte se décide à nous donner un coup de main pour trouver les derniers bons morceaux. On a l'impression d'être des petits enfants nourris par les adultes, mais ça nous permet de manger un bout de moelle qu'on a presque pas besoin de mâcher ! Et alors qu'on pensait que le repas était fini, voilà qu'arrive une soupe à la viande et aux nouilles ! Nous sommes sauvés ! Nous qui croyions que nous allions passer la nuit avec en tout et pour tout 50g de viande dans l'estomac, nous voilà devant un vrai repas !

Après le repas, c'est l'heure du digestif. Pour le prendre, faut aller casser la glace ! Ben oui, il est congelé dans un tonneau dehors. Et ce n'est rien de moins que du lait de jument fermenté ! Une fois l'airag décongelé, notre hôte en rempli une grande tasse, la passe à un de ses invités - toujours de la main droite à la main droite - qui doit ensuite la boire avant de la rendre notre l'hôte. Et on tourne comme ça dans le sens horaire. Pour les hommes, impossible de refuser, c'est dans leur culture. Un homme, ça boit, sinon, c'est une tapette (ceux qui parlent anglais utilisent le mot "pussy"). Heureusement, l'airag passe plutôt bien. J'y trouve une ressemblance avec les bières IPA pour le coté acide/vinaigré, mais adouci par le lait. Pour vous faire une idée, c'est une sorte de mélange de bière, de lait et de jus de cornichon.
Après 3 tournées d'airag, on décide de sortir la poire de la grand-mère de Marion. Elle remporte un grand succès, et on fini la bouteille dans la foulée et la bonne humeur. Leur petit garçon met aussi beaucoup d'ambiance, et alors qu'il n'a que 3 ou 4 ans, et toujours pas de pantalon ni de slip, il s'amuse à attraper tout et n'importe quoi avec un petit lasso. Dont mon bras. Je me prête au jeu jusqu'au moment ou il a voulu me chevaucher. Mais il n'avait toujours pas de slip...


\begin{figure}[h]
\centering
\includegraphics[height=6cm,width=9cm,keepaspectratio]{pb182441.jpg}
\caption*{ L'heure du digestif, ou pas...}
\end{figure}

Nous sommes donc 12 à dormir dans cette yourte. On s'installe serrés les uns contre les autres et on s'endort. On voit même les étoiles à travers le trou au sommet de la yourte !


\begin{figure}[h]
\centering
\includegraphics[height=6cm,width=9cm,keepaspectratio]{pb182454.jpg}
\caption*{ Dodo !}
\end{figure}

Le lendemain matin, c'est dur pour moi et Marion. On est exténués, et impossible d'avaler quoi que ce soit : l'airag a frappé ! On va mettre deux à trois jours avant de reprendre un peu d'appétit, pendant lesquels l'odeur de lait ou de mouton bouilli nous révulsera. Ça ne nous empêche pas d'aller faire un peu de cheval dans la neige et d'assister à une exécution de mouton surprenante : (Âmes sensibles, passez au paragraphe suivant) Ils mettent le mouton sur le dos, et incisent 10cm juste en dessous de la cage thoracique. Puis, ils rentrent une main dans le ventre du mouton, et vont bloquer une artère à la main pendant une 20aine de seconde, jusqu'à ce que le mouton soit pris de tremblements puis s'affaisse, mort. Le tout dure moins d'une minute, presque sans bruit, et sans qu'une seule goutte de sang soit versée.


\begin{figure}[h]
\centering
\includegraphics[height=6cm,width=9cm,keepaspectratio]{pb192465.jpg}
\caption*{ Balade du matin.}
\end{figure}

On assiste aussi à la traite des vaches. Ils ont des vaches normales, qui font 4 à 6L par jour, on ne voit presque pas leurs pis ! Ça change de nos usines sur pattes... On aurait aussi voulu voir la traite des juments, mais ce n'est possible qu'en été. En hiver, les chevaux sont en liberté dans la montagne, et c'est d'ailleurs une de leurs principales activité hivernale que de surveiller la position du troupeau.


\begin{figure}[h]
\centering
\includegraphics[height=6cm,width=9cm,keepaspectratio]{pb192456.jpg}
\caption*{ Il est pas mignon ?}
\end{figure}

On passe le reste de la journée dans la voiture. En fait, on passera beaucoup de temps dans la voiture : la Mongolie, c'est immense, et Naraa nous dira plus tard qu'il un peu sous-estimé les temps de trajet en hiver. On arrive devant un bâtiment qui s'avère être une église chrétienne, et qui nous servira d'abri pour deux nuits, le temps de faire l'aller retour vers un lac magnifique dans le Nord.


\begin{figure}[h]
\centering
\includegraphics[height=6cm,width=9cm,keepaspectratio]{pb202537.jpg}
\caption*{ Enfin au sommet !}
\end{figure}

Après encore une journée entière dans la voiture, on arrive dans un hôtel-restaurant. En fait, c'est une toute petite maison avec trois pièces : le restaurant (une table et 10 chaises), l'hôtel (deux grand lits) et la cuisine. Les toilettes sont à 100m de l'autre coté de la route : une cabane autour d'un trou, comme partout à la campagne. Vous devez vous dire que par -20\textdegree C, ça doit être l'horreur, mais on se dit qu'en plein été, ça doit être bien pire. En hiver au moins, le contenu est gelé !


\begin{figure}[h]
\centering
\includegraphics[height=6cm,width=9cm,keepaspectratio]{pb212584.jpg}
\caption*{ La chambre d'hôtel.}
\end{figure}

On visite le lendemain un immense volcan qui surplombe la plaine. Et signe de l'activité du volcan, on trouve de nombreuses "bouches" qui exhalent un air chaud et légèrement soufré. On les remarque facilement, car une magnifique cheminée de givre se forme autour.


\begin{figure}[h]
\centering
\includegraphics[height=6cm,width=9cm,keepaspectratio]{pb222587-panorama.jpg}
\caption*{ Panorama du volcan.}
\end{figure}


\begin{figure}[h]
\centering
\includegraphics[height=6cm,width=9cm,keepaspectratio]{pb222624.jpg}
\caption*{ Du givre partout.}
\end{figure}

On passe la nuit dans une vraie maison en dur, chez les parents d'un ami à Naraa. C'était la première fois que Naraa les rencontrait, mais ça marche comme ça en Mongolie : le petit frère de l'ami de la belle sœur de ton cousin débarque chez toi un soir avec 4 potes, pas de souci, tu les loges et les nourris sans poser de question ! On découvre alors la tradition du tabac à sniffer. Pour se souhaiter la bienvenue, point de poignée de main ni d'embrassade ni de bise. Non, on s'assoie et on échange ses fioles de tabac en signe de bienvenue. Ce sont des objets précieux, taillés d'un seul tenant dans des pierres fines, avec un bouchon en corail, or et argent. Les premiers prix commencent autour de 1000\euro et peuvent monter très haut ! Les fioles sont effectivement des œuvres d'art. On ouvre alors légèrement le bouchon et on sniffe un coup avec chaque narine. Ça sent un peu les herbes. Voilà, on s'est dit bonjour !


\begin{figure}[h]
\centering
\includegraphics[height=6cm,width=9cm,keepaspectratio]{pb222655.jpg}
\caption*{ C'est après l'avoir tripotée qu'on a appris sa vraie valeur...}
\end{figure}

On visite aussi des sources chaudes. Au milieu de nulle part (bon, OK, en Mongolie, on est toujours au milieu de nulle part), une station thermale s'est construite autour des sources chaudes. Et c'est là qu'on ne comprend pas : c'est fermé en hiver ! Il n'est ouvert qu'en été, quand la température peut dépasser les 35\textdegree C, mais en hiver, quand il fait -20\textdegree C, qui voudrait donc se baigner dans des sources chaudes, voyons... Tout ce qui est accessible, c'est un bassin à l'air libre dont l'eau est assez chaude pour cuire un œuf, donc on ne se trempe pas... Perso, j'ai eu ma dose d'eau bouillante !


\begin{figure}[h]
\centering
\includegraphics[height=6cm,width=9cm,keepaspectratio]{pb222646.jpg}
\caption*{ Voilà ce qui arrive quand on se balade par -20\textdegree C dans de la vapeur d'eau.}
\end{figure}

La nuit suivante se passe dans une toute petite yourte pour touriste. Pas grand chose à dire, si ce n'est qu'un mec est passé pour demander un fusil car il avait vu des loups à moins de 200m des yourtes. La routine on vous dit...

On nous avait promis la plus grande cascade de glace de toute la Mongolie. Perso, j'attendais un truc exceptionnel, pas un truc de 15m de haut. Je pense que le Saut du Gier en Ardèche est plus haut. Mais bon, ça leur suffit pour organiser une compétition tous les ans (elle aura lieu cette année une semaine après notre passage). Les compétiteurs sont assurés en moulinette, et les meilleurs mettent 15s. On devrait appeler du Ice Speed Climbing à mon avis ! La cascade était magnifique quand on l'a vue, car intacte. Après la compétition, elle ne doit plus ressembler à grand chose...


\begin{figure}[h]
\centering
\includegraphics[height=6cm,width=9cm,keepaspectratio]{pb242808.jpg}
\caption*{ La cascade de glace.}
\end{figure}

La vraie découverte se trouvait 100m plus loin en aval de la cascade : quelques chevaux paissaient tranquillement au bord de la rivière. Avec cette chaude lumière, la neige, les mélèzes dorés, la rivière... c'était enchanteur.


\begin{figure}[h]
\centering
\includegraphics[height=6cm,width=9cm,keepaspectratio]{pb242826.jpg}
\caption*{ Sûrement un fan de The Cure.}
\end{figure}

On a aussi fait un petit tour dans l'ancienne capitale de la Mongolie, Karakorum, dont la principale attraction est un ancien monastère bouddhiste. Ça nous a donné un avant-goût du Népal !


\begin{figure}[h]
\centering
\includegraphics[height=6cm,width=9cm,keepaspectratio]{pb242840.jpg}
\caption*{ Un mur et des moutons.}
\end{figure}

Puis, le moment que Naraa et Mandul attendaient depuis le début : la pêche sous glace ! On fait une première tentative de nuit, ils veulent chopper une race de poisson nocturne. Mandul et Naraa étaient un peu réticents à l'idée de nous emmener, car il faisait vraiment très froid. Sur le chemin, on regardait avec attention le thermomètre de la voiture, en espérant le voir passer sous les -30\textdegree C, quand Naraa nous a expliqué que c'était le seuil bas du thermomètre, et qu'en réalité, il faisait probablement entre -35\textdegree C et -40\textdegree C... Et bien une fois arrivés, malgré l'intégralité nos vêtements sur le dos, le fait qu'on bougeait et qu'on allait de temps en temps de réfugier dans la voiture pour tenter de se réchauffer, je peux vous dire qu'on a eu froid aux pieds ! Heureusement, avec le bout du nez, ce sont les seuls endroits qui ont souffert. Et tout ça pour rien, on n'a choppé aucun poisson.


\begin{figure}[h]
\centering
\includegraphics[height=6cm,width=9cm,keepaspectratio]{pb242859.jpg}
\caption*{ Pêche sous glace et sous les étoiles.}
\end{figure}

Après ça, quand on se réveille le lendemain matin dans la yourte, et qu'il fait -20\textdegree C (le feu s'est éteint et il y a un trou au sommet de la yourte je vous rappelle), on trouve ça plutôt agréable !


\begin{figure}[h]
\centering
\includegraphics[height=6cm,width=9cm,keepaspectratio]{pb252932.jpg}
\caption*{ De gauche à droite : une yourte, un bidon de bordel, les toilettes.}
\end{figure}

A propos du froid, quelques effets marrants : Tout gèle. Très vite ! Une heure de balade dehors, et la gourde est gelée. C'est thermos obligatoire si on ne veut pas se contenter de sucer des glaçons ! Les fruits gèlent bien évidemment, et font de la purée quand on les réchauffe. Les briquets à gaz ne marchent plus ! Les bières gèlent aussi (canette obligatoire : ça n'explose pas), mais pas la vodka. Le diesel gèle aussi, mais pas l'essence. Et les orteils de Marion gèlent aussi un peu : elle a eu une petite gelure sans conséquences.

La pêche sous-glace qu'on a faite le jour suivant s'est donc beaucoup mieux passée. On a choppé des poissons ! Et on a aussi eu un petit moment d'incertitude quand on a entendu la glace craquer. Je savais déjà que ça existait, les guides du lac Baïkal en parlent. Alors je m'attendais à entendre "Crac", voire "Craaaac", mais pas "CRRRRAAAAAKKKKKKRRRRBBBBBRAAAAAAAOUMMMMMMRRRRMMMMLLLLLLLL"... Le son, les vibrations viennent de tout le lac en même temps, la glace et tout notre corps vibrent pendant pas loin d'une minute, on a l'impression d'être sur le point d'être avalé par un animal tellement gigantesque qu'il ne remarquera même pas qu'il a humain coincé entre les dents...
Mais ça n'inquiète pas Naraa, donc nous non plus. Ce qui l'inquiète en revanche, c'est cette idiote de voiture qui roule en notre direction. Sur la glace. Si les 20cm d'épaisseur sont largement suffisants pour supporter quelques piétons, pour la voiture, c'est une autre histoire. On décide alors de plier les gaules (c'est la première fois que j'utilise cette expression littéralement !), car il y a un vrai risque que la voiture passe à travers la glace, et si elle le fait proche de nous, ça pourrait mal se passer... Tant qu'on est sur la glace, on espère vraiment que rien ne se passe, mais une fois à l'abri, cette fameuse curiosité macabre reprend le dessus. La voiture poursuit son chemin sous notre regard anxieux en provoquant craquement après craquement jusqu'à disparaitre... à l'horizon.


\begin{figure}[h]
\centering
\includegraphics[height=6cm,width=9cm,keepaspectratio]{pb252937.jpg}
\caption*{ D'abord on fait un trou.}
\end{figure}


\begin{figure}[h]
\centering
\includegraphics[height=6cm,width=9cm,keepaspectratio]{pb252943.jpg}
\caption*{ Puis on agite un ver en espérant que le poisson n'a pas vu qu'il était congelé.}
\end{figure}


\begin{figure}[h]
\centering
\includegraphics[height=9cm,width=12cm,keepaspectratio]{pb252945.jpg}
\caption*{ Enfin, on fait le fier !}
\end{figure}

Ah oui, et aussi, sur les derniers km, on a vu des chameaux se balader tranquilou dans la neige par -20\textdegree C. Et ce sont des chameaux indigènes, complètement adaptés au froid. On vend même des couettes très chaudes en laine de chameau ! Ça change de l'image d’Épinal du chameau dans le Sahara.


\begin{figure}[h]
\centering
\includegraphics[height=9cm,width=12cm,keepaspectratio]{pb252948.jpg}
\caption*{ Un chameau sous la neige, quoi de plus logique ?}
\end{figure}



\chapter{Beijing}
Nous arrivâmes donc à Beijing. Le choc fut rude. Pour mémoire, on venait de Mongolie, un pays complètement vide, avec des paysages à perte de vue, un ciel bleu quasi permanent. Et on arrive dans une agglomération grande comme la Belgique, et contenant l'équivalent de la moitié de la population de la France, le tout en plein pic de pollution.


\begin{figure}[h]
\centering
\includegraphics[height=6cm,width=9cm,keepaspectratio]{pb302971.jpg}
\caption*{ La place Tian'anmen. La moitié visible en fait.}
\end{figure}

C'est difficile à se représenter cette pollution, mais on va essayer quand même : je crois que l'OMS dit que quand le taux de PM2,5 dépasse 25, c'est pas bien. En France, on fait des plans pollution à partir de 80 (footing déconseillé, circulation alternée et tout le tsoin-tsoin). Les mesures de pollution en Chine vont jusqu'à 600, et ce jour là, les appareils de mesure étaient saturés et le bruit courait qu'on était à 750. Mais quand on est arrivés, on ne savait pas tout ça. Tout ce qu'on savait, c'était qu'il y avait vraiment beaucoup de Chinois en Chine, qu'à Pékin, le haut des immeubles disparaissait dans une brume jaune/maronnasse, et que l'air n'avait pas seulement une odeur, mais aussi comme qui dirait... de la texture, oui, une sorte de texture épaisse et rapeuse. Beaucoup de gens portent des masques. Différents types de masques : le vrai masque anti-pollution, qui épouse bien le visage, avec filtre à particules et valves d'expiration. Il y a aussi le masque en laine fantaisie, avec des animaux tricotés, dont l'efficacité nous semble incertaine, mais pourquoi pas, il y a peut-être un filtre sous la laine... Et enfin, il y a le masque que je qualifierais de "psychologique" dans ces conditions : un bête masque anti-postillon, celui que les soignants et les malades mettent pour éviter de refiler leurs germes aux autres, mais qui n'a certainement aucun effet pour éviter à la pollution d'être inspirée. Et on s'est renseigné : les pharmacies vendent effectivement ces masque en disant que ça protège de la pollution (dont les constituants les plus dangereux sont des particules de moins de 2,5\textmu{}m) tout en ayant des espaces de 1cm de chaque coté. Je ne cesse de m'étonner de la propension d'une partie de l'humanité à profiter de la crédulité de l'autre. Promis, un jour, j'ouvrirai les yeux et j'arrêterai de vivre dans un monde de Bisounours.


\begin{figure}[h]
\centering
\includegraphics[height=6cm,width=9cm,keepaspectratio]{pc013043.jpg}
\caption*{ Des tas de Français !}
\end{figure}

On rencontre Vaï et Guillaume, deux Français qui dorment aussi dans notre dortoir. Ils sont à la fin de leur voyage en Chine, ce qui fait d'eux une source de renseignements précieuse ! On tente un premier restaurant chinois (oui, bon, à partir de maintenant, tous les restaurants sont des restaurants chinois). Évidemment, pas de carte en anglais, pas d'images sur le menu, et le serveur ne parle pas anglais. La solution qu'on a trouvée sur le moment, c'est notre voisin de table qui parle un peu anglais : on a réussi à lui faire comprendre qu'on lui faisait entièrement confiance pour commander pour nous 4. Et ce fut une réussite ! On a mangé super bon pour pas cher : 15\euro pour 4, bières comprises. La bouffe en Chine, on va le découvrir, est très variée, aussi bien en terme de contenu de l'assiette que de prix. Histoire de bien nous mettre dans le bain, on passe rapidement devant une série de petit stands qui offrent tout un tas de brochettes plus délirantes les unes que les autres : poulpe, araignée, scorpion, scolopendre, hippocampe, étoile de mer... Après quelques renseignements, on teste le poulpe, qui est plutôt bon. Les autres, visiblement, sont surtout des attrape-touristes hors de prix : c'est rigolo/exotique de manger une tarentule, mais niveau gastronomique, aucun intérêt.


\begin{figure}[h]
\centering
\includegraphics[height=6cm,width=9cm,keepaspectratio]{pb302975.jpg}
\caption*{ Bon appétit !}
\end{figure}

Un deuxième restau avec Vaï et Guillaume nous amène dans un restaurant islamique (mais quand même chinois). Et ce coup-ci, il y a des images sur le menu, nous sommes sauvés ! On ne remarque pas de grande différence avec un restaurant non-islamique, sauf pour deux choses : il n'y a pas de porc, et les serveuses portent le voile. Par contre, aucun problème avec l'alcool.


\begin{figure}[h]
\centering
\includegraphics[height=6cm,width=9cm,keepaspectratio]{pc012992.jpg}
\caption*{ Je me fond dans la population en adoptant leurs mœurs.}
\end{figure}

Un des passages obligé de Beijing, c'est la place Tian'anmen, juste en face de la cité interdite. Alors un jour de pic de pollution, c'est dommage : on n'arrive pas à voir à l'autre bout de la place, ça gâche un peu... Pareil pour la cité interdite : l'immensité des lieux est gâchée par la visibilité. Mais quand même, c'est grand ! Et pour le coup, on a vraiment l'impression de rentrer dans ces films chinois, ou au moins d'inspiration chinoise genre tigre et dragon ou kung-fu panda.


\begin{figure}[h]
\centering
\includegraphics[height=6cm,width=9cm,keepaspectratio]{pc013004.jpg}
\caption*{ La citée interdite.}
\end{figure}


\begin{figure}[h]
\centering
\includegraphics[height=6cm,width=9cm,keepaspectratio]{pc013026.jpg}
\caption*{ Le degré d'importance des bâtiments est indiqué par le nombre d'animaux sur les arrêtes.}
\end{figure}

On découvre aussi qu'on est pas les seuls touristes, mais que la plupart des touristes sont chinois. Et que beaucoup des touristes chinois viennent de coins de la Chine où très peu d'Occidentaux se rendent. Et que quand ils visitent Beijing pour la première fois, non seulement ils voient la Cité Interdite pour la première fois, mais ils voient aussi des blancs pour la première fois : bref, on se fait dévisager, et mitrailler sans cesse, nous volons la vedettes aux autres attractions... Le plus polis/courageux, viennent nous demander s'ils peuvent nous prendre en photo avec eux. Ils ne veulent rien de plus, prennent leur photo, et repartent sans savoir nos prénoms ou d'où on vient. Et visiblement, pour le Chinois moyen (le pékin moyen, mouarfarfahah !) c'est la classe de montrer une photo de lui avec des blancs ! On accepte toutes les demandes, mais on veut une photo aussi, donc on fera un album de tous les Chinois qui nous ont demandé une photo !


\begin{figure}[h]
\centering
\includegraphics[height=6cm,width=9cm,keepaspectratio]{pc013001.jpg}
\caption*{ Ce vase est le summum de la technique de porcelaine chinoise. Une copie est à vendre pour 10000\euro.}
\end{figure}

Notre troisième jour à Beijing a été surprenant : le vent s'était levé dans la nuit et avait balayé toute la pollution : on voyait le ciel bleu et le soleil ! On est donc allé visiter le palais d'été. C'est un ensemble de palais dans un parc un peu à l'extérieur de la ville (enfin, c'était à l'extérieur de la ville il y a 200 ans, maintenant, c'est accessible en métro). Et tout est artificiel : Bien entendu, les bâtiments sont artificiels, mais aussi le lac et la montagne. Ils ont creusé un lac immense, et fait une montagne avec les débris, puis construit des palais sur la montagne. C'est démesuré, beau, et on remercie le ciel d'avoir chassé la pollution pour nous ce jour là !


\begin{figure}[h]
\centering
\includegraphics[height=6cm,width=9cm,keepaspectratio]{pc023055.jpg}
\caption*{ Le palais d'été... et du ciel bleu !}
\end{figure}


\begin{figure}[h]
\centering
\includegraphics[height=6cm,width=9cm,keepaspectratio]{pc023094.jpg}
\caption*{ La palais est désormais en pleine ville.}
\end{figure}


\begin{figure}[h]
\centering
\includegraphics[height=6cm,width=9cm,keepaspectratio]{pc023111.jpg}
\caption*{ Oui, c'est un bateau en marbre.}
\end{figure}


\begin{figure}[h]
\centering
\includegraphics[height=6cm,width=9cm,keepaspectratio]{pc023307.jpg}
\caption*{ C'est beau hein ?}
\end{figure}


\begin{figure}[h]
\centering
\includegraphics[height=6cm,width=9cm,keepaspectratio]{pc023308-panorama.jpg}
\caption*{ Encore une, parce que vraiment, c'est beau !}
\end{figure}

On découvre aussi les patates cuites dans la rue dans des petits poêles au charbon. On pensait que c'étaient des patates standards, mais en fait non, ce sont des patates douces super bonnes !


\begin{figure}[h]
\centering
\includegraphics[height=6cm,width=9cm,keepaspectratio]{pc023316.jpg}
\caption*{ Manger !}
\end{figure}

Le métro à Beijing est aussi une surprise pour nous : tout neuf, tout moderne, on a presque l'impression de rentrer dans un hôpital tellement tout est carrelé et lumineux. Le seul problème est qu'aucun des automates de vente de ticket n'accepte nos billets. Soit il y a un problème de calibration, soit on a un gros stock de faux billets à écouler ! Autre surprise : tous les bagages, tous les sacs, sont passés au scanner à chaque entrée dans le métro, sans exception. Le tout est opéré par des agents qui ne pourraient mieux incarner l'ennui. Non, sérieusement, vous imaginez ? Assis toute la journée à regarder une émission vraiment chiante, avec pas un seul retournement de situation, et comme point culminant de la journée, un mec qui n'a pas vu qu'il avait un couteau dans son sac... Au moins, dans les aéroports, le contenu des sacs est un peu plus varié, mais là, ils voient toute la journée des porte-feuilles et des parapluies.


\begin{figure}[h]
\centering
\includegraphics[height=6cm,width=9cm,keepaspectratio]{pc023320.jpg}
\caption*{ Il y a aussi des gens qui dansent dans le métro.}
\end{figure}

Sur les conseils de Vaï et Guillaume, on part visiter un pan isolé de la fameuse grande muraille. Plutôt que de payer une fortune pour accéder à un bout de mur tellement envahi par les touristes qu'on ne voit plus le sol, on prend des bus locaux pour aller dans un petit village qui a accès à une portion officiellement non ouverte au public. Le mur n'est pas restauré, en accès libre, le ciel est bleu et nous sommes seuls !


\begin{figure}[h]
\centering
\includegraphics[height=6cm,width=9cm,keepaspectratio]{pc033351.jpg}
\caption*{ Le mur... vide !}
\end{figure}

On n'aurait pas pu rêver de meilleures conditions. Par contre, le mur est abimé par endroit, c'est escarpé au point de devoir progresser avec les mains par moment (hein, quelqu'un a dit escalade ?), et la neige tombée quelques jours auparavant forme une couche de glace par endroit. Bref, on met deux heures pour faire deux km entre les difficultés du mur et les pauses paysages (et Kyle qui s'est rajouté à l'aventure avec des petites chaussures de ville pas vraiment adaptées).


\begin{figure}[h]
\centering
\includegraphics[height=6cm,width=9cm,keepaspectratio]{pc033363.jpg}
\caption*{ Ça glisse.}
\end{figure}

On a du mal à croire ce qu'on voit : ce mur rejoint l'horizon des deux cotés, en suivant un chemin compliqué qui descend dans les vallées et remonte sur la crête des montagnes. Il n'y a des créneaux que du coté Mongol, et perso, si j'avais été Mongol, je n'aurais pas su comment attaquer ce mur. La légende dit que le mur n'a jamais servi, mais peut-être était-ce justement parce qu'il était tellement impressionnant que les attaquants ont été dissuadés avant même de commencer ? Retour brutal à la réalité de la Chine pragmatique : a la fin de cette section, le mur a été abattu pour laisser la place à une route. 10m après être descendu du mur, un grand panneau bilingue nous rappelle que cette section est fermée au public. 10m après ce panneau, un petit panneau en bois gravé à la main posé sur le chemin indique selon nos meilleurs experts sinophones : "bla-bla-bla-en-chinois-3-yuans". 10m après ce panneau, une femme nous demande 3 yuans chacun. Oui, c'est aussi ça la Chine !


\begin{figure}[h]
\centering
\includegraphics[height=6cm,width=9cm,keepaspectratio]{pc043420.jpg}
\caption*{ Un moine.}
\end{figure}

Notre dernier jour à Beijing est occupé par le Lama Temple, un des plus grand temple bouddhiste du monde. La plupart des gens qui le visitent (et paient donc l'entrée comme tout le monde) ont l'air d'être des vrais bouddhistes, étant donné qu'ils ont l'air bien au courant des us et coutumes : ils savent comment et où se prosterner, quoi faire avec ces millions de bâtons d’encens.


\begin{figure}[h]
\centering
\includegraphics[height=6cm,width=9cm,keepaspectratio]{pc043400.jpg}
\caption*{ Ils font tous ça.}
\end{figure}


\begin{figure}[h]
\centering
\includegraphics[height=6cm,width=9cm,keepaspectratio]{pc043402.jpg}
\caption*{ Puis ça.}
\end{figure}


\begin{figure}[h]
\centering
\includegraphics[height=6cm,width=9cm,keepaspectratio]{pc043410.jpg}
\caption*{ Et enfin ça.}
\end{figure}

Du coup, j'ai des sentiments contradictoires. D'un coté, je me sens un intrus, comme si j'étais entré par erreur dans une église orthodoxe au milieu de la messe (exemple pris totalement au hasard), mais d'un autre coté, bordel, j'ai payé mon entrée comme tout le monde ! Pendant un moment, je trouve cocasse d'imaginer devoir acheter un ticket pour assister à la messe à Waldighoffen, puis je me rend compte qu'à Waldighoffen, ce n'est pas l'entrée mais la sortie qu'on paye... Une autre chose m'intrigue : devant les brûloirs d’encens, un panneau dit que c'est interdit d'en brûler les jours de pollution et les jours de vents. Étant donné que c'est le vent qui chasse la pollution, je me dis qu'ils ne doivent pas brûler de l’encens tous les jours... Cela dit, le temple est magnifique. L’encens, les moines, l'architecture, les fidèles perdus dans leurs prières, tout ça fait qu'on oublie qu'on est toujours dans le centre de Beijing. Le clou du spectacle est atteint avec le plus grand Bouddha du monde taillé dans un seul morceau de bois : 3 étages, 9m de haut !


\begin{figure}[h]
\centering
\includegraphics[height=6cm,width=9cm,keepaspectratio]{pc043415-panorama.jpg}
\caption*{ Beau morceau !}
\end{figure}

D'ailleurs : je commence à me rendre compte qu'il y a un truc avec le plus grand Bouddha du monde. Il y en a des tas, mais à chaque fois avec une condition : le plus grand debout en intérieur, le plus grand en pierre à l'extérieur, le plus grand couché, le plus grand en lotus... Je me demande combien il y a de Bouddhas le plus grand du monde.
Sur le retour, on fait un saut à la cité des antiquaires, remplies de merveilles en Jade. Hors de prix bien entendu. Je me demande du coup qui est capable de claquer plusieurs millions pour un service à thé, et surtout, est-ce que le domestique (oui, je pense que le mec qui achète cette théière a un domestique dédié au thé) a une assurance responsabilité civile suffisante en cas de bris.


\begin{figure}[h]
\centering
\includegraphics[height=6cm,width=9cm,keepaspectratio]{pc043424.jpg}
\caption*{ Je parie que la théière ne verse même pas bien.}
\end{figure}

Puis, c'est l'heure de prendre le train de nuit ! On nous avait dit d'être au moins 1h, voire plus, en avance à la gare, même en ayant déjà les billets. Ça nous a semblé un peu exagéré, mais on a suivi les conseils. Nous avons bien fait, une heure, ce n'est pas de trop ! La gare est conçue un peu comme un aéroport : il faut tout d'abord faire la queue à l'entrée de la gare, où le ticket et le passeport sont vérifiés, et les bagages scannés. Les accompagnants ne rentrent pas dans la gare. Les couteaux sont interdits en théorie, en pratique, si c'est un petit couteau suisse caché au fin fond du sac, ça peut passer. Mais on a vu des gens se les faire confisquer, et on n'a pas réussi à trouver de règle officielle. On a l'impression que les grands couteaux à lame non repliables sont interdits, et que les petits... ça dépend. Pour le moment, on est passé entre les gouttes. Une fois le sac vérifié, on doit trouver notre salle d'attente, qui est en fait plutôt une salle d'embarquement. Le moment venu, un portail s'ouvre au fond de la salle, et les passagers du train ont quelques minutes pour embarquer, et c'est seulement à ce moment là qu'on a accès au quai. Et je peux vous dire qu'il y a du monde qui rentre dans un train chinois !

Les trains chinois de nuit ont trois classes, presque comme les trains russes. Soft-sleeper : compartiments fermés de 4 personnes, hard-sleeper : compartiments ouverts de 6 personnes, hard-seat : des sièges tout bêtes. Vu les bons souvenirs des trains russes, on a pris du hard-sleeper, en se disant que ce serait similaire à la 3ième classe russe. Bon, en vrai, c'est un peu moins bien. C'est très jouable, mais quand même, c'est moins bien. Les couchettes sont empilées par groupe de 3, donc il y a moins de place pour les bagages. Il n'y a pas de draps propres pour chaque passager, les draps ne sont changés qu'une fois au départ du train, donc si on arrive en cours de route, il est possible qu'on ne soit pas le premier à dormir dedans. Les Chinois sont bruyants. Ça ne les dérange pas d'être à plusieurs avec leur petite enceinte à fond, en plus de la musique passée dans le train. Un nombre certain jettent leur ordures par terre comme ça, pouf, normal quoi. En particulier, ceux qui mangent des graines de tournesol laissent des gros tas de coquilles partout. Les Chinois ont une façon bizarre d'exprimer leur intérêt envers les occidentaux : ils les fixent, l’œil vide, la bouche ouverte, et ne bougent pas ni ne parlent, y compris si on les fixe en retour, jusqu'au moment où on leur fait un petit signe de la main. Là, deux réactions possibles : ils sont contents, sourient, et nous font aussi un petit signe, ou alors ils détournent les yeux, prennent un air détaché et font genre "hein, non, c'était pas moi, très intéressant cette tâche sur la peinture". Il est vrai que la barrière de la langue n'aide pas. Ils ne parlent ni anglais ni allemand ni rien, et nous parlons très exactement 2 mots de chinois. Ça limite les échanges...



\chapter{Pingyao}

\begin{wrapfigure}{l}{0.55\textwidth}
\centering
\includegraphics[width=0.5\textwidth]{pc063507.jpg}
\caption*{Les remparts, de jour.}
\end{wrapfigure}


Nous arrivâmes donc à Pingyao. A 5h du matin. Et à 5h du matin à Pingyao, il y a des hordes de tuk-tuk ! Ils nous assaillent comme il se doit, et nous les repoussons car d'une part, on n'aime pas être assaillis, et d'autre part, l'hôtel est à deux pas de la gare. Mais on se rend vite compte qu'à 5h du matin à Pingyao, c'est calme, sombre et désert. L'éclairage urbain est éteint, et personne ne se promène dans les rues, hormis dans un rayon de 5m autour de la sortie de la gare. A l'intérieur de la ville, on ne voit rien. Oui, l'intérieur, c'est l'intérieur de la muraille. Pingyao est une ancienne ville fortifiée. C'est en gros un carré de 2,5km de coté entouré d'une muraille de 10m de haut. Ce serait une des plus belle et mieux conservée des villes fortifiées du monde.



Et à l'ombre de cette muraille, de nuit, on y voit comme dans un vieux four à pain éteint. Donc on attend l'aube sur un banc un peu à l'extérieur, ce qui nous permet d'assister au réveil de la ville. On voit petit à petit apparaitre des gens qui fond leur footing, des gens qui vont au boulot en chantant, on voit un mec marcher à reculons, et on a l'impression d'entendre des "réveilleurs publiques" ? Aux premières lueurs de l'aube, on s'engage dans la ville déserte et on retrouve notre hôtel.


\begin{figure}[h]
\centering
\includegraphics[height=6cm,width=9cm,keepaspectratio]{pc053430.jpg}
\caption*{Pingyao, à 7h du matin.}
\end{figure}

Au passage, on prend un petit-déjeuner vraiment pas bon (c'est suffisamment rare en Chine pour que ça mérite d'être souligné) et beaucoup trop cher. Heureusement, notre hôtel rattrape le coup. Certes, il fait 7\textdegree C dans la chambre, mais on a, pour la première fois depuis le début du voyage, notre propre salle de bain ! Et la température de la chambre va rapidement remonter grâce au climatiseur, ne vous inquiétez pas.


\begin{figure}[h]
\centering
\includegraphics[height=6cm,width=9cm,keepaspectratio]{pc053463.jpg}
\caption*{On s'y croirait, non ?}
\end{figure}

La ville est vraiment magnifique, on dirait un musée habité. Pas de pub lumineuses, pas de néons, pas de buildings, et pas de voitures ! Le temps s'est arrêté avant la révolution industrielle et la ville a été petit à petit transformée en une sorte de grand parc d'attractions. Tout est en bois et en pierre sculpté. Le sol est pavé et les rues touristiques sont bardées d'échoppes d'artisanat, de restaurants à touristes (chinois) et de musées. L'entrée dans la ville est gratuite, et on peut acheter un billet d'entrée pour l'ensemble des musées de la ville, valable trois jours.


\begin{figure}[h]
\centering
\includegraphics[height=6cm,width=9cm,keepaspectratio]{pc063515.jpg}
\caption*{Pas de doute, c'est bien son signe !}
\end{figure}

Les musées sont en fait des ensembles de maison d'époque (courtyard en anglais). On peut en particulier visiter la première banque du monde ayant inventé le principe du chèque. La ville est en effet devenue très riche grâce à quelques marchants, qui ont profité de leur fortune pour créer leurs propres banques. Et dans cette ville, berceau des banques modernes, cruelle ironie, impossible de trouver un distributeur compatible Visa ou Mastercard...


\begin{figure}[H]
\centering
\includegraphics[height=6cm,width=9cm,keepaspectratio]{pc063531.jpg}
\caption*{Un vieux tout mignon qui joue un peu de guitare quand personne ne lui achète ses crêpes.}
\end{figure}

Nous sommes quasiment les seuls touristes occidentaux. En trois jours, on croisera moins de 10 blancs, et probablement 8 milliards de Chinois. Du coup, ils ne parlent pas un seul mot d'anglais, ni à l'hôtel, ni dans les restaurants. Il y a quand même quelques panneaux avec des explications en anglais, et même quelques fois en français ! Mais il y a tant de choses à dire sur ces traductions que ça fera l'objet d'un article dédié.
On verra aussi le fameux "Mickey Chinois". Qui fait un peu peur. Il a même essayé de me soutirer de l'argent pour avoir pris cette photo :


\begin{figure}[H]
\centering
\includegraphics[height=6cm,width=9cm,keepaspectratio]{pc053434.jpg}
\caption*{Mais quel rapport avec une ville fortifiée chinoise ?}
\end{figure}

J'ai hésité à lui parler de propriété intellectuelle, mais quelque chose me dit qu'en Chine, ils s'en contrefoutent (sans oublier que même moi, je n'y crois qu'à moitié).
On voit aussi que le problème de la pollution n'est pas cantonné aux grandes villes. A l'arrivée, le ciel était bleu. On pensait que c'était normal, on était quand même assez loin des grandes villes. Mais en fait, c'était à cause du vent des jours précédents. Durant nos trois jours à Pingyao, on a vu le ciel s’opacifier petit à petit, pollué par le chauffage et la cuisine au charbon omniprésents. Et l'armée de véhicules électriques n'y change rien (surtout si les centrales électriques tournent au charbon).


\begin{figure}[H]
\centering
\includegraphics[height=6cm,width=9cm,keepaspectratio]{pc063534.jpg}
\caption*{Il y en a un comme ça tous les 10m. Forcément, ça n'aide pas.}
\end{figure}

On a aussi eu un petit aperçu des conditions de travail en Chine. Mieux vaut illustrer par une photo :


\begin{figure}[H]
\centering
\includegraphics[height=6cm,width=9cm,keepaspectratio]{pc063504.jpg}
\caption*{Que fait le CHSCT ?}
\end{figure}

Perso, je trouve déjà que c'est un peu limite. Mais quand on sait que de l'autre coté, c'est juste un chinois tout maigre qui tient la corde à la main, ça devient carrément suicidaire...
Niveau nourriture, on a vite compris qu'il fallait sortir des remparts (enfin vite, le temps d'un petit déjeuner et d'un repas de midi). Et le fait de voir un restaurant rempli de chinois n'est pas un gage de qualité du restaurant : ce sont tout simplement des touristes chinois qui n'y connaissent pas plus que nous. Avec le temps, on apprend à faire la différence entre le touriste chinois et le local chinois, grâce à des indices subtils du genre sac à dos et gros appareil photo. Bref, toujours est-il que hors des remparts, nous trouvons beaucoup de bouffe de rue : des nouilles sautées, des soupes aux nouilles, des sortes de crêpes, des brochettes, du tofu grillé etc. Et aussi des nouilles en soupe : on commence par choisir des brochettes de viande/légumes/tofu/autres trucs bizarre, la cuisinière les cuit au wok en quelques secondes, et rajoute les nouilles et du bouillon. C'est très bon, on n'a pas besoin de parler la langue, et ça coute environ 1,5 \euro par personne.


\begin{figure}[H]
\centering
\includegraphics[height=6cm,width=9cm,keepaspectratio]{pc053478.jpg}
\caption*{Vous êtes sûre de ce que vous faites ?}
\end{figure}

On trouve enfin des trucs un peu sucrés : des sortes de beignets genre churros, qu'on peut saupoudrer au choix de piment ou du sucre (devinez ce qu'on a choisi). Tellement fan qu'on y retournera tous les jours.


\begin{figure}[H]
\centering
\includegraphics[height=5cm,width=9cm,keepaspectratio]{pc053475.jpg}
\caption*{Enfin du sucre (et du gras) !}
\end{figure}


\begin{wrapfigure}{l}{0.55\textwidth}
\centering
\includegraphics[width=0.5\textwidth]{pc0735391.jpg}
\caption*{Cet homme est une mitraillette !}
\end{wrapfigure}

En se baladant un peu plus loin dans les petites rues hors des murailles, un soir, on est absorbé par un cuisinier qui fait des nouilles à la baguette. Il a un plat de pâte à nouille dans un main, une grande marmite d'eau bouillante devant lui, et avec son autre main, il utilise une baguette pour, en un seul mouvement, détacher un long bout de pâte et le jeter dans la marmite. Et il fait ça si vite qu'on a du mal à suivre à l’œil nu. Sans même se concerter avec Marion, on décide de s'asseoir dans ce restaurant. Et on n'a pas regretté. On a vraiment très bien mangé, et quelle surprise lors de l'addition. On demande le prix, et avec les mains, la serveuse nous fait "11", c'est à dire 1,5\euro. Content de payer si peu cher, on lui donne 22, pensant que c'était le prix par personne. Mais non, c'était pour les deux !



Puis, c'est l'heure d'aller à Xi'an (prononcer chie-anne), et le tout, en train à grande vitesse, où plus communément "bullet train", à prononcer "belette trai-inne", et le train belette, ça nous fait beaucoup marrer. Blague à part, c'est un train très confortable, plus encore que nos TGV. Il s'arrête sur le quai pile à l'endroit prévu, pas besoin de faire la moitié du quai en courant pour rentrer dans le bon wagon. Le seul point négatif, c'est la télé. Oui, il y a une télé tous les 5m dans le train, qui passe des pubs en chinois, avec le son, pendant tout le trajet...



\chapter{Xi'an}
Nous arrivâmes donc à Xi'an. Et pour ceux qui n'ont pas lu le précédent article, ça se dit "chie-anne", avec probablement des tons en plus. Elle fait partie des 10 plus grandes villes chinoise avec 8 millions d'habitants, a été pendant longtemps la capitale de la Chine, et abrite un des candidats au titre de 8ème merveille du monde : le mausolée de l'empereur Qin. Et je n'en avais jamais entendu parler avant de commencer ce voyage ! Alors qu'à coté de ça, je connais, bien malgré moi, le prénom du bébé de Kayne West... (Aucun rapport, je cherchais juste l'information la plus inutile possible que je connaisse) (et pour ceux qui ne savent pas, c'est un prénom complètement con, quand on porte le nom West : North...)


\begin{figure}[h]
\centering
\includegraphics[height=6cm,width=9cm,keepaspectratio]{pc083590.jpg}
\caption*{Circulation dans Xi'an}
\end{figure}

 Petit tour rapide des trucs qu'on a vus dans la ville :
\textbf{La muraille autour de la ville.} On peut payer pour aller faire du vélo dessus, mais bon, on avait déjà vu la grande muraille de Pékin et la muraille de Pingyao, donc niveau muraille, on était plutôt bien, pas besoin d'en remettre une dose, on l'a vue vite fait d'en bas.
\textbf{ La Bell Tower.} C'est une tour au centre de Xi'an. Et elle a des cloches, comme son nom l'indique. On peut payer pour aller dedans, mais elle est plus jolie vue de l'extérieur.


\begin{figure}[h]
\centering
\includegraphics[height=6cm,width=9cm,keepaspectratio]{pc103650.jpg}
\caption*{PC103650.jpg}
\end{figure}La Drum Tower de nuit

\textbf{La Drum Tower.} C'est une tour presque juste au centre de Xi'an (la place au centre était déjà prise). Et elle a plein de gros tambours comme son nom l'indique. A notre grande déception, on ne les entend jamais, ce qui nous amène à nous poser la question suivante : "Mais quel intérêt ?". Nous n'avons malheureusement pas de réponse fiable, mais c'est probablement une raison du genre "conservation de reliques, pas les user, blablabla". On peut payer pour aller voir les tambours de près. On a craqué, on aurait pu s'en passer.


\begin{figure}[h]
\centering
\includegraphics[height=6cm,width=9cm,keepaspectratio]{pc1136611.jpg}
\caption*{Aucune oie en vue...}
\end{figure}

\textbf{La grande pagode de l'oie sauvage}. C'est une grande pagode, et comme son nom ne l'indique pas, aucune oie sauvage dans les parages. On peut payer pour rentrer dans le monastère construit autour il y a moins de 10 ans (gros intérêt historique), puis repayer pour monter au sommet et admirer d'un peu plus haut le nuage de pollution jaunâtre qui couvre Xi'an. Heureusement, nous faisons cette visite en compagnie d'Eve, une jeune chinoise ravie de pratiquer son français et de nous présenter un peu sa ville.


\begin{figure}[h]
\centering
\includegraphics[height=6cm,width=9cm,keepaspectratio]{pc113678.jpg}
\caption*{Merci pour la visite, Eve !}
\end{figure}

Il y a aussi la petite pagode de l'oie sauvage, mais on ne s'est pas donné la peine d'aller vérifier la présence d'oies dans le coin.

Sinon, se balader dans la ville est finalement plus intéressant que les monuments. On assiste à une compétition féroce entre plusieurs marques de téléphone et/ou d'opérateurs mobiles : les rues sont envahies de gros bonhommes en plastique contenant des petits bonhommes bien vivants. Ils font des processions devant les boutiques des concurrents, des haies d'honneur pour les consommateurs, et ralentissent beaucoup la circulation sur les trottoirs. Il y a aussi les grands ballons et les scènes temporaires. Les photos ne rendent pas compte du bruit : des haut parleurs poussés au maximum en permanence complètent le tableau.


\begin{figure}[h]
\centering
\includegraphics[height=6cm,width=9cm,keepaspectratio]{pc123694.jpg}
\caption*{Les armées sont en place !}
\end{figure}


\begin{figure}[h]
\centering
\includegraphics[height=6cm,width=9cm,keepaspectratio]{pc123695.jpg}
\caption*{Regardez bien, derrière la publicité, on remarque le symbole de Xi'an, la Drum Tower !}
\end{figure}

La rue musulmane fait aussi partie des sorties obligatoires. C'est une rue, en fait un petit quartier, pleine de street food, et de touristes. Chaque échoppe essaie de se démarquer, souvent en mettant l'atelier directement sur le trottoir, ce qui nous permet de voir des techniques de confections plutôt surprenantes.


\begin{figure}[h]
\centering
\includegraphics[height=6cm,width=9cm,keepaspectratio]{pc083558.jpg}
\caption*{Quoi, tu n'as jamais vu un vrai mec faire de la pâtisserie ?}
\end{figure}


\begin{figure}[h]
\centering
\includegraphics[height=6cm,width=9cm,keepaspectratio]{pc083560.jpg}
\caption*{Ça chauffe !}
\end{figure}

Il y a un monde fou, et pour couvrir le bruit et se faire entendre, les haut-parleurs sont inévitables ! On essaie d'acheter des bananes frites, car ça semble être un des trucs à tester ici. On demande le prix, le mec met d'abord les bananes à frire, puis nous dit "20 yuans". Que n'avait-il pas dit... Marion l'engueule comme il se doit, il ne comprend rien, et reste avec ses bananes sur les bras. 3\euro pour deux bananes, faudrait pas nous confondre avec des lapins de six semaines ! On fait la queue comme tout le monde pour avoir le privilège de manger le fameux hamburger chinois (de la viande d'agneau hachée dans un pain frit). La technique est bien rodée : dès qu'on commence à faire la queue, on doit acheter le bon pour le burger. Ce qui nous incite fortement à rester dans la file même si on attend longtemps. La file est donc toujours bien pleine, ce qui intrigue les autres passants qui se disent que ce burger doit être fantastique pour que la file soit si longue... bref, vous avez compris l'idée. Et oui, le burger était pas mal, bonne quantité de viande, et sans os ! Par la suite, on s'est rendu compte qu'il y en avait dans tous les restaurants, et que ce n'était pas la peine de faire la queue pour en avoir...


\begin{figure}[h]
\centering
\includegraphics[height=6cm,width=9cm,keepaspectratio]{pc123698.jpg}
\caption*{Mais puisque je vous dis que ça passe !}
\end{figure}

C'est aussi à Xi'an qu'on essaie d'envoyer pour la première fois un paquet en France. Trois semaines avant Noël, c'est le bon moment, et on va se débarrasser de quelques vêtements chauds puisqu'on a pris 20\textdegree C en deux semaines. A la poste chinoise, la première étape, c'est l'emballage : impensable d'arriver avec son paquet déjà fait ! Sinon, comment pourrait-ils vérifier le contenu ? C'est donc ce qu'ils font. On est assez confiant, que vont-ils trouver à redire à une paire de collant, des chaussettes en poils de chameau, quelques cartes postales et des petits jouets ? Rien. Ou si peu... Quoi ? Du bois ? Quelle insulte, quel affront fait à la sécurité de leur pays ! Oser envoyer un cerf-volant qui contient facilement de quoi faire 3 cure-dents, non, vraiment, ce n'est pas négociable. Idem pour les jetons de xiangqi. On repartira avec. La deuxième étape de l'envoi est marrante aussi : formulaires à remplir en triple exemplaire, photocopie des passeports, numéros de téléphones des destinataires à fournir obligatoirement etc. Bienvenu dans la maison qui rend fou ! (cf les 12 travaux d'Astérix) Bilan de l'opération : 1h d'efforts pour envoyer deux petits paquets.

On a fait la sortie obligatoire au Mausolée de l'empereur Qin, dit aussi, Armée de terre cuite, ou encore Terracotta Army. On trouve assez facilement le bus public qui nous y amène, c'est quand même à 2h de route, et nous sommes les seuls touristes occidentaux dans le bus. Il nous dépose sur un parking qui est encore à 10mn à pied des caisses. Sur le chemin, nous repoussons plusieurs vagues de guides touristiques anglophones à grands coups de "No, thank you". Mais ils sont tenaces, et tentent de nous convaincre de leur indispensabilité à grands coups de chiffres et de dates. Après les caisses, rebelote pour 10mn de marche et à nouveau quelques vagues de guides. Enfin, nous arrivons sur le site. Il a été découvert par des paysans en train de creuser un puits, forcément au milieu d'un champ. Un grand hall a été construit autour de la zone explorée, un grand rectangle autour duquel on peut se promener. Bien entendu, interdiction d'approcher les statues sans faire partie de l'équipe d'archéologues qui continue le travail d'exploration et de restauration.


\begin{figure}[h]
\centering
\includegraphics[height=6cm,width=9cm,keepaspectratio]{pc093635.jpg}
\caption*{Il y a encore un peu de boulot.}
\end{figure}


\begin{figure}[h]
\centering
\includegraphics[height=6cm,width=12cm,keepaspectratio]{pc093615-panorama.jpg}
\caption*{Mais quand même, c'est impressionnant !}
\end{figure}

C'est grand, mais on a vite fait le tour. Il y a beaucoup d'explications sur les techniques de restauration, mais moins sur comment le mec a pu se convaincre qu'une armée de terre cuite allait pouvoir l'aider une fois mort. Je pense qu'il faudrait enquêter du coté de la guilde des potiers, qui a sûrement réussi à corrompre les conseillers de l'empereur (suivez l'argent, comme on dit).
En conclusion, on est content d'avoir vu ça, c'est un site incroyable, toussa toussa, mais l'impression qu'il m'en reste, c'est surtout celle d'un empereur qui a probablement oublié de profiter de la vie à force de craindre la mort.


\begin{figure}[h]
\centering
\includegraphics[height=6cm,width=9cm,keepaspectratio]{pc083583.jpg}
\caption*{Le cerf-volant, ce n'est pas pour rigoler !}
\end{figure}

Finalement, la chose la plus intéressante qui nous est arrivée à Xi'an, c'est la rencontre avec Scott, un couchsurfer qui nous accueille deux jours chez lui. Il vit dans un nouvel ensemble d'immeubles comme il doit s'en construire un par semaine en Chine : 20 buildings d'une 60aines d'étages, la population d'Annonay dans un petit quartier. Il est spécialisé dans les énergies renouvelables et la transition énergétique mais gagne plus d'argent en faisant des traductions. Il nous apprend qu'au premier étage (donc le rez-de-chaussée pour les Français), il y a souvent des salles de jeux dédiées au majong, équipées de tables mécaniques qui mélangent les jetons et les placent sur la table automatiquement. Les chinois en sont fous, on en voit partout dans les rues, des jeunes, des vieux, des femmes, des hommes, et évidemment, ils jouent de l'argent.


\begin{figure}[h]
\centering
\includegraphics[height=6cm,width=9cm,keepaspectratio]{pc083611.jpg}
\caption*{Une partie de majong.}
\end{figure}

Il m'apprend à jouer à cet autre jeu qu'on voit partout dans les rues : le xiangqi, une sorte de jeu d'échec chinois. On découvre tout un tas de petits restaurants typiques dans les environs en sa compagnie. Quand on sait où chercher, c'est possible de manger très bon pour vraiment pas cher, mais il faut savoir, ou alors au moins lire le chinois (à ce propos, je recommande vivement l'application memrise qui m'a permis d'apprendre juste ce qu'il faut de caractères chinois pour savoir si on commandait un plat avec des nouilles ou du riz). Il raconte aussi la transition de la Chine vers le capitalisme telle qu'il l'a vécue alors qu'il était un enfant. Il vivait avec ses parents dans un appartement d'état d'une seule pièce avec cuisine et salle de bain commune. Toute l'économie était gérée par l’État : le travail, le logement, et ce qu'on avait le droit d'acheter, géré à base de coupons, depuis le sac de riz jusqu'à la télévision.
C'est avec ce témoignage qu'on commence à prendre la mesure du chemin parcouru par la Chine ces dernières années.


\begin{figure}[h]
\centering
\includegraphics[height=6cm,width=9cm,keepaspectratio]{pc123701.jpg}
\caption*{Ça ressemble à un épluche légume tout doux et sans lame. On l'a logiquement baptisé : Le Caresse-Carotte !}
\end{figure}



\chapter{Shanghai}
Nous arrivâmes donc à Shanghai. Et nous sommes accueilli comme des princes chez Charles, et en même temps, c'était en toute simplicité... bref, il est cool. C'est un ami d'un cousin qui vit en Chine depuis plus de 10 ans. C'est un bonheur de le laisser commander pour nous au restaurant ! Fini les interrogations sans fin : Est-ce qu'ils ont bien compris ? C'est quoi qu'on a commandé en vrai ? C'est du piment ou de la tomate ce truc là ? On va même se faire un restaurant japonais à volonté avec table chauffante (teppanyaki). Oui, à volonté. Sushi compris. Boisson comprise ! Pour 30\euro. A Shanghai, ce qui nous a plu, c'est que c'est un peu moins la Chine qu'ailleurs en Chine. Entendons nous bien : c'est la toujours la Chine hein ! Mais pour la première fois depuis qu'on est en Chine, on arrête de nous prendre en photo dans la rue, car on est plutôt commun parmi tous ces expats. On a même trouvé un restaurant savoyard ! Et des bières belges !!! Bref, on fait une pause quelques jours sur les nouilles sautées. Et ça nous fait comme une sorte de plaisir coupable de manger un burger et des frites au milieu de la Chine, comme si c'était honteux de ne pas profiter à fond de toutes les spécialités locales surprenantes... mais c'est bon la honte...


\begin{figure}[h]
\centering
\includegraphics[height=6cm,width=9cm,keepaspectratio]{pc133710.jpg}
\caption*{Shanghai, de nuit.}
\end{figure}

Pour être honnête, on n'a pas fait lourd à Shanghai en une semaine. On a pas mal glandé. On est allé voir un concert live de jazz. On s'est baladé dans la concession française. On est allé voir les lumières de la ville, et on a pris les mêmes photos du Bund que tout le monde. On sait bien que tout le monde a déjà pris cette photo, alors on cherche une autre possibilité, un angle auquel personne n'a pensé, et au bout d'un moment, faut se rendre à l'évidence, ce n'est pas pour rien que tout le monde a pris exactement cette photo et pas une autre.


\begin{figure}[h]
\centering
\includegraphics[height=6cm,width=9cm,keepaspectratio]{pc153758.jpg}
\caption*{Shanghai, le Bund, de nuit.}
\end{figure}

On a donc vu les trois grandes tours de Shanghai. Le fameux décapsuleur (World Financial Center), une autre grande tour tellement pleine de barres qu'Alain Robert a dit qu'il aurait pu la gravir d'une seule main (Jin Mao Tower), et la petite dernière, la Shanghai Tower, 632m, qui est en fait la 3ème plus grande tour du monde, et qui n'était malheureusement pas éclairée au moment où on est passés. Pour la petite histoire, non, le décapsuleur n'a pas un trou pour laisser passer les avions (hahaha), mais, d'après les architectes, pour laisser passer le vent... Genre. Le vent. Je ne saisis pas vraiment pourquoi ils ont besoin de se cacher derrières des pseudos raisons techniques plutôt que de juste avouer qu'ils voulaient faire un truc cool.


\begin{figure}[h]
\centering
\includegraphics[height=6cm,width=9cm,keepaspectratio]{pc143722.jpg}
\caption*{A gauche : le décapsuleur. En haut : la tour pleine de barre. En bas : le petite dernière.}
\end{figure}

En fait, une des raisons qui nous pousse à rester aussi longtemps à Shanghai, c'est le visa vietnamien. C'est une des seules ville en Chine où c'est possible de le faire faire. On essaie de trouver ce qu'il faut fournir comme papier, où est le formulaire à remplir... et rien. On se pointe à l'ambassade avec un passeport, du liquide et une photo d'identité, on revient 5 jours plus tard, et ça y est ! Ça change de la Russie ou de la Chine !


\begin{figure}[h]
\centering
\includegraphics[height=6cm,width=9cm,keepaspectratio]{pc153783.jpg}
\caption*{C'est pour moi qu'elle fait semblant de croire que les fleurs en plastique lumineuses ont une odeur...}
\end{figure}

On se perd un peu dans la ville, on arrive dans un quartier assez calme, propre, avec de jolis lampadaires, et d'un coup, paf, sans prévenir, côte à côte, les concessions de Maserati, puis, Mac Laren, et enfin Rolls Royce. Le genre de quartier où la bonne arrive en Audi... Et dans le genre moins luxueux, on a passé pas mal de temps à trouver le fameux marché du faux. Évidemment, ce n'est pas son nom officiel, ce serait vraiment trop provocateur. Mais tout le monde le connait sous cette appellation. C'est un grand centre commercial, plein de petites boutiques dans lesquelles on peut trouver des montres, des sacs à main, de l'électronique, des jouets, des bijoux. Et il faut négocier ! Bon, il est vrai qu'il faut négocier partout en Chine, mais là plus encore qu'ailleurs. C'est dur au début de négocier ! On ne sait pas quel est le vrai prix. La règle "diviser le prix de départ par 3" est connue des commerçants et donc ne marche pas ! Ils peuvent faire n'importe quoi entre fois deux et fois cent ! Alors on balance des prix un peu au hasard, ce qui nous semble correct pour nous, et on regarde ce qui se passe. Invariablement, le commerçant fait une tête comme si on venait de lui arracher un bras. L'affront le fait suffoquer. En tant que touristes faisant attention au maximum à rester respectueux des us des pays visités, le premier réflexe est de se dire qu'on a fait une bourde, et qu'il va penser que tous les Français sont vraiment des cons. Mais ce serait une erreur, il faut lutter contre ce réflexe, et se moquer gentiment de lui, rester ferme, et se barrer en dernier recours. Si le dernier prix offert laisse une marge au commerçant, il nous courra après. Pour l'anecdote, on est quand même tombé sur une vendeuse qui nous soutenait mordicus que ses boucles d'oreilles étaient serties de vraies perles... au marché du faux... à 5\euro la paire... mais bien sûr ! Elle n'en a pas démordu de toute la négociation, allant jusqu'à gratter une perle avec une lame pour faire de la poudre. On n'a toujours pas compris en quoi ça prouvait quoi que ce soit.


\begin{figure}[h]
\centering
\includegraphics[height=6cm,width=9cm,keepaspectratio]{pc153748.jpg}
\caption*{On trouve ça en plein centre de Shanghai, ça nous donnerait presque le mal du pays !}
\end{figure}


\begin{figure}[h]
\centering
\includegraphics[height=6cm,width=9cm,keepaspectratio]{pc143745.jpg}
\caption*{Toujours Shanghai. Encore de nuit.}
\end{figure}



\chapter{Huangshan, la montagne aux milles marches}
Nous arrivâmes donc à Huangshan. Huangshan est une des montagnes sacrées de Chine. Et visiblement, en Chine, sacré est synonyme de payant et blindé de touristes. Huangshan City est aussi l'autre nom de Tunxi, une ville à plus de 100km des montagnes, qu'il ne faut pas confondre avec Tangkou qui elle est vraiment la ville à coté des montagnes appelées Huangshan, mais Tangkou n'a pas de gare, et donc quand on veut visiter Huangshan, on prend un train pour Huangshan City, mais quand on arrive à la gare, on arrive à Tunxi, puis il faut prendre un bus pour Huangshan, mais on arrive à Tangkou, et enfin on prend un téléphérique pour les montagnes. Bref, il nous a fallu un peu de temps pour démêler tout ça, mais on y est arrivé ! Pour l’anecdote, Tunxi est une ville de plus d'un million d'habitants. Dans le Lonely Planet qui nous sert de guide, cette ville occupe 5 lignes  : elle n'existe que pour prendre le train, et ne mérite pas qu'on s'y arrête ne serait-ce qu'une heure. Je me demande comment le million d'habitants qui la composent le prendraient s'ils l’apprenaient un jour.

C'est la basse saison pour Huangshan. Seul un téléphérique sur deux est en fonctionnement, et on voit bien que la foule de touristes chinois qui nous entourent ne remplissent qu'une toute petite fraction des files d'attente à l'entrée du téléphérique. L'entrée est aussi à moitié prix. Il faut dire que la météo est loin d'être idéale. La couche de nuages est juste au dessus de nos têtes, une bruine froide et continue essaie désespérément de nous faire rentrer nous mettre au chaud sous la couette, mais on tient bon ! Après tout, peut-être que le téléphérique nous fera traverser la couche de nuages, et qu'au dessus, c'est grand ciel bleu ? Et en effet, on traverse rapidement la couche de nuages ! Mais... bon, il y en a une deuxième au dessus... On ne perd toujours pas espoir : peut-être qu'on va aussi la traverser ? On s'en approche, on rentre dedans, et... on est arrivés. Dans le nuage. Il pleut. Il fait froid. Il y a même de la neige et de la glace sur le chemin. Et c'est envahi de chinois.


\begin{figure}[h]
\centering
\includegraphics[height=6cm,width=9cm,keepaspectratio]{pc203811.jpg}
\caption*{Entre les deux couches de nuages.}
\end{figure}

D'ailleurs, on ne comprend pas : c'est très clairement la basse saison, et à vue de pif, il doit y avoir au moins 10 fois plus de gens en haute saison, mais dans ce cas, comment est-ce possible de tous les faire tenir ? Parce que là, c'est embouteillage permanent au sommet de la montagne, alors avec 10 fois plus de gens ? D'ailleurs, le sommet, c'est un peu plus qu'un sommet. Il y a tout un tas de petits sommets et de points de vue reliés entre eux par des chemins. Il doit y avoir au moins une dizaine d'hôtels et une cinquantaine de km de chemins qui n'ont rien à voir avec un sentier de montagne : tout est bétonné et empierré, avec des escaliers et des rambardes partout, d'où le titre de l'article. En réalité, le surnom de cette montagne c'est plutôt "les Monts Jaunes". C'est tellement accessible que les chinoises  se permettent d'y aller en robe et talons aiguille, c'est quand même mieux pour les photos.


\begin{figure}[h]
\centering
\includegraphics[height=6cm,width=9cm,keepaspectratio]{pc203820.jpg}
\caption*{Visibilité au top !}
\end{figure}

Il faut le savoir, les chinois ne font pas du tourisme pour voir des beaux paysages, ils font du tourisme pour ramener de jolies photos et se la péter auprès des amis et des collègues. Ça s'appelle "avoir la face", et c'est surtout une question de montrer qu'on a de la thune, donc ils aiment ce qui brille, et ce qui coûte cher, voire ce qui donne l'impression de couter cher. Un chinois peut tout à fait se prendre en photo devant un hôtel de luxe, histoire d'avoir la face quand il montrera les photos, pour ensuite aller dormir en dortoir. Tout ça pour dire qu'on est dans un des endroits les plus prisés des chinois pour prendre des photos pour montrer qu'ils ont la face. Et il pleut. Du coup, ils achètent tous un poncho jaune jetable, ainsi que le pantalon de pluie, les protège-chaussures et la canne assortis (tant pis pour la belle robe). Nous sommes pris dans une marée jaune qui veut absolument sa photo au bon endroit, quand bien même le paysage est inexistant car il y a une visibilité de 10m. Même ceux qui ont déjà des super vestes de montagne achètent aussi le poncho... On voit trois explications à ce comportement : soit on ne leur a pas expliqué le gore-tex et ils ont acheté la veste juste parce que c'était la plus chère, soit ils ne voudraient pas que les autres chinois pensent qu'ils sont trop pauvres pour s'acheter un poncho jetable, soit elle vient du fake market, et les coutures fondent quand il pleut.


\begin{figure}[h]
\centering
\includegraphics[height=6cm,width=9cm,keepaspectratio]{pc203818.jpg}
\caption*{Un porteur, et des chinois en ciré jaune !}
\end{figure}

On passe donc la journée à marcher doucement dans la pluie et les chinois, en voyant tout juste le bout de nos pieds, en essayant de ne pas trop penser au fait qu'on est en train de passer à côté des plus beaux paysages du monde. Et on part se coucher tôt sans avoir vu le fameux coucher de soleil, plein d'espoir dans la journée du lendemain, qui, selon certaines sources mais pas toutes, devrait être ensoleillée. A 6h du matin, le réveil sonne, et le dortoir se réveille. On a trente minutes pour rejoindre le sommet à coté de l’hôtel, où le lever de soleil promet d'être magnifique. Hélas, nous sommes toujours plongés dans les nuages.


\begin{figure}[h]
\centering
\includegraphics[height=6cm,width=9cm,keepaspectratio]{pc203822.jpg}
\caption*{Mi-figue, mi-raisin devant le lever de soleil.}
\end{figure}


\begin{figure}[h]
\centering
\includegraphics[height=6cm,width=9cm,keepaspectratio]{pc203823.jpg}
\caption*{Mi-raisin, mi-figue devant le lever de soleil.}
\end{figure}

Un peu dégoutés, on retourne se coucher sur les coups de 7h.

Et à 7h20, nos compagnons de dortoir nous re-réveillent pour dire que les nuages se sont dissipés, et que le ciel est grand bleu. Ils avaient raison : grand ciel bleu, mer de nuages en contrebas. Et vous savez le plus beau ? On était lundi, et tous les chinois étaient rentrés chez eux. On a eu la montagne pour nous.

Et maintenant, vous avez mérité quelques photos :


\begin{figure}[h]
\centering
\includegraphics[height=6cm,width=9cm,keepaspectratio]{pc213842.jpg}
\caption*{C'est beau !}
\end{figure}




\begin{figure}[h]
\centering
\includegraphics[height=9cm,width=12cm,keepaspectratio]{pc214178.jpg}
\caption*{Ces paysages ont servi d'inspiration pour les décors d'Avatar.}
\end{figure}


\begin{figure}[h]
\centering
\includegraphics[height=9cm,width=12cm,keepaspectratio]{pc2138381.jpg}
\caption*{C'est vraiment beau !}
\end{figure}


Dans le genre, on ne comprend pas les chinois, il y a le coup des porteurs. On a croisé des porteurs partout sur les chemins, avec tout et n'importe quoi : de la nourriture, de la boisson, des poubelles, des matériaux de construction, et même des pierres ! Ils ont l'épaule déformée par le poids des marchandises, qui doit dépasser 80kg pour certains. C'est sûr que vu les chemins, il n'y aurait guère que l'hélicoptère qui pourrait les remplacer, au prix de sérieuses nuisances sonores. Mais là où on ne comprend pas, c'est que le boulot des porteurs commence tout en bas de la montagne, juste à coté du départ du téléphérique. Ils préfèrent payer des gens une misère, à faire un boulot qui leur détruit la santé, plutôt que d'utiliser la machine toute neuve juste à coté...


\begin{figure}[h]
\centering
\includegraphics[height=9cm,width=12cm,keepaspectratio]{pc2138471.jpg}
\caption*{Non, ce n'est pas du polystyrène peint.}
\end{figure}


\begin{figure}[h]
\centering
\includegraphics[height=9cm,width=12cm,keepaspectratio]{pc2138501.jpg}
\caption*{Une petite pause.}
\end{figure}



\chapter{Hong Kong, Buildings et Dim Sum}


Nous arrivâmes donc à Shenzhen. Oui, je sais l'article s'intitule Hong-Kong, mais on n'entre pas dans Hong-Kong comme ça. On a d'abord passé une nuit à Shenzhen car notre train arrivait trop tard pour qu'on puisse passer la frontière le même jour. Le lendemain, on a passé la frontière, et ça nous change des frontières russes ou chinoises. Un passeport à montrer, pas besoin de visa, pas besoin de payer, et juste de l'autre coté de la frontière, le métro Hongkongais, un des métros les mieux conçus du monde, qui nous amène très rapidement au centre-ville.


\begin{figure}[h]
\centering
\includegraphics[height=6cm,width=9cm,keepaspectratio]{pc244404.jpg}
\caption*{Centre ville de Hong Kong}
\end{figure}

Nous arrivâmes donc à Hong-Kong. Par le métro. Qui est hyper bien fait. Le plus marquant je trouve, ce sont les correspondances : pas besoin de se perdre dans un labyrinthe de tunnels souterrains pour passer d'une ligne à l'autre. Non, la correspondance est toujours le quai en face,on n'a jamais plus de 10m à faire. Et ils ont aussi la fameuse carte octopuss. C'est une carte de paiement sans contact utilisable partout : tous les transports publics, les distributeurs automatiques de cartes sim (oui, ils ont ça aussi), les petits magasins, et certains restaurants. Bref, c'est tellement pratique que même pour 3 jours sur place, ça vaut le coup d'en prendre une chacun ! La taille des pièces de monnaie joue aussi en faveur de la carte. Si on ne fait pas attention à la petite monnaie, on risque vite d'en avoir 2kg dans les poches...


\begin{figure}[h]
\centering
\includegraphics[height=6cm,width=9cm,keepaspectratio]{pc244401.jpg}
\caption*{On voit un peu les décorations de Noël géantes sur certains immeubles.}
\end{figure}

L'ambiance est très différente de la Chine. On se fait nettement moins bousculer dans la rue, quand bien même la rue est bondée. Même tard dans le soir, les files d'attente aux arrêts de bus restent parfaitement alignées, personne ne resquille, tout le monde est calme, et quand le bus arrive, si il est plein, on ne bourre pas dedans comme un âne, mais on attend sagement le suivant.


\begin{wrapfigure}{l}{0.55\textwidth}
\centering
\includegraphics[width=0.5\textwidth]{pc234340.jpg}
\caption*{File d'attente à l'arrêt de bus.}
\end{wrapfigure}

On dort dans le dortoir le plus cher de tout notre voyage. Hong-Kong n'est pas vraiment la destination la plus abordable d'Asie, et ça ne s'arrange pas à Noël... Heureusement, la nourriture est plutôt peu chère comparée au logement. On découvre la spécialité du coin : les dim-sum. Et quand je dis \emph{la} spécialité, en fait, ce sont plutôt \emph{les} spécialités. Les dim-sum, on ne sait même pas vraiment à quoi ça fait référence, ce sont juste plein de petits plats, pour la plupart cuits à la vapeur, mais pas uniquement. Mais tout est extrêmement bon ! Il y a des petits trucs sucrés, salés, sucrés-salés, au poisson, à la viande, et même à la crème anglaise. La commande arrive petit à petit, dans le désordre : ce n'est pas la peine d'espérer que le dessert arrive à la fin. Vraiment aucun respect pour les bonnes manières !


\begin{figure}[h]
\centering
\includegraphics[height=6cm,width=9cm,keepaspectratio]{pc234327.jpg}
\caption*{Ces petits cochons mignons sont fourrés à la crème anglaise.}
\end{figure}

Pour continuer sur le thème de la nourriture,  on a mangé du \emph{congee} pour la veille de Noël. Mais qu'est-ce donc ? On n'en savait rien non plus, quand on est entrés dans le boui-boui à coté de notre hôtel.Pourquoi avons nous choisi ce restaurant ? C'est simple : il était plein de locaux en permanence.  J'avoue comme repas de Noël, on a déjà fait plus festif : c'est une sorte de bouillie de riz, mélangée avec ce qu'on veut : viande, poisson, herbes, épices. Ce n'est pas du tout appétissant, mais il ne faut pas s'arrêter à l'apparence ! C'est en fait plutôt bon, et très nourrissant.  Mais comparé aux photos du repas de Noël qu'on a reçu de la famille, comment dire...


\begin{figure}[h]
\centering
\includegraphics[height=6cm,width=9cm,keepaspectratio]{pc244414.jpg}
\caption*{Joyeux Noël !}
\end{figure}

Et puis, ce n'est pas comme s'ils fêtaient Noël là-bas. En voyant la ville, on peut penser que ça a de l'importance : il y a les mêmes décorations lumineuses dans les rues qu'en Europe, les devantures de magasins sont remplies de guirlandes et de sapins, et il y a même un show lumineux sur le thème de Noël, avec des lutins qui luttent pour apporter les cadeaux à temps (bon, cette partie, c'est peut-être simplement une allégorie de la Chine). Mais pour les Hong-kongais, c'est une journée comme les autres.


\begin{figure}[h]
\centering
\includegraphics[height=6cm,width=9cm,keepaspectratio]{pc2444111.jpg}
\caption*{Les lutins en action.}
\end{figure}

Dès le premier jour, nous rencontrons dans l'auberge de jeunesse un autre français qui habite à Hong-Kong depuis quelques mois. Il nous accompagne directement pour une visite guidée de la ville, avec trajet en ferry, volière géante, et legos star-wars. C'est une ville vraiment surprenante : une rangée de buildings coincés entre un bras de mer et une montagne.


\begin{figure}[h]
\centering
\includegraphics[height=6cm,width=9cm,keepaspectratio]{pc234242.jpg}
\caption*{Dans la volière, les oiseaux ne sont pas vraiment farouches.}
\end{figure}


\begin{figure}[h]
\centering
\includegraphics[height=6cm,width=9cm,keepaspectratio]{pc234217.jpg}
\caption*{Chantier d'extension sur la mer.}
\end{figure}

La ville est on ne peut plus active, les rues grouillent de gens à toute heure, les buildings poussent comme des petits pains, et pourtant la nature est toute proche. En une demie heure de transports publics, on se retrouve au pied du \emph{dragon back trail}, le sentier du dos du dragon, une magnifique rando qui suit la ligne de crête des montagnes de l'île. Et ce coup-ci, contrairement à ce qu'on trouve en Chine, c'est une vraie rando, un petit sentier avec des cailloux et de la boue. Pour la première fois depuis une éternité, on avait l'impression de retrouver de la nature sauvage ! On était tellement contents qu'on s'est mis à galoper sur le sentier en rigolant, complètement ivres d'oxygène.

Si on prend le bon ferry, on peut même visiter l'île de Lamma. Aucun rapport avec ce quadrupède cracheur, ni avec Serge. C'est une île sans voiture ! Il y a des sentiers, des villages de pêcheurs, des plages, et ... une énorme centrale électrique qui alimente tout Hong Kong, ce qui donne des paysages assez surprenants !


\begin{figure}[h]
\centering
\includegraphics[height=6cm,width=9cm,keepaspectratio]{pc244362.jpg}
\caption*{L'île de Lamma, avec la centrale au charbon.}
\end{figure}

Le temps était estival, 25\textdegree C pour le jour de Noël, ça ne nous était encore jamais arrivé. Et après le froid des semaines précédentes, ça faisait vraiment du bien. Au retour, si comme nous vous loupez le ferry, il n'y a plus qu'à attendre le prochain deux heures plus tard en admirant les fruits de mers encore vivants qui servent de cartes aux restaurants !


\begin{figure}[h]
\centering
\includegraphics[height=6cm,width=9cm,keepaspectratio]{pc244383.jpg}
\caption*{Bel appendice !}
\end{figure}


\begin{figure}[h]
\centering
\includegraphics[height=6cm,width=9cm,keepaspectratio]{pc244388.jpg}
\caption*{A voir son regard, on dirait qu'il sait qu'il va être mangé.}
\end{figure}



\chapter{Hainan, comme un parfum de tropiques}
Nous partîmes donc pour l'île de Hainan. Le point le plus au sud de la Chine, et, on l'espère, le plus chaud et ensoleillé de la Chine également ! C'est un peu la Corse de la Chine. Le trajet fut... comment dire... intéressant ! Il n'y avait plus de place en couchette dans le train, étant donné la proximité avec nouvel an, donc on a pris des places en \emph{hard seat} (un siège tout simple), pour un trajet de plus de 20h. Et ça a été long... Déjà, le dossier des sièges est parfaitement vertical, mais il y a quand même un petit appui-tête. Donc c'est impossible de dormir assis sans tomber en avant. Encore faudrait-il pouvoir s'endormir : les lumières sont allumées toute la nuit, et les chinois parlent, s'engueulent, mettent de la musique, regardent des films, toute la nuit durant. On a même vu deux chinois bourrés commencer à se taper dessus. Et la gestion des déchets est assez simple : tout fini par terre, même si une petite poubelle se trouve sur chaque tablette. Bref, on n'a pas beaucoup dormi. Et on est arrivé sur l'île. Dans le train ! Eh oui, le train embarque sur un ferry sans que les passagers aient à quitter leur compartiment. On voit la mer à travers les vitres du train puis les hublots du ferry, et on sent le train tanguer. Qui eut crut qu'il fût possible d'avoir le mal de mer en train ?


\begin{figure}[h]
\centering
\includegraphics[height=6cm,width=9cm,keepaspectratio]{pc294514.jpg}
\caption*{Marion dans le ciel.}
\end{figure}


Une fois arrivés sur l'île, on devait encore prendre un bus pour rejoindre le petit village un peu hors des sentiers touristiques qu'on avait ciblé : Wenchang. Une plage, un hôtel, et une plantation de cocotier, ça nous semblait bien comme programme. Par contre, après avoir un peu galéré, on finit par comprendre qu'il n'y a plus de bus, mais qu'il faut prendre le train. Eh oui, nos informations étaient un peu dépassées : le village était désormais accessible en train à grande vitesse.


\begin{figure}[h]
\centering
\includegraphics[height=6cm,width=9cm,keepaspectratio]{pc294491.jpg}
\caption*{Il y a quatre ans, seul l'hôtel depuis lequel je prend la photo existait, tous les autres bâtiments sont apparus depuis.}
\end{figure}

Bref, le village avait un petit peu changé, et les centres de vacances s'alignent maintenant les uns à coté des autres le long de la plage. On rencontre un couple germano-chinois qui vit sur cette île depuis quelques années et qui nous a raconté tous les changements depuis l'arrivée du train. La Chine, ça change très très vite ! Ils nous donnent aussi une petite leçon de cerf-volant, après nous avoir montré leur spectacle à quatre cerf-volants de toute beauté !


\begin{figure}[h]
\centering
\includegraphics[height=6cm,width=9cm,keepaspectratio]{pc294493.jpg}
\caption*{Il faut voir ça en mouvement...}
\end{figure}


\begin{figure}[h]
\centering
\includegraphics[height=6cm,width=9cm,keepaspectratio]{pc294501.jpg}
\caption*{Si je pratique tous les jours pendant un an, je pourrai ensuite attaquer le vol en formation !}
\end{figure}

Depuis le début de la Chine, on n'avait pas vu beaucoup de vie sauvage, mais on mettait ça sur le compte des grosses villes. Là, on était sur une île assez peu urbanisée, et en se baladant le long d'un massif de fleurs, ça nous a soudain frappé. Depuis un certain temps, on trouvait qu'il y avait un truc bizarre dans l'air, une ambiance inhabituelle, mais on a mis du temps à mettre le doigt dessus : il n'y a pas d'insectes dans les fleurs. Aucun bourdonnement, aucune activité, rien, comme si c'étaient des fleurs en plastique. On a cherché longtemps, et on a bien réussi à trouver une abeille par ci ou par là, souvent mourante d'ailleurs, mais sans plus. On savait déjà qu'à certains endroits de Chine, la pollinisation se faisait désormais à la main, mais constater l'absence des insectes par soi-même, ça fait bizarre.


\begin{figure}[h]
\centering
\includegraphics[height=6cm,width=9cm,keepaspectratio]{pc284467.jpg}
\caption*{Pas d'insectes, mais des \st{bernards l'hermites} \st{bernard-l'hermites} ... un bernard l'hermite et son copain !}
\end{figure}

Ici, nouvel an, tout comme Noël, c'est un truc de touristes, pas vraiment pour les chinois. Mais on arrive quand même à trouver un bar à touristes qui fait un peu la fête ce soir là. Et pour fêter ça comme il se doit, on se fait un burger ! Et c'est bon un burger après un mois de nouilles et de riz ! Le tout en compagnie d'un Camerounais et d'une Russe qui nous ont abreuvés de vin français tout au long de la soirée !


\begin{figure}[h]
\centering
\includegraphics[height=6cm,width=9cm,keepaspectratio]{pc284458.jpg}
\caption*{WAHOOOUUUUUU !}
\end{figure}

Pour fêter la nouvelle année, j'ai aussi décidé de mettre un peu d'ordre dans le bordel que certains audacieux pourraient appeler une chevelure et une barbe. Direction le village, il y aura bien un coiffeur compatissant. Un truc étonnant en Chine, c'est que les coiffeurs semblent être ouverts tout le temps. 


\begin{wrapfigure}{l}{0.55\textwidth}
\centering
\includegraphics[width=0.5\textwidth]{p1034546.jpg}
\caption*{Résultat des opérations ! (Un jour plus tard, dans une autre ville, pour ceux qui reconnaissent le paysage)}
\end{wrapfigure}

C'est le monde à l'envers : en fin de soirée il peut devenir très compliqué de manger un morceau ou boire un coup, mais pour se faire une nouvelle coupe, aucun souci ! On trouve donc ce qui semble être un salon de coiffure normal (oui, parce qu'il arrive aussi que le salon de coiffure soit juste une façade, et on se fait tailler d'autres trucs dedans...), et ils semblent comprendre que je veux me faire couper les cheveux et la barbe. Ils me font signe de suivre une jeune femme dans la salle du fond, elle me fait m'allonger sur un matelas, et c'est parti pour 30mn... de shampoing- massage ! Puis coupe de cheveux par un coiffeur obsessionnel compulsif à l'affut du moindre cheveu qui dépasse. Enfin il s'est intéressé à ma barbe : un petit coup de tondeuse pour commencer, et il a ensuite légèrement humidifié les 2mm de barbe restante et il a sorti une lame de rasoir neuve.

Ce fut un massacre.



Il ne m'a pas beaucoup coupé, mais j'en ai chié... Heureusement que la blouse me couvrait les mains, il n'a pas vu comment je m'agrippais aux accoudoirs à chaque fois qu'il faisait un mouvement. Je me suis mis à transpirer à cause de la douleur, ce qui a au moins eu le mérite de ramollir un peu les poils, mais ce fut quand même très long : 20mn pour me raser. Il galérait tellement qu'il a même sorti une deuxième lame neuve à mi-parcours. Heureusement pour moi et pour lui que, pour un français, je ne suis pas très barbu. Mais comparé au chinois moyen, ça a du lui faire bizarre !

Ça a aussi fait bizarre aux employés de l'auberge. A mon retour, ils ne m'ont pas reconnu et ils m'ont accueilli comme un nouveau client. J'ai joué le jeu avec un grand sourire, jusqu'au moment ils se sont dit que ce grand sourire n'était pas normal, et se sont mis à éclater de rire en réalisant leur méprise !


Sur cette île, on a vu que beaucoup de gens avaient les dents rouges. Mais ce n'est pas un problème d'hygiène dentaire : ils passent leurs journées à mâcher une sorte de petit fruit vert. Je crois que ça s'appelle noix de bétel en français, mais ça demande confirmation. On a voulu goûter, et on a essayé d'en acheter aux marchands dans les rues. Mais ils ont refusé en nous mimant la folie. En effet, c'est un psychotrope euphorisant, d'ailleurs interdit en France, et on dirait que les marchands ont voulu nous en protéger ! Et dire que tous les chauffeurs de taxis carburent à ça...


\begin{figure}[h]
\centering
\includegraphics[height=9cm,width=12cm,keepaspectratio]{coconut.jpg}
\caption*{A défaut de noix de betel, on se gave de noix de coco !}
\end{figure}


\begin{figure}[h]
\centering
\includegraphics[height=8cm,width=9cm,keepaspectratio]{pc274434.jpg}
\caption*{Notre dealer officiel.}
\end{figure}



\chapter{Yangshuo : enfin de la campagne  !}


Nous arrivâmes donc à Yangshuo, après avoir de nouveau pris le train qui prend le ferry. Mais ce coup-ci, fini le \emph{hard seat}, on voyage de nouveau en wagon lit ! Le train nous a d'abord amené à Guilin, une autre de ces villes gigantesques dont personne n'a entendu parlé, et de là, on a de nouveau pris un bus. A propos de ces bus, c'est toujours un peu compliqué de savoir quel bus on doit prendre. On essaie de privilégier les bus publics, mais les compagnies privées (qui sont parfois juste composées d'un bus avec un chauffeur et un rabatteur) font tout pour nous embrouiller. Les rabatteurs attendent à la sortie des gares ferroviaires et à l'entrée des gares routières, et nous parlent en continu pour surtout qu'on ait aucune chance de réfléchir et qu'on monte le plus vite possible dans leur bus. Si on les écoute, leur bus est toujours le seul disponible, et il va bientôt partir. Parfois, ils reprennent même les couleurs et les logos des bus publics. Bien entendu, tout est faux, le bus attend d'être plein pour partir, ce qui peut parfois prendre beaucoup de temps, les tarifs peuvent être beaucoup plus chers, et pour couronner le tout, il peut même arriver que la destination ne soit pas exactement la bonne.


\begin{figure}[h]
\centering
\includegraphics[height=6cm,width=9cm,keepaspectratio]{p1054700.jpg}
\caption*{Je n'avais pas de photo de bus...}
\end{figure}

Bref, on essaie de se concentrer, de les ignorer sans être trop impolis, mais parfois, c'est un peu dur car ils peuvent être très tenaces. Ils ne nous lâchent pas et on a très envie de les envoyer paître. C'est là que le temps passé à lire les guides et à parler aux autres voyageurs paye : on sait ce qui nous attend, on connait les arnaques avant d'y être confrontés, on connait les vrais prix. Si on était vraiment livrés à nous même, on se ferait arnaquer à tout bout de champ ! Tout ça pour dire qu'on arrive à Yangshuo avec le bon bus et à l'heure prévue.


\begin{figure}[h]
\centering
\includegraphics[height=6cm,width=9cm,keepaspectratio]{p1034540-panorama.jpg}
\caption*{On est monté au sommet pour trouver un mur, une porte, et une chinoise qui vendait un ticket pour avoir le droit de voir la vue. Je ne vous raconte pas l'état de Marion...}
\end{figure}

C'est une ville surprenante : selon le standard chinois, c'est minuscule et c'est aussi coincé entre les massifs karstiques. Les routes et les bâtiments se faufilent entre les pics calcaires et l'horizon est constellé de petites collines verdoyantes. Le coté touristique est toujours bien présent bien entendu : quand on longe la rivière connue pour ses paysages, les 500 premiers mètres sont tellement bordés de boutiques que la rivière est invisible. Mais si on a le courage de continuer un peu, on arrive dans une zone complètement différente. D'un seul coup, plus de bâtiments, plus de boutiques, plus de vendeurs : juste la rivière qui zigzague au milieu des rizières et des montagnes. En continuant à se balader au hasard, on a même finit pas se retrouver au milieu des champs de mandariniers. On avait l'impression de trouver pour la première fois en Chine de la vraie campagne !


\begin{figure}[h]
\centering
\includegraphics[height=6cm,width=9cm,keepaspectratio]{p1034561.jpg}
\caption*{Des mandarines !}
\end{figure}

Évidemment, qui dit massif de calcaire dit escalade ! C'est un des endroit les plus connus de Chine par les grimpeurs. Nous avons dormi dans une auberge de jeunesse spéciale grimpeurs : quand on arrive, les premières questions ne sont jamais "Vous venez d'où ?", ou bien "Vous allez où ?" comme c'est le cas d'habitude, mais directement : "Vous avez quel niveau ?". On s'est donc fait une petite journée de grimpe à Swiss Cheese, une falaise couverte de gros trous avec plein de voies faciles et une forêt de bambous au pied. Après 3 mois sans exercice, sur un mur mouillé et avec des chaussons empruntés, on n'a pas été méga performants, mais ça faisait bien plaisir quand même !


\begin{figure}[h]
\centering
\includegraphics[height=6cm,width=9cm,keepaspectratio]{p1044577.jpg}
\caption*{J'aurais volontiers pris une photo de Marion en train de grimper, mais elle était réticente à l'idée que je lâche la corde.}
\end{figure}

Évidemment, qui dit massif de calcaire dit grottes ! C'est une des activités classiques de Yangshuo. Tant qu'à faire, on a choisi une grotte connue pour ses bains de boue et ses bains d'eau chaudes. En France, quand on visite une grotte, souvent, le guide fait en même temps un cours de géologie, et nous explique la formation de la grotte, des stalactites, etc. Ici, pas du tout, la visite s'apparente à une observation de nuages par des maternelles : ici, vous pouvez voir un cochon bourré, là c'est un chameau, et enfin au fond à droite, c'est un Bouddha. On voit surtout des bouddhas et des animaux, et de temps en temps un choux-fleur, le tout éclairé par un mec qui a abusé du LSD.


\begin{figure}[h]
\centering
\includegraphics[height=6cm,width=9cm,keepaspectratio]{p10648172.jpg}
\caption*{Ça pique les yeux, non ?}
\end{figure}

Une fois passée la partie visite toute simple, on passe au bain de boue : une grande flaque de boue toute froide. On rentre prudemment dedans, on s'enfonce jusqu'à mi-cuisse, un peu inquiet à la fois de ne pas voir dans quoi on met les pieds (il y a surement des animaux bizarres et affamés au fond des flaques de boue), et aussi inquiet à l'idée de peut-être rester coincé dedans ! Tout ces doutes dissipés, on peut enfin se détendre, régresser de quelques années et se lancer enfin dans une bataille de boue de bon aloi ! Une fois bien recouverts de boue, et un peu refroidis, on peut attaquer le bain d'eau chaude, complètement artificiel, mais on s'en fout, c'est nécessaire. Tout le reste de la grotte est conçu pour nous faire dépenser encore un peu d'argent en plus du ticket : il y a des casiers payants à l'entrée, on peut acheter des petits marteaux pour taper sur des cloches dans la grotte, il y a trois magasins de souvenirs régulièrement espacés, et enfin, on peut se payer un nettoyage des pieds par des poissons avides de peaux mortes. Je trouve que ça fait beaucoup pour une seule grotte !


\begin{figure}[h]
\centering
\includegraphics[height=6cm,width=9cm,keepaspectratio]{p1064825.jpg}
\caption*{Jolie babiole en vente dans la grotte.}
\end{figure}

Les environs de Yangshuo sont aussi très agréables en vélo. En quelques minutes, on peut être au milieu de nulle part, quasiment seuls sur une petite route au milieu des rizières.


\begin{figure}[h]
\centering
\includegraphics[height=6cm,width=9cm,keepaspectratio]{p1054805.jpg}
\caption*{Demain, je ne sais pas où on va, mais ce sera à vélo !}
\end{figure}

C'est ainsi qu'on a aperçu deux magnifiques buffles dans une rizière au bord de la route, genre paysage de carte postale. Sans trop réfléchir, on s'arrête, on descend dans le champ pour prendre des photos. Et le plus gros des deux buffles s'est mis à s'approcher de nous doucement. En soufflant, la tête levée, probablement l'air un peu menaçant, mais c'est difficile à dire quand on n'en côtoie pas tous les jours. Le paysan un peu plus loin nous faisait signe et semblait rigoler et être content, mais quand le buffle a commencé à être vraiment proche de nous, on a quand même rejoint tant bien que mal nos vélos, le buffle sur nos talons (c'est boueux une rizière), et le paysan est finalement venu à notre secours en chassant son buffle à coups de bâtons...


\begin{figure}[h]
\centering
\includegraphics[height=6cm,width=9cm,keepaspectratio]{p1054692.jpg}
\caption*{Confiance ou pas confiance face à ce bestiau ?}
\end{figure}

La ville en elle même est une petite ville touristique. Il y a tout un quartier plein de restaurants allemand, où on peut boire les meilleures bières allemandes pour un prix supérieur à celui qu'on trouve en Allemagne. C'est difficile de se balader tranquillement, car les serveurs des restaurants peuvent être TRÈS insistants, certains nous ont tellement mis leur menu dans la figure qu'on a du les repousser physiquement à plusieurs reprises. On a du mal à croire qu'ils peuvent remplir leurs restaurants comme ça... Dans les restaurants, c'est aussi très intéressants. Alors qu'on mangeait un excellent porc caramélisé, on a vu passer un plat pour le moins original : un "pole dancing chicken", qu'on pourrait traduire par "poulet stripteaseur". Le plat comporte une barre verticale sur laquelle le poulet est empalé, le cou lascivement enroulé autour de la barre. De quoi donner du grain à moudre à une armée de psychanalystes !


\begin{figure}[h]
\centering
\includegraphics[height=9cm,width=12cm,keepaspectratio]{p1034544.jpg}
\caption*{Je n'ai pas de photo de poulet, et de toute façon, le karst, c'est plus joli !}
\end{figure}

Une autre tradition typiquement chinoise, c'est de commander beaucoup plus que ce qu'ils ne peuvent manger, surtout en public. C'est toujours cette histoire de face qui est en jeu, évidemment. C'est même parfois recommandé de toujours laisser un peu de nourriture dans son assiette à la fin du repas, particulièrement quand on est invité, car tout finir pourrait être interprété comme "il n'y en a pas assez", ce qui pourrait être vexant. Le gouvernement essaie de faire changer ces mentalités en instaurant une nouvelle habitude : le contrat de l'assiette vide. Moi, c'est ma maman qui m'a appris à vider mon assiette. Je me demande comment ça se serait passé si ça avait été le boulot d'un agent du gouvernement chinois. En parlant avec quelques Chinois, on se rend compte que c'est entrain de changer doucement mais il y a encore beaucoup de gaspillage : les plats au restaurant sont rarement finis, et on a même vu certains plats rester intacts sur les tables voisines...



\chapter{Lijiang et les gorges du saut du tigre}
Nous arrivâmes donc à Lijiang, dans le Yunnan. Nous sommes arrivés tard le soir à la gare, et après avoir attendu en vain un bus, on prend un taxi pour le centre-ville. Notre hôtel était déjà réservé comme d'habitude, et il se trouvait en plein dans la vieille ville. Après s'être un tout petit peu perdu dans le dédale de petites ruelles, on arrive à notre hôtel et on va directement se coucher.


\begin{figure}[h]
\centering
\includegraphics[height=6cm,width=9cm,keepaspectratio]{p1094828.jpg}
\caption*{Le centre-ville.}
\end{figure}

Le lendemain, on se balade un peu dans cette vieille ville, et on se rend compte que ce n'est qu'un grand centre commercial à ciel ouvert. C'est joli, certes, mais ça fait tellement faux ! Les mêmes boutiques se répètent tous les 50m : argenterie, babioles, djembé, vêtements, puis à nouveau argenterie, etc. De temps en temps un petit restaurant hors de prix, et puis voilà ! L'ambiance musicale est assurée par les magasins de djembés, qui passent tous les mêmes trois chansons, accompagnés à chaque fois par la vendeuse qui tape sur son instrument avec l'entrain d'un paresseux sous bêtabloquants.


\begin{figure}[h]
\centering
\includegraphics[height=6cm,width=9cm,keepaspectratio]{boutiqueslijiang.jpg}
\caption*{Ctrl+C, Ctrl+V}
\end{figure}

Et l'attraction du coin, (vous savez, là où il faut faire une photo absolument) c'est une roue à aubes : Même pas une vraie en plus, une roue à aubes de parc d'attractions, actionnée par un moteur électrique.


\begin{figure}[h]
\centering
\includegraphics[height=6cm,width=9cm,keepaspectratio]{p1094832.jpg}
\caption*{Marion fait la chinoise.}
\end{figure}

La cerise sur le pompon est atteinte quand, au moment de retourner vers l'hôtel, on nous demande les billets d'entrée. Eh ben oui, il faut payer pour entrer dans le centre-ville/centre-commercial. Et si on n'a pas payé le premier soir, c'est parce qu'on est arrivés tard par une petite ruelle. On ne se démonte pas, on fait demi-tour, et on a vite fait de trouver un passage non surveillé. Non mais, on ne va pas payer pour avoir le privilège de se balader dans un centre commercial en plastique !

On a quand même tenté une petite dégustation de thé dans une des boutiques. Ici, ils font le thé puer, qui se présente sous la forme d'un gros bloc. La cérémonie est très complexe, et nécessite une quantité impressionnante de vaisselle. Ils ont même une bouilloire et un robinet automatique : un bouton, et le robinet se place tout seul au dessus de la bouilloire, la rempli, et l'eau est chauffée à la température idéale. La tasse pour infuser, la petite théière pour servir, et les petites tasses sont ensuite rincées à l'eau chaude, puis le thé est rincé plusieurs fois, et l'eau de rinçage du thé est utilisée pour  rincer les tasse et la théière, et enfin on nous sert une toute petite tasse. On a vraiment l'impression de voir une petite fille jouer à la dinette. La première tasse bue, on nous en sert tout de suite une autre. On sent le gout du thé changer au fur et à mesure du nombre d'infusions. Boire le thé dans ces conditions est une opération à part entière : pas question de manger des petits gâteaux en même temps.


\begin{figure}[h]
\centering
\includegraphics[height=6cm,width=9cm,keepaspectratio]{p1094840.jpg}
\caption*{Dégustation de thé.}
\end{figure}

Notre hôtel aussi voulait nous vendre du rêve : International Youth Hostel qu'il s'appelle. On se dit qu'on va rencontrer des gens de partout, mais non, nous sommes les seuls touristes occidentaux, et à l'accueil, ils ne parlent pas anglais, même pas un mot, on doit communiquer à coups de traducteurs et de dessins. Ils arrivent même à nous envoyer à la mauvaise station de bus quand on veut partir pour les gorges du saut du tigre, notre étape suivante. Heureusement qu'il y avait plusieurs bus dans la journée, on a quand même pu arriver à la bonne station de bus assez tôt le matin.


\begin{figure}[h]
\centering
\includegraphics[height=6cm,width=9cm,keepaspectratio]{p1104846.jpg}
\caption*{La fin des gorges.}
\end{figure}



Nous arrivâmes donc dans les gorges du saut du tigre, après avoir traversé la ville de Lijiang en courant pour changer de station de bus. Les gorges du saut du tigre sont supposées être le canyon le plus profond du monde, mais c'est bizarre, car j'avais déjà visite le canyon le plus profond du monde au Pérou, mais bref... C'est de la montagne, c'est beau. Il y a deux chemins possibles pour visiter ce canyon : une route le long de la rivière accessible en bus, ou un vrai sentier de montagne qui passe plus haut. Devinez par où on est passés...


\begin{figure}[h]
\centering
\includegraphics[height=6cm,width=9cm,keepaspectratio]{p1104859.jpg}
\caption*{Vue de l'autre coté des gorges.}
\end{figure}



\begin{wrapfigure}{l}{0.55\textwidth}
\centering
\includegraphics[width=0.5\textwidth]{p1114984.jpg}
\caption*{Le bas des gorges, avec le chemin taillé dans la falaise.}
\end{wrapfigure}

Ça nous prend deux jours pour aller au fond des gorges, en passant par quelques petits villages dont les habitants ont vite compris que les guesthouses, c'est plus rentable que la culture du riz. On peut manger et dormir sur le sentier pour le même prix qu'en ville, ce qui surprend quand on est habitués aux tarifs des gîtes de montagne européens. Les paysages sont magnifiques, et on sympathise rapidement avec quelques autres marcheurs à force de les croiser sur le chemin et dans les guesthouses.
A la fin du chemin, on redescend le long de la rivière, et on atteint l'endroit où le tigre à sauté par dessus la rivière comme le dit la légende. Et ben ce sont de sacrés tigres qui vivaient là-bas !




\chapter{Yuanyang, là où Dieu pleure pour les photographes.}
Nous arrivâmes donc à Yuanyang. Là encore, il y a un piège : ça correspond à la fois à toute une zone, et à un village en particulier. Mais ce n'est pas tout : le village en particulier est en deux parties, la nouvelle ville et la vieille ville qui sont quand même à 30 km de distance. Et ce sont trente km de lacets dans la montagne. Encore une fois, il faut bien savoir où on veut aller, et nous, on voulait aller chez Jacky, au milieu des rizières. Le coin est connu pour avoir les plus belles rizières du monde. Jacky est retourné dans son village pour ouvrir cette guesthouse après avoir voyagé dans le monde entier comme photographe professionnel. Elle est magnifiquement décorée par ses photos des rizières. Il dit que Dieu pleure pour les photographes qui ne sont jamais venus à Yuanyang. On ne peut qu'acquiescer.


\begin{figure}[h]
\centering
\includegraphics[height=6cm,width=9cm,keepaspectratio]{p1135531.jpg}
\caption*{On commence en douceur.}
\end{figure}

J'ai aussi envie de vous parler un peu de l'envers du décors : Quand on voit ces magnifiques photos, on imagine souvent que c'est un endroit paisible, qu'on est seul au paradis, qu'il a fallut marcher longtemps, que ce fut épuisant de se lever si tôt pour avoir le temps de monter au sommet avant le lever du soleil. Ce n'est pas exactement comme ça ici : on peut acheter un ticket qui donne accès à plusieurs plateformes, certaines dédiées au lever, d'autres au coucher de soleil. Il faut venir tôt si on veut une bonne place pour poser son trépied.


\begin{figure}[h]
\centering
\includegraphics[height=6cm,width=9cm,keepaspectratio]{p1135614.jpg}
\caption*{Il y en a toujours un qui essaie de faire son intéressant...}
\end{figure}

Quand le lever commence, il doit y avoir 10 000 \euro de matos photo pour chaque mètre de rambarde. C'est à se demander quel est l'intérêt de prendre des photos si c'est pour avoir les mêmes que tout le monde. Mais on oublie vite tout ça devant la magnificence des paysages, et on se laisse absorber par le jeu de miroirs dans les rizières, les vagues dans les brumes, et les couleurs du lever de soleil. Voici une maigre tentative de rendre compte de l'ambiance :


\begin{figure}[h]
\centering
\includegraphics[height=6cm,width=9cm,keepaspectratio]{p1135514.jpg}
\caption*{Le détail.}
\end{figure}


\begin{figure}[h]
\centering
\includegraphics[height=6cm,width=9cm,keepaspectratio]{p1135515.jpg}
\caption*{La vue d'ensemble.}
\end{figure}


\begin{figure}[h]
\centering
\includegraphics[height=6cm,width=9cm,keepaspectratio]{p1135557.jpg}
\caption*{Ça marche aussi en noir et blanc.}
\end{figure}

Jacky, en plus d'être un très bon photographe, est un excellent cuisinier. Quand on mange chez lui, on a un seul choix à faire : avec ou sans viande, et ensuite, la valse des plats commence. Plus on est une grande tablée, mieux c'est, car il y a d'autant plus de plats différents à partager et à goûter. On a donc tôt fait de se faire plein de compagnons de repas. On ira jusqu'à en ramener depuis d'autres auberges. La plus grande surprise viendra d'un plat en apparence tout bête : des frites au poivre. Mais pas n'importe quel poivre ! Du poivre du Sichuan. J'avais déjà entendu parlé de ça, et je pensais que c'était du poivre normal qui poussait au Sichuan. Que nenni ! C'est une autre plante, dont l'intérêt est qu'elle anesthésie la bouche. La sensation est assez difficile à expliquer, ce n'est pas vraiment comme chez le dentiste, car on a encore quelques sensations en bouche, mais la sensation est plus diffuse, accompagnée de petits picotements et de l'impression étrange que cette langue n'a rien à faire ici.


\begin{figure}[h]
\centering
\includegraphics[height=5.5cm,width=9cm,keepaspectratio]{p1135541.jpg}
\caption*{Un futur ingrédient de la cuisine de Jacky.}
\end{figure}

On fait également un tour au marché, où on se frotte à la photographie de rue. On aimerait bien prendre des photos qui retranscrivent l'ambiance, mais on a l'impression d'être des voleurs si on ne demande pas l'accord d'abord. Mais si on demande l'accord, soit les gens disent non, soit ils demandent de l'argent, soit ils perdent totalement leur naturel. C'est un peu compliqué. On achète quand même des sauterelles, parce qu'on est curieux et que ça nous permet de faire des photos de la vendeuse. Mais au moment de gouter les sauterelles, on se rend compte qu'elles ne sont pas tout à fait mortes : dès qu'on en prend une dans la main, elle bouge un peu ! Ça nous a coupé l'appétit pour le coup.


\begin{figure}[h]
\centering
\includegraphics[height=5.5cm,width=9cm,keepaspectratio]{p1145820.jpg}
\caption*{En effet, c'est une perceuse.}
\end{figure}


\begin{figure}[h]
\centering
\includegraphics[height=5.5cm,width=9cm,keepaspectratio]{p1145828.jpg}
\caption*{Cuit ou cru, telle est la question.}
\end{figure}




\begin{figure}[h]
\centering
\includegraphics[height=5.5cm,width=9cm,keepaspectratio]{p1145846.jpg}
\caption*{Pour les moins de 20 ans : ceci est une télé !}
\end{figure}


\begin{figure}[h]
\centering
\includegraphics[height=5.5cm,width=9cm,keepaspectratio]{p1145835.jpg}
\caption*{Au marché, les hommes aussi font leur part !}
\end{figure}

On part ensuite pour le Vietnam. Nous faisons 200 km en six heures de bus. C'est lent. Et pour nous tenir compagnie dans le bus, une Israélienne et sont fils avec lesquels nous avions déjà sympathisé chez Jacky (les repas, souvenez-vous). Le trajet est lent, et on fait plusieurs pauses, dont une pour nettoyer le bus et refroidir les freins au jet d'eau, à mi-parcours d'une grande descente, c'est rassurant... Le seul intérêt de ce voyage : lors d'une pause, on achète des glaces un peu au hasard, et on tombe sur une glace... aux petit-pois !


\begin{figure}[h]
\centering
\includegraphics[height=5.5cm,width=9cm,keepaspectratio]{p1145901.jpg}
\caption*{Les rizières, c'est aussi ça.}
\end{figure}

\begin{figure}[h]
\centering
\includegraphics[height=6cm,width=9cm,keepaspectratio]{p1145904.jpg}
\caption*{Je vous jure, ce sont les vraies couleurs !}
\end{figure}

\begin{figure}[h]
\centering
\includegraphics[height=8cm,width=12cm,keepaspectratio]{p1135635.jpg}
\caption*{Une petite dernière pour la route !}
\end{figure}



\chapter{Sapa, premier aperçu du Vietnam.}
Nous arrivâmes donc au Vietnam. Le passage de la frontière est assez rapide, et nous voilà dans la rue, à quelques kilomètres de la station de bus qu'on doit rejoindre si on veut pouvoir aller à Sapa, notre prochaine étape. Et on se sent tout de suite très oppressés : des vendeurs nous collent aux basques d'une manière qui serait vraiment très impolie en France. Ils nous suivent d'un distributeur à l'autre, nous attendent à la sortie du magasin où on va faire un peu de monnaie, et se penchent sans aucune gêne sur notre épaule pour tenter de voir ce qu'on tripote sur le téléphone. Et ceci couplé à tous les avertissements qu'on a lus à propos des arnaques au Vietnam nous met un peu sous pression. Ça ne s'arrange pas quand on essaie de négocier un taxi, et qu'ils nous proposent tous un prix au moins deux fois supérieur à ce qui devrait être le prix normal d'après de multiples sources concordantes sur internet. Alors oui, on parle de payer 1,5\euro au lieu de 75 centimes pour faire 3km, mais pour le principe, on refuse et on part à pied. On finit par trouver un taxi avec des tarifs normaux quelques centaines de mètres plus loin, et on monte rapidement dans le bus pour Sapa.


\begin{figure}[h]
\centering
\includegraphics[height=6cm,width=9cm,keepaspectratio]{p1185946.jpg}
\caption*{Vue depuis Sapa.}
\end{figure}

Dans ce bus, on assiste à ce qui nous semble être du racisme : un groupe de jeunes femmes d'une minorité ethnique entre dans le bus avec des gros paniers, et semblent revenir du marché. Pendant le trajet, un panier tombe contre la porte du bus un peu avant que le chauffeur n'ouvre la porte, ce qui coince le panier derrière la porte. Plutôt que de fermer la porte pour décoincer le panier, le chauffeur a plutôt regardé la jeune femme batailler, jusqu'à ce qu'un autre homme vienne le lui déchirer pour le décoincer, et même si on ne comprenait pas le viet, on a eu la sérieuse impression que ça faisait bien marrer les mecs dans le bus...


\begin{figure}[h]
\centering
\includegraphics[height=6cm,width=9cm,keepaspectratio]{p1175938.jpg}
\caption*{On l'a trouvé perdu dans la rue !}
\end{figure}

Et c'est sur ces bonnes impressions qu'on arrive à Sapa. C'est une ville construite par les Français dans la montagne, pour fuir la chaleur de Hanoi. A ce qu'il parait, la ville a eu un air de village alpin. Mais actuellement, c'est une usine à touristes. Les magasins de contrefaçons Northface succèdent aux restaurants qui proposent tous les mêmes menus. Les motos-taxis nous hèlent dès qu'on fait mine de vouloir aller quelque part à pied, pour laisser la place rapidement à des femmes des minorités qui viennent nous accoster avec un grand sourire et un excellent niveau d'anglais, voire de français, et qui essaient de nous vendre leur artisanat, ou de se vendre comme guide pour nous accompagner dans leur village. Car c'est ça en fin de compte le vrai intérêt de Sapa, c'est ce qu'il y a autour : la montagne, et les villages des minorités.



\begin{wrapfigure}{l}{0.45\textwidth}
\centering
\includegraphics[width=0.4\textwidth]{p1185986.jpg}
\caption*{Le genre de petite fille qu'on laisse tranquille à la récré !}
\end{wrapfigure}


D'ailleurs, c'est ce qu'on va essayer de faire rapidement. On achète une carte de rando à l'office du tourisme, on se prépare une petite boucle pour la journée, et on part plein d'entrain. La première embûche nous tombe dessus très vite : on doit payer pour traverser le premier village si on veut atteindre le fond de la vallée. Le village en question est de nouveau à mi-chemin entre Disneyland et un centre commercial.

\begin{figure}[h]
\centering
\includegraphics[height=6cm,width=9cm,keepaspectratio]{p1175937.jpg}
\caption*{Ça fait bizarre quand on tombe dessus en se baladant dans le marché... C'est une viande recommandée pour les femmes enceinte !}
\end{figure}

Hors de question de payer pour ça, on va bien trouver un moyen de le contourner. On prend donc un autre chemin, et un peu plus tard, on croise d'autres touristes qui remontent. Quand on leur demande d'où ils viennent, on apprend qu'ils voulaient faire la même chose que nous, et qu'ils se sont fait refouler une deuxième fois plus bas, et qu'il n'y a pas moyen de passer sans acheter le ticket d'entrée. On finit donc pas céder tous les quatre, on traverse ce village, et nous voilà désormais à 4 cerveaux à essayer de lire cette carte. Notre GPS et toute notre expérience de rando en France n'y fait rien, on ne comprend rien à la carte, et on finit par se balader au hasard en suivant des chemins qui nous inspirent. On apprendra un peu plus tard que la carte serait volontairement faussée pour encourager l'embauche de guides. On finit par faire une chouette balade dans la forêt en compagnie de ce couple de Français qui nous racontent leur expérience au tribunal international des Kmers Rouges au Cambodge où ils viennent de bosser quelques mois. C'est un petit avant-goût de notre prochaine destination !


Marion, ne supportant pas tout ces décors en cartons, trouve les coordonnées d'un guide français, Olivier, et on part trois jours avec lui et sa famille. La première nuit on dort chez lui : il est marié avec une Dao, une des minorité ethnique du Vietnam. Ils vivent à Ta Phin, un petit village Dao traditionnel, avec leurs enfants. C'est à 2 heures de marche de la route, il n'y a ni électricité ni eau courante, mais le téléphone mobile passe ! L'eau est fournie par un petit ruisseau, et l'électricité fournie par la petite turbine en aval du même ruisseau permet d'alimenter deux ampoules et de recharger le téléphone portable. La maison en elle même ressemble plutôt à une grange. Le sol est en terre battue et il y a des animaux partout. C'est d'ailleurs l'impression que cet endroit nous laissera : ça grouille de vie. Tous les animaux qu'on croise sont accompagnés de leurs petits : les buffles et leurs petits veaux/bufflons, les porcs et les porcelets, les chiens et leurs chiots, les poules et leur innombrable marmaille piaillante. Tout ce petit monde semble vivre en harmonie.


\begin{figure}[h]
\centering
\includegraphics[height=6cm,width=9cm,keepaspectratio]{p1216681.jpg}
\caption*{Pièce principale.}
\end{figure}


\begin{figure}[h]
\centering
\includegraphics[height=6cm,width=9cm,keepaspectratio]{p1206477.jpg}
\caption*{Olivier, sa femme, son fils.}
\end{figure}

Plus tard dans l'après midi, on continue tout seul l'ascension de la colline qui surplombe le village, car il y a une chance qu'on arrive à traverser la couche de nuage dans laquelle on baigne depuis le matin. Pleins d'espoir, on monte, et on monte, et on monte encore dans la brume (ça vous rappelle une histoire ?). On arrive au sommet, toujours dans le brouillard. Ce n'est pas la première fois que ça nous arrive, on ne se démonte pas pour autant et on commence à faire des photos à la con, genre selfie faussement épaté par le paysage inexistant. Là, le miracle se produit. La couche de nuage semble s'affiner, on commence à sentir la chaleur du soleil, et soudain, les nuages se déchirent pour laisser apparaitre quelques instants une mer de nuages, avant de nouveau nous recouvrir. Puis à nouveau le ciel se déchire, et ça recommence : nous étions en fait tout pile à la limite des nuages, et les vagues de brumes nous passaient dessus au ralenti. Magique !


\begin{figure}[h]
\centering
\includegraphics[height=6cm,width=9cm,keepaspectratio]{p1196007-panorama.jpg}
\caption*{D'un coup, on a vu ça !}
\end{figure}

De retour chez Olivier, un bain nous attendait. Mais alors il faut se rendre compte de la quantité de travail pour faire ce bain : il faut aller couper du bois pour alimenter le feu qui va chauffer l'eau. Et ce n'est pas un simple bain, c'est un bain aux herbes traditionnel dao. Donc il faut aussi ajouter la corvée de cueillette des herbes ! Puis il faut remplir la baignoire, qui est un tonneau en bois, avec une bassine, sans se brûler de préférence, et ajuster la température avec de l'eau froide. Et enfin on peut aller macérer dans l'eau chaude parfumée.


\begin{figure}[h]
\centering
\includegraphics[height=6cm,width=9cm,keepaspectratio]{p1196463.jpg}
\caption*{L'eau du bain aux herbes en train de chauffer.}
\end{figure}

Olivier nous parle aussi de l'agriculture : ils dépendent de la culture du riz, et ici dans les montagnes, c'est dur ! Ils ne peuvent faire qu'une seule récolte par an, contre deux dans la vallée et même trois dans le delta du Mékong. Étant donné la configuration des rizières, pas de tracteur, tout est fait à la main, où à la rigueur avec un petit motoculteur qui commence à remplacer le buffle pour le labourage. Mais Montsanto est quand même présent : c'est cette entreprise qui fourni les graines et les produits chimiques de manière quasi obligatoire. Les paysans qui refusent et replantent leurs graines d'une année sur l'autre ont des amendes pour diverses raisons (corruption des dirigeants locaux, assimilation des gènes depuis les champs voisins qui sert de prétexte aux poursuites). C'est à tel point que les Dao ne savent plus, ou ne veulent plus faire sans la chimie. Olivier et sa femme ont chacun leur jardin potager : celui de sa femme, cultivé de manière "traditionnelle" avec des intrants, et celui d'Olivier, cultivé avec des engrais naturels, car il veut lui montrer que c'est possible de se passer en grande partie de chimie.


\begin{figure}[h]
\centering
\includegraphics[height=6cm,width=9cm,keepaspectratio]{p1196466.jpg}
\caption*{L'inévitable atelier nems !}
\end{figure}

Le lendemain, on passe la journée avec la belle-mère d'Olivier. Une toute petite Dao avec un sourire éclatant, et une pêche incroyable. J'aurais pu dire "une pêche incroyable pour son âge", car elle a déjà quelques années, elle est plusieurs fois grand-mère, mais non, elle a une pêche tout court : on avait rendez-vous à 9h en bas du village d'Olivier pour qu'elle nous amène chez elle, à trois heures de marche. Et elle était venue à pied : oui, 6h de marche dans la montagne à l'aube pour accompagner deux touristes ! Et arrivé chez elle, on est accueilli par le reste de la famille (fils, belles-filles, petits enfants) et l'odeur de l'alcool de riz en train d'être distillé dans la pièce principale. Les vapeurs nous enivrant tous, l'accueil a été très joyeux ! Pour ce qui est de la marche, ce n'était pas fini, on est allé dans la montagne avec elle pour faire la cueillette de plantes pour le bain du soir. Puis on est rentré, elle nous a dit de nous reposer pendant qu'elle repartait encore crapahuter car elle n'avait pas trouvé toutes les plantes qu'elle voulait ! Et pendant les temps morts, elle s'occupait de l'alambic. Nouvel an approchant, il fallait faire les réserves !


\begin{figure}[h]
\centering
\includegraphics[height=6cm,width=9cm,keepaspectratio]{p1206522.jpg}
\caption*{Parée pour la cueillette !}
\end{figure}


\begin{figure}[h]
\centering
\includegraphics[height=6cm,width=9cm,keepaspectratio]{p1206506.jpg}
\caption*{Rechargement de l'alambic avec du riz fermenté.}
\end{figure}

On a aussi pu visiter leur jardin : ils cultivent des orchidées spécialement pour le têt, le nouvel an vietnamien. Ces orchidées sont très populaires à cette période, et sur les routes, on voit beaucoup de motos transporter ces énormes pots de fleurs. Elle nous explique que ces fleurs mettent 3 ans à pousser, elles sont vendues pour le nouvel an, et elles meurent ensuite souvent rapidement.


\begin{figure}[h]
\centering
\includegraphics[height=6cm,width=9cm,keepaspectratio]{p1206509.jpg}
\caption*{Les toilettes au milieu des orchidées.}
\end{figure}

Les repas sont délicieux. La table est couverte de petits plats dans lesquels chacun se sert. Et pour la boisson, c'est aussi le partage qui prime : interdiction de siroter son alcool de riz dans son coin. Soit on trinque avec tout le monde, soit on ne boit pas ! Et leur alcool, il n'y a pas a dire, "\emph{ça reste une boisson d'homme"} : pur à la sortie de l'alambic, distillé le jour même, ça doit bien titrer dans les 70\textdegree . On l'a bien senti passer ! Alors qu'on venait de finir le repas du soir, et qu'on continuait frénétiquement à trinquer, on entend soudain une moto s'approcher. Et là, ce fut la panique. La belle mère nous fait signe de venir avec elle, et on part se cacher dans une chambre, en silence, tandis qu'on entend ses enfants parler avec quelqu'un. On finit par sortir un peu plus tard, pour apprendre qu'en fait, elle n'a pas encore officiellement le droit d'accueillir des touristes chez elle, et elle avait peur que ce visiteur inattendu puisse être la police ou un voisin curieux et dénonciateur. Mais non, c'était juste un voisin perdu. Apparemment, le suivi des touristes est très sérieux au Vietnam : seuls les endroits autorisés peuvent accueillir des touristes, et souvent, seulement des touristes. Il existe donc des hôtels pour Vietnamiens, et des hôtels pour étrangers, ce qui peut poser problème aux couples mixtes face à un employé d'hôtel bureaucratiquement zélé.


\begin{figure}[h]
\centering
\includegraphics[height=6cm,width=9cm,keepaspectratio]{p1216693.jpg}
\caption*{Petite séance photo !}
\end{figure}

On s'est fait un petit échange de cadeaux à la fin : avec les moyens du bord, on a fait un petit mobile en origami pour le dernier petit bébé de la maison, arrivé fraichement dans la semaine. En fin de compte, c'est son grand frère que ça a vraiment intéressé. Et Marion a eu droit à un vrai foulard Dao ! C'est le genre de truc qu'on ne voit pas habituellement sur les têtes des touristes, et beaucoup de Vietnamiens nous ont posés des questions sur l'origine du foulard une fois revenus en ville. On a l'impression d'avoir eu un cadeau plutôt rare !


\begin{figure}[h]
\centering
\includegraphics[height=6cm,width=9cm,keepaspectratio]{p1216686.jpg}
\caption*{Le grand frère est convaincu !}
\end{figure}


\begin{figure}[h]
\centering
\includegraphics[height=6cm,width=9cm,keepaspectratio]{p1216709.jpg}
\caption*{Marion avec son nouveau foulard.}
\end{figure}

\chapter{Ba Be lake, le paradis, ou presque...}
Nous partîmes donc pour Ba Be Lake. En bus de nuit ! Notre premier bus de nuit en Asie. Les premières impressions ont été mitigées : un petit van avec des sièges confortables et légèrement inclinables, mais on était serrés dedans. Mais en fait non ! C'était juste une navette qui nous amenait au point de rendez-vous avec le vrai bus. Le point de rendez-vous, c'était sous un pont pas éclairé, et il faisait nuit. Le genre d'endroit où on attend plutôt son dealer, mais non, nous, on attendait notre bus ! Et le vrai bus était nettement mieux : de vraies couchettes dans lesquelles on est quasiment allongés. Bon, ce bus étant au format local, faut pas s'attendre à pouvoir déplier les jambes si on fait plus de 160cm, mais il est néanmoins possible de dormir pour de vrai.


\begin{figure}[h]
\centering
\includegraphics[height=6cm,width=9cm,keepaspectratio]{p1246786.jpg}
\caption*{Au bord du lac.}
\end{figure}

Et nous arrivons donc le matin dans une autre ville, on reprend un bus pour quelques heures, puis on fait 1h de taxi, et enfin 20mn de marche. Disons qu'on ne passe pas là-bas par hasard. Mais une fois sur place, ça présente de fortes similitudes avec ce que d'aucuns appelleraient "le paradis sur Terre". Un grand lac vert émeraude, bordés d'arbres immenses, les rares endroits plats sont des rizières qui alimentent un petit village paisible. Plus de klaxons, plus de rabatteurs, plus de bus. Le calme. Et en plus, il y fait bon toute l'année. Enfin... la plupart de l'année. Nous, on y est resté juste le temps de profiter de la super vague de froid descendue de Chine. Il a fait 4\textdegree C. Alors, vous allez me dire, pour des gars qui on fait du -35\textdegree C quelques semaines avant en Mongolie, 4\textdegree C, c'est du pipi de chat ? Non ? Et ben en fait, pas vraiment. Tout d'abord, l'humidité : 100\%, ou presque. Ça donne un froid qui rentre partout, sous les vêtements, et il n'y a pas de soleil, tout est toujours brumeux, rien ne sèche ! Ensuite, il faut savoir que les bâtiments sont conçus pour des températures oscillant entre 20\textdegree C et 35\textdegree C, donc il y a des trous partout, entre les planches du mur, sous la porte, entre les lattes du plancher, et il fait donc 4\textdegree C jusque dans la chambre, sous la couette. On avait de la chance d'avoir une chambre avec un climatiseur qui pouvait aussi faire plus ou moins radiateur, mais sinon, aucun chauffage n'est prévu ! De même, la salle à manger est dehors, les plats sont froids en 2mn, impossible de se réchauffer en mangeant ! Les locaux n'ont même pas de pull à mettre, on les voyait greloter toute la journée, alors que nous avions ressorti la doudoune ! Ils en ont été réduits à faire un feu dans une bassine en métal pour tenter de se réchauffer, ce qui a surtout eu pour effet d'enfumer notre chambre juste au-dessus (les trous entre les lattes, souvenez-vous). Croyez le ou non : on a eu plus froid ici qu'en  !




\begin{figure}[h]
\centering
\includegraphics[height=6cm,width=9cm,keepaspectratio]{p1246745.jpg}
\caption*{Petite balade en bateau.}
\end{figure}

Ça ne nous a pas empêché de nous balader un peu. On a vu quelques grottes absolument magnifiques. J'avais l'impression qu'à tout moment, Indiana Jones allait sortir en courant, pourchassé par une armée de sauvages, mais après un peu d'exploration, on se rend compte qu'il y a beaucoup plus de chauve-souris que d'archéologues kleptomanes dans ces grottes.


\begin{figure}[h]
\centering
\includegraphics[height=6cm,width=9cm,keepaspectratio]{p1246804-panorama.jpg}
\caption*{Pour ne rien gâcher, nous étions seuls dans la grotte !}
\end{figure}

A certains endroits dans la forêt, on se serait crus dans chérie j'ai rétréci les gosses, tellement les plantes, les arbres, et parfois les insectes étaient gros ! Et à propos d'insecte, ou plutôt, d'arthropode, voici la photo d'un centipède (comme son nom l'indique, ça a beaucoup trop de pattes pour être un insecte) qui squattait juste à coté de l'interrupteur de la salle de bain. On a beau savoir que ça ne pique pas, ça ne mord pas, ça ne vole pas, et que ça bouge très lentement quand il fait froid, ça fait quand même bizarre de tomber dessus au milieu de la nuit.


\begin{figure}[h]
\centering
\includegraphics[width=0.5\textwidth]{p1236728.jpg}
\caption*{Il me regarde, non ?}
\end{figure}


\begin{wrapfigure}{l}{0.55\textwidth}
\centering
\includegraphics[height=6cm,width=9cm,keepaspectratio]{p1236717.jpg}
\caption*{Enfin une feuille de la bonne taille !}
\end{wrapfigure}

On a aussi fait une balade en bateau jusqu'à l'autre bout du lac, et on a pu visiter la fameuse île de la veuve. Il y a une légende qui dit qu'avant, ce lac n'existait pas, c'était une plaine. Un jour, alors que personne ne demandait rien, une vieille mendiante est apparue, et elle a commencé à faire ce que toutes les mendiantes font : demander de la bouffe ! Évidemment, la plupart des gens n'étaient pas dupes, et avaient bien vu que c'était du flan cette histoire de mendiante, et ne lui donnaient rien. Jusqu'à ce que la mendiante arrive à la maison de cette veuve, qui, probablement accablée de chagrin et surchargée de boulot, lui donna un peu à manger, histoire de s'en débarrasser rapidement. Et là, grosse surprise, on apprend que la mendiante est en fait une fée ! Oh, ben ça alors, tu parles d'une surprise ! Et que fait la fée pour remercier la veuve ? Elle lui dit : "Hop, monte dans ton panier que je viens de transformer en barque, et accroche-toi, je vais faire un rien de terrassement". Sur ces paroles, elle transforme la plaine en lac entouré de montagnes, et la maison de la veuve se retrouve isolée sur une minuscule île pleine de cailloux. Sympa la vieille !


\begin{figure}[h]
\centering
\includegraphics[height=9cm,width=9cm,keepaspectratio]{p1256818.jpg}
\caption*{Là, c'est trop grand !}
\end{figure}

\begin{figure}[h]
\centering
\includegraphics[height=9cm,width=9cm,keepaspectratio]{p1246792.jpg}
\caption*{Sur l'île de la veuve. Cosy, non ?}
\end{figure}

J'imagine que la morale de cette histoire c'est : il ne faut pas aider les vieilles un peu louches, sinon, on va perdre ses champs et se retrouver seul au monde ? Bref, je ne suis pas sûr d'avoir bien compris cette histoire, mais ils en ont l'air très fier là-bas. Si vous avez une meilleure idée sur ce qu'il y a à comprendre, n'hésitez pas !


\begin{figure}[h]
\centering
\includegraphics[height=9cm,width=9cm,keepaspectratio]{p1236723.jpg}
\caption*{Un nénufar ? (comment convertir un innocent article de voyage en article politiquement engagé en un seul mot)}
\end{figure}


\begin{figure}[h]
\centering
\includegraphics[height=9cm,width=9cm,keepaspectratio]{p1256823.jpg}
\caption*{S'il y a un biologiste qui me lit, je veux bien savoir ce que c'est !}
\end{figure}



\chapter{Hanoï}
Nous arrivâmes donc à Hanoï, la capitale du Vietnam. Et non, aucune trace de moines qui passent leur temps à empiler des disques sur des tours, c'est une légende inventée de toutes pièces ! C'est la deuxième plus grande ville du Vietnam, donc ça nous fait l'impression d'un village chinois en terme de taille. Et comme on reste dans le vieux centre, on a presque l'impression d'être en France ! Il y a des petites ruelles fleuries, des boulangeries avec des vraies baguettes et des petits pains, ce qui nous permet de nous faire un petit repas nostalgie avec des goûts qu'on n'avait plus eu depuis longtemps. En revanche, niveau circulation, pas de doute, on est bien au Vietnam.


\begin{figure}[h]
\centering
\includegraphics[height=6cm,width=9cm,keepaspectratio]{p1276839.jpg}
\caption*{Et pour ne rien arranger, on a l'impression que certains conducteurs ne voient vraiment rien.}
\end{figure}

On progresse encore d'un cran dans l'anarchie du trafic, avec un flux ininterrompu de scooters qui prend d'assaut toutes les rues. On doit re-apprendre à traverser en ignorant tous ces signaux d'avertissement que nos parents se sont échinés à mettre en place dès notre plus tendre enfance. Ici, si on attend que les véhicules s'arrêtent pour nous laisser traverser, on risque fort d'y passer la journée. Il faut avancer doucement, sans changer de vitesse ni de direction, et faire confiance aux scooters pour nous éviter et ça marche !


\begin{figure}[h]
\centering
\includegraphics[height=6cm,width=9cm,keepaspectratio]{p1276835.jpg}
\caption*{Et si on en a marre des scooters, il y a cette rue...}
\end{figure}

Une particularité des bâtiments du Vietnam commence à nous sauter franchement aux yeux : tous les bâtiments sont très étroits. Notre hôtel par exemple, ne doit faire qu'environ 4m de large, et c'est le cas de la plus grande partie des bâtiments à Hanoï et dans le Vietnam. Après enquête, il s'avère qu'il y a une taxe qui dépend de la largeur. Ça me fait penser au Pérou, ou la plupart des maisons ont un étage en construction, car les impôts commencent uniquement quand les travaux sont finis. C'est fou les conséquences d'une loi mal faite...


\begin{wrapfigure}{l}{0.55\textwidth}
\centering
\includegraphics[width=0.5\textwidth]{p1286874.jpg}
\caption*{Exemple typique d'optimisation fiscale.}
\end{wrapfigure}

On va visiter comme il se doit le mémorial d'Ho Chi Minh, ou, comme ils l'appellent là-bas : "Oncle Ho". C'est un peu le Genghis Khan vietnamien : il a bouté les américains hors du pays, réunifié le Nord et le Sud, il apparait sur tous les billets, toutes les villes ont baptisé leur plus belle avenue en son honneur, et chaque ville se doit également d'avoir un mémorial qui lui est dédié. Celui de Hanoï est bien sûr le plus grand : en plus d'être construit autour de sa dernière résidence (il s'est fait construire une maison traditionnelle de pêcheur, sur pilotis, pour bien montrer son origine populaire), on peut aussi voir le mausolée qui, dans la tradition communiste la plus pure, abrite son corps momifié, contre sa volonté bien entendu. Évidemment, on s'est levé trop tard pour voir son corps, qui n'est visible que le matin. Quel dommage...


\begin{figure}[h]
\centering
\includegraphics[height=6cm,width=9cm,keepaspectratio]{p1276860.jpg}
\caption*{PEUGEOT 404 : PRESIDENT NOT FOUND}
\end{figure}

En fait, ce qui nous plait le plus à Hanoï, ce sont les restaurants. On commence à découvrir vraiment la richesse de la gastronomie vietnamienne, et vu le prix des restaurants, on ne se prive pas ! Il y a les classiques : les nems de toutes sortes, frits ou frais, le pho (à prononcer feu, comme pot au feu, héritage des Français), et tout plein de plats avec plein d'épices : citronnelle, gingembre, coriandre, piment... Plein de saveurs, beaucoup de produits frais, des mélanges surprenants pour nous, bref, on s'éclate !


\begin{figure}[h]
\centering
\includegraphics[height=6cm,width=9cm,keepaspectratio]{p1286867.jpg}
\caption*{Je ne sais plus ce que c'était, mais qu'est-ce que c'était bon !}
\end{figure}


\begin{figure}[h]
\centering
\includegraphics[height=6cm,width=9cm,keepaspectratio]{p1286888.jpg}
\caption*{Hanoï de nuit.}
\end{figure}



\chapter{La baie d'Halong, des cailloux, de l'eau et des calamars.}
Nous partîmes donc pour la baie d'Halong, un des incontournables du Vietnam, et ce d'autant plus que depuis quelques années, elle fait partie du patrimoine mondiale de l'UNESCO. Comme la grande majorité des gens, on prend un tour organisé au départ de Hanoï, pour deux jours et une nuit sur un bateau.


\begin{figure}[h]
\centering
\includegraphics[height=6cm,width=9cm,keepaspectratio]{p1297269-panorama.jpg}
\caption*{La baie d'Halong.}
\end{figure}


\begin{wrapfigure}{l}{0.45\textwidth}
\centering
\includegraphics[width=0.4\textwidth]{p1297785.jpg}
\caption*{Un calamar curieux.}
\end{wrapfigure}

Même le Lonely Planet, chantre du voyage individuel, recommande de prendre un tour plutôt que de tenter d'y aller tout seul. On avait le choix entre 3 gammes de tarifs : pour 45\euro, les repas sont à peine mangeables, les activités promises ne sont pas faites, et les guides sont désagréables. Pour 200\euro, c'est repas gastronomique, cours de Tai-Chi le matin, chambres climatisées, et chaises longues avec un matelas sur le toit du bateau. Nous on a pris le forfait à 130\euro, (en fait 100\euro, merci Marion), et on a eu des repas très bons, une chambre avec salle de bain privée, et toutes les activités promises ont été faites : on a visité une caverne, on a fait du kayak même si il a fallut se lever tôt.



A la fin du repas du soir, un des cuisinier est venu nous montrer les calamars qu'il venait de pêcher, c'était le signal pour aller tenter ça nous-même : on met un grosse lumière à l'avant du bateau, et on agite un appât en plastique vert garni d'hameçons à l'aplomb de la lumière. Le calamar, animal curieux et naïf s'il en est, s'en approche alors  histoire d'en savoir un peu plus. Le pêcheur attentif, aux réflexes intactes malgré la bière, peut alors tirer sur sa ligne d'un coup sec, espérant ainsi embrocher le calamars qui verra sa curiosité récompensée et découvrira un monde nouveau, à base de citronnelle et d'huile chaude. Enfin, ça c'est la théorie. Mais quand même, que d'émotions quand un calamar s'approche de l'appât, puis s’enfuit à toute vitesse  en lâchant un nuage d'encre, alors que, les yeux rivés sur le calamar fuyard, on prie pour que les hameçons retombent dans l'eau plutôt que sur nos têtes...


\begin{figure}[h]
\centering
\includegraphics[height=6cm,width=9cm,keepaspectratio]{p1297791.jpg}
\caption*{Un Gérald patient.}
\end{figure}


\begin{figure}[h]
\centering
\includegraphics[height=6cm,width=9cm,keepaspectratio]{p1297260.jpg}
\caption*{Le guide a insisté : c'est un doigt ! Mais un doigt sacrément drôle car il continue de pouffer comme un ado à chaque visite quotidienne.}
\end{figure}


On a aussi visité une ferme de perle, dont la partie la plus intéressante était probablement l'apparition d'une énorme méduse, puis sa capture par des gens du cru. Les perles, c'était intéressant aussi hein ! On a vu toutes les étapes, de l'insertion du noyau à l'ouverture d'une huitre pour y trouver, oh surprise, une perle, en passant par la boutique de souvenirs ou je n'ai jamais vu autant de zéros sur une étiquette de prix ! Mais ça, comparé à un mec en chaussures de ville, en équilibre sur une latte, en train passer une corde autour d'une méduse de 30kg, j'avoue, j'hésite, et je ne sais plus très bien si je dois continuer la visite guidée ou rester pour voir qui va manger qui !


\begin{figure}[h]
\centering
\includegraphics[height=6cm,width=9cm,keepaspectratio]{p1307808.jpg}
\caption*{A table !}
\end{figure}




\begin{figure}[h]
\centering
\includegraphics[height=6cm,width=9cm,keepaspectratio]{p1307806.jpg}
\caption*{Oh, une perle !}
\end{figure}

Je parle des activités mais on en oublierai presque le plus important ! La baie en elle-même ! On n'est quand même pas venus là juste pour pêcher ! Alors : c'est beau. Et on a eu de la chance, car les précédents jours, les sorties en bateau étaient interdites à cause du mauvais temps. Nous, on a bénéficié d'un temps potable : très nuageux, pas un seul rayon de soleil, mais pas de pluie. Et malgré ces conditions moyennes, c'est magnifique ! On arrive même à faire abstraction du coté usine à touristes, et à oublier les déchets qui flottent un peu partout.


\begin{figure}[h]
\centering
\includegraphics[height=9cm,width=12cm,keepaspectratio]{p1297574.jpg}
\caption*{Notre bateau.}
\end{figure}

\chapter{Ninh Binh, là où les pelleteuses chassent les barques.}
Nous arrivâmes donc à Ninh Binh. Il faisait nuit, la gare était déserte, et un mec nous attendait avec un panneau à nos noms. C'était notre taxi, grande classe non (enfin, pour des backpackers) ? On avait fait une exception à nos habitudes, et réservé le trajet jusqu'à notre auberge de jeunesse. Et on a bien fait, elle est perdue au milieu des rizières, sur une route en terre sans éclairage. On a découvert les paysages le lendemain matin : c'est magnifique. Le coin ne vole pas son nom de Baie d'Halong terrestre : des pics karstiques entourés de rivières et de rizières. C'est calme, beau, et entièrement couvert de brouillard...


\begin{figure}[h]
\centering
\includegraphics[height=6cm,width=9cm,keepaspectratio]{p1318117.jpg}
\caption*{C'est le paradis des oiseaux !}
\end{figure}




\begin{figure}[h]
\centering
\includegraphics[height=6cm,width=9cm,keepaspectratio]{p2029405.jpg}
\caption*{Le bâtiment à droite caché dans les arbres, c'est notre guesthouse.}
\end{figure}

On loue des vélos sur place, et on part visiter le coin. Faut voir les vélos : ce sont des vélos de ville au format Vietnamien, donc on peut ne peut déplier les jambes qu'à moitié. Heureusement, tout est quasiment plat, et presque sans s'en rendre compte, on a fait 50km dans la journée. Vers midi, en passant le long d'une digue, deux hommes nous font signe de venir les rejoindre. Ils étaient en train de mettre une toute petite barque à l'eau, et nous invitent à monter avec eux. On leur demande combien ils veulent, et non, ils ne veulent pas d'argent. Deux inconnus dont vous ne parlez pas la langue, vous invitent à les accompagner vers une destination inconnue au milieu d'un marécage : vous faites quoi ? La bonne réponse est, "On dit gentiment non, et on se barre". Nous, confiants, on y est allés ! A quatre dans une barque pour deux, l'eau passait doucement par dessus bord, et la barque restait souvent empêtrée dans les algues, mais on a quand même fini par arriver à la pelleteuse flottante des mecs ! On a réussi à comprendre deux mots : Hollywood, et King Kong 2. Et on est allé vérifier, et oui, King Kong 2 est bien en tournage au Vietnam, et les mecs étaient super fiers à l'idée d'y contribuer. Ils sont montés sur leur pelleteuse, et nous ont donné les clés de la barque, histoire qu'on aille se balader pendant qu'ils bossent. On s'est retrouvé à ramer dans un marécage, poursuivi par une pelleteuse qui élargissait le passage et nous rattrapait petit à petit. On se serait cru dans un film d'action, même si je n'ai jamais vu une poursuite entre une pelleteuse flottante et une barque.


\begin{figure}[h]
\centering
\includegraphics[height=6cm,width=9cm,keepaspectratio]{p1318127.jpg}
\caption*{Fuyez, pauvres fous !}
\end{figure}


\begin{figure}[h]
\centering
\includegraphics[height=6cm,width=9cm,keepaspectratio]{p1318130.jpg}
\caption*{Comme on était trop mauvais pour ramer, il nous a ramené, mais en cassant une rame au passage.}
\end{figure}

On est aussi allé faire un tour en bateau plus classique, dans un circuit bien organisé, avec achat de billets, et portique d'entrée. Juste après le portique d'entrée, on arrive sur le quai, où on voit 20 bateaux et leurs rameuses qui attendent sagement, mais on nous fait signe d'attendre car ce sont des bateaux de 4 places. Après avoir attendu 30mn et vu plusieurs groupes de 4 déjà formés nous passer devant, nous finissons par aller demander si on ne pourrait pas partir quand même, et ils acceptent ! La balade est plutôt sympa, nous sommes deux dans une barque pour 4 personnes, et comme pour tous les bateaux dans le coin, c'est une femme qui rame. Elle ne parle pas anglais, mais elle semble quand même contente et souris chaque fois qu'on est ébahi par un canard, une fleur ou un caillou. Mais ce qu'on préfère, c'est traverser les grottes creusées par la rivière à travers les pics. Notre rameuse se transforme en pilote pour éviter les nombreuses stalactites, et on doit souvent se coucher dans le bateau quand le plafond baisse un peu trop. Ravis de la balade, on sort du bateau, et on lui donne le pourboire recommandé. Et là, elle fait la gueule, et commence à parler en viet avec les autres gens, l'air visiblement énervée. On s'est renseigné un peu plus tard, et on apprend qu'on aurait du donner un double pourboire vu qu'on était que deux dans le bateau... Ce fonctionnement m'énerve sur tellement de niveaux, que je ne sais pas par où commencer ! La prochaine fois, s'ils veulent des trucs, ils n'ont qu'à l'expliquer. C'est dommage, à part ça, la balade était sympa.


\begin{figure}[h]
\centering
\includegraphics[height=6cm,width=9cm,keepaspectratio]{p2018478.jpg}
\caption*{Un échantillon des rameuses désœuvrées. La file de bateaux continue, il y en a au moins 500 (cinq cents, ce n'est pas une faute de frappe).}
\end{figure}


\begin{figure}[h]
\centering
\includegraphics[height=6cm,width=9cm,keepaspectratio]{p2019391.jpg}
\caption*{Petit temple au milieu de l'eau.}
\end{figure}

Le soir, on discute un peu avec le propriétaire de l'auberge et on apprend qu'il a commencé à mettre en place des cours d'anglais gratuits pour les gamins du village. Il fournit un local, et cherche des volontaires parmi ses clients pour donner un cours un soir. Les élèves qui viennent sont volontaires (ou en tout cas, quelqu'un les encourage fortement à être volontaire sinon tu vas devoir aller désherber la rizière de nuit), et ils sont quand même accompagnés par leur prof d'anglais de l'école, histoire de ne pas laisser un pauvre gars sans formation ni expérience seul face à un groupe de gamins (on a l’Éducation Nationale pour ça...). Le jour où ce fut mon tour (et ne me dites pas que vous ne le sentiez pas venir), j'ai eu de la chance : on a dépassé le record de fréquentation : 21 gamins, enfin, 19 gamines et 2 gamins, deux fois plus que d'habitude ! Le principale apport du voyageur, c'est l'accent. Et oui, même un français peut aider. J'avais même du mal à comprendre la prof d'anglais officielle par moment. Par contre, niveau motivation, rien à voir avec mes souvenirs du collège... Elles se battent presque pour poser des questions. Elles parlent déjà toutes anglais suffisamment bien pour avoir une petite conversation (oui, il y avait aussi deux garçons, mais ils semblaient nettement moins motivés).


\begin{figure}[h]
\centering
\includegraphics[height=9cm,width=12cm,keepaspectratio]{p2019393.jpg}
\caption*{Les filles.}
\end{figure}


\begin{figure}[h]
\centering
\includegraphics[height=9cm,width=9cm,keepaspectratio]{p2019395.jpg}
\caption*{Les garçons.}
\end{figure}


\begin{figure}[h]
\centering
\includegraphics[height=9cm,width=12cm,keepaspectratio]{p1318116.jpg}
\caption*{J'adore cette ambiance post-apocalyptique !}
\end{figure}



\chapter{Hué, là où on a appris ce que survivre veut dire.}
Nous partîmes donc pour Hué. Après nos expériences de train de nuit, on s'est dit qu'on se devait de comparer aussi le train de nuit du Vietnam. On a pris la 1ère classe. Et bien en terme de confort, ça ne vaut pas la 3ème classe russe. Le lit est ok, même si on n'est pas sûr d'être les premiers à dormir dedans. Les toilettes ne sont vraiment pas terribles (toilettes occidentales, rarement lavées, dans un train qui bouge, je vous laisse imaginer), les lavabos sont utilisés pour stocker des plantes, et on se fait réveiller au milieu de la nuit par des gens pas foutus de voir que le téléphone qu'ils cherchent est sur leur lit. Sans oublier les contrôleurs qui se mettent devant notre compartiment avec des grosses lampes de poche pour se raconter des blagues au milieu de la nuit. Ah oui, il y a aussi des blattes... Le tout à une vitesse moyenne de 40km/h. Faut dire que ce sont les français qui ont construit tout le réseau, et que le matériel n'a pas changé depuis, donc on voit partout les bricolages, les rustines et les tentatives de modernisation du genre "et si on rajoutait une télé ?". Bref, on prendra désormais le bus pour les longs trajets.


\begin{figure}[h]
\centering
\includegraphics[height=6cm,width=9cm,keepaspectratio]{p2039409.jpg}
\caption*{Vous croyez que ce sont des couleurs naturelles ?}
\end{figure}

On finit par arriver à Hué, l'ancienne capitale du Vietnam. Il a fait mauvais tout notre séjour, et on n'avait pas envie de visiter encore des monuments, donc on a fait l'impasse sur la cité interdite de Hué, qui serait presque aussi bien que la cité interdite de Pékin. On les a crus sur parole. Par contre, on a pris un tour d'une journée pour aller visiter la DMZ (celle du Vietnam bien sûr, rien à voir avec les Corées). Cette Zone Démilitarisée (DMZ) séparait le Vietnam du Nord et du Sud, et est rapidement devenue une des zones les plus militarisée du monde. Le bus nous emmène à plein d'endroits ou ne voit plus rien. Les bombes, ça n'aide pas à la conservation : On voit le début du sentier de Ho Chi Minh, qui s'avère être un simple sentier qui disparait dans la forêt, et une colline où il y avait une base d'observation américaine : une colline toute bête. Très instructif...


\begin{figure}[h]
\centering
\includegraphics[height=6cm,width=9cm,keepaspectratio]{p2059424.jpg}
\caption*{En effet, c'est une colline.}
\end{figure}

On visite un musée de la guerre, avec des reconstitutions de tranchées, quelques engins exposés mais guère plus. Et enfin, on arrive à la partie vraiment intéressante du tour : les tunnels de Vinh Moc. Dans la DMZ, les gens qui voulaient survivre n'ont eu qu'un seul choix : vivre sous terre. Plus de 120 villages dans la zone ont créé un grand réseau de sous-terrains pour abriter la population et les soldats. Le village qu'on a visité comptait 600 personnes sous terre, vivant dans des conditions incroyables. Dans le noir complet et l'humidité toute la journée, ils ne sortaient de temps en temps que tôt le matin ou tard le soir. Il y a même 17 bébés qui sont nés dans ces conditions. En visitant les tunnels, on a vu la "maternité" : une cavité d'un mètre sur deux, sur 1,6m de hauteur.




\begin{figure}[h]
\centering
\includegraphics[height=6cm,width=9cm,keepaspectratio]{p2059587.jpg}
\caption*{La lumière et les marches, c'est un ajout postérieur pour les touristes.}
\end{figure}

Dans les rues de la ville, c'est l'effervescence : tout le monde est en train de préparer têt qui arrive dans quelques jours. Il y a des marchés aux fleurs de partout, et des scooters qui transportent des énormes pots de fleurs, voire des orangers ! L'oranger, c'est un peu leur arbre de Noël : tout le monde se doit d'en avoir un, il est posé au milieu du salon et décoré avec des guirlandes brillantes.
Un soir, le proprio de la guesthouse nous invite, ainsi que tous les autres clients, pour un repas de pre-têt, deux jours avant la vraie date. Le dîner fut somptueux, avec beaucoup plus à manger que ce qui était possible, même avec 20 convives ! Et il a fait tout son possible pour vider sa grande réserve de bière. Il surveillait les niveaux dans les cannettes, et gare à celui qui en avait une vide !


\begin{figure}[h]
\centering
\includegraphics[height=6cm,width=9cm,keepaspectratio]{p2049422.jpg}
\caption*{Tout doit disparaitre !}
\end{figure}

Mais ce n'est pas à Hué qu'on a passé le têt, mais dans une ville proche, à Hoi An. La veille du dernier jour de leur année, on est parti en train, et on a mis 4h pour faire 100km. Réellement, je vous jure. Outre la vétusté du matériel, il n'y a qu'une seule ligne, donc les trains ne peuvent se croiser qu'en de rares endroits, et on peut se retrouver à attendre longtemps à l'arrêt dans une gare abandonnée qu'un train passe dans l'autre sens.

C'est dans ce train que, intrigué par les cris d'un gamin capricieux, nous avons découvert non pas un gamin relou, mais un petit singe en laisse !

\emph{Imaginez ici une conclusion appropriée, rigolote mais néanmoins profonde, car là tout de suite, je n'en ai pas. Une chose est sûre en revanche, c'est la fin de l'article.}


\begin{figure}[h]
\centering
\includegraphics[height=6cm,width=9cm,keepaspectratio]{p2069610.jpg}
\caption*{Le singe, pendant un des rares moments où il ne courrait pas dans tous les sens}
\end{figure}



\chapter{Hoi An, la ville aux 1000 tailleurs.}

\begin{wrapfigure}{l}{0.55\textwidth}
\centering
\includegraphics[width=0.5\textwidth]{p2099955.jpg}
\caption*{Boutique de lanternes dans Hoi An.}
\end{wrapfigure}

Nous arrivâmes donc à Hoi An, la ville que nous avons choisie pour passer la semaine de Têt. Il faut savoir que la grosse fête de fin d'année est suivie d'une semaine de vacance pour tout le pays. La majeure partie des commerces sont fermés, les transports sont blindés de gens qui rentrent dans leur famille, et globalement, tout est plus cher. C'est pourquoi nous avons décidé de nous offrir une semaine à la plage, dans une ville assez touristique pour que certains restaurants et hôtels restent ouverts.





Une autre caractéristique de Hoi An, ce sont les tailleurs : apparemment, dans les années 2000, il y avait une vingtaine de tailleurs à Hoi An, capables de faire des vêtements sur-mesure de bonne qualité pour un prix défiant toute concurrence. Pour une raison ou une autre, ça a grandi en popularité et d'autres tailleurs sont arrivés. En 2016, il y aurait plus de 1000 boutiques de tailleurs se battant pour attirer les touristes. On n'a pas compté les boutiques, mais on a pu constater que des rues entières, longues de plusieurs kilomètres, alignent les tailleurs les uns à coté des autres. Il y a des ateliers un peu partout pour réaliser les commandes, et ils peuvent faire un costume en quelques heures en s'y mettant à 5 ou 6 si besoin : un ouvrier sur une manche, un sur une jambe etc. Évidemment, il n'y a pas assez de touristes pour faire vivre toutes ces boutiques, ce qui tire les prix vers le bas, et les commissions vers le haut (commissions qui vont sans problème jusqu'à 40\%) ! On a bien essayé nous-même de faire un devis dans une boutique, mais pendant la seule semaine de congé nationale, disons que les prix étaient un tout petit peu moins intéressants. Au regret de Marion, je ne continuerai pas le voyage en costume sur mesure.




\begin{figure}[h]
\centering
\includegraphics[height=6cm,width=9cm,keepaspectratio]{p2079648.jpg}
\caption*{C'est bientôt Têt, donc les gens transportent des fleurs. Au fait, vous avez vu le deuxième gars ?}
\end{figure}

Le soir de têt, on demande à l'accueil de l'hôtel s'il se passe quelque chose dans les rues, et on a une comme réponse "ça va être amusant". Têt a la réputation d'être une fête plutôt familiale, donc on ne s'attendait pas forcément à voir des trucs incroyables. Mais finalement, le centre ville était très animé, avec beaucoup de stands de nourriture, une sorte de fête foraine à l'ancienne, et plein d'illuminations. Ils aiment aussi beaucoup les loteries façon bingo, mais c'est aussi un spectacle : plusieurs présentateurs se relaient pour chanter en permanence. Sans comprendre le vietnamien, on avait l'impression que c'était une sorte d'improvisation sur le thème des symboles tirés au sort. Le chant ne s'arrête pas une seule fois, les enchainements sont parfaits ce qui a un coté hypnotisant.




\begin{figure}[h]
\centering
\includegraphics[height=6cm,width=9cm,keepaspectratio]{p2079706.jpg}
\caption*{Illuminations le soir de Têt.}
\end{figure}




\begin{figure}[h]
\centering
\includegraphics[height=6cm,width=9cm,keepaspectratio]{p2089730.jpg}
\caption*{De l'autre coté de la rivière.}
\end{figure}


\begin{figure}[h]
\centering
\includegraphics[height=6cm,width=9cm,keepaspectratio]{p2079726.jpg}
\caption*{Les vendeuses de lanternes s'ennuient...}
\end{figure}

Vers 22h, un peu fatigués, on fini par rentrer après avoir tout vu. Là, on vide nos poches de tous les prospectus reçus, et on voit une référence à un feu d'artifice à minuit ! Ben on est ressortis aussi sec, on a trouvé la place idéale, et on a regardé le feu d'artifice le plus impressionnant qu'on a jamais vu, juste au dessus de nos têtes. On a cru plusieurs fois voir le bouquet final tellement ça pétait dans tous les sens, mais ça recommençait encore plus fort, à tel point qu'on s'est même senti "menacés" et qu'on a eu un mouvement de recul. Je vous assure, d'habitude, quand des trucs brillent, explosent, ou brûlent, j'ai plutôt tendance à me sentir attiré...




\begin{figure}[h]
\centering
\includegraphics[height=6cm,width=9cm,keepaspectratio]{p2089738.jpg}
\caption*{C'est beau !}
\end{figure}

Hué est aussi connu pour ses grandes plages, et comme il faisait enfin de nouveau un peu chaud, et qu'on avait une semaine à passer dans le coin, on a pas mal larvé sur le sable, une fois la bonne plage trouvée : la première était plutôt dans un sale état, et on aurait dit qu'un banc de baleines s'était échoué.




\begin{figure}[h]
\centering
\includegraphics[height=6cm,width=9cm,keepaspectratio]{p2079641.jpg}
\caption*{C'est vendeur comme plage, non ?}
\end{figure}


\begin{figure}[h]
\centering
\includegraphics[height=6cm,width=9cm,keepaspectratio]{p2079644.jpg}
\caption*{La plage se fait grignoter.}
\end{figure}

On a pu admirer les pêcheurs locaux en train de manœuvrer leurs grosses bassines qu'ils appellent des bateaux. J'ai du mal à comprendre comment, parmi toutes les formes possibles pour faire un bateau, ils ont choisi la bassine... Mais ça marche, et ils reviennent sur la plage avec quelques poissons, vendus quelques instant après aux vacanciers.




\begin{figure}[h]
\centering
\includegraphics[height=6cm,width=9cm,keepaspectratio]{p2130011.jpg}
\caption*{Je vous jure, ça flotte !}
\end{figure}

Il y a aussi la même vieille qui passe toutes les 20 min pour essayer de nous vendre des bières de sa glacière. A chaque fois, elle s'assoit à un mètre de nous, et nous répète "Beer ? Beer ?" puis repart en maugréant des trucs qui ne nous sonnaient vraiment pas gentil à l'oreille. Et elle nous a fait ça toute la semaine, et a dû nous proposer sa bière une bonne cinquantaine de fois, passant rapidement de la catégorie "je fais mon boulot", à "je suis reloue" !


\begin{figure}[h]
\centering
\includegraphics[height=9cm,width=12cm,keepaspectratio]{p20796391.jpg}
\caption*{Elle par contre, elle était tellement sympa et commerciale, qu'elle a réussi à nous vendre des souvenirs...}
\end{figure}

Par contre, dans la catégorie, "je suis hyper sympa", il y a cette famille (étendue) vietnamienne qui s'était installée pas loin de nous pour pique-niquer. Quand ils nous ont vus, ils ont insisté pour nous inviter à partager leur pique nique. On a bu quantité de bière, et mangé du poisson séché. Comme quoi mon degré de sympathie envers les inconnus dépend fortement du prix de leur bière. Après avoir fait les désormais traditionnels selfies avec tous les membres de la famille, ils ont même tenus à me donner de l'argent ! Je me suis donc retrouvé avec un dollar dans la main ! Visiblement, ça porte bonheur chez eux de donner de l'argent à des étrangers pendant Têt. Mais pourquoi donc avoir choisi un riche occidental ?


\begin{figure}[h]
\centering
\includegraphics[height=9cm,width=12cm,keepaspectratio]{p2079645.jpg}
\caption*{Nouveau sport extrême : la cueillette de noix de coco !}
\end{figure}

En sa baladant dans la rue, soudainement, mon regard est interpellé par une silhouette qui me semble familière. L'intérêt de la silhouette en question semble réciproque, et on reste tous les deux plantés un quart de seconde avant le cerveau accepte la conclusion de l'analyse visuelle après l'avoir re-vérifiée, mais oui, c'est bien Cyril, ami d'enfance, qui se balade dans la rue, et qui n'en revient pas lui non plus de me trouver là ! Quand je pense à toutes les façons possibles qu'on aurait eu de se louper, je n'en reviens pas qu'on ait réussi à boire un coup ensemble à l'improviste à l'autre bout du monde.


\begin{figure}[h]
\centering
\includegraphics[height=6cm,width=9cm,keepaspectratio]{p2079636.jpg}
\caption*{Alsace power !}
\end{figure}

C'est aussi à Hoi An que j'ai fait du scooter pour la première fois de ma vie ! On a commencé par en louer un, car on voulait visiter les ruines de My Son (non, rien à voir avec mon fils, c'est le nom du lieu en vietnamien). Marion a fait les premiers kilomètres puis on a trouvé par hasard un parking abandonné (je dis par hasard pour ne pas dire qu'on s'était planté de rue), ce qui m'a permis d'apprivoiser l'engin à l'abri de la circulation anarchique propre au Vietnam. Et ce mode nous a tellement plu à tout les deux que  depuis ce moment, c'est moi le chauffeur officiel ! Nous avons donc foncé à la vitesse ébouriffante de 35km/h, avec des pointes à 45 par moment, vers le site de My Son. Il y avait des vieux cailloux, des vieilles briques et des grosses araignées : c'était plutôt sympa, mais la découverte de la journée, ça reste quand même le scooter !


\begin{figure}[h]
\centering
\includegraphics[height=6cm,width=9cm,keepaspectratio]{p2099907.jpg}
\caption*{Faut aimer les briques...}
\end{figure}

On avait bien anticipé la semaine de Têt, et pour la première fois depuis le voyage, on avait tout réservé deux semaines en avance. Par contre, on n'avait pas anticipé que les gens allaient aussi repartir à la fin de la semaine... Résultat, tout est complet. Faisant contre mauvaise fortune bon cœur (Oh mince, encore de la plage et des supers bons restaurants...), nous avons rallongé notre séjour et réservé un bus un peu plus tard, pour seulement trois fois plus cher que le reste de l'année. C'était encore un bus de nuit, tellement plein que les gens n'arrivaient même plus à s'allonger dans les couloirs. Il semble que la stratégie ici est de d'abord remplir les vraies places avec des occidentaux qui paient cher, puis remplir les espaces restants avec des locaux pour maximiser la marge. Au petit matin, on a une correspondance, et on monte dans un mini-bus pour la fin du trajet. J'hésite à raconter cette dernière partie car nos mères nous lisent, mais tant pis, je me lance : (mères, sachez que tout s'est bien passé) le chauffeur, pensant sans doute que conduire au Vietnam manquait de piment, s'est mis à regarder un film sur l'écran du tableau de bord ! Les mobylettes n'ont qu'à faire gaffe...


\begin{figure}[h]
\centering
\includegraphics[height=6cm,width=9cm,keepaspectratio]{p2129989.jpg}
\caption*{Et ça, c'est un spectacle de marionnettes dans l'eau. En vietnamien s'il vous plait !}
\end{figure}


\begin{figure}[h]
\centering
\includegraphics[height=12cm,width=12cm,keepaspectratio]{p2120007.jpg}
\caption*{Encore une vendeuse de lanternes.}
\end{figure}



\chapter{Mui Ne, sable rouge et grosses crevettes.}
Nous arrivâmes donc à Mui Ne, une des stations balnéaires les plus connues du Vietnam. Au programme : plage, piscine, fruits de mer, et dunes de sable.


\begin{figure}[h]
\centering
\includegraphics[height=6cm,width=9cm,keepaspectratio]{p2170086.jpg}
\caption*{Je n'avais jamais vu de si gros bernard-l'hermite (oui, c'est invariable).}
\end{figure}


Niveau plage, on est mitigés : la plage a disparu sur la moitié de la côte, le sable a été emporté par les vagues, et il ne reste plus qu'une digue en béton qui protège la rangée d'hôtels. On doit marcher quelques km pour trouver une plage digne de ce nom.


\begin{figure}[h]
\centering
\includegraphics[height=6cm,width=9cm,keepaspectratio]{p2170105.jpg}
\caption*{L'horizon est rempli de voiles.}
\end{figure}

Mais cette plage est le royaume du kite-surf, ce qui implique deux choses. Premièrement, il faut faire attention de ne pas se prendre un kite-surfeur dans les gencives - non qu'ils soient mal-intentionnés, mais une proportion non négligeable des pratiquants est en phase d'apprentissage, et leurs trajectoires peuvent, si le vent est suffisamment blagueur, ponctuellement coïncider avec la trajectoire des gencives d'un baigneur trop confiant. Donc aller dans l'eau, c'est un coup de poker. Deuxièmement, si les kite-surfeurs aiment ce coin, c'est parce qu'il y a un vent très régulier, et fort. Qui soulève le sable. Et le redépose partout, comme par exemple sur une mangue fraichement découpée, ou bien sur l'objectif de l'appareil photo qui se met à faire un son louche ! Bref, rester sur la plage, c'est nul. C'est pourquoi nous avons été ravis de profiter de la magnifique piscine de notre auberge de jeunesse, qui, outre le fait d'être exempte de kite-surfeurs, de sable et de vent, comporte un bar adjacent proposant des mojitos très convenables à 1\euro, sirotables dans une chaise longue ombragée !



\begin{figure}[h]
\centering
\includegraphics[height=6cm,width=9cm,keepaspectratio]{p2170094.jpg}
\caption*{Le retour de la pêche.}
\end{figure}


Pour ce qui est des fruits de mer, on est apparemment dans un des meilleurs endroits du Vietnam. De nombreux restaurants exposent la pêche du jour sur le trottoir, il suffit de pointer les victimes du doigt puis d'aller s'asseoir (en vrai faut aussi dire quelle cuisson, et s'enquérir un minimum du prix). On s'est ainsi fait une langouste entière, mais la révélation, ce furent les petites coquilles saint-jacques à l'échalote, cuites au barbecue ! Des étoiles s'allument dans les yeux de Marion à la simple évocation de ce plat.


\begin{figure}[h]
\centering
\includegraphics[height=6cm,width=9cm,keepaspectratio]{p2170082.jpg}
\caption*{Le tri de la pêche.}
\end{figure}


\begin{figure}[h]
\centering
\includegraphics[height=6cm,width=9cm,keepaspectratio]{p2170067.jpg}
\caption*{Un bol de poulpes pour le petit déjeuner ?}
\end{figure}


\begin{figure}[h]
\centering
\includegraphics[height=6cm,width=9cm,keepaspectratio]{p2170069.jpg}
\caption*{Les langoustes le matin.}
\end{figure}


\begin{figure}[h]
\centering
\includegraphics[height=6cm,width=9cm,keepaspectratio]{p2170409.jpg}
\caption*{Les langoustes le soir.}
\end{figure}

Enfin les dunes de sable, autre attraction fameuse de Mui Ne. Le circuit classique que tous les touristes font commence par le "Fairy stream", ou ruisseau des fées. C'est un ruisseau qui a creusé des paysages similaires au grand canyon dans du sable, et c'est une balade sympathique et rafraichissante de remonter le cours d'eau pieds nus.


\begin{figure}[h]
\centering
\includegraphics[height=6cm,width=9cm,keepaspectratio]{p2160028.jpg}
\caption*{Le ruisseau et ses étranges sculptures.}
\end{figure}

Ensuite on attaque les choses sérieuses : les dunes blanches ! C'est simple, on se croit dans le désert. Sur un ou deux kilomètres, rien d'autre que du sable, des dunes et des dunes de sable. Ça pourrait être un endroit fantastique, propre à l'évasion, à la méditation, face à l'immensité désertique ! Mais certaines personnes ont jugé pertinent de proposer aux touristes la location de quads. On est donc entourés de touristes stupides sur des quads poussifs et bruyants, dont les émissions ajoutent une couche de poussière noire sur le sable blanc. En plus, ces quads ne sont même pas amusants, leur vitesse de pointe plafonne à 10km/h, et ils n'ont que deux roues motrices, donc s'ensablent à la première montée. Il y a sûrement quelque chose qui nous échappe.


\begin{figure}[h]
\centering
\includegraphics[height=6cm,width=9cm,keepaspectratio]{p2160048.jpg}
\caption*{Je vous jure, c'est au Vietnam !}
\end{figure}

La fin de la journée nous amène aux dunes rouges, pour voir le coucher de soleil, et là, avec la couleur du sable accentuée par le soleil couchant, on en prend plein les yeux !


\begin{figure}[h]
\centering
\includegraphics[height=9cm,width=12cm,keepaspectratio]{p2160059.jpg}
\caption*{On en prend plein les yeux : de la lumière, mais aussi du sable !}
\end{figure}


\begin{figure}[h]
\centering
\includegraphics[height=9cm,width=12cm,keepaspectratio]{p2170084.jpg}
\caption*{J'adore cet animal !}
\end{figure}



\chapter{Ho Chi Minh Ville (Saigon), petite pause urbaine.}
Nous arrivâmes donc à Ho Chi Minh Ville, anciennement Saigon. C'est la plus grande ville du Vietnam, et sa capitale économique, et niveau circulation, ça se ressent ! On progresse encore d'un cran dans le chaos, avec des trois voies saturées de motos. On est content d'avoir eu de l'entrainement dans des villes plus faciles, sinon, je pense qu'on aurait du prendre un tuk-tuk pour traverser la rue !


\begin{figure}[h]
\centering
\includegraphics[height=6cm,width=9cm,keepaspectratio]{p2190668.jpg}
\caption*{Ho Chi Minh Ville de nuit.}
\end{figure}

Pour ce séjour, on fait une pause dans le tourisme/voyage. Tout d'abord, on va se faire un cinéma ! On était très frustrés d'avoir loupé Star Wars à cause des décalages de dates de sorties (on était trop tôt en Chine, puis trop tard au Vietnam), et on s'est vengé sur Zootopia que l'on vous recommande chaudement. Ce fut la première fois qu'on regardait un film en Anglais sous-titré vietnamien, et étant donné qu'on riait tous les deux de concert avec le reste de la salle, on se dit que pas grand chose du film n'a du nous échapper.




\begin{figure}[h]
\centering
\includegraphics[height=6cm,width=9cm,keepaspectratio]{p2190683.jpg}
\caption*{Est-ce que vous savez pourquoi les crocodiles ont le ventre bleu ? Une carte postale au plus rapide !}
\end{figure}

On a aussi fait un peu de ménage dans les affaires : les vêtements chauds sont dans un paquet pour la France, et on s'est équipé un peu mieux pour le chaud, et Marion s'est fait faire une nouvelle paire de lunettes de soleil à sa vue : arrivé chez l'opticien le matin à 9h avec l'ordonnance, quelques minutes pour choisir la monture, et on a récupéré les lunettes finies le soir à 8h. Le tout pour la somme modique de 26\euro. Va falloir que j'en parle à mon opticien de frangin pour comprendre...




\begin{figure}[h]
\centering
\includegraphics[height=6cm,width=9cm,keepaspectratio]{p2190672.jpg}
\caption*{Une fontaine pile au moment où la lumière change de couleur.}
\end{figure}

Je ne sais pas si on vous avait déjà parlé du café Vietnamien, mais dans le doute, je vais en remettre une couche : c'est le meilleur café du monde (selon moi). Il a beaucoup de gout, mais sans être amer, et, plutôt que de sentir le brulé, a un gout très rond. Bref, il est bon. Et pour ne rien gâcher, c'est facile à faire : il faut juste avoir le petit filtre métallique à poser sur la tasse, et faire un petit café filtre individuel. Libre ensuite d'ajouter de la glace et/ou du lait concentré sucré, toutes les combinaisons sont bonnes ! Histoire de pousser l'expérience du bon café vietnamien jusqu'au bout, nous sommes allés dans le meilleur café de la ville, goûter le meilleur café de la carte, fait avec le café le plus cher du monde (je commence à trouver pénible le fait qu'en français, café puisse signifier "le lieu", "les grains", "la poudre" ou "la boisson"). J'ai nommé : le "café belette".

Petite leçon d'histoire coloniale : des colons français ont apporté la culture du café au Vietnam, mais évidemment, le café avait beaucoup trop de valeur pour qu'on laisse les locaux en consommer, il était réservé à l'exportation. Les locaux ont alors remarqué qu'une sorte de belette mangeait les fruits, dont les graines étaient ensuite retrouvées intactes dans ... bon, je ne vais pas vous faire un dessin ! Il faillait bien laver les grains évidemment, mais ça marchait ! Et ça marchait même drôlement bien, les sucs digestifs de la belette font une torréfaction beaucoup plus douce, et ajoutent même des arômes de vanille. Évidemment, on est rapidement passé de "je cherche des crottes dans la forêt" à "j'élève mes belettes", puis à "je synthétise les sucs digestifs de la belette" et enfin pour certains audacieux "je colle une photo de belette sur mon pot de café à 3\euro le kg, ces c**s de touristes n'y verront que du feu". En effet, si d'un coté vous avez des vendeurs qui vendent le café à ce prix, et que de l'autre une tasse du supposé même café est à 8\euro, il y a forcément une arnaque quelque part. Nous, on a fait le pari que l'arnaque était du coté café pas cher, et on est allé claquer le prix de deux repas dans une tasse de café belette, accompagnée, à titre de café témoin, par une tasse de bon café. Mes papilles en tremblent encore...


\begin{figure}[h]
\centering
\includegraphics[height=6cm,width=9cm,keepaspectratio]{p2190410.jpg}
\caption*{On a religieusement suivi les conseils du guide en allant visiter la poste. Voilà, c'est fait. Bon, ben on va rentrer !}
\end{figure}

Il y a quelques buildings qui poussent à Saigon, et ils sont idéaux pour admirer un coucher de soleil. En faisant un peu de recherche, on se rend compte que c'est moins cher d'aller boire une bière au sommet du Sheraton (*****) que de payer un billet pour accéder à la plateforme d'observation à peine plus haute de l'immeuble d'à coté. Tant qu'à faire, autant boire de la bière ! Nous voici donc dans le bar du Sheraton, et deuxième bonne nouvelle : en happy hour, pour un cocktail acheté, le deuxième à 1\$ ! Dans ce cas, autant boire des mojitos plutôt que de la bière pas bonne ! Le bar est quasiment vide, on a une table face au soleil couchant, et le cocktail est servi format saladier. Tout est parfait ! Le soleil fini par se coucher, et on se dit qu'on va faire de même. L'addition arrive. Ah, petit problème ! L'offre happy hour n'a pas été prise en compte, ils veulent nous faire payer les deux boissons au plein tarif. On appelle la serveuse, puis son manager, qui nous explique que l'offre est valable "par personne", que c'est uniquement la deuxième boisson d'un seul consommateur qui est à 1\$. Si on veut profiter de l'offre, libre à nous de commander chacun une deuxième boisson. L'offre étant ambiguë, ce n'était pas ce qu'on avait compris, et étant de bonne fois, dans un établissement de ce standing, alors que nous sommes les seuls consommateurs, on se dit qu'ils vont faire un geste ! Eh bien non, le manager ne voulait rien savoir. Toujours en gardant les formes, en s'excusant à chaque phrase, le mieux qu'il nous offrait était un bon pour deux consos à 1\$ à utiliser un autre jour, mais il voulait quand même qu'on paye les deux premières conso (22\$ quand même). Et il me disait qu'il était coincé, qu'ils avaient ouvert une table pour deux clients sur l'ordinateur, et que le logiciel n'autorisait pas cette promo dans ces conditions. Coincé par un ordinateur. Me dire ça... à moi... Après 20mn de pourparlers, et seulement après que j'ai demandé à parler à son supérieur, il a enfin consenti à un geste. Mais cette lutte de marchand de tapis nous a semblé malvenue ici, même les bouibouis dans la rue proposent un meilleur service !




\begin{figure}[h]
\centering
\includegraphics[height=6cm,width=9cm,keepaspectratio]{p2200688.jpg}
\caption*{Cours de peinture.}
\end{figure}

Il y a quand même un endroit de Ho Chi Minh Ville qu'on était obligé de visiter en tant que touriste : le musée de la guerre. Il est rempli de photos et d'histoires qui décrivent les horreurs de la guerre du Vietnam. Et les horreurs n'ont pas seulement eu lieu pendant la guerre, mai continuent maintenant encore : le pays est encore loin d'être déminé (vous saviez que les Américains avaient plus balancé de bombes sur le Vietnam que l'ensemble des bombes balancées pendant toute la seconde guerre mondiale par l'ensemble des belligérants ?), et l'agent orange est encore présent dans l'eau, et continue de provoquer des malformations chez les nouveaux-nés, 40 ans plus tard...




\begin{figure}[h]
\centering
\includegraphics[height=6cm,width=9cm,keepaspectratio]{p2190680.jpg}
\caption*{Je ne sais pas ce que c'est, mais c'est joli !}
\end{figure}







\chapter{Vinh Long, le calme du delta du Mékong.}
Nous arrivâmes donc à Vinh Long, plus précisément sur l'île de An Binh. Après avoir pris trois bus successifs puis un ferry, et enfin une moto qui a suivi une route, puis un chemin bétonné, puis un chemin en gravier, nous arrivâmes à notre guesthouse. On peut dire qu'elle était un rien isolée : au milieu d'une île du détroit du Mékong, perdue entre une plantation de bananier et un petit canal de navigation. Et pourtant, dans la guesthouse, que des Français... Et que font des Français qui se rencontrent à l'étranger ? Eh bien ils parlent de bouffe ! Donc on a passé les quelques jours sur place à se raconter à quel point le fromage nous manquait...


\begin{figure}[h]
\centering
\includegraphics[height=6cm,width=9cm,keepaspectratio]{p2230709.jpg}
\caption*{Coucher de soleil sur le Mékong.}
\end{figure}




\begin{figure}[h]
\centering
\includegraphics[height=6cm,width=9cm,keepaspectratio]{p2230729.jpg}
\caption*{Bon appétit !}
\end{figure}

Le delta du Mékong est énorme : le fleuve se divise en 9 bras, chacun bien plus gros que le Rhône. C'est un endroit parfait pour cultiver du riz : avec 3 récoltes par an, la zone produit plus de riz que le Japon et la Corée réunis, ce qui explique pourquoi on a vu autant de bateaux remplis de riz pendant le tour en barque qu'on a fait sur le Mékong. Mais cette zone est en danger : les différents barrages en amont du fleuve régulent le débit et limitent les inondations nécessaires pour fertiliser les terres. Les terres qui ne sont plus inondées deviennent impropre à la culture. L'autre danger, est la baisse du débit : la Chine, puis le Laos, la Thaïlande et le Cambodge puisent dedans pour l'irrigation et l'industrie, ce qui laisse l'eau de mer remonter plus haut dans le delta, salinisant les terres par la même occasion. Nous étions à plus de 100km de la cote, et la marée faisait quand même varier le niveau d'eau de presque deux mètres !


\begin{figure}[h]
\centering
\includegraphics[height=6cm,width=9cm,keepaspectratio]{p2230722.jpg}
\caption*{Une cargaison de riz.}
\end{figure}

Le tour en barque, c'est le truc à faire dans le coin, et ils vendent ça surtout comme une visite au marché flottant. Ah, le fameux marché flottant, tellement typique avec ses gens qui se baladent de barque en barque, tellement à l'aise qu'ils nous font oublier que le sol ondule sous leurs pas... et bien depuis l'arrivée de la route, ce marché a été réduit à la portion congrue. Il ne reste que quelques bateaux qui font de la vente en gros de tubercules, le reste du marché flottant est maintenant bien ancré sur la terre ferme. C'est principalement un marché alimentaire, et on y voit des choses vraiment étranges, voire choquantes pour un occidental au cœur sensible : un premier stand de grenouilles les vendait entières, mortes et dépecées. Contrairement aux Français, ici, ils mangent les grenouilles entières et pas uniquement les cuisses. Un deuxième stand les vendait vivantes, intactes, attachées par paquets de 5. Vous sentez venir le troisième stand ? Je conseille aux âmes sensibles de passer directement au paragraphe suivant, et de ne pas essayer d'analyser la photo qui suit. Vous êtes sûrs de continuer ? Bon, vous ne pourrez pas dire que je ne vous ai pas prévenus... Le troisième stand, donc, vend les grenouilles vivantes, mais dépecées, sans la peau. Elles sont dans une grande assiette métallique, baignant dans leur sang, et essaient de temps en temps de sauter par dessus le rebord de 3 cm, mais leurs moignons glissent, elles retombent dans l'assiette, et restent là, la bouche ouverte dans un grand cri silencieux.


\begin{figure}[h]
\centering
\includegraphics[height=6cm,width=9cm,keepaspectratio]{p2230758.jpg}
\caption*{...}
\end{figure}


\begin{figure}[h]
\centering
\includegraphics[height=6cm,width=9cm,keepaspectratio]{p2230757.jpg}
\caption*{Un poulpe bien frais.}
\end{figure}

Et nous continuons cette visite, et cet article, sur une note plus joyeuse avec la visite d'une manufacture de bonbons !  On voit la fabrication de caramel à la noix de coco, de feuilles de riz, et une démonstration de popcorn de riz, ou poprice, avant de passer à la dégustation de diverses infusions d'alcool de riz. D'après le guide, le classique alcool fermenté au cobra/scorpion est un puissant aphrodisiaque, à réserver aux hommes donc, parce que sinon, oulala, je ne vous raconte pas... Pour les femmes, le guide recommande l'alcool gout banane, idéal pour bien dormir. Marion fait alors justement remarquer que c'est quand même stupide de vouloir faire dormir sa femme à ce moment là, c'est un coup à dormir sur la béquille... Ils n'ont rien compris ces Vietnamiens !


\begin{figure}[h]
\centering
\includegraphics[height=6cm,width=9cm,keepaspectratio]{p2230830.jpg}
\caption*{Ça fait quand même moins peur sans les dents, et dans une bouteille, et mort...}
\end{figure}

Après ces derniers jours au calme, c'est le moment pour nous de dire au revoir au Vietnam, et nous allons passer une dernière nuit à la ville frontière de Chau Doc. Pas grand chose à raconter, mis à part les 100 moustiques qu'on a tué dans la chambre, avant de monter la moustiquaire de la tente sur le lit. Le lendemain, passage de frontière en bateau, et rien, aucune tentative de corruption, malgré tous les avertissements qu'on avait lus à propos des compagnies de transports et des douaniers, personne ne nous a demandé plus d'argent que ce qui était prévu, et c'est sur ces bonnes impressions que nous entrons au Cambodge !


\begin{figure}[h]
\centering
\includegraphics[height=6cm,width=9cm,keepaspectratio]{p2250901.jpg}
\caption*{La brigade fluviale est impressionnante !}
\end{figure}


\begin{figure}[h]
\centering
\includegraphics[height=6cm,width=9cm,keepaspectratio]{p2210699.jpg}
\caption*{Ces petites filles ont tenu à jouer les modèles pour nous.}
\end{figure}



\chapter{Phnom Penh, chute et reconstruction du Cambodge.}
Nous arrivâmes donc à Phnom Penh, la capitale du Cambodge. Et on progresse encore d'un cran dans l'anarchie circulatoire : les scooter y roulent aussi mal qu'au Vietnam, mais en plus, il y a masse de 4x4, ce qui est nettement plus compliqué à esquiver. Les rares trottoirs sont utilisés pour le commerce ou le stationnement. Hormis dans quelques rues touristiques, c'est extrêmement désagréable, voire dangereux de s'y promener. D'ailleurs, personne ne le fait, mis à part les mendiants et quelques touristes. Il est inconcevable pour un Cambodgien d'aller quelque part à pied, il perdrait la face, et il se déplace donc dans le plus gros véhicule qu'il peut se permettre. Avec la progression du tourisme, et le développement urbain qui l'accompagne, de nombreux paysans se sont retrouvés très riches quand leur rizières ont été déclarées constructibles. Et que fait un Cambodgien riche à votre avis ? D'après des sources bien informées, il se contente d'acheter un plus gros 4x4 que son voisin.


\begin{figure}[h]
\centering
\includegraphics[height=6cm,width=9cm,keepaspectratio]{p2261229.jpg}
\caption*{L'entrée du Palais Royal.}
\end{figure}




\begin{figure}[h]
\centering
\includegraphics[height=6cm,width=9cm,keepaspectratio]{p2261226.jpg}
\caption*{Le palais est plein de moines. Le plus dur, c'est d'en trouver qui ne sont pas entourés de touristes !}
\end{figure}

Nous visitons le palais royal, en fait un parc rempli de palais, de pagodes, et de temples de toute tailles. Le clou du spectacle est la pagode d'argent. Elle est appelée ainsi car son sol est recouvert de dalles en argent, il y en a pour plusieurs tonnes, une vraie fortune. Comme dans toutes les pagodes, il faut enlever ses chaussures avant d'entrer, et c'est pieds nus que nous entrons. Et nous voyons... des tapis ! Plein de tapis qui recouvrent la quasi intégralité du sol. Seul un petit coin protégé par des barrières est visible. On voit donc quelques dalles ternes, dont les joints ont été protégés il y a déjà quelque temps par du gros scotch marron qui part à présent en lambeaux...


\begin{figure}[h]
\centering
\includegraphics[height=6cm,width=9cm,keepaspectratio]{p2261242.jpg}
\caption*{Heureusement, l'extérieur du palais est bien plus beau !}
\end{figure}

L'autre endroit à visiter, c'est la prison S21. C'est une ancienne école que les Khmers Rouges ont, comme beaucoup d'autres écoles, transformé en prison pour les opposants au régime. Et cette prison était leur prison VIP, là où ils envoyaient leurs ennemis les plus puissants, les plus éduqués. Nous avons passé 4h ici, à apprendre l'histoire de cet endroit et du régime Khmer Rouge, à faire connaissance avec les victimes et les bourreaux (rôles qui se confondent parfois), à visiter les cellules et voir les instruments de torture, et finir par voir les restes des victimes dont les ossements portent des stigmates visibles. On apprend comment le directeur de la prison, un professeur de mathématique adoré de ses élève, a pris des jeunes des campagnes pour les former correctement à la torture. Il n'était pas question de torturer pour le plaisir, il fallait soutirer des aveux : si "l'organisation" arrêtait quelqu'un, c'est qu'il était coupable et il ne restait plus qu'à trouver de quoi. On a ainsi découvert qu'un grand nombre de Cambodgiens étaient des espions de la CIA. Si un bourreau faisait mourir un prisonnier avant qu'il ait signé ses aveux, il risquait fortement de passer lui aussi à la question, donc tout était mis en place pour garder les prisonniers en "bonne santé" le plus longtemps possible. Le but des Khmers Rouges était de créer une utopie communiste basée sur une société agricole. Quelques jours après leur victoire, ils ont vidé toutes les villes, et envoyé leurs habitants travailler dans les champs. Ils ont aboli la monnaie, la propriété et l'éducation. Ils se sont débarrassés des intellectuels, notions très vaste pour eux : avoir des lunettes, un stylo, savoir faire du vélo, savoir écrire son nom, ou avoir un parent dans ce cas était un motif amplement suffisant. En quelques année, ils ont détruit toute l'élite du pays, et toute sa culture. Le pays panse encore aujourd'hui ses blessures.


\begin{figure}[h]
\centering
\includegraphics[height=6cm,width=9cm,keepaspectratio]{p2271252.jpg}
\caption*{Les barbelés, c'est pour empêcher les prisonniers de se suicider.}
\end{figure}

Et puis, d'un coup, comme ça, pouf, sans prévenir, une envie subite de vélo nous a saisi ! Et nous avons acheté des vélos. De jolis vélos chinois, bien adaptés à la ville, pour une somme très raisonnable. Vous devinez la suite ? Oui ? Non ? Quoi qu'il en soit, ce sera dans le prochain article !


\begin{figure}[h]
\centering
\includegraphics[height=6cm,width=9cm,keepaspectratio]{p2271253.jpg}
\caption*{Mais qu'allons nous donc bien pouvoir faire avec ces bolides ?}
\end{figure}


\begin{figure}[h]
\centering
\includegraphics[height=9cm,width=12cm,keepaspectratio]{p2250906.jpg}
\caption*{Les moines sont censés renoncer à leurs possessions, mais le téléphone, c'est dur...}
\end{figure}


\begin{figure}[h]
\centering
\includegraphics[height=9cm,width=12cm,keepaspectratio]{p2261245.jpg}
\caption*{De l’encens.}
\end{figure}



\chapter{A vélo vers les temples d'Angkor.}
Et nous partîmes donc en direction de Siem Reap. A vélo ! 350km sur un vélo de ville chinois, avec des températures prévues entre 28\textdegree C et 40\textdegree C, ça vend du rêve, non ?


\begin{wrapfigure}{l}{0.65\textwidth}
\centering
\includegraphics[width=0.6\textwidth]{p3021286.jpg}
\caption*{Lever de soleil sur la route.}
\end{wrapfigure}

Mais qu'est-ce qui a bien pu se passer dans nos têtes pour qu'on en arrive là ? En fait, depuis quelques semaines déjà, on commençait à trouver qu'on s'était installé dans une sorte de routine. Oui, je sais, c'est bizarre de parler de routine quand on change d'endroit en permanence. Mais la routine, c'était trouver notre prochaine étape, puis trouver un moyen de transport, puis trouver un logement. Toujours les mêmes sites d'information sur le web, toujours les mêmes bus, les mêmes guesthouses... Avec Internet, toutes ces étapes sont faciles à faire, toutes les informations disponibles en avance, et tout est prévu pour le touriste. Sérieusement, c'est très (trop?) facile de voyager dans ces conditions. Alors on a eu envie de quelque chose de différent, et vu que le pays est plat, pourquoi ne pas faire du vélo ? En plus, les travaux de la route entre Phnom Penh et Siem Reap sont quasiment finis, et la route est comme neuve !


\begin{figure}[h]
\centering
\includegraphics[height=6cm,width=9cm,keepaspectratio]{p3011268.jpg}
\caption*{Nos bolides !}
\end{figure}

N'empêche que ça nous a fait un peu sortir de notre zone de confort : les jours précédents, nous étions tour à tour un peu tendus à l'idée du trip qui nous attendait, ce qui ne nous était plus arrivé depuis un petit bout de temps.


\begin{figure}[h]
\centering
\includegraphics[height=6cm,width=9cm,keepaspectratio]{p3021282.jpg}
\caption*{Parfois, c'est de la piste !}
\end{figure}

Le matin du départ, on est sur le pied de guerre à 5h. Le temps de ranger nos affaires, de réveiller le veilleur de l'hôtel, puis de trouver comment sangler le sac à dos sur le porte bagage, il est 6h au moment où on commence à pédaler, pile aux premières lueurs de l'aube. Vu la chaleur qu'il fait en ce moment, on s'est dit qu'il fallait vraiment qu'on profite au maximum de la fraicheur du matin.


\begin{figure}[h]
\centering
\includegraphics[height=6cm,width=9cm,keepaspectratio]{p3021288.jpg}
\caption*{Quand on sort de la route principale, on se perd...}
\end{figure}

Et ce fut génial. En quelques minutes, on est déjà dans un endroit où les seuls touristes habituellement visibles le sont subrepticement à travers la vitre d'un bus. Alors quand les gens voient deux couillons tout blanc rouge passer à 15km/h, ça interpelle ! On va passer tout le trajet à répondre aux "hello" tous les 50m. Ce n'est pas grand chose, mais tous ces gens qui nous saluent, ça encourage sérieusement ! Résultat, on a la banane. Et c'est encore plus marrant pendant les entrées et sorties des écoles : on se retrouve à pédaler au milieu de centaines d'enfants à vélo, qui ont un moment de doute quand ils réalisent que, par le plus grand des hasards, je suis habillé comme eux : pantalon noir et chemise blanche.


\begin{figure}[h]
\centering
\includegraphics[height=6cm,width=9cm,keepaspectratio]{p3011270.jpg}
\caption*{C'est quand j'ai crevé qu'on a compris que les gens ne mettaient pas des piles de pneus sur le bord de la route pour faire joli, mais pour indiquer qu'ils réparaient les pneus crevés !}
\end{figure}

On trouve des petits stands de nourriture un peu partout, les noix de coco et les jus de canne à sucre nous aident à combattre la déshydratation. Il faut dire que le soleil n'est pas tendre. A partir de 9h du matin, il fait déjà trop chaud (genre, 35\textdegree C à l'ombre), alors à midi, je vous laisse imaginer... Donc l'après-midi, c'est sieste dans la première guesthouse qu'on trouve. La réaction des tenanciers  est généralement une surprise mêlée d'incrédulité : ils nous demandent si on travaille dans le coin, semblent parfois douter qu'on veuille vraiment dormir ici, et il y en a même un qui a cru comprendre qu'on voulait la chambre à l'heure... Mais ils se détendent rapidement une fois qu'on leur montre nos vélos.


\begin{figure}[h]
\centering
\includegraphics[height=6cm,width=9cm,keepaspectratio]{p2291262.jpg}
\caption*{Hello, hellooo, HELLOOOOO !}
\end{figure}

Pour ne pas perdre de temps et profiter au maximum du frais du matin, on se contente de quelques petites bananes au petit déjeuner. Le midi et le soir, on mange dans toutes sortes de cantines locales : habituellement, ce sont quelques casseroles de plats mijotés au choix, et du riz à volonté. On a même mangé un excellent poulet aux fourmis (bien que la cuisinière ait tenté de secouer le poulet) et Marion vous parlera encore des courgettes à la viande hachée. Même en sachant que la faim est la meilleure des épices, on a très bien mangé tout le long du trajet !


\begin{figure}[h]
\centering
\includegraphics[height=6cm,width=9cm,keepaspectratio]{p30212831.jpg}
\caption*{La photo est trompeuse, ces vaches sont très lentes...}
\end{figure}

Sur le trajet, il y a quand même quelques trucs à voir. Le premier fut une montagne avec un temple envahi de singes. Et oui, dans un pays plat comme le Cambodge, un caillou de 50m de haut s'appelle une montagne !


\begin{figure}[h]
\centering
\includegraphics[height=6cm,width=9cm,keepaspectratio]{p3021341.jpg}
\caption*{Oui, appuie toi sur la tête, l'autre main sur la hanche, et maintenant, fais moi ton regard langoureux... parfait, c'est dans la boite !}
\end{figure}


\begin{figure}[h]
\centering
\includegraphics[height=6cm,width=9cm,keepaspectratio]{p3021354.jpg}
\caption*{Il y a aussi des femmes-moines.}
\end{figure}


\begin{figure}[h]
\centering
\includegraphics[height=6cm,width=9cm,keepaspectratio]{p3021362.jpg}
\caption*{Si seulement les humains savaient poser aussi bien que ces singes...}
\end{figure}

Après quelques aller-retour, on fini aussi par trouver la ferme de ver à soie dont tout le monde parle, mais que personne n'est capable de placer sur une carte. C'est en fait un atelier de tissage plus qu'une ferme. Un vétéran du Vietnam, ancien employé d'ONG, s'est mis à son compte pour créer cet atelier. Il a formé des dizaines de jeunes femmes au tissage, et vend leur production aux touristes. On peut voir les jeunes femmes travailler, et se faire prendre en photo avec une de leur production. La partie "ferme" est toute petite. C'est juste pour la démonstration : on voit quelques vers et quelques cocons, et on a une démo de tout le processus, depuis l’œuf, jusqu'au sacrifice de la larve et au filage des cocons. La soie est en fait importée de Chine, ici, il n'y aurait pas assez d'eau pour faire pousser les muriers qui nourrissent les vers.


\begin{figure}[h]
\centering
\includegraphics[height=6cm,width=9cm,keepaspectratio]{p3021302.jpg}
\caption*{Le fondateur, Bud.}
\end{figure}


\begin{figure}[h]
\centering
\includegraphics[height=6cm,width=9cm,keepaspectratio]{p3021365.jpg}
\caption*{Sachant qu'il faut au moins 20 cocons pour faire un fil, au moins ouate-mille fils pour une robe,  et qu'un ver à soie mange 1,345 feuilles par jour, combien de kilos de murier faut-il pour avoir la classe en soirée ?}
\end{figure}

Enfin, histoire de se mettre en jambes pour tous les temples qui nous attendent à Angkor, on visite le site de Sambor Prey Kuk, un des sites de temples pré-angkoriens les plus connus. Nous improvisons un groupe avec Laura et Woussi, rencontrés sur le site, et nous embauchons un guide. En plus de nous fournir quelques données historiques, il évite aussi qu'on se perdre : dans une forêt toute plate, remplie de ruines toutes semblables, c'est vite arrivé ! On voit plein de petits temples en brique, certains bâtiments sont envahis par les arbres, et d'autres par des ... cobras ! Non sérieusement, le guide nous a fortement déconseillé de rentrer dans certains endroits. Vous imaginez bien qu'on est allé voir !


Au détour d'un sentier, on remarque un tronc calciné. Et d'ailleurs, à y regarder de plus près, c'est un tronc non seulement calciné, mais aussi calcinant ! Il fume et il y a des braises ! Et on est au milieu de la saison sèche, dans une forêt pleine de feuilles mortes ! La conversation qui suit est surréaliste, mais néanmoins véridique :

\emph{Nous, inquiets :} "Euh, il y a un arbre qui brûle là !"

\emph{Le guide, calme :} "Ah oui, j'ai vu, les habitants font parfois ça pour déloger des oiseaux dans le tronc."

\emph{Nous, anxieux :} "Mais il faut faire quelque chose !"

\emph{Le guide, mou :} "Oui, il faudrait."

\emph{Nous, limite paniqués :} "La forêt risque de prendre feu !"

\emph{Le guide, amorphe :} "Oui, ça arrive de temps en temps."

\emph{Nous, incrédules :} "Alors... on appelle les pompiers ? Quelqu'un ?"

\emph{Le guide, fatigué :} "Oh bah, il y a bien quelqu'un qui va le faire."

\emph{Nous, résignés :} "C'est pas nous qui travaillons ici, dans une heure, on sera loin..."

\emph{Le guide, plein d'entrain :} "On continue la visite ?"


Je dois aussi vous raconter ce qui peut arriver quand on ne prend pas soin de ses affaires : le matin de notre journée de pause pédalage au milieu du trajet, Marion, prête avant moi comme souvent, part prendre des photos de l'aube. Elle revient avec un masque soucieux sur le visage, et me dit d'une petite voix que la mise au point ne marche plus, tout est flou, et il y a une barre noire dans un coin. Un rapide coup d’œil à l'appareil me permet de constater que l'angle de l'objectif par rapport au boitier est passé d'un standard 90\textdegree  à un inquiétant 75\textdegree /80\textdegree ... C'est pas un angle naturel, et ça ne s'était pas fait sans dégâts :


\begin{figure}[h]
\centering
\includegraphics[height=5cm,width=9cm,keepaspectratio]{p3031370.jpg}
\caption*{Je ne sais pas vous, mais moi, ça me brise le cœur.}
\end{figure}

Pourtant, impossible de se souvenir d'un quelconque évènement traumatisant, ni choc ni chute. Juste le souvenir d'avoir peut-être bourré une bouteille d'eau un peu fort dans le sac à dos la veille au soir. Quoi qu'il en soit, le résultat est là : un objectif inutilisable. Je monte l'objectif à portrait pour tester le boitier, et au moins ça, ça marche. Et on part pour la journée en essayant de penser à autre chose. Mais c'est dur de penser à autre chose quand on a l'appareil avec pas le bon objectif toute la journée dans la main...


\begin{figure}[h]
\centering
\includegraphics[height=5cm,width=9cm,keepaspectratio]{p3031401.jpg}
\caption*{Ne pas trembler...}
\end{figure}

Doutant de pouvoir trouver un réparateur d'objectif, n'ayant plus rien à perdre, on achète un set de mini-tournevis. Je vais tenter la réparation moi-même ! On prépare l'atelier dans la chambre d’hôtel : ventilateur coupé pour éviter de faire circuler la poussière, mains propres, masque anti-postillons, outils prêts. Je commence par dévisser proprement la monture qui a été arrachée, et là, c'est le drame : je vois qu'une nappe électronique a été déchirée, et ce n'est pas avec un mini tournevis que je vais pouvoir réparer ça... Sans plus y croire, je rebranche quand même ce qu'il reste de la nappe, remets les 25 rondelles un peu au hasard, et j'arrive à refermer tout ça proprement. Je remonte l'objectif, et là, il y a ça qui se passe :


\begin{figure}[h]
\centering
\includegraphics[height=5cm,width=9cm,keepaspectratio]{p3031412.jpg}
\caption*{Rare exemplaire de Marion ne souriant pas, ce qui montre le niveau de tension à ce moment là. Mais le fait est que l'objectif remarche !}
\end{figure}

Ça marche. Contre toute attente, alors j'étais prêt à parier que non, l'objectif marche ! Il est toujours un peu tordu (genre 89.5\textdegree ) car j'ai remis les rondelles n'importe comment, problème que je corrigerai un peu plus tard, mais l'autofocus, l'ouverture, tout marche ! C'est un miracle ! On va pouvoir prendre des photos !

Temples d'Angkor, nous voici !


\begin{figure}[h]
\centering
\includegraphics[height=5cm,width=9cm,keepaspectratio]{p3031425.jpg}
\caption*{Ça marche !}
\end{figure}



\chapter{Siem Reap et les temples d'Angkor.}
Nous arrivâmes donc à Siem Reap, après une semaine de vélo. C'était un samedi matin, et après quelques jours sur une route plutôt calme, on retrouve l'agitation d'une grande ville. On se pose dans un bar avec du wifi, et on décide de se récompenser de cette semaine d'efforts en réservant un chouette hôtel, avec piscine, et bien oui,  parce que merde quoi !


\begin{figure}[h]
\centering
\includegraphics[height=6cm,width=9cm,keepaspectratio]{p3061684.jpg}
\caption*{Angkor Vat, le symbole du Cambodge.}
\end{figure}

Les temples d'Angkor, c'est, comme on va le découvrir bientôt, bien plus que l'Angkor Vat. Il y a des centaines de monuments, répartis autour de Siem Reap. Plusieurs jours ne sont pas de trop pour visiter tout ça, et ça tombe bien, ils vendent des tickets de plusieurs jours, et confort ultime : les jours ne sont pas à utiliser d'affilé, on peut par exemple répartir 3 jours sur une semaine, ce qui est bien pour nous, car on a une piscine à rentabiliser !


\begin{figure}[h]
\centering
\includegraphics[height=6cm,width=9cm,keepaspectratio]{p3061694.jpg}
\caption*{Le dernier étage d'Angkor Vat.}
\end{figure}

Un bonheur n'arrivant jamais seul, un couple de français nous aborde dans la rue, et nous demandent : vous voulez un livre sur les temples ? Et nous tendent le livre dont je parlais à Marion le matin même. Et ils s'en vont, nous laissant à peine le temps de leur dire merci. Nous sommes donc fin prêt pour aller nous perdre dans ces vieux cailloux !


\begin{figure}[h]
\centering
\includegraphics[height=6cm,width=9cm,keepaspectratio]{p3061431.jpg}
\caption*{Lever de soleil. Ça ne valait vraiment pas le coup de mettre le réveil...}
\end{figure}

On commence par le plus connu : Angkor Vat, tôt le matin, car le lever de soleil est fameux. On est toujours sur le rythme du vélo, donc pas de souci pour se lever à 5h. A 6h, on est avec la foule, devant le temple le plus connu du Cambodge, et ... rien. Il y a du brouillard, on ne voit pas le soleil, les couleurs restent blafardes, aucun intérêt... On visite le temple jusqu'à ce que la chaleur devienne insupportable, on rentre se rafraichir dans la piscine, et on ressort le soir pour aller voir le tout aussi fameux coucher de soleil. Mais le temple ayant le meilleur point de vue est limité en places, et évidemment, on le découvre seulement une fois au sommet de la colline où se trouve le temple. Et la queue est longue, et les gens qu'on voit au sommet du temple donnent vraiment l'impression de ne pas bouger, les places ne risquent pas de se libérer. On est déçus dans un premier temps, mais quand le soleil disparait dans la brume comme le matin, on se dit qu'on n'a pas loupé grand chose.


\begin{figure}[h]
\centering
\includegraphics[height=6cm,width=9cm,keepaspectratio]{p3061689.jpg}
\caption*{Le temple n'a jamais cessé d'être en activité.}
\end{figure}

Et ce sera ça toute la semaine qu'on va passer à Siem Reap :  pas de lever ni de coucher de soleil, mais un soleil de plomb le reste du temps. Je n'ai pas pu faire les photos dont je rêvais, tant pis...


\begin{figure}[h]
\centering
\includegraphics[height=6cm,width=9cm,keepaspectratio]{p3061691.jpg}
\caption*{Intérieur d'Angkor Vat.}
\end{figure}

Malgré cette (petite) déception niveau éclairage, il faut admettre que les temples sont vraiment impressionnants. Et j'avoue, là, je ne sais pas par où commencer pour raconter tout ce qu'on a vu. Des gens très compétents ont passé des années à étudier ces ruines, à écrire des gros bouquins détaillants le moindre bas-relief (et il y a des kilomètres de bas reliefs !), à compiler des listes de noms de rois terminant en -avarman, et à tenter de retrouver qui a construit quoi.

Comme à beaucoup d'autres endroits dans le monde, ce fut la course à qui a la plus grosse. Et ici, c'est la plus grosse citadelle. Chaque roi, ou presque, a donc construit une ville plus grande que ses prédécesseurs : une grande muraille carrée en pierre pour protéger la ville, puis une plus petite muraille pour protéger le temple, et enfin le temple. Seuls ces éléments ont survécus, car ce sont les seuls construits en pierre : Même les palais royaux étaient en bois. Les temples qui restent, toujours au centre de la ville, étaient considérés comme de véritable demeures divines. Ce ne sont pas des lieux où les fidèles se rassemblent comme nos églises, seuls quelques moines peuvent y entrer. Pas besoin de faire des grandes salles, les temples sont très grands, mais il n'y a que des petites salles et des couloirs étroits. Ils n'ont jamais utilisé la voûte, et on peut se poser la question : Ils ne connaissaient pas la voûte, donc ils n'ont jamais pu construire de grandes salles, ou bien, ils n'ont jamais voulu construire de grande salle, donc ils n'ont jamais eu besoin de voûtes ?

Bref, trêve de bavardages inutiles, place aux photos !


\begin{figure}[h]
\centering
\includegraphics[height=6cm,width=9cm,keepaspectratio]{p3081868.jpg}
\caption*{Ce temple a servi de décor à Tomb Raider.}
\end{figure}


\begin{figure}[h]
\centering
\includegraphics[height=6cm,width=9cm,keepaspectratio]{p3081883.jpg}
\caption*{Les restaurateurs ont décidé de laisser certains temples dans un état de restauration partielle. C'est probablement une des meilleure idées qu'ils aient jamais eu !}
\end{figure}


\begin{figure}[h]
\centering
\includegraphics[height=6cm,width=9cm,keepaspectratio]{p3081902.jpg}
\caption*{Certains murs ne tiendraient plus sans les arbres.}
\end{figure}


\begin{figure}[h]
\centering
\includegraphics[height=6cm,width=9cm,keepaspectratio]{p3081971.jpg}
\caption*{Le temple du Bayon, couvert de visages géants.}
\end{figure}


\begin{figure}[h]
\centering
\includegraphics[height=6cm,width=9cm,keepaspectratio]{p3081977.jpg}
\caption*{Ces danseuses peu habillées sont des Apsaras, des nymphes célestes de la mythologie bouddhiste. A mon avis, c'est surtout un prétexte pour décorer le palais avec des femmes à poil.}
\end{figure}

On fera aussi une balade à vélo loin des temples, sur les conseils avisés de Jean-Louis, le patron de l'hôtel. On voit des champs de lotus, des laotiens très pauvres et très souriants, une ferme de crocodile et un énorme centre de vacance qui semble abandonné.


\begin{figure}[h]
\centering
\includegraphics[height=6cm,width=9cm,keepaspectratio]{p3092037.jpg}
\caption*{Il y avait des canards à vendre, pour donner à manger aux crocos. On n'a pas osé...}
\end{figure}


\begin{figure}[h]
\centering
\includegraphics[height=6cm,width=9cm,keepaspectratio]{p3091993.jpg}
\caption*{Un lodge abandonné.}
\end{figure}


\begin{figure}[h]
\centering
\includegraphics[height=6cm,width=9cm,keepaspectratio]{p3081920.jpg}
\caption*{Encore quelques racines pour la route, mais avec des gens pour donner l'échelle.}
\end{figure}



\chapter{Kratie et les dauphins du Mékong.}
Nous partîmes donc pour Kratie, dans l'est du Cambodge. Et on prend un bus local. Ce bus, c'est le sang du pays : il sert à tout transporter. Les soutes sont pleines de planches de bois, le toit est chargé de tuyaux, et le chauffeur sert aussi de facteur, il s'arrête partout pour charger ou décharger des paquets, et bien sûr, il transporte aussi des gens. Et des poules, bien entendu. Pour les gens qui n'ont jamais ouvert un guide de voyage, sachez-le : ce n'est pas un bus local authentique si il n'y a pas de poules. Cela fait partie des figures imposées avec le vieux édenté souriant, la mémé qui persiste à te parler dans sa langue même si tu ne comprends rien, et les gamins cul-nus. Bref, on a fait 350 bornes en 9h...


\begin{figure}[h]
\centering
\includegraphics[height=6cm,width=9cm,keepaspectratio]{p3132174.jpg}
\caption*{En attendant le ferry.}
\end{figure}




\begin{figure}[h]
\centering
\includegraphics[height=6cm,width=9cm,keepaspectratio]{p3152195.jpg}
\caption*{Elle doit être habituée, elle n'a pas fait un bruit du voyage.}
\end{figure}

En arrivant à Kratie, le transport n'est pas fini, on veut traverser le Mékong pour rejoindre notre homestay sur l'île située en face de la ville. L'endroit est encore à peu près épargné par le monde moderne, et il y règne une ambiance très paisible. La seule route de l'île n'est bétonnée que sur 500m, et permet à peine à deux motos de se croiser. Il n'y a pas encore l'électricité, mais c'est en train de changer : nous avons vu des ouvriers planter les premiers poteaux électriques. Notre homestay est assez moderne : afin d'accueillir des touristes, ils ont investi dans un groupe électrogène qui tourne quelques heures le soir, et ils ont même installé une vraie douche ! Mais la pression est tellement faible qu'on a hésité à leur demander si on pouvait se laver comme eux, dans un bac dehors à l'arrière de la maison, mais je pense qu'ils n'auraient pas compris. Les gens là-bas ont parfois des idées bien arrêtées sur les besoins des touristes...


\begin{figure}[h]
\centering
\includegraphics[height=6cm,width=9cm,keepaspectratio]{p3142187.jpg}
\caption*{Quand il fait chaud, il faut que les vaches se baignent.}
\end{figure}

Kratie est connue surtout pour les dauphins du Mékong. Nous partons donc en scooter pour quelques kilomètres, et arrivons vers la zone de la rivière où ils vivent. Après moult hésitations et discussions, nous décidons de payer un bateau pour nous amener plus près des dauphins, en se disant que si le bateau commence à chasser/embêter les dauphins, on fera demi-tour. Mais tout c'est très bien passé : la bateau a à peine fait 200m avant de couper le moteur et continuer à la rame, et nous étions soudain au milieu des dauphins ! Ce n'est pas non plus Aqualand hein ! Il faut bien tendre l'oreille pour localiser les dauphins quand ils respirent, et espérer qu'ils restent dans le même coin jusqu'au prochain souffle. Les apparitions sont fugaces, et les dauphins ne font pas de saltos ! Mais ils ont l'air de tolérer les bateaux de touristes et restent autour de nous pendant plus de 45 min. On aime à croire que le tourisme a contribué à empêcher l'extinction de ces rares dauphins d'eau douce.


\begin{figure}[h]
\centering
\includegraphics[height=6cm,width=9cm,keepaspectratio]{p3132112.jpg}
\caption*{La petite tâche sombre, c'est un dauphin !}
\end{figure}

En rentrant, toujours en scooter, nous décidons de faire un petit détour, histoire de sortir de la route principale et de voir la campagne. On trouve sur le GPS une petite boucle qui a l'air inoffensive... Disons que j'ai gagné quelques points de compétence en scootercross. On a mis une heure pour faire 4km. Les nids de poules faisaient parfois, sans mentir, 50cm de profondeur sur toute la largeur de la route. Ça ne nous a pas empêché de croiser plusieurs autres mobylettes, confirmant que nous n'étions pas perdus !


\begin{figure}[h]
\centering
\includegraphics[height=6cm,width=9cm,keepaspectratio]{p3132170.jpg}
\caption*{Nid-de-poulesque, non ?}
\end{figure}


\begin{figure}[h]
\centering
\includegraphics[height=6cm,width=9cm,keepaspectratio]{p3142192.jpg}
\caption*{Un lézard s'imaginant que je ne l'avais pas vu !}
\end{figure}



\chapter{Mondolkiri, kiri !}
(J'aurais aussi pu faire Mondolkiri, Mondolkipleure, alors ne vous plaignez pas trop...)

Nous partîmes donc pour Sen Monorom, ville principale de la région de Mondolkiri. Nous nous étions levés très tôt pour être sûrs de ne pas louper le bac nous permettant de traverser la rivière : il y en a un par heure, mais on ne sait jamais quand dans l'heure : pas très pratique quand on a des horaires à tenir ! A 7h30, le bus part comme prévu, fait des détours, récupère des gens et des paquets, et... revient à la station de bus ! D'autres gens montent, on attend encore un peu, et on finit par partir vraiment à 8h30... On a vraiment l'impression de s'être fait voler une heure de sommeil !


\begin{figure}[h]
\centering
\includegraphics[height=6cm,width=9cm,keepaspectratio]{p3162213.jpg}
\caption*{Appel aux biologistes/fleuristes : c'est quoi ?}
\end{figure}

La province du Mondolkiri est un des rares endroits du Cambodge à avoir de la montagne. Enfin, on voit un peu de relief, et de la forêt tropicale, même si le coté luxuriant n'est pas au top pendant la saison sèche. Les températures redeviennent acceptables, il ne fait plus que 30\textdegree C et on peut dormir sans ventilo ! Nous logeons dans un bungalow à quelques mètres de la forêt, avec tout ce que ça implique en terme d'araignées, de geckos et autres bestioles curieuses qui n'ont pas compris que c'est notre bungalow. Il y a aussi les cigales aux hormones. C'est un peu comme une cigale normale, mais qui a fait de la muscu, a probablement pris des produits dopants, afin de pouvoir GUEULER plus fort que ses 30000 voisines. Elles font tellement de bruit que les conversations tournent à la version dialogue de sourd en boite de nuit genre :

\emph{"Bonne ambiance, non ?"}

\emph{"Quoi ?"}

\emph{"Je disais : BONNE AMBIANCE !"}

\emph{"Hein, ah non, pas du tout, c'est juste un gecko"}

\emph{"Ah ouais, moi pareil, je l'adore"}


\begin{figure}[h]
\centering
\includegraphics[height=6cm,width=9cm,keepaspectratio]{p3172429.jpg}
\caption*{Un gecko plutôt maousse !}
\end{figure}




\begin{wrapfigure}{l}{0.45\textwidth}
\centering
\includegraphics[width=0.4\textwidth]{p3172235.jpg}
\caption*{A moi, la banane !}
\end{wrapfigure}

Bon, pour être honnête, on n'est pas venu par ici pour le frais, ni pour les cigales, mais pour les éléphants ! C'est devenu le truc touristique du coin, et il y a plusieurs agences/sanctuaires/fondations dédiées au financement de la sauvegarde des éléphants par le tourisme. Après un peu de recherche, on choisit une agence qui caresse notre conscience dans le bon sens du poil, en nous garantissant que chez eux, c'est vachement mieux, les éléphants sont libres, ne travaillent plus, ne portent pas de gens, et ne viennent voir les touristes que s'ils le veulent bien, et que les autres agences, c'est le diable, ils mangent même des éléphanteaux crus.

On va donc joyeusement rendre visite aux éléphants dans la forêt, sans oublier de prendre au passage un bon stock de bananes. Faut pas rêver, les éléphants ne vont pas venir pour nos beaux yeux. Et soudain, au milieu de la forêt, on voit deux éléphants qui se dirigent nonchalamment vers nous. Pas de barrière, pas de fossé, pas de vitre, juste deux éléphants qui se dirigent vraiment dans notre direction, ou tout du moins en direction des bananes qui sont dans nos mains. C'est impressionnant, ça fait une boule au ventre, on les entend respirer, on sent leur pas, on les entend comme "ronronner", ils font des sons tellement graves qu'on les ressent plus qu'on ne les entend.


\begin{figure}[h]
\centering
\includegraphics[height=6cm,width=9cm,keepaspectratio]{p3172288.jpg}
\caption*{Voici "Princesse", une éléphante tellement classe qu'elle ne va pas s'abaisser à mettre elle même de la nourriture dans sa bouche : elle a des employés pour ça. Mieux : des gens paient pour avoir ce privilège !}
\end{figure}

Quelques instants plus tard, nous voilà avec des bananes cachées dans le dos, à nourrir les éléphants une banane après l'autre. Tous n'ont pas la même méthode : il y a celle (il n'y a que des éléphantes) qui prend la banane délicatement avec sa trompe, celle qui ne mange que ce qu'on lui met directement dans la bouche, celle qui essaie de chopper directement toutes les bananes dans ton dos, et la boss qui bouscule sans ménagement les autres éléphantes afin d'accéder aux bananes en premier. Petit à petit, on s’enhardit, on s'approche, on les touche, on sent la force impressionnante de leur trompe et la rugosité de leur peau. Les éléphants sont très doux dans leurs gestes, et certains semblent même un peu craintifs, et reculent si l'on s'approche d'eux.


\begin{figure}[h]
\centering
\includegraphics[height=6cm,width=9cm,keepaspectratio]{p3172275.jpg}
\caption*{Alors, elle sont où les bananes, hein, elle sont où ? Elles sont parties les bananes ?}
\end{figure}

Ce sont tous des éléphants qui ont été dressés pour travailler. Et il ne faut pas y aller de main morte pour dresser un animal qui peut tuer un homme d'un seul coup ! Ils ont été rachetés par le sanctuaire pour leur permettre de vieillir tranquillement et leur rendre un peu de liberté dans la forêt que le sanctuaire a aussi rachetée. Nous n'avions aucune idée du prix d'un éléphant : autour de 50000\$ ! Nous ne savions pas non plus que la dernière naissance d'éléphant au Cambodge remonte à une vingtaine d'année, car personne ne peut se permettre l'investissement en temps d'éléphant : deux ans de gestation puis deux ans d'allaitement, ça fait 4 années sans travail pour une éléphante, sans compter l'année nécessaire pour que les deux éléphants tombent amoureux, dixit le guide. C'est pourquoi ce sanctuaire est en train de se lancer dans la reproduction d'éléphant et cherche activement un mâle. Si vous connaissez un éléphant célibataire, faites passer le mot !


\begin{figure}[h]
\centering
\includegraphics[height=6cm,width=9cm,keepaspectratio]{p3172353.jpg}
\caption*{"Mes nouvelles copines !"}
\end{figure}

Ensuite, c'est l'heure du bain : on se jette dans la rivière, et on attend les deux éléphants les plus sympas, les seuls assez doux et attentifs pour pouvoir se baigner avec des touristes. On nous donne des balais-brosse et des seaux, et c'est parti pour frotter. Et il faut y aller, ils aiment qu'on frotte fort, et si on ne frotte pas assez fort à leur gout, ils se détournent de nous et cherchent quelqu'un de plus vigoureux !


\begin{figure}[h]
\centering
\includegraphics[height=6cm,width=9cm,keepaspectratio]{p3172365.jpg}
\caption*{Et que ça brille !}
\end{figure}

Et pour finir notre séjour à Mondolkiri, un petit trek dans la jungle ! Notre guide était super et s'arrêtait tous les 5m pour nous montrer des champignons bizarres, des fourmilières, des noix de cajou... on a même deviné un toucan au loin ! Et son secret pour faire des kilomètres dans la jungle sans trop souffrir de la chaleur ? C'est simple : il suffit de connaitre les rivières, les cascades et se baigner dès qu'on peut ! 3 baignades en 20km, ça aide !


\begin{figure}[h]
\centering
\includegraphics[height=6cm,width=9cm,keepaspectratio]{p3182452.jpg}
\caption*{Une cascade perdue au milieu de la jungle.}
\end{figure}


\begin{figure}[h]
\centering
\includegraphics[height=6cm,width=9cm,keepaspectratio]{p3172216.jpg}
\caption*{Ils font beaucoup de culture sur brûlis.}
\end{figure}


\begin{figure}[h]
\centering
\includegraphics[height=6cm,width=9cm,keepaspectratio]{p3162211.jpg}
\caption*{Ce qui est bien avec les libellules, c'est que c'est assez facile à prendre en photo.}
\end{figure}







\chapter{Arrivée au Laos : calme et chaos.}
\subsubsection{Le Chaos}
Nous partîmes donc en direction du Laos. Après une nuit à Stung Treng, dont le seul intérêt pour le touriste est ... d'être à coté du Laos, nous prenons un bus pour les 4000 îles, au Laos. Nous avions lu que la frontière était gangrénée par la corruption et, bien en avance, nous avons décidé d'essayer d'y résister. C'est de la petite corruption omniprésente : chaque coup de tampon est soumis à une taxe de 1 ou 2\$, et ce, des deux cotés de la frontière. C'est bien entendu complètement illégal, mais la plupart des touristes ne se posent aucune question. Dans le bus qui nous amène aux 4000 îles, nous sommes sept. Nous discutons bien sûr de cette histoire, et trouvons une autre fille qui a envie de lutter un peu contre ce système, les autres ne sont pas au courant ou ne veulent pas prendre de risques avec la douane.


\begin{figure}[h]
\centering
\includegraphics[height=6cm,width=9cm,keepaspectratio]{p3212508.jpg}
\caption*{Allégorie (subtile) de la corruption.}
\end{figure}

Première étape : tampon de sortie du Cambodge. Il faut tout d'abord trouver où obtenir ce tampon, car rien n'est indiqué. Il y a bien une cabane encerclée de touristes, mais quand on leur demande, ils nous disent qu'ils rentrent au Cambodge, alors que nous, on en sort ! Bon, en fait, c'est au même endroit. Il n'y a pas de queue, les gens arrivent de partout, et on joue des coudes pour arriver à passer le passeport à travers un guichet. On entend "two dollars". Évidemment, on ne voit rien ni personne : les douaniers ont ajouté des caches derrière les vitres...


\begin{figure}[h]
\centering
\includegraphics[height=6cm,width=9cm,keepaspectratio]{p3212466.jpg}
\caption*{Allégorie du douanier qui se cache.}
\end{figure}

On essaie de parlementer, de dire que non, mais c'est difficile quand on est entouré de gens qui crachent la thune. Malgré tout, à force d'insister, on ne paye que 1\$ pour deux passeports.

Deuxième étape : obtention du visa Laotien. Il y a une affiche très claire qui indique le prix par pays, et indique qu'un dollar en plus est demandé pour frais de services. On paie sans discuter et on obtient le visa.


\begin{figure}[h]
\centering
\includegraphics[height=6cm,width=9cm,keepaspectratio]{p3232578.jpg}
\caption*{Allégorie du touriste plein de principes qui se prend pour un prédateur alors qu'il n'est qu'un tout petit gecko de rien du tout.}
\end{figure}

Troisième étape : le tampon d'entrée au Laos. C'est un autre bureau juste à coté du bureau des visas, et ils sont en possession de nos passeports. Ils nous appellent un par un, et demandent 2\$ à chaque fois. Manque de bol, ce sont les 4 qui ne veulent pas négocier qui sont appelés les premiers. Quand arrive notre tour, nous tentons de dire que nous n'allons pas payer. Ça les fait doucement sourire, ils ne répondent pas et mettent ostensiblement nos passeports dans un tiroir. Marion s'occupe de la négociation, et décide de les faire un chier un peu, et leur parle encore et encore, sans réponse. On essaie de négocier un dollar pour les deux passeports, mais non. Pendant ce temps, le bus nous attend. Une lueur d'espoir apparait quand une nouvelle douanière apparait : peut-être a-t-elle une position différente vis-à-vis de la corruption ? Nos espoirs sont rapidement douchés quand elle commence à nous expliquer que si on persiste à ne pas vouloir payer, le bus va partir sans nous. Ah ! A d'autres ! Il est clair qu'elle bluffe, on ne va pas se laisser avoir si facilement. Et en effet, au bout de bien une heure de négociation, elle nous explique qu'elle accepte, à condition que ceux qui ont déjà payé aillent en direction du bus pour ne pas être témoins de notre arrangement. On fait donc ça, et on rejoint les 4 autres un instant plus tard, avec nos passeports tamponnés, tout content d'avoir réussi à un peu résister. Et nous avons économisé deux dollars ! Ce n'est pas grand chose, mais c'est pour le principe !


\begin{figure}[h]
\centering
\includegraphics[height=6cm,width=9cm,keepaspectratio]{p3212469.jpg}
\caption*{Allégorie du piège à fric qu'est la douane.}
\end{figure}

Et là, c'est le drame : pas de bus. Il est vraiment parti sans nous ! Il y a vraiment peu de circulation, et les quelques véhicules qui passent refusent de nous amener à la prochaine ville 10km plus loin. Après 4h d'attente, et après avoir vu plusieurs fois les douaniers passer devant nous, l'air satisfait, on finit par prendre un taxi. Comme on se sentait coupables pour les autres qui étaient aussi coincés là par notre faute, on a payé leur part. Résultat : quinze dollars et une après-midi de perdue. Le seul point positif : pendant l'attente, on a rencontré Stéphane, un photographe cyclotouriste très sympa.


\begin{figure}[H]
\centering
\includegraphics[height=5cm,width=9cm,keepaspectratio]{p3222529.jpg}
\caption*{Allégorie du soleil couchant de nos illusions déçues...}
\end{figure}


\subsubsection{Le calme}


\begin{wrapfigure}{l}{0.55\textwidth}
\centering
\includegraphics[width=0.5\textwidth]{p3222530.jpg}
\caption*{On tente de se fondre dans le paysage.}
\end{wrapfigure}


Nous arrivâmes donc aux 4000 îles, en essayant d'oublier le passage de la frontière, histoire de profiter un peu du Laos. On passe sans s'arrêter à travers l'île la plus connue, Don Det, non sans profiter des odeurs disons... illégales, émanant des party-hostels, et on se dirige vers Don Khon, une île bien plus calme. Et quand on dit calme, au Laos, comment dire, c'est TRÈS calme ! Le sport national, c'est la sieste ! Bon, ça c'est une blague. En réalité, c'est plutôt la pétanque le sport national, ce qui vous donne aussi une idée du niveau d'énergie du pays. On a l'impression de passer notre temps à réveiller des gens : il y a des hamacs partout, sous leur maisons sur pilotis, à l'arrière des tuk-tuks, derrière les comptoirs des magasins...


Avec la chaleur qui règne, c'est facile de se mettre au même rythme. On passe quelques jours sur place à boire des fruitshakes glacés, et à faire un peu de vélo pour aller visiter les quelques chutes d'eau autour des îles. Nous sommes rejoints par Stéphane, le photographe rencontré à la frontière Cambodge/Laos. Quand on en a marre du vélo et que la lumière est mauvaise pour les photos, on barbote dans le Mékong, en analysant bien les tourbillons afin de ne pas de faire emporter.


\begin{figure}[h]
\centering
\includegraphics[height=9cm,width=12cm,keepaspectratio]{p3222546.jpg}
\caption*{Stéphane en plein travail.}
\end{figure}

En découvrant le Mékong, les premiers explorateurs l'imaginaient comme le futur axe de transport principal de l'Asie. C'est en arrivant aux 4000 îles que leurs espoirs ont été anéantis : les rapides sont impossibles à traverser en bateau. Quelques années plus tard, les Français se sont mis en tête de construire une ligne de chemin de fer à la place. Le projet n'a jamais abouti, mais il reste encore quelques ponts destinés à faire partie du projet, dont celui qui lie l'île de Don Det à Don Khon. Du coup, un mec a tenté de mettre un péage pour les touristes sur ce pont, mais vu l’énergie qu'il met à sortir de son hamac, c'est plus facile de continuer son chemin en l'ignorant...


\begin{figure}[h]
\centering
\includegraphics[height=9cm,width=12cm,keepaspectratio]{p3222569.jpg}
\caption*{Du coton et du soleil.}
\end{figure}



\chapter{Tad Lo, le temps ralenti et ... s'arrête ?}
Nous arrivâmes donc à Tad Lo, petit village du plateau de Bolaven. Après le rythme effréné des 4000 îles, ça nous fait du bien de nous poser un peu. Oui, non, je sais, vous allez me dire que vous avez lu le précédent article, et qu'aux 4000 îles, ce n'est certainement pas un rythme effréné qui règne, alors faudrait voir à pas prendre ses lecteurs pour des truffes ! Et vous n'avez pas tort. Mais le fait est que le niveau moyen d'énergie baisse encore un peu.


\begin{figure}[h]
\centering
\includegraphics[height=6cm,width=9cm,keepaspectratio]{p3262808.jpg}
\caption*{Se baigner, ou ne pas se baigner ? Les avis divergent...}
\end{figure}

Par exemple, la guesthouse qui nous héberge ferme le portail d'entrée sur les coups de 20h30... Ou encore la fameuse cascade située à quelques kilomètres du village qu'on est allé visiter à pied, et bien cette cascade ne s'est même pas donné la peine de faire couler un peu d'eau pour nous : nous avons admiré à la place une grande falaise toute sèche !


\begin{figure}[h]
\centering
\includegraphics[height=6cm,width=9cm,keepaspectratio]{p3242602.jpg}
\caption*{Regardez bien : à droite, on voit un petit filet d'eau !}
\end{figure}


\begin{figure}[h]
\centering
\includegraphics[height=6cm,width=9cm,keepaspectratio]{p3242606.jpg}
\caption*{Une maison à l'ancienne.}
\end{figure}


\begin{figure}[h]
\centering
\includegraphics[height=6cm,width=9cm,keepaspectratio]{p3242608.jpg}
\caption*{Ils n'ont pas l'eau courante, mais ils ont la télévision satellite !}
\end{figure}

Vu la température ambiante, une des activités favorites des touristes aussi bien que des locaux, c'est la baignade dans la rivière. Ça change des 4000 îles non ? Mais pour le moment, on n'est pas lassé. D'autant plus qu'ici, il y a aussi les éléphants qui se baignent.


\begin{figure}[h]
\centering
\includegraphics[height=6cm,width=9cm,keepaspectratio]{p3252682.jpg}
\caption*{"Bonne nouvelle : mes vieux m'ont lâché les clés de l'éléphant pour la soirée. Je passe te prendre vers 20h ?"}
\end{figure}

Ce n'est pas vraiment la même ambiance qu'à Mondolkiri en revanche. Un hôtel de luxe propriétaire de deux éléphants les fait se baigner tous les jours. Cet hôtel, on y avait fait un petit tour à notre arrivée alors qu'on cherchait une chambre : on en est vite ressorti en essayant de garder un air digne après avoir vu les tarifs, et on est allé se cacher dans un bungalow tout à fait confortable et surtout 25 fois moins cher. Et cet hôtel donc, possède deux éléphants que les touristes peuvent louer le temps d'une balade. La plupart du temps, ils sont attachés à une chaine de 10m de long, et 10m, pour un éléphant, c'est pas énorme.


\begin{figure}[h]
\centering
\includegraphics[height=6cm,width=9cm,keepaspectratio]{p3242610.jpg}
\caption*{On t'a vu derrière ton arbre ! (J'essaie de blaguer, mais ça fait mal au cœur en vrai...)}
\end{figure}

Leur seul autre moment de liberté, c'est la baignade : ils viennent l'un après l'autre avec leur mahout sur le cou, qui les dirige en appuyant sur leurs oreilles avec les pieds, vers un petit endroit de la rivière où l'eau est un peu plus profonde. Et là, désormais debout sur l'éléphant dans l'eau, le mahout les fait s'immerger en entier et les frotte un peu. Puis les éléphants retournent sur la berge où ils sont récompensés avec de la canne à sucre et des bananes.

Le propriétaire, un expat américain, est dans les parages et raconte avec une verve toute ... américaine, à quel point ses éléphants sont bien traités et heureux, tout en démontrant qu'ils sont gentils et dociles en leur attrapant la trompe. Dans le doute, je lui fais part d'une observation : j'ai remarqué qu'un des mahouts avait dans la main une pointe fichée dans un petit morceau de bois. Se pourrait-il que ce soit la version portable/discrète du fameux instrument de dressage tant décrié ? Sûr de lui, il me répond que j'ai sûrement vu autre chose, probablement une brindille que le mahout a enlevé du dos de l'éléphant, ces grosses bêtes ayant tendance à se jeter sur le dos poussière, herbes sèches et autre bidules pour se protéger du soleil. N'étant pas là pour polémiquer, et après tout, je n'ai pas vu le mahout utiliser le pic, je lâche l'affaire, quand bien même je suis sûr de savoir faire la différence entre un pic et une brindille à 2m de distance. De plus, je n'y connais pas grand chose en éléphant, peut-être est-ce une mesure de sécurité : un éléphant pourrait si facilement blesser un touriste...


\begin{figure}[h]
\centering
\includegraphics[height=6cm,width=9cm,keepaspectratio]{p3242619.jpg}
\caption*{Un éléphant dans une flaque.}
\end{figure}

Le lendemain, à notre grande surprise, le propriétaire nous hèle en nous voyant nous promener, et vient nous présenter ses excuses ! Après notre conversation, il est allé voir son mahout qui utilisait effectivement un pic pour forcer l'éléphant à s'immerger en entier dans l'eau, ce qui les a amenés à avoir une petite discussion sur ce qu'on peut faire ou pas avec un éléphant devant des touristes ! J'avoue, pendant un petit moment, j'ai eu l'impression d'être Brigitte Bardot sauvant des bébés phoques ! Mais rassurez vous, ça m'a passé.


\begin{figure}[h]
\centering
\includegraphics[height=6cm,width=9cm,keepaspectratio]{p3252679.jpg}
\caption*{Je n'ai même pas eu besoin de zoom.}
\end{figure}

En remontant un peu la rivière, on voit à quel point c'est une partie intégrante de la vie du village. Pour beaucoup de monde, c'est leur salle de bain : ils s'y lavent et font aussi leur lessive. C'est aussi le domaine des pêcheurs. Chacun sa méthode : il y a ceux qui choppent les poissons à la main sous les rochers, en s'aidant d'un masque de plongeur, il y a les pièges permanents, ceux qui tendent des filets, ceux qui jettent des filets, et ceux qui s'assoient en haut d'une cascade pour pêcher à la ligne. On a frissonné de nombreuses fois en voyant des petits gamins traverser le courant un mètre en amont de la cascade pour aller rejoindre un meilleur spot de pêche !


\begin{figure}[h]
\centering
\includegraphics[height=6cm,width=9cm,keepaspectratio]{p3262807.jpg}
\caption*{Il y a dix mètres de chute d'eau, juste là, à quelques pas...}
\end{figure}


\begin{figure}[h]
\centering
\includegraphics[height=6cm,width=9cm,keepaspectratio]{p3262812.jpg}
\caption*{Ici, certains gamins font leur lessive à la main, dans la rivière.}
\end{figure}


\begin{figure}[h]
\centering
\includegraphics[height=6cm,width=9cm,keepaspectratio]{p3252703.jpg}
\caption*{Le fameux  lancer de filet.}
\end{figure}


\begin{figure}[h]
\centering
\includegraphics[height=6cm,width=9cm,keepaspectratio]{p3262796.jpg}
\caption*{Ici, un appareil photo est une machine à faire sourire les gens :-)}
\end{figure}


\begin{figure}[h]
\centering
\includegraphics[height=6cm,width=9cm,keepaspectratio]{p3262766.jpg}
\caption*{Un bout de bambou, un fil, un hameçon, et un short. What else ?}
\end{figure}



\chapter{Thakhek et "La Boucle"}
Nous arrivâmes donc à Thakhek, ville dont le seul intérêt est d'être le point de départ de la fameuse "boucle". Cette boucle est un circuit classique parmi les backpackers : entre 3 et 5 jours à scooter dans les montagnes et les karsts, avec comme point d'orgue la visite de la fameuse grotte de Kong Lore !


\begin{figure}[h]
\centering
\includegraphics[height=6cm,width=9cm,keepaspectratio]{p33028371.jpg}
\caption*{Grotte de Kong Lore.}
\end{figure}

On se prend un scooter un peu classe ce coup-ci. Si on a des bornes à faire, autant ne pas les faire avec la première bouse chinoise venue. On hérite donc d'un Honda 125cm3 rutilant ! Eh ouais, fini de rire maintenant ! Et on part pour la boucle. Les premiers kilomètres se font dans un paysage de karsts qui nous rappellent étrangement le Vietnam. Bon, pas si étrangement que ça en fin de compte, à y regarder de plus près, on n'est qu'à quelques kilomètres des karsts vietnamiens.


\begin{figure}[h]
\centering
\includegraphics[height=6cm,width=9cm,keepaspectratio]{p33028661.jpg}
\caption*{A la sortie de la grotte de Kong Lore.}
\end{figure}

Il y a des tas de cavernes à visiter, toutes payantes, et on fait notre première pause à la caverne du Bouddha, un peu au hasard. On se retrouve dans une toute petite caverne, de peut-être 20m\textsuperscript{2}, remplie de statues de Bouddha. On dirait une brocante, c'est bordélique, moche, et on essaie d'oublier qu'on a payé et fait 1h de piste poussiéreuse pour ça...


\begin{figure}[h]
\centering
\includegraphics[height=6cm,width=9cm,keepaspectratio]{p32928131.jpg}
\caption*{Voilà ce qui se passe quand on inonde une forêt !}
\end{figure}

Heureusement, les paysages sont vraiment magnifiques, et on se retrouve rapidement sur des portions de routes plutôt tranquilles. Après les karsts, ça commence à monter, et on passe à coté d'un barrage tout neuf. Comme tout ce qui est neuf au Laos, c'est construit par des étrangers, ici, des Chinois. Des informateurs, qui préfèrent rester anonymes, nous ont confié que les Français avaient fait une proposition pour le barrage, mais qu'elle avait été refusée car elle impliquait d'apprendre aux Laotiens à exploiter le barrage, alors que les Chinois s'occupaient de tout. Difficile de savoir quelle est la part de vérité dans cette explication, mais l'idée colle bien à la façon dont on perçoit le Laos (et c'est un peu triste). Le fait est que la Chine, la Thaïlande et le Vietnam se battent pour exploiter au mieux les ressources du Laos, pour le seul bénéfice d'une petite élite corrompue (et ça c'est encore plus triste).


\begin{figure}[h]
\centering
\includegraphics[height=6cm,width=9cm,keepaspectratio]{p32928161.jpg}
\caption*{Ambiance !}
\end{figure}

La route serpente ensuite sur un dédale de digues, entre les nouvelles zones inondées. Des milliers de squelettes d'arbres pointent leur troncs hors de l'eau de part et d'autre de la route. L'ambiance est halloweenesque, il ne manque qu'un voile de brume sur l'eau et des cris de corbeaux pour compléter l'ambiance. (C'est mon blog, j'invente des mots si je veux)


\begin{figure}[h]
\centering
\includegraphics[height=6cm,width=9cm,keepaspectratio]{p33028321.jpg}
\caption*{A pied dans la grotte.}
\end{figure}

Enfin, nous arrivons à Kong Lore. La grotte se traverse en bateau. Elle n'est pas éclairée, donc tout le monde a sa petite lampe frontale. J'ai bien essayé de prendre des photos, mais dans un bateau qui bouge dans une grotte, une frontale, ça ne fait pas assez de lumière. Croyez moi cependant : c'est classe ! Le seul moment où j'ai pu prendre quelques photos, c'est à la pause à mi-parcours. Le bateau nous débarque, et on suit un petit chemin parmi des concrétions, et c'est le seul endroit éclairé. Arrivés de l'autre coté, on fait une petite pause, et ... on repart dans l'autre sens. La grotte est pour le moment le seul passage pour rejoindre cette partie du monde.


\begin{figure}[h]
\centering
\includegraphics[height=6cm,width=9cm,keepaspectratio]{p33028741.jpg}
\caption*{On se croirait presque en boite de nuit.}
\end{figure}

Après trois jours et 500 km de scooter, nos fesses en compote sont contentes de prendre le bus de nuit pour Vientiane. Un siège de bus, même de bus local, c'est quand même plus confortable qu'un scooter...


\begin{figure}[h]
\centering
\includegraphics[height=6cm,width=9cm,keepaspectratio]{p3302831.jpg}
\caption*{Notre guide/capitaine de bateau.}
\end{figure}



\chapter{Vientiane, une capitale à l'image de son pays}
Nous arrivâmes donc à Vientiane, la capital du Laos. D'après la légende, c'est la capitale la plus relax du monde. Déjà, elle ne fait "que" 200 000 habitants, et en plus, elle est pleine de Laotiens. Le centre ville se traverse en une petite demi-heure à pied, et la circulation est plutôt calme. Au programme : famille, paperasse et shopping.


\begin{figure}[h]
\centering
\includegraphics[height=6cm,width=9cm,keepaspectratio]{p4032917.jpg}
\caption*{Le frangipanier, un des symboles nationaux du Laos.}
\end{figure}

Nous sommes restés quelques jours ici pour diverses raisons techniques. Tout d'abord, refaire le plein de pain au chocolat. Ensuite, faire le visa thaïlandais. D'ailleurs, l'ambassade de Thaïlande à Vientiane est probablement le lieu le plus actif de tout le Laos. Alors que les rues alentours étaient désertes, on a trouvé un lieu grouillant de vie : c'est un des endroits favoris des touristes pour faire un "visa run" depuis la Thaïlande, la ville n'est qu'à 20km de la frontière. C'est donc rempli de gens qui ne veulent qu'une chose : rentrer le plus vite possible en Thaïlande !


\begin{figure}[h]
\centering
\includegraphics[height=6cm,width=9cm,keepaspectratio]{p4012896.jpg}
\caption*{Coucher de soleil sur la Thaïlande, de l'autre coté du Mékong.}
\end{figure}

On a aussi cherché une sacoche d'appareil photo : la sangle a lâché quelques jours avant, et je ne peux plus le porter en bandoulière. Cela c'est avéré plus difficile que prévu : malgré tous nos efforts, impossible de trouver une boutique photo. Des téléphones, des laves-linges et des voiture autant qu'on veut, mais d'appareil photo point. Soit. Une sacoche toute bête fera aussi l'affaire.


\begin{figure}[h]
\centering
\includegraphics[height=6cm,width=9cm,keepaspectratio]{p4042924.jpg}
\caption*{Pha That Luang, la stupa la plus célèbre du Laos.}
\end{figure}

On se confronte une fois de plus à l'esprit Laotien quand on essaie d'aller voir un film. Il n'y a qu'un seul cinéma au Laos, dans un centre commercial construit par des chinois. On arrive une heure avant la séance, et on nous dit que c'est trop tôt. Bon, on attend. 20mn avant la séance, on retente, et là, le mec semble un peu gêné. On ne comprend pas trop au début, mais à force d'insister, il finit par nous dire en passant par plusieurs chemins détourné qu'il n'y a pas de séance, mais qu'on peut revenir pour la prochaine dans 4h... (Soit disant qu'il n'avait pas la clé, parce que la belle-sœur de son chat a eu une attaque de lait sur le feu en prison, ou un truc dans le genre). C'est un peu le problème de la culture asiatique, où le fait de dire non est très difficile, et de notre coté, en gros lourdauds d'européens, on a du mal à comprendre leurs messages subtils. Résultat, on insiste encore et encore jusqu'à les acculer et les obliger à dire un gros "non" bien franc.


\begin{figure}[h]
\centering
\includegraphics[height=6cm,width=9cm,keepaspectratio]{p4012913.jpg}
\caption*{C'est la saison des fourmis volantes !}
\end{figure}

La grosse raison qui nous a fait rester à Vientiane, c'est aussi l'arrivée du papa de Marion. On l'a retrouvé à l'aéroport de Vientiane et il nous a suivi pendant quelques jours dans le nord du Laos. Mais on a commencé par un petit tour en scooter autour de Vientiane, en direction du Bouddha Parc : une sorte de palais idéal du facteur cheval, mais rempli de sculptures en béton sur de thème de la mythologie Bouddhiste. C'est kitch à souhait, il y en a dans tous les coins, et on n'a pas assez d'yeux pour remarquer tous les détails.


\begin{figure}[h]
\centering
\includegraphics[height=6cm,width=9cm,keepaspectratio]{img_20160405_135553091.jpg}
\caption*{A fond dans la campagne !}
\end{figure}


\begin{figure}[h]
\centering
\includegraphics[height=6cm,width=9cm,keepaspectratio]{p4052933.jpg}
\caption*{Échantillon du Bouddha Parc.}
\end{figure}

Sur le trajet, on croise aussi l'arc de triomphe de Vientiane. Il a été copié ostensiblement sur celui de Paris, mais en plus grand (prend ça dans tes prérogatives post-coloniale, France !), et en béton.


\begin{figure}[h]
\centering
\includegraphics[height=6cm,width=9cm,keepaspectratio]{p4042930.jpg}
\caption*{Sans les palmiers, on se croirait à Paris, non ?}
\end{figure}

Et le soir venu, ce fut un peu Noël pour nous : le papa de Marion a ouvert sa hotte, et nous a sorti fromage, saucisson et chocolat. Depuis le temps qu'on en rêvait !





\chapter{Luang Prabang, des moines, et des cascades.}
Nous arrivâmes donc à Luang Prabang. En tant qu'ancienne capitale du Laos, c'est une ville riche en histoire, de plus, les paysages alentours sont magnifiques et il y a quelques chutes d'eau qui valent vraiment le voyage. Et croyez-nous, depuis qu'on est parti, on en a vu quelques unes des chutes d'eau... Tout ceci fait de la ville un incontournable du Laos.


\begin{figure}[h]
\centering
\includegraphics[height=6cm,width=9cm,keepaspectratio]{p4073069.jpg}
\caption*{C'est biquet, non ?}
\end{figure}




\begin{figure}[h]
\centering
\includegraphics[height=6cm,width=9cm,keepaspectratio]{p4062972.jpg}
\caption*{A vélo dans la jungle !}
\end{figure}

Nous avons fait le voyage en bus de nuit. Un première pour le papa de Marion ! Nous voilà à 6h du matin dans les rues de la ville à chercher notre hôtel. Comme d'habitude, nous l'avions réservé en avance et soigneusement noté l'emplacement sur le GPS. Mais ce coup ci, impossible de le trouver. Nous qui comptions impressionner mon beau père avec notre expérience de voyageur aguerri, c'est l'échec... Rassurez-vous, on n'a pas dormi dans la rue pour autant. Dans ce coin, il suffit de secouer un lampadaire et dix hôtels en tombent.


\begin{figure}[h]
\centering
\includegraphics[height=6cm,width=9cm,keepaspectratio]{p4062960.jpg}
\caption*{Je n'avais pas de banane pour donner l'échelle, mais je vous jure, c'est un p****n de gros papillon !}
\end{figure}

Nous louons des vélos, et nous voilà partis vers une petite chute d'eau dans la jungle. Bon, en vrai, c'est la saison sèche, et la chute d'eau... voilà quoi. L'avantage, c'est qu'on était les seuls sur le sentier, et que la jungle est magnifique malgré tout, surtout une fois que le leader de la troupe a enlevé les toiles d'araignées qui barrent le chemin.


\begin{figure}[h]
\centering
\includegraphics[height=6cm,width=9cm,keepaspectratio]{p4062966.jpg}
\caption*{Tout ça pour une cascade à sec...}
\end{figure}

Le soir, c'est l'heure pour un petit sandwich avec un milkshake au marché de nuit. Il y a des dizaines de stands alignés, mais ils font tous les mêmes sandwiches et milkshakes, alors on choisi le stand ou on a les plus grands sourires :-)
Nous avions presque fini le repas quand soudain, un petit coup de vent ! Mais un petit coup de vent un peu louche, genre, il a une odeur de pluie, et il y a une rumeur qui agite encore l'air après son passage, les arbres bruissent plus fort que d'ordinaire. On sent un changement d'ambiance très clair, et certain marchands ayant plus de nez que leurs voisins commencent à ranger leur stand. Ils font bien, car quelques instants plus tard, le grand frère du coup de vent débarque et ce coup-ci, retourne quelques chapiteaux ! Et là, ce fut une mini panique, et deux stratégies s'opposaient :
\emph{"On va tous mourir, donc je range mes affaires avant que tout s'envole en poussant des petits couinements"}
contre
\emph{"On va tous mourir mais je n'ai pas encore fait mon chiffre de la soirée alors je m'accroche à mon chapiteau pour l'empêcher de s'envoler en poussant des petits couinements".}

Je vous laisse admirer la stratégie de notre stand sur la photo suivante.


\begin{figure}[h]
\centering
\includegraphics[height=6cm,width=9cm,keepaspectratio]{p4062975.jpg}
\caption*{C'est le moment d'être lourd !}
\end{figure}

Le pic de couinements a été atteint quand la lumière a été coupée pendant quelques minutes. Quand on ne voit plus rien, qu'on est accroché à 20m\textsuperscript{2} de toile et que le vent fait s'envoler les chapiteaux adjacents, j'avoue, on ne fait pas trop le malin. Finalement, tout s'est calmé, et je ne crois pas qu'il y ai eu de dégâts autres que matériels.


\begin{figure}[h]
\centering
\includegraphics[height=6cm,width=9cm,keepaspectratio]{p4072983.jpg}
\caption*{L'attente.}
\end{figure}

Le lendemain, nous nous sommes levés à l'aube pour aller assister à l'offrande aux moines. Tous les matins, les moines font le tour de Luang Prabang à pied, et de nombreux bouddhistes les attendent agenouillés sur les trottoirs pour leur donner de la nourriture et recevoir une bénédiction. C'est devenu très connu et fait désormais partie du circuit touristique classique de la ville. Et comme vous le savez bien : il est impossible d'observer un système sans le modifier. Ici, la modification vient de gens qui vont vendre des offrandes aux touristes désireux de faire plus que simplement observer. Il y a eu plusieurs histoires de moines rendus malades par de la mauvaise nourriture obtenue par ce biais. Les moines s'en sont plaint et ont menacé d'aller ailleurs. Le gouvernement leur a dit que s'il faisaient ça, ils embaucheraient des acteurs pour les remplacer, tellement c'est important pour le tourisme. Je ne sais pas quelle a été le résultat de la négociation, tout ce que je peux vous dire, c'est qu'on a vu des mecs rasés en robe orange sortir en file d'un temple, et recevoir de la nourriture de la part de Laotiens, mais aussi de quelques touristes qui avaient plus l'air hippies que bouddhistes. Donc je vous le dis : à moins d'être bouddhiste, contentez vous d'observer les moines de loin !


\begin{figure}[h]
\centering
\includegraphics[height=6cm,width=9cm,keepaspectratio]{p4072997.jpg}
\caption*{Les moines sont sortis.}
\end{figure}

Les moines, c'est bien, mais selon moi, les cascades de Luang Prabang, c'est mieux. Et je crois que ça se passe de mots :


\begin{figure}[h]
\centering
\includegraphics[height=6cm,width=9cm,keepaspectratio]{p4073074.jpg}
\caption*{C'est fou, non ?}
\end{figure}


\begin{figure}[h]
\centering
\includegraphics[height=6cm,width=9cm,keepaspectratio]{p4073043.jpg}
\caption*{On peut aussi s'y baigner, mais c'est un peu frais !}
\end{figure}


\begin{figure}[h]
\centering
\includegraphics[height=6cm,width=9cm,keepaspectratio]{p4073057.jpg}
\caption*{:-)}
\end{figure}


\begin{figure}[h]
\centering
\includegraphics[height=6cm,width=9cm,keepaspectratio]{p4073087.jpg}
\caption*{En route vers les sources de la cascade.}
\end{figure}


\begin{figure}[h]
\centering
\includegraphics[height=6cm,width=9cm,keepaspectratio]{p4073106.jpg}
\caption*{Ambiance Tahiti douche !}
\end{figure}

Seul regret : à cause de deux chinoises qui partageaient notre tuktuk, on n'a pas pu y rester aussi longtemps qu'on voulait. Ils ne comprennent rien ces chinois : un selfie, et ça repart !


\begin{figure}[h]
\centering
\includegraphics[height=6cm,width=9cm,keepaspectratio]{p4073150.jpg}
\caption*{Un papillon qui a bien vécu !}
\end{figure}

Petite dernière surprise sur le chemin du retour : un parc aux papillons ! On y a découvert l'existence de chrysalides dorées ! C'est étonnant, car il me semble qu'une chrysalide devrait se camoufler plutôt que de porter des dorures façons rappeur américain, non ? Une hypothèse suggérée par le guide est que ça camoufle peut-être en imitant les gouttes d'eau qui brillent elles aussi au soleil ? Chers lecteurs, si vous avez des connaissances sur ce sujet, je prends !


\begin{figure}[h]
\centering
\includegraphics[height=6cm,width=9cm,keepaspectratio]{p4073120.jpg}
\caption*{Dans le genre sobre et discret, on a vu mieux !}
\end{figure}



\chapter{Muang Ngoy, le village où il pleut des cendres.}
Nous partîmes donc pour Muang Ngoy. Si vous n'en avez jamais entendu parler, c'est normal ! Il n'y a même pas de route qui y va : quatre heures de bus pour aller tout au nord du Laos, jusqu'au bout de la route, puis une heure de bateau pour aller encore un peu plus au nord, et on arrive dans un tout petit village en train de s'ouvrir au tourisme !


\begin{figure}[h]
\centering
\includegraphics[height=6cm,width=9cm,keepaspectratio]{p4093239.jpg}
\caption*{Bain de boue pour tous !}
\end{figure}

Pendant le trajet, on croise plusieurs bateaux couverts d'une armée de rameurs : dans quelques jours, chaque village sur le fleuve va envoyer une équipe de ses meilleurs rameurs pour la compétition annuelle. La rivière est alors réservée aux compétitions. On a eu de la chance de passer à coté d'un bateau à l'entrainement. Pendant la compétition, on aurait du rester sur la rive, mais là, ils sont passés si près qu'on les entendait même respirer à l'unisson !


\begin{figure}[h]
\centering
\includegraphics[height=6cm,width=9cm,keepaspectratio]{p4083164.jpg}
\caption*{26 personnes sur un frêle esquif.}
\end{figure}

A peine débarqués, un charmant jeune homme, au teint qu'on devine très scandinave malgré le soleil laotien, nous accoste. Ou plutôt nous aborde, étant donné qu'on avait encore un pied dans le bateau. Il propose de nous loger dans des bungalows avec vue sur la rivière. En chemin, on apprend qu'il est suédois, et qu'il est arrivé ici en voyageant il y a quelques années déjà. Et le voilà désormais marié, père, et propriétaire d'une auberge dans le village ! En visitant un peu le village, on comprend en partie ce qui l'a retenu ici. L'ambiance est paisible au possible, et les paysages semblent magnifiques. Un peu plus tard, on comprend l'autre partie qui l'a fait rester ici en rencontrant sa femme, puis sa fille.


\begin{figure}[h]
\centering
\includegraphics[height=6cm,width=9cm,keepaspectratio]{p4083181.jpg}
\caption*{La rue principale.}
\end{figure}

Dans le paragraphe précédent, vous l'avez sûrement remarqué, j'ai écrit : "Les paysages semblent  magnifiques". Mais pourquoi donc "semblent" ? Et bien parce que nous n'avons fait que les deviner : avril, c'est le dernier mois avant la mousson, c'est le mois le plus chaud et le plus sec de l'année, et c'est logiquement la saison des brûlis ! Le ciel est rempli de cendres qui nous tombent dessus, et la visibilité est réduite à quelques centaines de mètres. L'ambiance est vaporeuse, et les montagnes disparaissent très vite dans la brume.


\begin{figure}[h]
\centering
\includegraphics[height=6cm,width=9cm,keepaspectratio]{p4093241-modifier1.jpg}
\caption*{A ce qu'il parait, il y aurait encore d'autres montagnes derrière ces collines.}
\end{figure}

Nous faisons une petite balade dans les rizières derrière le village, ce qui nous permet de nous éloigner encore un peu de la civilisation (occidentale). On dirait que le temps s'est arrêté, et que rien ne va bouger avant l'arrivée de la pluie : les rizières sont craquelées, les gens passent leurs journées à l'ombre, à bouger le moins possible, tant la chaleur est écrasante. On dépasse les 40\textdegree C. Les buffles passent le temps en tentant de se rafraichir, entassés dans la dernière mare de boue du village. J'ai l'impression que si la pluie n'arrive pas rapidement, ils vont se transformer en statue de terre cuite...




\begin{figure}[h]
\centering
\includegraphics[height=6cm,width=9cm,keepaspectratio]{p4093240.jpg}
\caption*{On n'est pas bien, là ?}
\end{figure}


\begin{figure}[h]
\centering
\includegraphics[height=6cm,width=9cm,keepaspectratio]{p4093213.jpg}
\caption*{Dans les rizières.}
\end{figure}


\begin{figure}[h]
\centering
\includegraphics[height=6cm,width=9cm,keepaspectratio]{p4093222.jpg}
\caption*{Un artisan en plein travail.}
\end{figure}


\begin{figure}[h]
\centering
\includegraphics[height=6cm,width=9cm,keepaspectratio]{p4093231.jpg}
\caption*{C'est donc une boite pour cuire du riz collant.}
\end{figure}

Nous serions volontiers restés ici quelques jours de plus, mais la vraie vie se rappela à nous. Ne dit-on pas : "Chassez la vie réelle, elle revient au galop" ? Nous galopâmes donc en direction de l'aéroport de Vientiane, non sans nous arrêter quelques jours à mi-chemin, dans l'éminente ville de Vang Vieng, Eldorado de la fête en Asie pour les backpackers de tout poils jusqu'à ces dernières années.


\begin{figure}[h]
\centering
\includegraphics[height=6cm,width=9cm,keepaspectratio]{p41132631.jpg}
\caption*{Par 40\textdegree C, c'est une activité conseillée !}
\end{figure}

Il n'y a pas si longtemps, c'était un passage obligatoire pour tout backpacker qui se respecte. Voici les ingrédients : une belle rivière, des chambres à air, des bars pas chers, quelques dealers (à prononcer "dilair" sinon ça ne rime pas).
On ajoute quelques jeunes, et ça donne des descentes de rivières sur des grosses bouées, le tout avec un esprit qui a du mal à évaluer les risques. Entre les noyades et les sauts dans les rochers (notez bien, j'ai dit "dans les rochers", et non "depuis les rochers"), l'endroit a commencé à avoir mauvaise réputation, au point d'attirer l'attention des médias internationaux. Il y a eu 20 morts au cours de l'année 2011, et plus de 3000 fêtards sont passés par l’hôpital. Et avant que CNN aie eu le temps d'envoyer un reporter sur place, quasiment tous les bars avaient été fermés par le gouvernement. L'ambiance est désormais plus calme. Il est toujours possible de descendre la rivière en bouée, mais le Coréen prudent en gilet de sauvetage a remplacé le jeune Australien en quête de nouvelles expériences.


\begin{figure}[h]
\centering
\includegraphics[height=6cm,width=9cm,keepaspectratio]{p4113253.jpg}
\caption*{Je suis le John Travolta des grottes !}
\end{figure}

Nous avions tellement lus de choses négatives sur cette rivière, qu'on n'y est même pas allé, préférant chercher les grottes loin des sentiers battus. Voici le résultat :


\begin{figure}[h]
\centering
\includegraphics[height=6cm,width=9cm,keepaspectratio]{p4113254.jpg}
\caption*{Même pas besoin de LSD pour planer !}
\end{figure}




\begin{figure}[h]
\centering
\includegraphics[height=6cm,width=9cm,keepaspectratio]{p4113252.jpg}
\caption*{Même le papa de Marion se met au light painting.}
\end{figure}


\begin{figure}[h]
\centering
\includegraphics[height=6cm,width=9cm,keepaspectratio]{p4123270.jpg}
\caption*{P4123270.jpg}
\end{figure}






\chapter{Vientiane, le retour : la fête de l'eau du nouvel an !}
Nous retournâmes donc à Vientiane, juste avant le début de Pimai, aussi appelé "fête de l'eau", aussi appelé nouvel an Lao. Rassurez-vous, au Laos, on fête aussi le nouvel an chinois, ainsi que le nouvel an occidental. Pas con !


\begin{figure}[h]
\centering
\includegraphics[height=6cm,width=9cm,keepaspectratio]{p1060335.jpg}
\caption*{A l'assaut !}
\end{figure}




\begin{figure}[h]
\centering
\includegraphics[height=6cm,width=9cm,keepaspectratio]{p1060322.jpg}
\caption*{Bouddha aussi a chaud.}
\end{figure}

A l'origine, la fête de l'eau est une fête bouddhiste : on arrose les mains et la tête des bouddhas avec de l'eau parfumée aux fleurs. On arrose aussi les grand-mères et grand-pères en signe de respect, et en fin de compte on arrose un peu tout le monde, parce qu'après tout, c'est la fin de la saison chaude, il fait une chaleur à crever, et c'est bien agréable un peu d'eau fraîche !


\begin{figure}[h]
\centering
\includegraphics[height=6cm,width=9cm,keepaspectratio]{p1060310.jpg}
\caption*{Oh oh, on dirait que c'est moi le prochain...}
\end{figure}

Bref, de proche en proche, c'est devenu une grosse bataille d'eau à l'échelle du pays. Les gens installent des piscine gonflables partout sur les trottoirs, qui vont servir de réserves d'eau. Et vers 12h, ça commence : les pick-up chargés eux aussi de bassins circulent dans la ville, et les passagers arrosent les piétons, les piétons ripostent, et ce sera à celui qui aura le plus gros débit/bassin/pistolet à eau, le tout dans une ambiance bon enfant ! Un peu partout dans la ville, il y a des concerts live, avec arroseurs automatiques au dessus de la foule, voire une machine à mousse. On a même vu un camion citerne équipé d'une lance à incendie circuler dans les rues : c'était vraiment le roi de la fête ! On s'est fabriqué des pistolets à eau avec des bouteilles percées, mais niveau puissance de feu, on était vraiment ridicules...


\begin{figure}[h]
\centering
\includegraphics[height=6cm,width=9cm,keepaspectratio]{p1060304.jpg}
\caption*{C'est moi qui ai la plus grosse (réserve d'eau) !}
\end{figure}

Il est très difficile de rester au sec, et de toute façon, personne n'en a vraiment envie. Mais il est quand même fortement recommandé d'avoir toutes ses affaires dans des sacs étanches pendant ces quelques jours. Oui, j'ai bien dit quelques jours : ça dure en théorie 4 jours, mais en vrai, il y a toujours quelques gamins qui commencent un peu avant. Il faut d'ailleurs être prudent en scooter à cette période de l'année, car un bac d'eau et de glaçons attend peut-être au prochain virage... Et si en plus le pilote est bourré (mais non, qui donc boit à nouvel an ?), je vous laisse imaginer les conséquences ! (Oui, un pilote trempé faisant la fête avec un arroseur, c'est aussi une conséquence possible.)


\begin{figure}[h]
\centering
\includegraphics[height=6cm,width=9cm,keepaspectratio]{p10603081.jpg}
\caption*{Il peut y avoir un tireur embusqué dans la moindre poubelle !}
\end{figure}

Cela dit, la plupart des gens sont très respectueux, et sur un simple geste, ils comprennent si on ne veut pas être arrosé, parce que, par exemple, on a un appareil photo pas étanche en main. Dans ce cas, ils nous arrosent seulement un peu les mains ou le dos, en nous expliquant que c'est une tradition, que c'est pour porter bonheur pour la nouvelle année. Comment dire non à une bénédiction d'eau fraiche quand il fait 40\textdegree C à l'ombre ?


\begin{figure}[h]
\centering
\includegraphics[height=6cm,width=9cm,keepaspectratio]{p10603311.jpg}
\caption*{Trois hypothèses : 1- il n'avait pas vu que j'avais un appareil photo, 2- il l'avait vu mais il s'en foutait, 3- il avait vu que l'appareil était étanche.}
\end{figure}

Le petit problème avec la fête du nouvel an, c'est que la plupart des restaurants et boutiques sont fermés, la majorité des chauffeurs de tuktuks et taxis complètement démotivés pour travailler (démotivé est probablement en dessous de la réalité). La dernier journée avec mon papa fut un peu difficile : la fatigue des transports, les crises de nerfs d'incompréhension avec les laotiens qui ne comprennent pas la nécessité de respecter les horaires quand on a un avion à prendre, et le pincement au cœur de dire au revoir à mon papa.




\begin{figure}[h]
\centering
\includegraphics[height=6cm,width=9cm,keepaspectratio]{p1060280.jpg}
\caption*{Dernière journée avec Jean-Michel à Vientiane.}
\end{figure}

Et voilà nous sommes à l'aéroport. Nous sommes tristes de te voir partir et tellement heureux d'avoir pu partager un peu de ce voyage avec toi papa !



\chapter{Arrivée... et traversée de la Thaïlande, direction la plongée}
Nous quittâmes donc le Laos. D'abord parce que notre visa arrive à expiration, et ensuite parce que... parce qu'on envie de faire autre chose ! On aurait bien voulu partir avant, mais pendant la fête de l'eau, impossible de circuler. Après 4 jours de bataille d'eau et de musique à fond, les pistolets à eau sont rangés, et les bus ressortis, et nous voilà enfin en route vers la Thaïlande.


\begin{figure}[h]
\centering
\includegraphics[height=6cm,width=9cm,keepaspectratio]{p1060405.jpg}
\caption*{Alors comme on n'a pris aucune photo du trajet ou de Bangkok, on va tout illustrer avec des photos sous-marines (ce sont des vers "sapin de noël").}
\end{figure}

Le passage de la frontière est facile, même s'ils ont des taxes supplémentaires pour les jours fériés (non non ce n'est pas de la corruption, c'est obligatoire).
Notre arrivée en Thaïlande a commencé par un petit autocollant que l'on nous a collé sur le t-shirt (il ne faudrait pas que l'on se perde tout de même) et puis une gentille dame nous a conduit jusqu'au bus. C'est bien organisé la Thaïlande, ils ont l'habitude des touristes un peu hagards. Même si la moitié de notre petit groupe a disparu entre la frontière et le bus...?


\begin{figure}[h]
\centering
\includegraphics[height=6cm,width=9cm,keepaspectratio]{p1060410.jpg}
\caption*{Attention : trop de snorkelling peut faire pousser la moustache !}
\end{figure}

Petite déception quand même en entrant dans le bus : point de couchette, juste des sièges inclinables. Quelques heures de mauvais sommeil plus tard, nous voilà donc au petit matin, mal réveillés, dans la banlieue d'une ville immense. Après quelques mois loin des grandes agglomérations, et bien ça fait étrangement du bien. On a été très surpris, mais le fait de retrouver un métro, des centres commerciaux, ça nous a fait du bien les premières heures. Ensuite, les embouteillages, le bruits, les odeurs se rappellent aussi à notre bon souvenir... Nous sommes repartis le soir même en direction des îles du sud, sans même passer une seule nuit en ville. On a donc enchainé deux bus de nuit, c'est vous dire la motivation des troupes !


\begin{figure}[h]
\centering
\includegraphics[height=6cm,width=9cm,keepaspectratio]{p1060426.jpg}
\caption*{Ça vaut le coup de se dépêcher, non ?}
\end{figure}

Oui, nous avons fait complètement l'impasse sur le nord de la Thaïlande. Laissez moi vous expliquer pourquoi : ça faisait 4 mois en gros qu'on visitait des petits villages, des temples, des grottes et des cascades dans la forêt. Et on trouve dans le nord de la Thaïlande... (suspense à deux balles) des petits villages, des temples, des grottes et des cascades dans la forêt. Bref, après un plongeon profond dans le regard l'un de l'autre, nous nous sommes rendu compte à temps qu'on avait tous les deux envie de plage, et nous avons changé nos plan pour foncer vers la mer à toute vitesse.


\begin{figure}[h]
\centering
\includegraphics[height=6cm,width=9cm,keepaspectratio]{p4284283-modifier.jpg}
\caption*{De nuit aussi c'est pas mal.}
\end{figure}

Nous arrivâmes donc à Koh Tao, une petite île située dans le golfe de Thaïlande. C'est l'île de la plongée par excellence ! Il y a plus de 70 clubs de plongée, et la concurrence tire les prix vers le bas, aussi bien pour les fun-dive (plongée pour le plaisir) que pour les formations. Après une petite piqure de rappel (brancher ça, là, et ça, ici, mettre le truc dans le machin et ensuite vérifier que quand on appuie sur ce truc, ça fait des bulles !), et quelques fun dives, on attaque la formation advanced ! A l'issue de cette formation, on ne sera plus juste des poussins qui suivent sagement la maman canard jusqu'à 18m de profondeur : non non non, on sera désormais des poussins qui suivent sagement la maman canard jusqu'à 30m de profondeur !


\begin{figure}[h]
\centering
\includegraphics[height=6cm,width=9cm,keepaspectratio]{g0181052.jpg}
\caption*{C'est nous avec Jérôme, notre maman canard !}
\end{figure}

Autour de Koh Tao, il y a déjà des plongées très sympathiques à faire : nous qui n'avions plongé qu'en Méditerranée, nous avons découvert un monde de corail et de poissons colorés. Mais... pour être tout à fait honnête, on en attendait un peu plus. Les émissions de Cousteau ont peut-être mis la barre un peu haut dans nos esprits ? C'est pourquoi nous sommes allés à Sail Rock : un rocher de deux mètres perdu au milieu de l'océan, à deux heures de bateau des côtes. Et là, on a vu du poisson. Du gros, et plein. Au tout début de la plongée, Marion, qui fermait la marche, enfin, la palme, nous la tire (la palme donc) pour attirer notre attention sur un mérou. On venait de passer un mètre au dessus sans le voir. Un mérou... LE mérou ! Deux mètres de long ! La taille d'un gros veau ! Ensuite, ce fut le festival : des bancs de poissons argentés énormes, des bat-fishs, des barracudas... Là, on a commencé à se croire vraiment dans un Cousteau !


\begin{figure}[h]
\centering
\includegraphics[height=6cm,width=9cm,keepaspectratio]{gopr1076.jpg}
\caption*{Un petit mérou.}
\end{figure}

On peut aussi faire des plongées de nuit ! Alors déjà que la plongée, ce n'est pas un truc de claustrophobique, je vous laisse imaginer de nuit... Et comme je ne suis pas sûr que vous imaginiez comme il faut, je vais aussi raconter cette plongée : on part à la tombée de la nuit, et on se met à l'eau quand il commence à faire sombre. Une fois dans l'eau, la lampe est indispensable sinon, on ne voit rien. Si on tourne la tête, sans tourner la lampe, et qu'il n'y a aucun autre plongeur dans le petit champ de vision que nous laisse le masque, on se retrouve dans un monde noir d'encre (et pas que de seiche), on ne sait plus où sont le haut et le bas, le manque de repères visuels donne le vertige. Pour ne rien simplifier, ce soir là, il y avait pas mal de courant et une visibilité un peu moins bonne. Bref, on s'accroche fort à la lampe, et on ne quitte notre guide des yeux que le temps de regarder un poisson, et encore, pas trop longtemps. Mais alors, pourquoi donc plonger de nuit ? Parce que les poissons ne sont pas les mêmes : on en voit certains dormir, d'autres chasser, et on a vu un énorme bernard-l'hermite, qui trainait une coquille de presque 30cm ! Et on plonge aussi de nuit à cause de l'ambiance. L'impression d'être un étranger est encore plus forte que de jour. Des ombres, des reflets se manifestent en permanence dans le coin de l’œil, mais le temps de tourner la lampe, plus rien...


\begin{figure}[h]
\centering
\includegraphics[height=6cm,width=9cm,keepaspectratio]{p1060623.jpg}
\caption*{Le fameux poisson chirurgien !}
\end{figure}

Il y a aussi le snorkelling (masque/tuba). Plusieurs avantages à la plongée : c'est gratuit, quantité d'air inépuisable, et on est silencieux, donc on fait moins peur à certains poissons, en particulier aux requins. On a ainsi pu voir de nombreux petits requins à pointe noire à quelques mètres à peine du rivage ! Parfois, jusqu'à 5 ou 6 d'affilée. On a beau savoir qu'ils sont inoffensifs, on a tendance à se sentir tout nu quand ils se montrent un peu curieux.


\begin{figure}[h]
\centering
\includegraphics[height=6cm,width=9cm,keepaspectratio]{p1060394.jpg}
\caption*{Ça fait bizarre de nager à coté d'un requin, même un petit...}
\end{figure}

Ah, et j'ai failli oublier de vous parler de la température : il fait chaud. Et la mer aussi est chaude : 30\textdegree C de manière uniforme, même à 20m sous l'eau ! En fin de journée, dans les baies peu profondes, l'eau chauffée toute la journée en devient même inconfortable en montant à 36 ou 37\textdegree C, c'est impossible de se rafraichir en se baignant... On n'a vraiment pas une vie facile !


\begin{figure}[h]
\centering
\includegraphics[height=6cm,width=9cm,keepaspectratio]{p1060458.jpg}
\caption*{Le nom de ce coquillage est "boring clam", qu'on peut traduire par "moule ennuyeuse". Ça ne s'invente pas !}
\end{figure}


\begin{figure}[h]
\centering
\includegraphics[height=6cm,width=9cm,keepaspectratio]{p1060483.jpg}
\caption*{Le poisson perroquet, qui passe sa journée à ronger du corail et à chier du sable : jusqu'à 3kg par jour et par individu. Sans eux, pas de plage de sable blanc !}
\end{figure}


\begin{figure}[h]
\centering
\includegraphics[height=6cm,width=9cm,keepaspectratio]{p1060549.jpg}
\caption*{Le poisson demoiselle, qui est tellement territorial qu'il attaque tout ce qui passe, même un humain environ mille fois lourd !}
\end{figure}


\begin{figure}[h]
\centering
\includegraphics[height=6cm,width=9cm,keepaspectratio]{p4284240.jpg}
\caption*{Des cailloux. Parce qu'il n'y a pas que les poissons dans la vie !}
\end{figure}


\begin{figure}[h]
\centering
\includegraphics[height=6cm,width=9cm,keepaspectratio]{p4284261.jpg}
\caption*{La plage de nuit.}
\end{figure}



\chapter{Koh Chang en famille}
Nous partîmes finalement de Koh Tao, un peu tristes de quitter cette île très agréable, mais tout excités à l'idée de retrouver la maman de Marion ! Nous les retrouvons dans leur hôtel (un vrai hôtel, pas une guesthouse !), et c'est parti pour une petite semaine en leur compagnie !


\begin{figure}[h]
\centering
\includegraphics[height=6cm,width=9cm,keepaspectratio]{p5087096.jpg}
\caption*{Jongleurs de feu sur la plage.}
\end{figure}

Le lendemain, direction l'île de Koh Chang, en voiture privée, s'il vous plait ! On arrive ensuite dans un vrai centre de vacances : deux grandes piscines, une chambre immense avec tout le confort moderne, un accès direct à la plage, et la cerise sur le gâteau (en tout cas pour moi) : le buffet petit déjeuner ! 20m de long, avec du vrai pain, des céréales, des œufs sous toutes leurs formes, des nouilles, du bacon, des dim-sums, du muesli, de la salade de fruits... J'avais l'impression de ne jamais avoir assez faim ! Chaque soir, je m'endormais en pensant déjà avec quelle stratégie j'allais attaquer le buffet du lendemain. Est-ce qu'il valait mieux attaquer par le lourd : œuf/bacon, et finir sur du muesli, où bien attaquer en douceur par une salade de fruit, et monter en régime progressivement ? Que de choix, que de goûts, que de combinaisons possibles !


\begin{figure}[h]
\centering
\includegraphics[height=6cm,width=9cm,keepaspectratio]{p5076318-modifier.jpg}
\caption*{Ciel nocturne au bord de la plage.}
\end{figure}

Malgré un poids sur l'estomac, en tout cas pour certains d'entre nous, on essaie de faire un peu de snorkelling, mais impossible de retrouver la qualité et l'ambiance qu'on avait à Koh Tao. Tout est gris, l'eau est trouble, les poissons se cachent, les coraux sont morts. Dominique se fait même attaquer lâchement par un oursin, ce qui lui fera un tatouage au bas du mollet. Tout ceci nous convaincs de faire une sortie snorkelling digne de ce nom. On se retrouve tous les quatre dans un grand bateau, en compagnie d'environ 200 personnes, et nous sommes les seuls touristes occidentaux !


\begin{figure}[h]
\centering
\includegraphics[height=6cm,width=9cm,keepaspectratio]{p1060910.jpg}
\caption*{Et nous ne sommes pas le seul bateau !}
\end{figure}

On apprend un peu plus tard que ce sont les vacances thaïlandaises, voilà pourquoi le bateau est plein de touristes thaïs. On se dit que ça va être surpeuplé une fois dans l'eau, mais non, on s'en sort plutôt bien : la plupart des thaïs se savent pas nager, ils ont donc un gilet de sauvetage et ne bougent pas trop une fois à l'eau. Certains embauchent même un guide empalmé qui va remorquer une bouée à laquelle les gens s'accrochent, afin qu'ils puissent se balader un peu. Tout ceci fait qu'on peut facilement s'écarter de la foule et trouver des coins tranquilles. Et enfin, on retrouve des poissons, des coraux multicolores, et on a même la chance de voir des seiches et des gros crabes !


\begin{figure}[h]
\centering
\includegraphics[height=6cm,width=9cm,keepaspectratio]{p1070018.jpg}
\caption*{Il y a des touristes qui ont du pain dans les poches. Ça augmente fortement la concentration locale de poissons !}
\end{figure}


\begin{figure}[h]
\centering
\includegraphics[height=6cm,width=9cm,keepaspectratio]{p1060918.jpg}
\caption*{Agile comme un mérou !}
\end{figure}


\begin{figure}[h]
\centering
\includegraphics[height=6cm,width=9cm,keepaspectratio]{p1070073.jpg}
\caption*{Un gros crabe !}
\end{figure}

Le débarcadère qui nous mène jusqu'aux bateaux est une attraction à lui tout seul : il doit faire dans les 400m de long, mais on ne voit pas la mer : une file ininterrompue de boutiques bordent le chemin. On entend la mer, on sent la mer, et c'est tout !


\begin{figure}[h]
\centering
\includegraphics[height=6cm,width=9cm,keepaspectratio]{p5034973.jpg}
\caption*{Quand on prend de l'essence, on a le choix entre menthe, orange et grenadine.}
\end{figure}

La mer, c'est bien, mais cette île est montagneuse, ce serait dommage de ne pas profiter de la vue ! On essaie plusieurs fois d'aller se balader tout seul, mais c'est compliqué ici aussi : aucune carte disponible, aucun panneau, tout est fait pour encourager l'embauche de guides locaux ! On apprend même qu'il y avait des sentiers de randonnée bien équipés, signalés, mais tous les panneaux ont été arrachés il y a quelques années déjà.


\begin{figure}[h]
\centering
\includegraphics[height=6cm,width=9cm,keepaspectratio]{p5055495.jpg}
\caption*{Le lézard, nettement plus facile à photographier que les singes !}
\end{figure}

En se baladant au hasard, on voit tout de même de belles choses : des forêts de bambou, des termitières, et même des singes sauvages ! Par contre, dans certains endroits de la jungle, c'est délicats de trouver un endroit ou s'arrêter tellement il y a de fourmis et de termites qui nous grimpent dessus à la moindre pause. Par endroits, nous nous sommes même retrouvés à devoir courir sur quelques mètres pour passer par-dessus des 2x8 voies de fourmis affamées.


\begin{figure}[h]
\centering
\includegraphics[height=6cm,width=9cm,keepaspectratio]{p5055484.jpg}
\caption*{La rando c'est bien, mais ça donne chaud...}
\end{figure}

Le rythme n'est pas le même que quand on est juste Marion et moi. Nous, on se la coule douce, et on ne va certainement pas faire du kayak une heure avant de partir pour l'aéroport ! Mais toutes les bonnes choses ont une fin (sauf le saucisson bien entendu), et nous prenons la route de l'aéroport. Les adieux sont tristes, mais nous repartons tous regonflés du plaisir d'avoir passé un peu de temps avec la famille.


\begin{figure}[h]
\centering
\includegraphics[height=6cm,width=9cm,keepaspectratio]{p5055444.jpg}
\caption*{On a aussi vu des bébés éléphants !}
\end{figure}


\begin{figure}[h]
\centering
\includegraphics[height=6cm,width=9cm,keepaspectratio]{p5045125.jpg}
\caption*{Et on a vu des jolis coucher de soleil !}
\end{figure}


\begin{figure}[h]
\centering
\includegraphics[height=6cm,width=9cm,keepaspectratio]{p5045437.jpg}
\caption*{L'anniversaire qui fait du bien aux papilles !}
\end{figure}

J'aurais bien voulu mettre un lien vers plus de photos, mais google refuse de coopérer depuis le Myanmar... Revenez ici vers le 15 juillet pour le reste des photos !

Oh, vous êtes revenus ? Pour de vrai ? Oh ben ça fait plaisir :-)



\chapter{Plongée dans les îles Similan}
Nous partîmes donc pour les îles Similan. Pour une croisière liveaboard. Oui, vous avez bien lu ! C'est le gros craquage budgétaire du voyage. C'est pour ce genre de choses qu'après chaque pays, on faisait les comptes pour savoir si on avait dépensé plus ou moins que prévu : la différence allait dans le budget plongée en Thaïlande. Mais quel rapport, allez-vous me dire, entre une croisière et la plongée ? Réponse : dans une croisière "liveaboard" on ne fait que de la plongée :-)


\begin{figure}[h]
\centering
\includegraphics[height=6cm,width=9cm,keepaspectratio]{similan-2016-433.jpg}
\caption*{Le poisson juste en dessous de la raie fait presque un mètre de long...}
\end{figure}

Vous vous en doutez, autant faire ce genre d'activité dans un endroit qui vaut un peu le coup : les îles Similan sont parmi les meilleures destinations de plongée du monde. Bref, on est super excités. Depuis qu'on a réservé cette croisière, tous les gens à qui on en parle rêvent soit d'y aller, soit d'y retourner, et des étoiles s'allument dans leurs yeux.


\begin{figure}[h]
\centering
\includegraphics[height=6cm,width=9cm,keepaspectratio]{similan-2016-483.jpg}
\caption*{Dive . Eat . Sleep . Repeat}
\end{figure}

On se retrouve à 30 plongeurs sur un bateau de 35 mètres, avec un programme en 4 mots : \emph{dive, eat, sleep, repeat} (plonge, mange, dors, recommence). Quinze plongées en quatre jours et quatre nuits : il s'agirait de ne pas chômer ! Toute une équipe est là pour nous faciliter la vie. La journée commence par le réveil en fanfare, on a alors 30mn pour manger un morceau et se préparer, avant le briefing. Quelques explications sur le site de plongée plus tard, et on descend s'équiper. On a chacun sa place qui ne bougera pas de toute la croisière. Les membres d'équipage se chargent de remplir les bouteilles au fur et à mesure, et nous aident même à enfiler les palmes !


\begin{figure}[h]
\centering
\includegraphics[height=6cm,width=9cm,keepaspectratio]{similan-2016-569.jpg}
\caption*{C'est nous, juste sous les lettres T et A.}
\end{figure}

Une fois en position sur la plateforme, on attend la sonnerie du capitaine qui nous autorise à sauter, et là, c'est comme des parachutistes qui sautent d'un avion : les uns derrières les autres, quatre par quatre, sous les encouragements de l'équipe, on saute dans le bleu.


\begin{figure}[h]
\centering
\includegraphics[height=6cm,width=9cm,keepaspectratio]{similan-2016-563.jpg}
\caption*{Le programme de la troisième journée.}
\end{figure}

Et on en prend plein les mirettes. Pour illustrer mon propos, rien ne vaut quelques images (n'étant pas équipé pour la photo à cette profondeur, cet article est illustré avec les photos de Kenneth, un photographe Hong-Kongais qui a fait la croisière avec nous).


\begin{figure}[h]
\centering
\includegraphics[height=6cm,width=9cm,keepaspectratio]{similan-2016-118.jpg}
\caption*{Un poisson lion.}
\end{figure}


\begin{figure}[h]
\centering
\includegraphics[height=6cm,width=9cm,keepaspectratio]{similan-2016-591.jpg}
\caption*{Un hippocampe !}
\end{figure}


\begin{figure}[h]
\centering
\includegraphics[height=6cm,width=9cm,keepaspectratio]{similan-2016-606.jpg}
\caption*{On a retrouvé Némo :-)}
\end{figure}


\begin{figure}[h]
\centering
\includegraphics[height=6cm,width=9cm,keepaspectratio]{similan-2016-603.jpg}
\caption*{Des petites crevettes !}
\end{figure}


\begin{figure}[h]
\centering
\includegraphics[height=6cm,width=9cm,keepaspectratio]{similan-2016-639.jpg}
\caption*{Des tas de poissons !}
\end{figure}


\begin{figure}[h]
\centering
\includegraphics[height=6cm,width=9cm,keepaspectratio]{similan-2016-731.jpg}
\caption*{Ce truc violet aux bouts oranges, c'est une sorte de limace de mer.}
\end{figure}


\begin{figure}[h]
\centering
\includegraphics[height=6cm,width=9cm,keepaspectratio]{similan-2016-658.jpg}
\caption*{Une sorte de petit mérou.}
\end{figure}

Environ une heure sous l'eau plus tard, on remonte, le bateau vient nous chercher et on remonte pour un petit-déjeuner/déjeuner/goûter/diner (rayer la mention inutile en fonction du nombre de plongées déjà effectuées). Une petite sieste, ou une grosse nuit plus tard, on recommence ! Les repas sont supers bons et variés, on passe la journée en maillot de bain quand on n'est pas au milieu des poissons.


\begin{figure}[h]
\centering
\includegraphics[height=6cm,width=9cm,keepaspectratio]{similan-2016-537.jpg}
\caption*{C'est grand !}
\end{figure}


\begin{figure}[h]
\centering
\includegraphics[height=6cm,width=9cm,keepaspectratio]{similan-2016-546.jpg}
\caption*{On se sent tout petit quand un engin comme ça nous passe au-dessus.}
\end{figure}

Dès le deuxième jour, nous avons la chance de voir des Raies Manta ! C'est énorme, impressionnant, majestueux, lent et léger à la fois. L'émotion est si forte que le souvenir de ce moment est gravé dans mon esprit. En imaginant ces raies manta, toutes les sensations du moment revivent : je respire à travers un détendeur, je sens le gout du sel, j'entends les bulle, et j'ai l'impression d'avoir la poitrine qui me serre et de ne plus savoir comment respirer ! Nous sommes tentés de les suivre après leur passage, mais nos guides sous-marins nous font signe que non, on attend au même endroit. Quelques instants plus tard, les mantas reviennent puis repartent : elles tournent en rond ! Toutes les palanquées sont aux aguets, le premier qui en voit une tape sur sa bouteille pour attirer l'attention des autres. Ce qui n'empêche pas certains plongeurs d'être à coté de la plaque par moment : Roger par exemple, a mis quelques instants à comprendre pourquoi tous les autres plongeurs le montrait subitement. C'est quand il s'est retourné et a vu la manta deux mètres au dessus de lui qu'il a compris. Découvrir un poisson de 6m juste dans son dos, ça doit faire un choc !


\begin{figure}[h]
\centering
\includegraphics[height=6cm,width=9cm,keepaspectratio]{p1070166.jpg}
\caption*{Petite pause sur une île paradisiaque.}
\end{figure}

Un des sites les plus attendu, c'est le rocher de Richelieu. Nous avons de la chance car nous sommes le seul bateau de plongée sur le site, alors qu'il y en a d'habitude une dizaine. Il se trouve que nous sommes sur la dernière croisière de la saison. Les îles Similan sont dans un parc national qui ferme 6 mois par ans, à la fois pour permettre à la nature de se régénérer, et aussi parce que c'est la saison de la mousson, donc pas génial pour la plongée. Ce n'est pas parce qu'on aime pas être mouillé, mais parce que le vent, la pluie et les vagues remuent la mer, et diminuent la visibilité sous l'eau. Nous avons même droit à une petite dérogation : le parc est officiellement fermé, mais les autorités nous laissent terminer la croisière. Donc nous somme seuls ! Cet endroit grouille de vie, et on voit pour la première fois autant de petites bêtes partout, comme la fameuse crevette mante, capable de donner des coups de patte supersoniques. La vitesse est telle que le mouvement crée des bulles de vides qui en s'effondrant font des petits flashs de lumière ! Il est même compliqué de la garder en aquarium : souvent, le verre ne lui résiste pas. On voit aussi des hippocampes, tout petits et immobiles, camouflés dans leur environnement, presque invisibles sans l’œil aiguisé de Nick notre guide russe.


\begin{figure}[h]
\centering
\includegraphics[height=6cm,width=9cm,keepaspectratio]{p1070198.jpg}
\caption*{Ça sent quand même la fin de la saison...}
\end{figure}

Nous sommes le seul bateau de plongée sur le site, mais il y a aussi, malheureusement, des bateaux de pêche. On nous explique que malgré le statut de parc national, les gardiens se font facilement corrompre avec une caisse de bière. Résultat, pendant la fermeture du parc, les poissons sont pêchés, et c'est pendant la saison touristique qu'ils sont le plus tranquilles...


\begin{figure}[h]
\centering
\includegraphics[height=6cm,width=9cm,keepaspectratio]{p5177112.jpg}
\caption*{Ils étaient à deux mètres de nous : on avait l'impression qu'on pouvait les toucher en tendant un peu plus le bras (et en tombant du bateau accessoirement).}
\end{figure}

Sur le chemin du retour, petit bonus inattendu, des dauphins font leur apparition et viennent jouer dans les vagues à la proue du bateau. Là, on s'est vraiment crus dans un épisode de Cousteau !



\chapter{Où on fait un tour sur les autres îles de Thaïlande}
Nous arrivâmes donc à Phuket, les pieds sur terre, mais la tête encore au milieu des poissons. Nous avons fait le trajet avec Edwige et Antoine, un couple de Français rencontré pendant la croisière. Ils voyagent depuis bien plus longtemps que nous : 3 ans (et là, je pense qu'avec cette phrase, je fais peur aux mamans...). Donc nous voilà à Phuket, temple du tourisme de masse en Thaïlande.


\begin{figure}[h]
\centering
\includegraphics[height=6cm,width=9cm,keepaspectratio]{p5238341.jpg}
\caption*{Avec un voilier, ce serait du William Turner.}
\end{figure}

Une grande plage de sable blanc sans le moindre rocher ni poisson attire les touristes la journée. L'activité phare est le parachute ascensionnel. C'est un parachute tiré par un hors-bord, un peu comme un enfant qui fait voler un cerf-volant en courant, sauf qu'un touriste a payé une place dans le cerf-volant. C'est pratiqué bien entendu en suivant toutes les règles de sécurité et de bon sens : par exemple, les hors-bord ont un accès réservé à la plage. Ce serait quand même super dangereux de les faire passer au milieu des baigneurs, non ? De la même manière, le pilote du parachute - car le touriste ne sait pas comment atterrir en douceur sur la plage - est correctement attaché. Ce serait, situation prise au hasard, complètement con et suicidaire qu'il se contente de s'agripper à la main aux suspentes, ne pensez-vous pas ?


\begin{figure}[h]
\centering
\includegraphics[height=6cm,width=9cm,keepaspectratio]{p5197179.jpg}
\caption*{Il y a aussi de jolis couchers de soleil.}
\end{figure}

Derrière la plage, la ville pleine de rues, des rues pleines de bars, des bars pleins d'Australiens, des Australiens plein de bière. On prend quand même le temps de visiter un peu les lieux, plus par curiosité que par envie, après tout, c'est aussi représentatif de la Thaïlande ! Dans la rue la plus active, on remarque un décalage certain entre l'ambiance festive et l'expression désabusée des femmes qui essayent d'attirer le chaland vers un ping-pong show.


\begin{figure}[h]
\centering
\includegraphics[height=6cm,width=9cm,keepaspectratio]{p5187166.jpg}
\caption*{"Swamp eel show" ??!!!? Le show anguille des marais ? Quelqu'un sait comment éteindre l'imagination ? Si ça se trouve, c'est une anguille qui joue au ping-pong...}
\end{figure}


\begin{figure}[h]
\centering
\includegraphics[height=6cm,width=9cm,keepaspectratio]{p5187175.jpg}
\caption*{Le mec, il regarde la télé. Genre, il est venu là pour voir du foot !}
\end{figure}

Après un jour à Phuket, ce qui est bien suffisant pour nous, direction Krabi, autre lieu touristique s'il en est ! Les plages sont un peu plus jolies, surtout grâce aux karsts qui agrémentent l'horizon. Mais il n'y a toujours pas de poissons à voir, alors on va plutôt visiter un temple en haut d'une montagne ! Il fait très chaud, et l'escalier est tellement raide qu'on voit de nombreux touristes se sentir mal dans la montée, certains au point d'en vomir. Et comme dans les tous les temples dans la montagne, il est colonisé par les singes, ainsi que par les vendeurs de nourriture pour singe, qui ont l'avantage, par rapport aux singes, d'éviter de monter sur ton sac à dos pour te piquer des trucs de valeur comme un sachet en plastique vide. A par ça, Krabi c'est assez semblable à Phuket, donc on passe aussi rapidement à la prochaine étape : Railay !


\begin{figure}[h]
\centering
\includegraphics[height=6cm,width=9cm,keepaspectratio]{p5207903.jpg}
\caption*{Vue après quelques centaines de marches.}
\end{figure}




\begin{figure}[h]
\centering
\includegraphics[height=6cm,width=9cm,keepaspectratio]{p5207646.jpg}
\caption*{En plus de sentir bon, ça éloigne les moustiques.}
\end{figure}


\begin{figure}[h]
\centering
\includegraphics[height=6cm,width=9cm,keepaspectratio]{p5207668.jpg}
\caption*{Une photo de singe, ça ne fait pas de mal !}
\end{figure}


\begin{figure}[h]
\centering
\includegraphics[height=6cm,width=9cm,keepaspectratio]{p5207672.jpg}
\caption*{Une photo de s... oh, salut Antoine ! Je ne t'avais pas reconnu ! (on ne dirait pas comme ça, mais je l'aime bien en vrai, et en plus on gagne à la coinche ensemble)}
\end{figure}


\begin{figure}[h]
\centering
\includegraphics[height=6cm,width=9cm,keepaspectratio]{p5197184.jpg}
\caption*{Bière, burger, coinche ! Le bonheur !}
\end{figure}

Railay, c'est une petite crique juste à coté de Krabi. Elle a l'insigne avantage d'être uniquement accessible en bateau, et c'est aussi et surtout le paradis des grimpeurs. Imaginez un peu : une crique entourée de falaises de calcaire, une plage, et des bungalows pleins de hippies/grimpeurs ! Malheureusement pour nous, il a plu tout notre séjour. Il restait bien quelques voies au sec, mais que du dévers. Nous avons donc fait l'impasse sur l'escalade, et plutôt fait du kayak, pour aller tourner autour de quelques îles en face de la crique. Nous avons pu admirer de loin quelques voies de "deep water soloing" (il s'agit de grimper sans corde au dessus de l'eau) qui ont fait la réputation de Railay.


\begin{figure}[h]
\centering
\includegraphics[height=6cm,width=9cm,keepaspectratio]{p5218288.jpg}
\caption*{Le moteur du bateau. Confiance !}
\end{figure}


\begin{figure}[h]
\centering
\includegraphics[height=6cm,width=9cm,keepaspectratio]{p5218290.jpg}
\caption*{Le capitaine du bateau. Confiant.}
\end{figure}


\begin{figure}[h]
\centering
\includegraphics[height=6cm,width=9cm,keepaspectratio]{p1070222.jpg}
\caption*{Le début d'une voie...}
\end{figure}


\begin{figure}[h]
\centering
\includegraphics[height=6cm,width=9cm,keepaspectratio]{p1070230.jpg}
\caption*{C'est tout petit un kayak !}
\end{figure}

Après quelques jours pluvieux à baver devant des voies inaccessibles, on passe à l'île suivante : Koh Lanta ! Autant le dire tout de suite : rien à voir avec l'émission. Il y a tout un chapelet d'îles qui portent ce nom, et nous logeons sur la plus grande, plutôt bien équipée pour la survie des aventuriers. Nous regrettons rapidement le climat pluvieux de Railay, car ici, c'est plutôt la tempête ! Le début de la mousson a chassé la plupart des touristes, beaucoup de commerces sont fermés, bref, c'est l'échec. Le seul avantage est que la température redevient acceptable, désormais, si on est dehors sans bouger, on ne transpire plus ! Une petite accalmie nous permet de faire un tour sur l'île, mais rien de bien folichon : il pleut toujours, nous repartons directement vers Koh Tao.


\begin{figure}[h]
\centering
\includegraphics[height=6cm,width=9cm,keepaspectratio]{p5238330.jpg}
\caption*{Un petit crabe sur la plage de Koh Lanta.}
\end{figure}


\begin{figure}[h]
\centering
\includegraphics[height=6cm,width=9cm,keepaspectratio]{p5248359.jpg}
\caption*{Pêcheurs à Koh Lanta.}
\end{figure}

Il faut savoir que depuis la fin de la croisière, nous avions envie de retourner à Koh Tao, mais on culpabilisait un peu de ne pas visiter d'autres îles. Et si nous allions louper quelque chose d'incroyable ? Si tout le monde va à Krabi/Phuket/Koh Lanta, il doit bien y avoir une raison ? Mais après une semaine à courir d'île en île, c'est sûr : Koh Tao, nous revoilà :-)



\chapter{Koh Tao, bis repetita placent}
Nous arrivâmes donc à Koh Tao. "Encore ?" allez-vous me dire. "Oui" vais-je vous répondre. Mais plutôt que de faire un copier-coller de l'article précédent, je vais vous parler d'un de nos poisson préféré : le trigger-fish, ou baliste pour les francophones.


\begin{figure}[h]
\centering
\includegraphics[height=6cm,width=9cm,keepaspectratio]{p5318583.jpg}
\caption*{Sombres présages !}
\end{figure}

\textbf{Note pour alléger ma conscience :} je me rends compte que, de plus en plus, mon propos est émaillé de mots anglais qui pourraient très bien être traduits en français. Snorkeling, guesthouse, fruitshake, trigger-fish... Mais d'une part la traduction est parfois malaisée (palme/masque/tuba = snorkelling), et d'autre part, tout le monde utilise ces mots par ici, et on se met, même entre nous, à parler un franglais de voyage peu esthétique mais plus intuitif. Les mots nous sortent comme ça. J'ai horreur de ça, je trouve que ça fait snob et/ou mec qui se la pète genre \emph{"han, je suis complètement trop jet-laggé, faut que je call mon boss asap pour postponer le meeting"}. Je sais, personne ne s'en était plaint, mais moi, je m'énerve tout seul quand je fais ça, et ça me fait du bien d'en parler !


\begin{figure}[h]
\centering
\includegraphics[height=6cm,width=9cm,keepaspectratio]{p6018591.jpg}
\caption*{Le volley, c'est pour ceux qui ne savent pas sauter.}
\end{figure}

Trêve de jérémiades : le trigger-fish, pourquoi est-il cool ? Déjà, quand un plongeur indique un trigger-fish, il a l'air d'un gangster en mimant un pistolet avec sa main. Top méga cool ! Ensuite, c'est un des seuls poissons (un peu) dangereux qu'on peut croiser sous l'eau. Il est plutôt territorial, et si on passe à proximité, et en particulier au-dessus de son nid, il peut s'énerver. Un signe qui ne trompe pas, c'est sa nageoire dorsale qui se dresse, comme une gâchette (trigger) avant une attaque. A ce moment, il faut se mettre en position de pouvoir donner des coups de palmes au poisson en cas de besoin, car il va tenter de nous foncer dessus à toute vitesse pour nous donner des coups de dents ! Bon, il faut relativiser, il fait environ 50cm, quelques kilos à tout péter, donc il ne va tuer personne. Mais il peut sonner un plongeur, ou lui casser son masque, et sous l'eau, ça peut être gênant. Bref, on a eu droit à un petit briefing spécialement sur ce poisson, et ça fait un petit quelque chose quand on le voit la première fois.


\begin{figure}[h]
\centering
\includegraphics[height=6cm,width=9cm,keepaspectratio]{p1070351.jpg}
\caption*{C'est notre meilleure photo du poisson en question.}
\end{figure}

Quand il ne charge pas des plongeurs, on dirait qu'il se contente de ronger du corail avec ses gros chicots. Mais quand on y regarde de plus près, (ce que j'ai fait pendant un petit bout de temps en snorkelling) on s'aperçoit qu'il creuse le corail pour y déloger de délicieux coquillages qu'il va directement consommer en sashimi. C'est un poisson raffiné. Quand je pense qu'en tant que plongeurs on n'a même pas le droit de toucher au corail, il y a vraiment deux poids deux mesures sous l'eau !


\begin{figure}[h]
\centering
\includegraphics[height=6cm,width=9cm,keepaspectratio]{p5298515.jpg}
\caption*{De gauche à droite : trigger-fish, tortue, requin léopard, poulpe. (je parle des signes de main bien sûr, pas des gens)}
\end{figure}

Sinon, on a aussi fait beaucoup de coinche, grâce à Edwige et Antoine qu'on a encore une fois retrouvé ici. C'est con hein, mais taper le carton en buvant des bières avec des potes jusqu'à point d'heure, mais qu'est-ce que ça fait du bien !


\begin{figure}[h]
\centering
\includegraphics[height=6cm,width=9cm,keepaspectratio]{p5288481.jpg}
\caption*{Jongleurs de feu tous les soirs sur la plage.}
\end{figure}


\begin{figure}[h]
\centering
\includegraphics[height=6cm,width=9cm,keepaspectratio]{p5288409.jpg}
\caption*{Photo réussie complètement par accident. Ce n'était pas du tout ça que j'essayais de faire !}
\end{figure}

J'ai aussi envie de vous raconter un plat improbable : Salade de porc à l'ail et au citron. Ce jour là, j'avais envie d'un truc un peu plus léger, alors pourquoi pas une salade. Quand le plat est arrivé, j'ai cherché la verdure : il n'y en avait pas... C'était un tas de morceaux de porc grillé, baignant dans une sauce au citron, accompagné, sans mentir, d'au moins 20 gousses d'ails émincées, soit l'équivalent d'une tête. C'était excellent et surprenant. Chaque bouchée me faisait pleurer, à chaque fois je me disais que c'était la dernière, mais quelques secondes plus tard le piquant de l'ail s'estompait, et je repartais pour la bouchée suivante. J'ai senti l'ail pendant deux jours...


\begin{figure}[h]
\centering
\includegraphics[height=6cm,width=9cm,keepaspectratio]{p5318564.jpg}
\caption*{C'est vraiment l'enseigne d'un restaurant. Je... je n'ai pas d'explications...}
\end{figure}

Cet article est complètement décousu, je passe du coq(uillage) à l'ail sans arriver à trouver de liaison fluide entre tous les sujets dont au sujet duquel que j'ai envie que je m'exprime. Et les photos n'ont rien à voir. Ma foi, tant pis, je continue : Bien que Koh Tao soit réputée être une île plus calme que ses voisines Koh Phangan et surtout Koh Samui, on voit quand même des trucs assez rigolos/tristes. Le grand classique : le tatouage bourré de 2h du matin ! J'imagine le retour : "Tu veux vraiment garder ton col roulé ?". On voit les gens qui arrivent à peine à marcher, mais qui se disent, plein d'espoir, que ça ira mieux une fois sur le scooter. Ils enlèvent la béquille, et paf, la gravité l'emporte. Il faut savoir que sur Koh Tao, les loueurs de scooter ont des règles très strictes, et peut-être justement dues aux mecs bourrés : la moindre rayure implique de payer la pièce de rechange neuve, et de nombreux touristes se retrouvent à payer des fortunes pour des petites rayures qu'ils ne sont même pas sûrs d'avoir faites. De notre coté, vu que l'île est toute petite, on a choisi de tout faire à pied. Ainsi, on culpabilise moins de glander à la plage si on a marché une heure pour y arriver. Et de toute façon, nos plages préférées ne sont même pas accessibles en scooter. Tranquillité assurée :-)


\begin{figure}[h]
\centering
\includegraphics[height=6cm,width=9cm,keepaspectratio]{p5278398.jpg}
\caption*{:-)}
\end{figure}

Sans transition : fin de l'article !



\chapter{Hpa An, entrée au Myanmar.}
Nous partîmes donc pour le Myanmar. Par la frontière terrestre s'il vous plait ! Eh oui, depuis quelques années, le Myanmar s'ouvre de plus en plus au tourisme, et il est maintenant possible d'y rentrer à pied, par le pont de l'amitié. L'entrée dans le pays est une simple formalité. Le douanier prend même le temps de nous parler un peu de la coupe d'Europe qu'il suit depuis son boulot. On est accueilli ensuite par un jeune qui parle plutôt bien anglais, qui nous guide vers un DAB qui me file des chocs électriques plutôt que du cash. Tant pis, on paiera le taxi en dollars !


\begin{figure}[h]
\centering
\includegraphics[height=6cm,width=9cm,keepaspectratio]{p6160437.jpg}
\caption*{Environ 0,3\% du jardin des Bouddhas.}
\end{figure}

Ah, le taxi. Quelle belle entrée en matière ! Il restait deux places dans le taxi partagé. Pile pour nous. En fait, on manquait d'imagination : il restait trois places ! A quatre à l'arrière plus les bagages, et si l'on rajoute, on plutôt, si on soustrait les suspensions, on comprend d'un coup la relativité : le temps s'étire, et plus on s'approche de notre destination, plus les secondes durent ! Heureusement, une crevaison nous a permis de faire une petite pause.


\begin{figure}[h]
\centering
\includegraphics[height=6cm,width=9cm,keepaspectratio]{p6159125.jpg}
\caption*{Le mauvais type de pneu clouté.}
\end{figure}


\begin{figure}[h]
\centering
\includegraphics[height=6cm,width=9cm,keepaspectratio]{p6159117.jpg}
\caption*{Ça marche moins bien comme ça.}
\end{figure}

Nous arrivons à Hpa An (prononcer "pah aaane"), et nous visitons d'un seul coup 50\% des guesthouses de la ville. Fini le choix de la Thaïlande. Ici, on fait avec ce qu'on trouve. Pour le petit déjeuner, ça se passe dans un "tea shop". Pas de carte : on s’assoit, on nous demande si on veut du thé. "Heu, oui ?" C'est bizarre comme question, car il y a déjà une thermos de thé noir en libre service sur toutes les tables... Le thé arrive : il est au lait concentré et accompagné par des assiettes de petites pâtisseries frites. On n'avait pas commandé ça, mais bon, tout le monde à l'air d'avoir les mêmes assiettes... C'est excellent, et la première bouchée est à chaque fois une surprise : noix de coco, banane, poulet, légumes, curry, il y a tous les gouts ! Au moment de l'addition, facile : ils comptent ce qui manque des assiettes et voilà !


\begin{figure}[h]
\centering
\includegraphics[height=6cm,width=9cm,keepaspectratio]{p6169502.jpg}
\caption*{C'est bon, mais (donc ?) c'est gras.}
\end{figure}

Puis on prend les vélos et on va se balader dans la campagne. L'accueil des gens est fabuleux : tout le monde nous hèle à grand coups de "hello" ou de "bye bye" (certains sont encore en train d'apprendre les bases de l'anglais). On leur répond des "Mingalaba" qui les font marrer. Ceux qui savent parler un peu plus anglais nous accueillent avec "Where do you come from?". En entendant "From France", on a invariablement droit à "Oooooohhh, Fraaaance ! Boujourcomançavaaaa ?". On nous avait dit que la Thaïlande était le pays du sourire. Nous avons trouvé leur maître :-)


\begin{figure}[h]
\centering
\includegraphics[height=6cm,width=9cm,keepaspectratio]{p6169510.jpg}
\caption*{Aucun rapport : des termites et une chenille !}
\end{figure}

Les paysages sont magnifiques. La mousson a commencé, et c'est plutôt une bonne chose : la température est supportable, les paysages sont verts, et le ciel est toujours plein de gros cumulus bien joufflus. Idéal pour les photos, il n'y a rien de plus ennuyeux qu'un ciel bleu uniforme...


\begin{figure}[h]
\centering
\includegraphics[height=6cm,width=9cm,keepaspectratio]{p6160445.jpg}
\caption*{On a désespérément attendu une prise...}
\end{figure}

Comme partout dans le monde, les hommes construisent des trucs pas possibles dans des endroits pas croyables. Ici, ça ne loupe pas : A quelques kilomètres de la ville, on trouve un parc rempli de statues de Bouddha. On ne les a pas comptées, mais de source sûre, il y en aurait 1000 !


\begin{figure}[h]
\centering
\includegraphics[height=6cm,width=9cm,keepaspectratio]{p6169508.jpg}
\caption*{1, 2, 3, ... 1000 !}
\end{figure}

Puis ça continue avec un monastère au sommet d'un piton karstique de 700m de haut. Après quelques heures de montée sans croiser un chat, on arrive au point de vue. Imaginez un peu : on est au bord d'une falaise qui tombe à pic jusqu'à la plaine inondée en contrebas, et il nous semble qu'il nous suffirait de lever le bras pour toucher les nuages. On en voit d'autres au loin qui se laissent aller sur les champs, tout en avançant doucement. Juste à coté de nous, un moine et une fille étendent du linge en rigolant. On a mal aux mollets, mais qu'est-ce qu'on est bien ici !


\begin{figure}[h]
\centering
\includegraphics[height=6cm,width=9cm,keepaspectratio]{p6169768.jpg}
\caption*{Dire qu'il y a des gens qui vivent ici toute l'année !}
\end{figure}


\begin{figure}[h]
\centering
\includegraphics[height=6cm,width=9cm,keepaspectratio]{p6160407.jpg}
\caption*{Le moine (non, les moines ne sourient pas sur les photos).}
\end{figure}


\begin{figure}[h]
\centering
\includegraphics[height=6cm,width=9cm,keepaspectratio]{p6160409.jpg}
\caption*{Mais les filles peuvent sourire !}
\end{figure}


\begin{figure}[h]
\centering
\includegraphics[height=6cm,width=9cm,keepaspectratio]{p6160410.jpg}
\caption*{Un crabe trouvé en haut de la montagne !}
\end{figure}



 Je n'ai pas eu envie d'en faire un article : on a mangé des pizzas et visité des temples.

\chapter{Bagan, le temple des pagodes}
Nous partîmes donc pour Bagan. 24h de bus d'affilée ! On aurait pu s'arrêter à Yangon (Rangoun), mais ... mais non, pas envie. On se retrouve donc dans ce bus surclimatisé sans pouvoir rien faire quant à la température glaciale : on voyait tous les birmans autour de nous tenter de s'emmitoufler, un moine s'est fait rembarrer par le chauffeur après avoir visiblement montré l'aération. Des vis ont même été rajoutées pour bloquer les aérations en position grande ouverte. Peut-être que la climatisation, c'est un truc hyper classe dans un pays où il fait toujours chaud : "Ouah, c'était hyper classe ce trajet, tellement classe que mon thé a gelé !". Il y a aussi les vidéos de karaoké amateur toute la nuit à fond. Bref, c'était long...


\begin{figure}[h]
\centering
\includegraphics[height=6cm,width=9cm,keepaspectratio]{p6191364_hdr.jpg}
\caption*{Ça vaut le coup d'endurer un peu de mauvais karaoké...}
\end{figure}

A la sortie du bus à notre correspondance, nous sommes accueillis comme il se doit par une armée de rabatteurs. Mais... ce ne sont pas les mêmes rabatteurs que dans les autres pays. Ceux-ci nous ont vraiment aidé en nous amenant à la banque, puis à la bonne agence de bus, alors qu'ils nous proposaient des taxis à la base. Ils sont partis sans qu'on ait eu le temps de les remercier. Ils ont sûrement réussi à chopper une commission quelque part, mais honnêtement, je n'en suis même pas si sûr, tellement ils étaient gentils et avaient l'air content de nous aider...


\begin{figure}[h]
\centering
\includegraphics[height=6cm,width=9cm,keepaspectratio]{p6180451.jpg}
\caption*{Les nonnes aussi sont rasées.}
\end{figure}

On arrive à Bagan à 3h du matin. C'est l'endroit le plus connu de tout le Myanmar, et ici, ils connaissent les touristes : On repousse un taxi qui voulait 10 dollars pour faire 5km, et on part à pied vers notre hôtel pendant que le taxi nous hurle qu'il y a des serpents sur la route...


\begin{figure}[h]
\centering
\includegraphics[height=6cm,width=9cm,keepaspectratio]{p6191342.jpg}
\caption*{Il y a encore quelques années, le cheval était le principal moyen de transport.}
\end{figure}

Le soleil se lève pendant qu'on marche dans la campagne, et on devine au loin les premières des 4000 pagodes qui ont été construites pendant une période de frénésie immobilière et spirituelle il y a quelques siècles. En arrivant dans le village au petit matin, on voit les nonnes, tout de rose vêtues, parcourir les rues à la recherche d'offrandes. Contrairement aux moines qui reçoivent de la nourriture prête à être consommée, les nonnes reçoivent surtout du riz cru. La raison est qu'un homme, tout moine qu'il soit, ne peut pas faire la cuisine, c'est un boulot de femme. Soupir.


\begin{figure}[h]
\centering
\includegraphics[height=6cm,width=9cm,keepaspectratio]{p6191374.jpg}
\caption*{Des grands temples.}
\end{figure}


\begin{figure}[h]
\centering
\includegraphics[height=6cm,width=9cm,keepaspectratio]{p6201593.jpg}
\caption*{Mais aussi des tout petits !}
\end{figure}

Et si nous arrêtions de tourner autour du pot : les temples ! Ils sont pleins, ils sont partout, ils de toutes les tailles, et ils sont surtout en brique rouge, ce qui s'accorde parfaitement au vert des arbres et au bleu du ciel. Ils sont remplis de statues Bouddha et de peintres sur sable. Le ciel est magnifique avec tous ces petit nuages. On passe quelques jours géniaux à se balader à vélo dans des petits chemins plein de poussière, au milieu de tous ces temples. Quand vient le coucher de soleil, on grimpe au sommet d'un temple, on voit alors les ombres des pagodes et des arbres qui s'allongent et on essaie de penser à garder la bouche fermée. C'est dur.


\begin{figure}[h]
\centering
\includegraphics[height=6cm,width=9cm,keepaspectratio]{p6180682.jpg}
\caption*{Oiseau en bonus !}
\end{figure}


\begin{figure}[h]
\centering
\includegraphics[height=6cm,width=9cm,keepaspectratio]{p6180900.jpg}
\caption*{Les temples ont toujours été en activité, et continuent aujourd'hui à accueillir des croyants.}
\end{figure}


\begin{figure}[h]
\centering
\includegraphics[height=6cm,width=9cm,keepaspectratio]{p6180910.jpg}
\caption*{Une des rares pagode peinte.}
\end{figure}


\begin{figure}[h]
\centering
\includegraphics[height=6cm,width=9cm,keepaspectratio]{p6181249.jpg}
\caption*{Arc en ciel et pluie en bonus.}
\end{figure}

\textbf{Petite anecdote inutile :} on s'est dit qu'on a passé trop de temps dans les grottes et les temples abandonnés quand on s'est rendu compte qu'on était capables d'identifier l'odeur des crottes de chauve souris avant même de rentrer dans les endroits qui leurs servent d'habitat. Ouais, on peut carrément faire ça ! Et ça ne sert à rien !


\begin{figure}[h]
\centering
\includegraphics[height=6cm,width=9cm,keepaspectratio]{p6191277.jpg}
\caption*{Un caillou, un temple.}
\end{figure}

A quelques kilomètres de Bagan, il y a aussi le fameux mont Popa. Encore un temple construit au sommet d'un caillou. Mais à la différence de celui de Hpa An, on peut atteindre le sommet en 20 minutes par un escalier protégé des intempéries. Il y a donc beaucoup plus de monde, et de singes ! Et comme tout le caillou est considéré comme un temple, on doit y aller pieds nus. Vous croyez que ça porte chance la crotte de singe ?


\begin{figure}[h]
\centering
\includegraphics[height=6cm,width=9cm,keepaspectratio]{p6191324.jpg}
\caption*{Les singes volent des graines.}
\end{figure}


\begin{figure}[h]
\centering
\includegraphics[height=6cm,width=9cm,keepaspectratio]{p6191315.jpg}
\caption*{Donc ils se font descendre.}
\end{figure}


\begin{figure}[h]
\centering
\includegraphics[height=6cm,width=9cm,keepaspectratio]{p6191313.jpg}
\caption*{Et ils finissent en prison !}
\end{figure}



\chapter{Kalaw, trois jours de voyage dans le temps}
Nous arrivâmes donc à Kalaw, ville surtout connue pour être le départ d'un trek de trois jours en direction du lac Inle. Et ... nous partîmes donc pour un trek de trois jours en direction du lac Inle, accompagnés d'un guide et de 8 autres randonneurs !


\begin{figure}[h]
\centering
\includegraphics[height=6cm,width=9cm,keepaspectratio]{p6242126.jpg}
\caption*{Notre hôtesse le temps d'une nuit.}
\end{figure}

Notre guide, bien que sympathique, ne parlait pas beaucoup. Son mutisme fut toutefois largement compensé par la taille du groupe : Nous étions un peu déçus au début d'être dans un groupe si inhabituellement grand, mais s'il n'y a que des gens sympathiques, c'est un plaisir !




\begin{figure}[h]
\centering
\includegraphics[height=6cm,width=9cm,keepaspectratio]{p6242138.jpg}
\caption*{La bande au complet. Le guide est celui qui a des moustaches sur le ventre.}
\end{figure}

Les paysages étaient magnifiques tout le long du trajet, et changeaient en permanence. Nous avons alterné entre sentier un peu montagneux, petits champs cultivés, et enfin forêt et terre rouge. Si l'on rajoute de gros nuages, et des paysans qui font tout à la main, on obtient ce genre de paysages.




\begin{figure}[h]
\centering
\includegraphics[height=6cm,width=9cm,keepaspectratio]{p6242156.jpg}
\caption*{La mémé n'approuve pas qu'on sourie aux touristes.}
\end{figure}


\begin{figure}[h]
\centering
\includegraphics[height=6cm,width=9cm,keepaspectratio]{p6242142.jpg}
\caption*{Il est beau mon buffle !}
\end{figure}


\begin{figure}[h]
\centering
\includegraphics[height=6cm,width=9cm,keepaspectratio]{p6231726.jpg}
\caption*{C'est inquiétant tout ce gris...}
\end{figure}


\begin{figure}[h]
\centering
\includegraphics[height=6cm,width=9cm,keepaspectratio]{p6242167.jpg}
\caption*{Vous vous imaginez aussi ce qui passe par la tête du taureau à ce moment ? (Indice, c'est une charrue au premier plan)}
\end{figure}

Voir ces paysans cultiver leur terre nous fait voyager dans le temps : pas de tracteur, aucune machine, et énormément de main d’œuvre pour désherber en permanence. Les champs sont labourés avec des buffles et des charrues en bois ! Seule concession visible à la modernité : quelques sacs d'engrais qui trainent par ci par là. Et malgré ce retour dans le temps (juste pour nous, pas pour eux), tous ces gens sont hyper souriants, et ont l'air de passer, pour la plupart, des super journées entre potes : ils bossent en petits groupes, papotent et rigolent toute la journée. (A moins que ce ne soit juste les touristes qui les fassent rire, si ça se trouve, ils font la gueule le reste du temps... éternel problème de l'observateur qui influence l'expérience). Cette ambiance a par moments un goût de paradis perdu : et si ces gens étaient les plus heureux du monde ? Doit-on les envier ? Voudraient-ils changer de vie s'ils le pouvaient ? Nous envient-ils ? Je n'ai pas de certitudes, mais il me semble qu'il y a un couac dans ce tableau : le choix. Leur mode de vie est subi, que feraient-ils s'ils avaient le choix ?




\begin{figure}[h]
\centering
\includegraphics[height=6cm,width=9cm,keepaspectratio]{p6242160.jpg}
\caption*{Ils posent même pour les photos.}
\end{figure}


\begin{figure}[h]
\centering
\includegraphics[height=6cm,width=9cm,keepaspectratio]{p6242163.jpg}
\caption*{On dirait qu'un des gamins est en train de... disons fertiliser le champ !}
\end{figure}


\begin{figure}[h]
\centering
\includegraphics[height=6cm,width=9cm,keepaspectratio]{p6252333.jpg}
\caption*{Ouh les belles grimaces !}
\end{figure}


\begin{figure}[h]
\centering
\includegraphics[height=6cm,width=9cm,keepaspectratio]{p6231746.jpg}
\caption*{Coucher de soleil.}
\end{figure}

La première nuit se passe dans une guesthouse : à l'étage, un grand dortoir, et une petite pièce pour la famille. Les repas sont excellents, on découvre des nouvelles saveurs, comme des plants de potirons ou de moutarde (pas le fruit hein, toute la plante donc !). La salle de bain est dehors : trois vagues panneaux de bambou tressés autour d'un bac d'eau, dans la boue, ça fera bien l'affaire non ? Certains ne sont pas à l'aise en s'y lavant, mais on se rend compte que les locaux sont très respectueux, ou s'en foutent, et ne regardent pas.




\begin{figure}[h]
\centering
\includegraphics[height=6cm,width=9cm,keepaspectratio]{p6242122.jpg}
\caption*{Voici la salle de bain. Intimité au top !}
\end{figure}

Petite anecdote du prof : alors que nous mangions tous des bananes, tout le groupe s'est mis à parler des multiples qualités de ce fruit courbé, et j'ai balancé au hasard qu'il était plutôt radioactif. Ça a intrigué tout le monde, et ils se sont mis à me poser plein de questions sur les bananes et la radioactivité. Ceux qui me connaissent doivent imaginer la scène... Bref, ils ont décrété que je serais le prof du groupe, et m'ont demandé une autre leçon sur le sujet de mon choix lors du repas du soir ! Si si, pour de vrai ! Ils ont eu le droit à l'histoire de l'Univers.




\begin{figure}[h]
\centering
\includegraphics[height=6cm,width=9cm,keepaspectratio]{p6242214.jpg}
\caption*{Quoi de mieux qu'une photo de la voie lactée pour illustrer le cours ?}
\end{figure}


\begin{figure}[h]
\centering
\includegraphics[height=6cm,width=9cm,keepaspectratio]{p6242208.jpg}
\caption*{Le diner aux chandelles.}
\end{figure}

Nous passons la deuxième nuit dans un monastère. Il est rempli de gamins qui courent dans tous les sens, et nous demandent de jouer au foot avec eux. J'ai donc gagné un point hippie en tapant le ballon en tong dans le gravier avec un petit moine qui portait sa toge orange sur l'épaule. Tellement cliché qu'on aurait pu tourner une pub pour Benetton. Ces mêmes petits moines (ce sont en réalité des novices, mais appelons les quand même des moines) se sont levés le lendemain à 5 heures pour la prière. Comme nous dormions dans la salle de prière, juste séparés de l'autel par un drap tendu, nous avons pu admirer ces petits crânes rasés psalmodier en cœur leur prières.




\begin{figure}[h]
\centering
\includegraphics[height=6cm,width=9cm,keepaspectratio]{p6242201.jpg}
\caption*{Notre dortoir.}
\end{figure}

Le plus étonnant dans ce monastère, c'est que les adultes sont quasiment invisibles. Il nous a semblé que ces gamins de 5 à 10 ans étaient complètement autonomes, s'occupant tout seuls de la prière, des repas, et des douches. Ah oui, les douches, parlons-en : les moines se lavent autour d'un grand réservoir dans la cour, en gardant leur toge. Comme les touristes n'ont pas les mêmes habitudes ou vêtements, pour notre plus grand confort ils ont ajouté 4 murets à l'autre bout de la cour. On peut donc se laver à l'abri des regards en utilisant parcimonieusement l'eau des seaux qu'on a apporté.




\begin{figure}[h]
\centering
\includegraphics[height=6cm,width=9cm,keepaspectratio]{p6242206.jpg}
\caption*{En route pour la douche.}
\end{figure}


\begin{figure}[h]
\centering
\includegraphics[height=6cm,width=9cm,keepaspectratio]{p6242202.jpg}
\caption*{La salle de bain.}
\end{figure}

Enfin, nous arrivâmes en vue du lac. Il ne ressemble pas du tout à un lac comme chez nous, avec, par exemple, une berge. Celui-là n'en a pas : il est bordé de marécages, plein d'herbes hautes, et on ne peut naviguer en pleine eau qu'après avoir zigzagué dans des canots pendant une dizaine de minutes au milieu des villages flottants et de leurs champs ... flottants eux aussi ! Ils cultivent toutes sortes de légumes qui ne disposent pas de traduction officielle en français, mais on a quand même reconnu des tomates ! Après avoir passé trois jours dans cette campagne magnifique, voir ces gens vivre sur le lac, toujours sur un bateau ou des pilotis, avec en fond des nuages énormes qui se préparaient, je ne sais pourquoi, mais je me suis senti super ému à l'idée que ce mode de vie était peut-être entrain de disparaitre...






\begin{figure}[h]
\centering
\includegraphics[height=6cm,width=9cm,keepaspectratio]{p6252392.jpg}
\caption*{Village sur le lac.}
\end{figure}


\begin{figure}[h]
\centering
\includegraphics[height=6cm,width=9cm,keepaspectratio]{p6252410.jpg}
\caption*{Transport public sur le lac.}
\end{figure}


\begin{figure}[h]
\centering
\includegraphics[height=6cm,width=9cm,keepaspectratio]{p6262423.jpg}
\caption*{Petite dégustation de vin à l'arrivée au lac.}
\end{figure}





 (Il y a plein de belles photos, j'ai vraiment eu du mal à choisir...)



\chapter{Hsipaw, trek, et rebelles dans la montagne.}
Nous arrivâmes donc à Hsipaw. A 3 heures du matin, c'est tellement plus sympa. Mais le tuk-tuk du Red Dragon attendait déjà, et réussi à attirer en son antre les quelques backpackers encore hagards, grâce à la promesse d'une portion de nuit gratuite.


\begin{figure}[h]
\centering
\includegraphics[height=6cm,width=9cm,keepaspectratio]{p7013352.jpg}
\caption*{C'est la mousson !}
\end{figure}

Après une demi-nuit, à moitié reposés, on commence à organiser le trek suivant, car c'est pour ça que les gens viennent à Hsipaw. Comme on a lu qu'il y avait une sorte de mafia du trek dans le village, on essaie de la contourner. En effet, contrairement à d'autres villages où les rabatteurs se jettent sur nous comme des canards sur du pain sec, ici, rien. C'est simple : Les treks sont tous organisés par Mr. Charles. Tous ? Non : Un petit village... euh non... une petite agence résiste encore et toujours à l'hégémonie de l'empire du mal... euh... de l'empire de Charles. Le nom de l'agence ne s'invente pas : "Ma Boat Boat". Sans savoir où elle se trouve, on la cherche et on tombe dessus en 30 secondes. Trop facile ! Il s'avère que Ma Boat Boat est le nom de la gironde propriétaire de l'agence, et elle nous organise rapidement un trek de trois jours, sous réserve de - détail rassurant - l'absence de combats dans le coin.


\begin{figure}[h]
\centering
\includegraphics[height=6cm,width=9cm,keepaspectratio]{p6293189.jpg}
\caption*{Le triage du riz.}
\end{figure}

Alors que nous discutions organisation de trek, une bande de jeunes nonnes se sont retrouvées coincées par la pluie avec nous sous l'auvent. Je dois avouer que quand on est entourés de petites filles tout de rose vêtues, qui piaillent et nous observent du coin de l’œil en se poussant du coude, c'est difficile de ne pas sourire bêtement.


\begin{figure}[h]
\centering
\includegraphics[height=6cm,width=9cm,keepaspectratio]{p7023570.jpg}
\caption*{Elle est pas TROP MIMI ?}
\end{figure}


\begin{figure}[h]
\centering
\includegraphics[height=6cm,width=9cm,keepaspectratio]{p7033572.jpg}
\caption*{Il y en a des tas ! Partout !}
\end{figure}

On ne part pas tout de suite pour le trek : on se garde quand même une journée entière pour visiter la ville. Ce sera notre "journée échec". D'abord, on commence par se lever hyper tôt pour aller voir le fameux marché à la bougie : A 5h du mat, on est en route pour le marché, et... il fait déjà jour ! Bref, on a vu une bougie, et on est retourné se coucher. En fait, quand on est arrivés avec notre bus de nuit, ça aurait été la bonne heure. Mais franchement, quelle idée de faire des marchés à 3h du matin ?


\begin{figure}[h]
\centering
\includegraphics[height=6cm,width=9cm,keepaspectratio]{p6303192.jpg}
\caption*{La seule bougie du marché.}
\end{figure}

L'échec se poursuit avec notre tentative ratée de marcher jusqu'aux fameuses sources chaudes. Notre GPS indiquait bien un sentier, mais après avoir mis 30mn pour faire 500m au milieu des rizières inondées, à ne pas savoir si on marche sur un chemin ou dans un ruisseau, on a fait demi-tour. Et comme souvent, c'est quand on décide d'arrêter d'essayer de faire des trucs que ça commence à s'arranger : on tombe d'abord sur "little bagan". Ce sont 4 petites pagodes en ruines. C'est joli et paisible, même si le nom est un peu usurpé (ce serait comme si on baptisait "Little Paris" un pylône de haute tension), mais surtout, nos pas nous amènent directement vers un restaurant tout aussi joli et paisible dans lequel on a pu manger des schniztels avec de la purée ! RRRRRAAAAAaaaaaaahhhhh enfin des gens qui savent cuisiner la viande et les patates !!!! Si vous allez dans le coin, ça s'appelle "popcorn garden".


\begin{figure}[h]
\centering
\includegraphics[height=6cm,width=9cm,keepaspectratio]{p6303200.jpg}
\caption*{Voilà ce qui arrive quand on ne désherbe pas.}
\end{figure}

Le lendemain, nous partons pour le trek. Nous avons deux guides pour nous : Pioupiou, le guide officiel et Danny, son pote en formation, 34 ans à eux deux. Oui, 34 ans, c'est aussi mon âge à moi tout seul (Aïe). On se balade de petits villages en petits villages, et les langues changent tous les quelques kilomètres : birman, puis palaung, puis shan. On voit des champs d'ananas, et aussi des champs de thé qui ici, ne sont cultivés que par des Palaung, et on devine en voyant les maisons que le thé est plus rentable que les ananas. Qui l'eut cru ?


\begin{figure}[h]
\centering
\includegraphics[height=6cm,width=9cm,keepaspectratio]{p7013358.jpg}
\caption*{Une mémé dans le village.}
\end{figure}


\begin{figure}[h]
\centering
\includegraphics[height=6cm,width=9cm,keepaspectratio]{p7013526.jpg}
\caption*{Notre hôtesse et sa mère.}
\end{figure}

En arrivant au village où nous devons dormir, on apprend deux choses : Premièrement, la maison dans laquelle on dort appartient à Mr Charles. Deuxièmement, il y a des combattants plus loin dans la vallée, donc hors de question d'aller plus loin, on fait demi-tour demain matin. Dommage... Mais cela ne nous empêche pas d'adorer ce petit village et même de se faire inviter pour le thé chez la maman de notre hôte !


\begin{figure}[h]
\centering
\includegraphics[height=6cm,width=9cm,keepaspectratio]{p7013529.jpg}
\caption*{Le village en question.}
\end{figure}

Le lendemain, ce sera donc retour anticipé, heureusement par un autre chemin. Le point d'orgue de la journée (d'après notre guide) : une cascade ! On a essayé d'être enthousiastes, et de lui dire que oui oui, elle était très belle, et très haute, et très... humide. Je ne sais pas si on a été très convaincants.


\begin{figure}[h]
\centering
\includegraphics[height=6cm,width=9cm,keepaspectratio]{p7043577.jpg}
\caption*{Sur le quai de la gare.}
\end{figure}



\chapter{Kyaukme, un village occupé, un moine nominé}
Nous arrivâmes donc à Kyaukme (prononcez "Chie au met"), après un petit voyage en train sur la voie la moins bien entretenue du monde ! Dès qu'on dépasse les trente à l'heure, le train se met à tanguer de manière impressionnante, et les abords des voies sont si mal entretenus (entendre, pas du tout), que c'est le train lui-même qui élague approximativement les branches et les herbes hautes à chaque passage. Marion l'apprendra à ses dépends, il est dangereux de se tenir près d'une fenêtre ouverte : c'est un coup à se prendre une branche dans la margoulette !


\begin{figure}[h]
\centering
\includegraphics[height=6cm,width=9cm,keepaspectratio]{p7053598.jpg}
\caption*{C'est un forgeron. Vous comprendrez plus tard.}
\end{figure}

Kyaukme s'ouvre doucement au tourisme : il n'y a que deux guesthouses autorisées pour les touristes. Nous faisons une petite balade dans la ville, qui nous amène dans un temple au sommet d'une petite colline. Le jardinier nous accueille chaleureusement et s'improvise immédiatement guide. Moment notable de la visite : au moment de redescendre de l'autre coté de la colline, on se fait presque encercler par une bande de chiens errants et - une fois n'est pas coutume - agressifs. Demi-tour.


\begin{figure}[h]
\centering
\includegraphics[height=6cm,width=9cm,keepaspectratio]{p7053631.jpg}
\caption*{La seule qui ne s'arrête pas de bosser pour la photo, c'est la femme enceinte. Si, je vous jure.}
\end{figure}


\begin{figure}[h]
\centering
\includegraphics[height=6cm,width=9cm,keepaspectratio]{p7053634.jpg}
\caption*{Là, vous me croyez ?}
\end{figure}

La ville est surtout connue pour être le point de départ des treks organisés par Thura. Nous le rencontrons le soir même, ainsi qu'un autre couple de Français, et décidons de partir tous les cinq le lendemain ! Une condition : je vais devoir apprendre, sur le tas, à conduire une moto semi-automatique, car le début du trek est loin du centre, et c'est le seul moyen de transport possible.


\begin{figure}[h]
\centering
\includegraphics[height=6cm,width=9cm,keepaspectratio]{p7053645.jpg}
\caption*{Ici, on prépare...}
\end{figure}


\begin{figure}[h]
\centering
\includegraphics[height=6cm,width=9cm,keepaspectratio]{p7053647.jpg}
\caption*{... des "nouilles shan"...}
\end{figure}


\begin{figure}[h]
\centering
\includegraphics[height=6cm,width=9cm,keepaspectratio]{p7053646.jpg}
\caption*{...et de la salade de thé fermenté !}
\end{figure}

Après quelques aller-retour dans une rue déserte, me voici avec deux packs d'eau entre les jambes, tout fier au guidon de ma moto, en train de rouler sur les petites routes birmanes ! Étant le pilote débutant, je suis le seul des trois motos à ne pas avoir de passagère dans le dos.



Le rythme est cool, et nous faisons de nombreuses pauses. Par exemple pour tester la noix de bétel à chiquer : on a eu droit à une version pour touristes, avec plein d'épices et peu de noix. Malgré ces artifices, j'ai envie de dire que "ça reste une boisson d'homme". Ce n'est pas très bon, j'ai surtout l'impression de mâcher du bois vert, et avec le peu de noix, difficile de sentir un quelconque effet...


\begin{figure}[h]
\centering
\includegraphics[height=6cm,width=9cm,keepaspectratio]{p7053621.jpg}
\caption*{L'usine de papier de bambou.}
\end{figure}

Petite pause également chez un forgeron local, chez qui on fera une partie de billard indien, avant d'aller visiter deux usines de papier de bambou : une manuelle et une automatique.


\begin{figure}[h]
\centering
\includegraphics[height=6cm,width=9cm,keepaspectratio]{p7053613.jpg}
\caption*{Le billard indien.}
\end{figure}

Après quelques kilomètres sur les petits chemins dans la montagne, on comprend mieux l'intérêt de la moto : route défoncée, mares de boue, cailloux... on est tout le temps en train de slalomer, et plus on s'approche de notre destination, pire c'est. Pour le dernier kilomètre, les passagères deviennent des piétons et partent devant, pendant qu'on galère à 2 km/h sur le sentier. Et enfin, nous arrivons : des maisons en bambou, des rues pleines de gamins qui courent et de vieux qui errent, nous voici dans un village palaung.


\begin{figure}[h]
\centering
\includegraphics[height=6cm,width=9cm,keepaspectratio]{p7053676.jpg}
\caption*{Quatre touristes devant le nez, et pas une qui le lève de ses leçons !}
\end{figure}


\begin{figure}[h]
\centering
\includegraphics[height=6cm,width=9cm,keepaspectratio]{p7053686.jpg}
\caption*{Un petit garçon qui a tout compris à la composition !}
\end{figure}

Après une petite visite de l'école, très studieuse, on va rendre visite ... roulement de tambours ... aux militaires ! (Je crois que j'ai rarement utilisé des roulements de tambours de manière autant appropriée) Hé oui, le village est occupé par l'armée Shan. Pour faire simple, et parce que je ne suis pas sûr d'avoir compris beaucoup plus, les Shans forment la minorité la plus importante du Myanmar, et certains voudraient l'indépendance de l'état Shan, et considèrent que les Palaungs sont sur leur territoire.


\begin{figure}[h]
\centering
\includegraphics[height=6cm,width=9cm,keepaspectratio]{p7053739.jpg}
\caption*{Au milieu, c'est le chef !}
\end{figure}

On fait la rencontre d'une bande de jeunes qui squattent au sommet du village, et ont l'air ravi de rencontrer des occidentaux ! Sérieusement, cette rencontre a été surréaliste : on a pris des selfies avec des rebelles armés ! Et ils étaient super contents ! Ils sont même allés chercher leur pétoire pour poser avec, avant de nous les prêter ! On apprend qu'ils n'ont pas vraiment le choix d'être ici : de temps en temps, l'armée Shan débarque dans un village, et demande tant de volontaires. Le village a intérêt à les fournir.


\begin{figure}[h]
\centering
\includegraphics[height=6cm,width=9cm,keepaspectratio]{p7053727.jpg}
\caption*{Les journées sont longues. L'un d'entre eux nous a avoué que ça faisait deux ans qu'il n'était pas rentré chez lui, alors que c'est à moins d'une journée de marche.}
\end{figure}

Ces "volontaires" sont ensuite catapultés dans un village à occuper. Sans solde, sans nourriture... Du coup, les volontaires catapultés se retrouvent à devoir demander au village de les nourrir. Difficile de promouvoir l'amitié des peuples dans ces conditions, mais malgré tout j'ai eu l'impression que la plupart des gens (villageois et militaires) étaient conscients qu'ils étaient tous plus des victimes qu'autre chose. On a vu par exemple les militaires descendre de leur colline le soir pour aller aider les villageois à rouler le thé fermenté...


\begin{figure}[h]
\centering
\includegraphics[height=6cm,width=9cm,keepaspectratio]{p7053755.jpg}
\caption*{Ici, le thé se boit un peu, mais se mange surtout fermenté. Première étape : le cuire à la vapeur.}
\end{figure}


\begin{figure}[h]
\centering
\includegraphics[height=6cm,width=9cm,keepaspectratio]{p70638211.jpg}
\caption*{Puis il est roulé/froissé avec ces machines.}
\end{figure}


\begin{figure}[h]
\centering
\includegraphics[height=6cm,width=9cm,keepaspectratio]{p7053764.jpg}
\caption*{C'est convivial, on peut s'y mettre à quatre.}
\end{figure}


\begin{figure}[h]
\centering
\includegraphics[height=6cm,width=9cm,keepaspectratio]{p7053689.jpg}
\caption*{Et on obtient ça ! C'est un peu comme des épinards, mais amer. C'est leur spécialité locale, ils en sont très fiers, alors je ne vais rien dire.}
\end{figure}

Nous commençons à marcher seulement le lendemain, en direction d'un autre village. Ce ne sera pas très intense : 8km en 4 heures, ce n'est pas trop violent. Thura connait tout le monde, et chaque petite cabane est l'occasion d'une pause : on s'arrête manger du jack-fruit tout juste cueilli, on rencontre plusieurs de cueilleurs de thé, et on fait une pause chez un pépé qui a un fusil ! Mais quel fusil ! Pas de cartouche, c'est à l'ancienne : un bout de coton, de la poudre et des morceaux de plomb taillés au canif, 10mn en tout pour recharger. A ce qu'il parait, il chasse le sanglier avec !


\begin{figure}[h]
\centering
\includegraphics[height=6cm,width=9cm,keepaspectratio]{p7063826.jpg}
\caption*{Rien à voir avec le durian. Le jack-fruit, c'est bon ! Le seul problème : c'est hyper collant. Si on le touche une fois, les mains collent pour la journée. Le dépiauter à la cuillère est devenu un art pour certains !}
\end{figure}


\begin{figure}[h]
\centering
\includegraphics[height=6cm,width=9cm,keepaspectratio]{p7063834.jpg}
\caption*{Et là, je hurle :"ATTENTION, UN OURS" ! Haha, non, je déconne...}
\end{figure}

Ce deuxième village ressemble beaucoup au premier, si ce n'est qu'il est libre de militaires. Les gens sont très curieux et très accueillants : on se fait tellement inviter à boire le thé qu'on ne fait plus que ça de la journée. Le moment est bien choisi : on nous explique que ce soir c'est la fête, car le lendemain matin, un novice sera intronisé moine, et prendra la direction du monastère qui était orphelin depuis le départ du moine précédent. C'est très rare, et le village est en effervescence, et tout le monde est en train de se préparer (entendre, les femmes cuisinent et les hommes boivent et fument). On passe ainsi la journée à boire du thé, et à prendre des photos avec tout le monde. Alex et moi sommes même pris à part un moment, et amené dans une petite pièce avec que des hommes, où on nous a discrètement offert de l'alcool de riz et de la viande fermentée. On avait l'impression d'être des ados en train de boire en cachette des parents !


\begin{figure}[h]
\centering
\includegraphics[height=6cm,width=9cm,keepaspectratio]{p7063871.jpg}
\caption*{Oui, le chef du village me caresse le genou... "Marion, fait quelque chose !"}
\end{figure}


\begin{figure}[h]
\centering
\includegraphics[height=6cm,width=9cm,keepaspectratio]{p7063842.jpg}
\caption*{Une petite fille très photogénique.}
\end{figure}

Le soir arrive, et la fête commence au son des tambours et des cymbales. "Tsoum tsoum tsoum BAM" en boucle. Pendant des heures. C'était tellement simple que j'ai réussi à en jouer du premier coup. C'était très monotone, mais ça les éclatait ! La petite particularité marrante, c'est qu'ils étalent du riz cuit sur la peau du tambour. Le son serait meilleur. On a surtout la main qui colle à la fin de la session.


\begin{figure}[h]
\centering
\includegraphics[height=6cm,width=9cm,keepaspectratio]{p7064401.jpg}
\caption*{Ça vaut 10 point hippies d'un coup !}
\end{figure}


\begin{figure}[h]
\centering
\includegraphics[height=6cm,width=9cm,keepaspectratio]{p7064388.jpg}
\caption*{Peu de gens le savent : c'est comme ça qu'on fait des galettes de riz soufflé !}
\end{figure}

Nous attendions surtout la danse, alors on a été un peu déçu quand la messe a commencé... C'était long ! Déjà que c'est long quand on comprend la langue, mais quand on ne comprend ni la langue, ni la religion... Au bout d'une heure dans le brouillard (il était 22h), on est rentré faire une sieste en se disant qu'on ressortirait quand on entendrait la musique. Et contre toute attente, nous l'avons vraiment fait ! A 23h, après deux heures de messe, les tambours sont chauds, et nous aussi, et on assiste à la fameuse danse. Une file de filles tourne en rond autour d'un poteau. Puis une file de garçons tourne autour des filles. Voilà. 23h30 nous étions couchés à nouveau.


\begin{figure}[h]
\centering
\includegraphics[height=6cm,width=9cm,keepaspectratio]{p7064416.jpg}
\caption*{La danse tant attendue !}
\end{figure}

La procession nous a réveillé à l'aube. Tambour, cymbales et la moitié du village juste sous sa fenêtre, c'est plutôt efficace ! Ils ont fait quelques tours autour du monastère, en montrant à tous les offrandes, et ils ont planté un mât en l'honneur du moine avant d'attaquer une nouvelle messe. Ce coup-ci, on n'a pas attendu et on s'est éclipsés discrètement, pour retourner à la civilisation.


\begin{figure}[h]
\centering
\includegraphics[height=6cm,width=9cm,keepaspectratio]{p7074418.jpg}
\caption*{La procession matinale.}
\end{figure}


\begin{figure}[h]
\centering
\includegraphics[height=6cm,width=9cm,keepaspectratio]{p7074433.jpg}
\caption*{C'est la fête !}
\end{figure}


\begin{figure}[h]
\centering
\includegraphics[height=6cm,width=9cm,keepaspectratio]{p7074445.jpg}
\caption*{"Je peux vous prendre en photo ?" Les avis semblent partagés...}
\end{figure}


\begin{figure}[h]
\centering
\includegraphics[height=6cm,width=9cm,keepaspectratio]{p7074453.jpg}
\caption*{Elles sont vraiment habillées comme ça tous les jours. La tradition, ça veut dire quelque chose ici !}
\end{figure}

 En particulier pas mal de portraits.

\chapter{Le choc de l'Inde}
Nous arrivâmes donc à Delhi. Par avion. L'échec. Je sais, c'est triste. Mais pour notre défense, on n'avait pas le choix si on voulait visiter l'Inde cette année : les règles d'obtention du visa indien ayant changé au début de notre voyage - je vous passe les détails - la seule possibilité était le visa électronique, facile à obtenir, mais nous imposant de rentrer dans le pays par un aéroport et nous limitant à 30 jours dans le pays. On est arrivé en fin d'après midi, et Sophie, la copine de Marion, arrivait au même aéroport 5h plus tard.


\begin{figure}[h]
\centering
\includegraphics[height=6cm,width=9cm,keepaspectratio]{p7214486.jpg}
\caption*{C'est quand même pratique l'avion !}
\end{figure}

On est sorti faire un tour, histoire de se dégourdir les jambes, et face à la chaleur ambiante, on se précipite pour rentrer et attendre les 5h à venir au frais. Non. Comment ça non ? Ben non, on ne rentre pas dans l'aéroport sans billet d'avion. Ah, c'est ballot. Bon, ben c'est pas grave, on va se poser sur un morceau de trottoir brûlant en attendant Sophie...


\begin{figure}[h]
\centering
\includegraphics[height=6cm,width=9cm,keepaspectratio]{p7214492.jpg}
\caption*{Complètement blasés de prendre l'avion...}
\end{figure}

Le premier contact avec l'Inde est rude : déjà, la circulation ! Je sais, j'ai déjà dit beaucoup de mal de la circulation de beaucoup de pays. Mais. Et je pèse mes mots. C'est. L'Inde. Le. Pire. Tout le monde force le passage en permanence, les motos, les tuk-tuks, les piétons, les voitures, les chariots, les vélos et les camions slaloment entre les vaches qui s'en foutent et les cratères de la route défoncée. Les gens se bloquent mutuellement en refusant de céder un centimètre, et foncent dès qu'ils ont deux mètres de libre devant eux, le tout en utilisant leurs putains de bordel de klaxons modifiés de merde qui défoncent les oreilles ! Autre exemple : je suis dans une rue large comme un couloir, une moto arrive par derrière en ronflant, je me pousse, mais il klaxonne quand même ! Je HAIS les klaxons, merde !!!! (pfiou, désolé que vous ayez du endurer tout ça, mais ça fait un bien fou).


\begin{figure}[h]
\centering
\includegraphics[height=6cm,width=9cm,keepaspectratio]{p7295285.jpg}
\caption*{Normal.}
\end{figure}

Et puis les arnaques. Le chauffeur a tenté tout ce qu'on avait lu sur le web : heureusement qu'on était préparés. Le coup classique, c'est le touriste qui débarque à l'aéroport avec sa réservation d'hôtel. Le chauffeur, sympa, offre de passer un coup de fil pour prévenir de l'arrivée du voyageur. Et là, pas de bol, au téléphone, un mec explique au touriste qu'il y a eu un problème, blablabla, bref, l'hôtel est complet. Mais heureusement, le taxi a un pote... et le touriste se retrouve dans un autre hôtel. Évidemment, c'était un coup monté, et c'est à un complice, et non à l'hôtel que le touriste a parlé par le téléphone.


\begin{figure}[h]
\centering
\includegraphics[height=6cm,width=9cm,keepaspectratio]{p7224496.jpg}
\caption*{Quelques friandises apportées par Sophie pour adoucir le choc.}
\end{figure}

Il y a aussi, "le centre ville est fermé la nuit", "tous les hôtels de Delhi sont complets, mais je peux te conduire au Taj-Mahal", "c'est la révolution", "les extra-terrestres ont débarqué, mais tu peux dormir chez ma belle sœur" ou encore, l'hôtel porte le bon nom, mais c'est juste une mauvaise copie d'un hôtel qui marche mieux la rue d'à coté. Il faut toujours refuser les propositions de taxis, rester ferme sur la destination, et si possible, ostensiblement lui montrer qu'on sait où on va avec le GPS du téléphone. Et par défaut, quand un mec qui bosse dans le tourisme s'adresse à un touriste, c'est une bonne pratique de tout mettre en doute.


\begin{figure}[h]
\centering
\includegraphics[height=6cm,width=9cm,keepaspectratio]{p8056623.jpg}
\caption*{Quand on voit la tête de certains conducteurs, faut pas s'étonner du bordel dans les rues...}
\end{figure}

On fini par arriver sain et sauf à notre hôtel, après avoir cru mourir trois fois sur la route. Là, ils n'ont pas notre chambre, qu'on avait pourtant réservé. Il est 2h du matin, on en a marre, on se retrouve donc entassés dans la pire chambre de tout notre voyage. Le lendemain, on attaque l'organisation du reste du voyage, car il y a du boulot. On comptait aller dans les montagnes pour échapper à la mousson, mais la mousson en avait décidé autrement : inondations, glissements de terrain, ce n'est pas le bon moment... Direction le Taj-Mahal, et pour ça, il nous faut des billets de train. C'est là, qu'on a commencé à visiter des agences de voyage. La première était conseillée  par notre hôtel. Quand on en est sortis, un homme qui passait par là, nous a dit que cette agence était une arnaque, qu'ils mentent sur le fait de travailler avec le gouvernement ! Ah ben oui, c'est aussi ce qu'il nous semblait. Et ce mec nous propose de nous amener à la vraie agence. Tu parles. On en a fait trois comme ça, avant de comprendre qu'il fallait arrêter d'essayer d'être poli avec les gens qui faisaient semblant de vouloir nous aider. On a pris une carte, marché en ignorant les gens qui nous parlaient, et fini par arriver dans une agence. Ils nous ont dit que les trains pour le Taj Mahal étaient complets pour les 2 prochains jours. Et ils ont commencé à dérouler leur speech, et peu après, on ressortait de l'agence en compagnie de Karma, notre chauffeur pour les 14 prochains jours :-)


\begin{figure}[h]
\centering
\includegraphics[height=6cm,width=9cm,keepaspectratio]{p7244836.jpg}
\caption*{Sophie, Marion, et Karma, notre chauffeur.}
\end{figure}

Au programme : le Rajasthan ! Normalement, cet état aurait du être inondé à cause de la mousson, mais pas cette année. Ben oui, les nuages sont allé directement dans les montagnes, là où justement on ne peut plus aller. C'est vraiment le monde à l'envers... Mais avant, un petit tour par Agra pour visiter le Taj Mahal, et le départ c'est tout de suite ! Et rassurez-vous, même si ce premier article sur l'Inde semble un peu négatif, il n'y a pas que des arnaques et des klaxons en Inde...


\begin{figure}[h]
\centering
\includegraphics[height=6cm,width=9cm,keepaspectratio]{p8056621.jpg}
\caption*{Non, pas que des arnaques et des klaxons, il y a aussi des déchets partout... mais rassurez-vous, il n'y a pas que ça !}
\end{figure}

Et non, pas de photos pour le moment. Je n'ai pas pris une seule photo de Delhi, faudra attendre les articles suivants !

\chapter{Le Taj-Mahal, et comment j'ai noyé mon guide.}
Nous arrivâmes donc à Agra, berceau du Taj-Mahal. C'était le soir, et Karma, notre chauffeur, nous a tout de suite conduit à un endroit un peu secret pour admirer le coucher du soleil sur le "Taj" - comme on dit par ici - mais sans payer l'entrée du parc. C'est facile, on longe le parc jusqu'à la rivière, on contourne le champs de fouille archéologique, et on tombe là-dessus :


\begin{figure}[h]
\centering
\includegraphics[height=6cm,width=9cm,keepaspectratio]{p7224493.jpg}
\caption*{La photo de rend pas très bien compte de la taille. C'est grand !}
\end{figure}

Karma nous conduit ensuite à un hôtel de son choix. Il nous avait dit qu'il connaissait tous les hôtels, et qu'on avait qu'à donner notre budget. On s'est laissé tenté. L'hôtel est très bien, on a même du, pour la première fois, donner un pourboire au mec qui nous a monté nos bagages dans la chambre afin qu'il accepte de nous laisser tranquille ! Le seul truc qui nous chiffonne, c'est que le prix de l'hôtel est, à la roupie près, le budget maximum qu'on avait donné à Karma. On demandera moins cher la prochaine fois. Malgré ce petit détail, c'est très agréable d'avoir un chauffeur qui s'occupe de tout. Pas besoin de renseigner sur les transports, pas d'horaire à respecter, et en prime, on a même droit à quelques bons plans. Il se charge aussi de nous prévenir des différentes arnaques à touriste ayant cours ici et là. Bref, c'est les vacances !


\begin{figure}[h]
\centering
\includegraphics[height=6cm,width=9cm,keepaspectratio]{p7234504.jpg}
\caption*{L'entrée du Taj.}
\end{figure}


\begin{figure}[h]
\centering
\includegraphics[height=6cm,width=9cm,keepaspectratio]{p7234514.jpg}
\caption*{:-)}
\end{figure}

Levé à l'aube le lendemain, on part en direction du Taj-Mahal. Au lever de soleil, c'est triplement mieux : déjà, ben c'est le lever de soleil donc c'est joli, c'est déjà pas mal, ensuite, il fait frais, et ça, c'est appréciable, et enfin, la plupart des touristes sont des grosses flemmes qui ont du mal à se lever le matin (j'en sait quelque chose...) et donc il y a moins de monde, et ça c'est cool. Bref, c'est beau, il y a un petit parc, plein de marbre, et chose étonnante : pour la première fois depuis notre arrivée en Inde, aucun déchet en vue !




\begin{figure}[h]
\centering
\includegraphics[height=6cm,width=9cm,keepaspectratio]{p7234528.jpg}
\caption*{Sophie tente de méditer.}
\end{figure}


\begin{figure}[h]
\centering
\includegraphics[height=6cm,width=9cm,keepaspectratio]{p7234747.jpg}
\caption*{Vue depuis le Taj.}
\end{figure}


\begin{figure}[h]
\centering
\includegraphics[height=6cm,width=9cm,keepaspectratio]{p7234745.jpg}
\caption*{Il y a foule !}
\end{figure}

Deuxième chose étonnante : l'audio-guide est très bon ! Il y a des anecdotes, de la mise en scène, de l'histoire, c'est raconté par de vrais acteurs. Mais... mais comment raconter ça. Le plus simple, c'est encore de faire la scène au ralenti, étape par étape :
\begin{itemize}
	\item \textbf{7:31:22,03} : Alors que j'étais immobile, à profiter de la vue en écoutant l'histoire du Taj, le velcro de l'étui qui contient le boitier de l'audio guide se défait. Le boitier commence à glisser vers le bas.
	\item \textbf{7:31:22,16} : Le câble audio tente de ralentir la descente du boitier. Sans succès. Le câble se détache. Le son se coupe. Le boitier tombe.
	\item \textbf{7:31:23,20} : La coupure du son m'étonne. Je suis presque sûr que le fronton n'est pas couvert de \emph{"gra"}. Je baisse le regard vers l'étui vide.
	\item \textbf{7:31:24,89} : J'amorce un hoquet de surprise. Le boitier heurte le chemin en pierre.
	\item \textbf{7:31:24,94} : Sous la violence du choc, le boitier s'ouvre. Un pigeon s'envole. A cause du bruit bien entendu. C'est une très mauvaise idée d'essayer de faire tenir un pigeon dans un audio guide.
	\item \textbf{7:31:25,56} : Le boitier ouvert rebondi. Je vois une carte électronique scintiller au soleil. C'est beau. L'eau scintille également. L'eau ? ... Oh putain, l'eau !
	\item \textbf{7:31:26,56} : Le boitier ouvert tombe dans 10cm d'eau.
	\item \textbf{7:31:28:02} : Les dernières bulles s'échappent du boitier, emportant mes derniers espoirs. Ma main atteint le boitier.
	\item \textbf{7:31:33} : Je vide et sèche tant bien que mal le boitier.
	\item \textbf{7:32:03} : Le boitier marche !
	\item \textbf{7:32:06} : Le boitier fume ?
	\item \textbf{7:32:07} : Bon, ben tant pis alors...
	\item \textbf{9:14:30} : On rend nos trois audio-guides en gardant un air innocent tout à fait convaincant, l'employé, teste le premier boitier. OK. Il teste le deuxième. OK. Il... range le troisième sans le tester.
	\item \textbf{9:14:45} : Marion me dit de recommencer à respirer. On fait demi-tour en marchant aussi vite qu'il est possible de marcher sans donner l'impression de fuir.
\end{itemize}



\begin{figure}[h]
\centering
\includegraphics[height=6cm,width=9cm,keepaspectratio]{p7234521.jpg}
\caption*{Le lieu du crime.}
\end{figure}


\begin{figure}[h]
\centering
\includegraphics[height=6cm,width=9cm,keepaspectratio]{p7234738.jpg}
\caption*{Il est interdit de prendre des photos à l'intérieur du Taj, alors gardez cette photo pour vous, et n'oubliez pas d'admirer la finesse des sculptures !}
\end{figure}

Sinon, le Taj-Mahal est un grand tombeau, symbole de l'amour indéfectible pour sa defunte épouse, d'un mec très riche, qui n'a pas hésité une seule seconde à ruiner son pays, se mettre ses héritiers à dos, déclencher une guerre de succession et plus ou moins détruire sa dynastie afin que nous, pauvres mortels puissions nous dire quelques siècles plus tard que, quand même, c'est beau l'amour.


\begin{figure}[h]
\centering
\includegraphics[height=6cm,width=9cm,keepaspectratio]{p7234759.jpg}
\caption*{Dernier coup d’œil en partant.}
\end{figure}





\chapter{Jaipur, la patrie des éléphants partiellement roses.}
Nous partîmes donc pour Jaipur, toujours avec Karma notre chauffeur. On fait une première pause à l'entrée de la ville au fameux temple des singes. Dans la religion hindou, Hanuman, le dieu singe, est un des dieux principaux, et est très aimé, d'où la présence de nombreux temples qui lui sont dédiés. Ce temple s'avère être une colline couverte de bâtiments et de singes. L'entrée est gratuite, alors on fait un geste en achetant un petit peu de nourriture auprès du "maitre des singes" comme il s'est présenté, très fier. On vient d'entrer, et tout de suite, un jeune nous fait un petit signe amical et nous invite à le suivre dans son temple.


\begin{figure}[h]
\centering
\includegraphics[height=6cm,width=9cm,keepaspectratio]{p7234786.jpg}
\caption*{Un singe et sa mangue.}
\end{figure}

Il nous présente rapidement son temple et commence à nous bénir en nous attachant un bracelet en ficelle (hop, 1 point hippie) et un coup de peinture rouge sur le front. C'est sympa, c'est typique, même si, n'étant pas croyant, je trouve ça un peu, voire carrément hypocrite. Comme il se doit, on se penche pour faire une petite offrande de 10 roupies. Et là, il nous arrête : "ah non, pour les étrangers, c'est 100 roupies". Bon, mon petit gars, c'est pas des manière de procéder, ce n'est pas ça le principe d'une donation. Mais bon, on n'est pas chez nous, on vient d'arriver, alors on va dire que c'est le prix de la leçon. Je donne 100 roupies. On fait mine de repartir, mais il nous arrête encore : "C'est 100 roupies : par personne". Ah. Eh bien ! Rien de tel qu'une bénédiction de bon matin pour se mettre en joie ! On le regarde dans les yeux, et on rajoute deux billets de 10 roupies. Ce sera notre dernière bénédiction en Inde.


\begin{figure}[h]
\centering
\includegraphics[height=6cm,width=9cm,keepaspectratio]{p7234782.jpg}
\caption*{Le prêtre scellant notre destin !}
\end{figure}

On continue quand même notre visite, tout en ignorant les appels émanant des nombreux temples qui émaillent le chemin grimpant la colline. On donne prudemment des cacahouètes une par une aux singes, qui hardis, préfèrent essayer de piquer tout le paquet d'un coup en venant par derrière ce qui a le don de nous rendre super nerveux.


\begin{figure}[h]
\centering
\includegraphics[height=6cm,width=9cm,keepaspectratio]{p72348191.jpg}
\caption*{Les singes font les cons dans l'eau.}
\end{figure}

Des locaux qui parlent un peu anglais commencent à faire le chemin avec nous en papotant de tout et de rien. Et là, on est dans un situation délicate : on sait que certaines personnes font ça juste pour pouvoir demander de l'argent en tant que guide un peu plus tard. Il y en a aussi qui sympathisent pour augmenter leurs chances de vente de bracelets/bénédictions/service de guide etc. Alors comment se comporter avec ces gens qui semblent super accueillants ? Un peu refroidi par la fausse bénédiction, on reste sur nos gardes, distants, en répondant de manière évasives aux questions. Au moment de partir, le plus "collant" prend son courage à deux mains, et demande si on peut prendre une photo tous ensemble. Ce n'était donc que ça ? Évidemment qu'on va prendre une photo ! Deux même ! Et là, je me sens un peu con/mal/frustré : on s'ouvre, on se fait arnaquer, et quelques instant plus tard, on est distants et méfiants avec des gens qui voulaient sincèrement faire notre connaissance...


\begin{figure}[h]
\centering
\includegraphics[height=6cm,width=9cm,keepaspectratio]{p7234789.jpg}
\caption*{Ambiance Indiana Jones/Tomb Raider (à choisir en fonction de votre génération).}
\end{figure}

Enfin c'est le moment de découvrir l'hôtel que Karma nous a sélectionné. On a bien précisé qu'on voulait moins cher ce coup-ci. Et on a ... 100 roupies de moins ! Youpie ! Bref, il voit qu'on est pas content, et nous conduit dans un autre hôtel, où, après d'âpres négociations, on obtient le même prix qu'au premier hôtel. Super. Le problème que l'on commence à entrevoir, est que Karma touche une commission quasiment partout où il nous conduit : restaurants, hôtels, boutiques, ce qui rend les prix plus difficiles à négocier. On va donc changer de stratégie à l'avenir, mais aujourd'hui, on est fatigués.


\begin{figure}[h]
\centering
\includegraphics[height=6cm,width=9cm,keepaspectratio]{p7244835.jpg}
\caption*{Un vendeur de \emph{massala tea} : un thé épicé excellent qu'on trouve partout.}
\end{figure}


\begin{figure}[h]
\centering
\includegraphics[height=6cm,width=9cm,keepaspectratio]{p7244851.jpg}
\caption*{Sur la route, priorité aux éléphants.}
\end{figure}

Tout ceci ne nous empêche pas de profiter du magnifique Fort d'Amber. C'est un grand fort construit sur une colline. Sa particularité est qu'on peut y monter à dos d'éléphant ! Une file continue de pachydermes multicolores chargés de touristes radieux jusqu'aux oreilles alimente l'immense porte d'entrée du fort. Il y a même des embouteillages ! Oui madame, des embouteillages d'éléphants roses (et bleus, et verts) ! Et maintenant, juste pour rire, imaginez Lyon, le tunnel de Fourvière, un week-end de départ en vacances. Ne pensez pas à des éléphants roses.


\begin{figure}[h]
\centering
\includegraphics[height=6cm,width=9cm,keepaspectratio]{p7244882.jpg}
\caption*{Ne vous inquiétez pas, les gens à jeun voient probablement la même chose que vous}
\end{figure}


\begin{figure}[h]
\centering
\includegraphics[height=6cm,width=9cm,keepaspectratio]{p7244875.jpg}
\caption*{Heureusement, les éléphants ont un rayon de braquage pas trop mauvais.}
\end{figure}


\begin{figure}[h]
\centering
\includegraphics[height=6cm,width=9cm,keepaspectratio]{p7244870.jpg}
\caption*{Coucou !}
\end{figure}


\begin{figure}[h]
\centering
\includegraphics[height=6cm,width=9cm,keepaspectratio]{p7244898.jpg}
\caption*{Petite pause au calme.}
\end{figure}

Et comme les palais, ça commence à bien faire, plutôt que d'aller ensuite visiter le classique palais de Jaipur, on opte pour l'observatoire, le Jantar Mantar, qui est très bien selon notre guide, et qui montre une collection impressionnante d'instruments de mesure. Et ça tombe bien, il commence à pleuvoir, alors on se précipite acheter nos billets, et on rentre à l'intéri... comment ça pas d'intérieur ? On découvre alors que les instruments de mesure sont grands comme des maisons et sont dans un parc à l'extérieur, et il pleut un peu comme si c'était la mousson en Inde.


\begin{figure}[h]
\centering
\includegraphics[height=6cm,width=9cm,keepaspectratio]{p72448991.jpg}
\caption*{C'est bien beau, mais sans soleil, c'est dommage...}
\end{figure}

J'avoue, j'exagère un peu afin d'augmenter l'intensité dramatique, mais on finit quand même par nous indiquer au bout d'un moment une petite salle qui projette un film explicatif. La salle est pleine comme un tuk-tuk, mais nous permet d'attendre au sec. On découvre ensuite une collection de super sextants et de gnomons géants. La pièce la plus impressionnante est un cadran solaire de 27 mètres de haut, précis à deux secondes près. J'apprécie d'autant plus la compacité de ma montre, qui a aussi l'autre avantage de marcher même à l'ombre. Parce que là, même si la pluie a cessé, le soleil est toujours caché. Dommage, alors qu'on vient de passer toute la matinée à cuire au soleil en visitant le Fort...


\begin{figure}[h]
\centering
\includegraphics[height=6cm,width=9cm,keepaspectratio]{p7254918.jpg}
\caption*{L'inévitable palais des vents, construit pour garder un harem au frais, tout en gardant ses occupantes invisibles, mais diverties par le spectacle de la rue.}
\end{figure}





\chapter{Pushkar, ses collines et son lac sacré.}
Nous arrivâmes donc à Pushkar. On sent bien que notre chauffeur est un peu inquiet : on a réservé nous-même notre hôtel via internet. Pas de commission pour Karma, mais on a un meilleur hôtel pour deux fois moins cher que les précédents choisis par Karma, ce qui nous amène à nous poser des questions sur le montant des commissions ! Mais que cela ne nous empêche pas de profiter de la visite : Pushkar est une petite ville sainte construite autour d'un lac. Ce dernier et sacré, et de nombreux Indiens y viennent en pèlerinage.


\begin{figure}[h]
\centering
\includegraphics[height=6cm,width=9cm,keepaspectratio]{p7265002.jpg}
\caption*{Le temple Sikh.}
\end{figure}

Là encore, il faut naviguer entre les arnaques. La plus connue : un passant sympathique te donne quelques pétales de fleurs à jeter dans le lac, car ça porte chance. Au bord du lac, les touristes sont aussitôt repérés par de faux prêtres qui vont faire pression jusqu'à obtenir une sérieuse donation. Bref, c'est de l'extorsion. Donc on envoie balader sans ménagement tous les gens qui nous offrent des fleurs, et on remarque bien qu'ils ne ciblent que les touristes ! Ce lac, parlons-en : on pensait voir un joli petit lac, et on découvre un grand bassin bétonné entouré d'escaliers et d'hôtels. Il n'y a quasiment pas de végétation, c'est un peu triste...


\begin{figure}[h]
\centering
\includegraphics[height=6cm,width=9cm,keepaspectratio]{p7264994.jpg}
\caption*{Dans les rues de Pushkar.}
\end{figure}

Karma nous fait découvrir le temple Sikh. Son temple en quelque sorte, étant donné qu'il est lui-même Sikh. Le temple est tout neuf, construit il y a à peine quelques années, le marbre est immaculé, bref, c'est magnifique. On y entre pieds nus et tête couverte. On observe d'abord un prêtre récitant des prières au micro avant de descendre d'un étage, rejoindre la cantine : tous les temples Sikhs font office de soupe populaire, et fournissent un repas simple à qui le demande, sans conditions. Là au moins, on sait pourquoi on fait une donation !


\begin{figure}[h]
\centering
\includegraphics[height=6cm,width=9cm,keepaspectratio]{p7254935.jpg}
\caption*{Le prêtre qui s'occupe de la distribution des repas.}
\end{figure}

On se lève tôt le matin pour aller observer le lever de soleil depuis une petite colline proche. On marche, on s'installe, et on observe. Le lever de soleil est naze, mais le spectacle est fourni par les paons en contrebas qui font la court à grand coups de "LEON" sonores. Le suspens est à son comble quand un mâle faisant la roue, entouré de 5 ou 6 femelles, se retrouve face un compétiteur... Mais nous sommes dérangés par un prêtre qui sort soudain du tout petit temple un peu plus loin. Ben oui, une colline : un temple ! Sympa, il nous invite à voir la cérémonie du matin. On refuse gentiment, parce qu'on n'a jamais de combat de paon, alors que des temples... Un peu plus tard, on amorce la descente, et là, le prêtre ressort, vraiment en colère, et commence à nous insulter, à nous dire qu'il faut venir voir le temple, qu'il faut faire une donation, que l'Inde ce n'est pas gratuit, et qu'on n'a pas intérêt à revenir ! C'est sûr qu'ils vont attirer des gens comme ça...


\begin{figure}[h]
\centering
\includegraphics[height=6cm,width=9cm,keepaspectratio]{p7264959.jpg}
\caption*{Vous en comptez combien ?}
\end{figure}

Pour Marion, cette journée fut un peu éprouvante : déjà, on la fait lever aux aurores, alors qu'elle avait été malade toute la journée précédente. Ensuite, elle se fait engueuler par le prêtre. Et ce n'est pas fini : un peu plus tard, alors que nous étions entrain de regarder nos parantas cuire, et de papoter avec le cuistot, une vache a soudain jugé opportun de passer une de ses cornes dans une anse du sac de Marion, et de continuer son chemin comme si de rien était, sans se rendre compte qu'elle trainait une Marion dans son sillage... Et elle a fini la journée avec une Sophie sur le dos.


\begin{figure}[h]
\centering
\includegraphics[height=6cm,width=9cm,keepaspectratio]{img_paranta.jpg}
\caption*{Quelques instants avant l'encornage.}
\end{figure}


\begin{figure}[h]
\centering
\includegraphics[height=6cm,width=9cm,keepaspectratio]{p7265013.jpg}
\caption*{Je ne sais pas laquelle est la plus contente...}
\end{figure}

Le soir, coucher de soleil depuis la colline de l'autre coté du village. C'est encore un échec car il pleut, mais on peut admirer le plus petit téléphérique du monde : à vue de pif, il remplace 10mn de marche et 50m de dénivelé. Et ce coup-ci, pas de prêtre pour nous agresser, mais des gens qui ne comprennent pas pourquoi on tient tant à marcher alors qu'il y a un téléphérique tout neuf.


\begin{figure}[h]
\centering
\includegraphics[height=6cm,width=9cm,keepaspectratio]{p7264975.jpg}
\caption*{Même les arbres en béton ont perdu leurs feuilles...}
\end{figure}


\begin{figure}[h]
\centering
\includegraphics[height=6cm,width=9cm,keepaspectratio]{p7264964.jpg}
\caption*{Couleurs à vendre. Mais on n'a pas compris pourquoi en faire des tours.}
\end{figure}



\chapter{Bundi, Udaipur et Jodhpur.}
Nous arrivâmes donc à Bundi. C'est une petite ville connue pour son joli centre ville, son Fort, et le fait que Rudyard Kipling a vécu ici quelques mois le temps d'écrire Kim. Soyons honnête un instant : qui a lu ce livre ? Personnellement, je n'en avais jamais entendu parler avant d'arriver dans cette ville. En revanche, tout le monde connait le livre de la jungle, non ? Ben voilà, vous êtes à Bundi.


\begin{figure}[h]
\centering
\includegraphics[height=6cm,width=9cm,keepaspectratio]{p7275032.jpg}
\caption*{Les indiens adorent qu'on les prennent en photo !}
\end{figure}

Non, bon OK, il n'y a pas de boa ni de tigres dans le centre ville. Mais allez visiter le fort de Bundi ! Il est à moitié envahi par la végétation, et complètement envahi par les singes. Il ne faut pas beaucoup d'imagination pour sentir la présence du roi Louis ou de Sherkhan.


\begin{figure}[h]
\centering
\includegraphics[height=6cm,width=9cm,keepaspectratio]{p7285226.jpg}
\caption*{D'intrépides exploratrices.}
\end{figure}


\begin{figure}[h]
\centering
\includegraphics[height=6cm,width=9cm,keepaspectratio]{p7285224.jpg}
\caption*{Un ancien réservoir.}
\end{figure}


\begin{figure}[h]
\centering
\includegraphics[height=6cm,width=9cm,keepaspectratio]{p7285239.jpg}
\caption*{Vue depuis le sommet du fort.}
\end{figure}

On avait rencontré un peu avant d'autres touristes qui nous avait mis en garde contre les singes du fort qui étaient "agressifs" et avaient de "grandes dents", au contraire des autres petits singes sympas du centre ville. Ils nous conseillaient donc de prendre un guide qui sait comment chasser les singes. On imaginait des babouins... Bon, en fait, ces singes sont des simples macaques, les mêmes que partout ailleurs, il suffit de jeter un cailloux ou d'agiter un bâton pour qu'ils se barrent, du coup, on s'est retrouvé avec un guide un peu inutile sur les bras, qui a chassé un pauvre singe et nous a raconté à chaque nouvelle ruine que qu'il fallait imaginer ici des femmes en sari qui chantaient et jouaient de la musique.




\begin{figure}[h]
\centering
\includegraphics[height=6cm,width=9cm,keepaspectratio]{p7285250.jpg}
\caption*{Rappel d'un ancien commentaire : ça souri une chauve-souris ?}
\end{figure}


\begin{figure}[h]
\centering
\includegraphics[height=6cm,width=9cm,keepaspectratio]{p7275040.jpg}
\caption*{Une famille sympathique.}
\end{figure}


\begin{figure}[h]
\centering
\includegraphics[height=6cm,width=9cm,keepaspectratio]{p7285045.jpg}
\caption*{C'est une ville très colorée !}
\end{figure}


\begin{figure}[h]
\centering
\includegraphics[height=6cm,width=9cm,keepaspectratio]{p7285242.jpg}
\caption*{La palais de Bundi est lui aussi plutôt impressionnant.}
\end{figure}

Nous arrivâmes ensuite à Udaipur. Appelée aussi la "Venise du Rajasthan". Et c'est effectivement une très jolie ville : Il y a un palais sur le lac qui a servit de décors pour un vieux James Bond : Octopussy. Depuis cette époque glorieuse, la plupart des hôtels passent le film tous les soirs !


\begin{figure}[h]
\centering
\includegraphics[height=6cm,width=9cm,keepaspectratio]{p7295448.jpg}
\caption*{A croire que ces gens posaient pour moi !}
\end{figure}


\begin{figure}[h]
\centering
\includegraphics[height=6cm,width=9cm,keepaspectratio]{p7295455.jpg}
\caption*{:-)}
\end{figure}


\begin{figure}[h]
\centering
\includegraphics[height=6cm,width=9cm,keepaspectratio]{p7295466.jpg}
\caption*{Coucher de soleil sur le lac.}
\end{figure}

Plutôt qu'un film, nous avons vu un magnifique spectacle : De la danse avec des robes qui brillent, des marionnettes agiles et rigolotes, des instruments de musique improbable. Chose plutôt rare : les danseuses étaient de tous les âges, et on a pu constater que l'expérience, ça compte. En particulier quand elles se sont mises à jouer des sortes de clochettes : imaginez un petit disque en métal attaché au bout d'une ficelle. Le but est maintenant de frapper avec ce disque différentes clochettes attachées un peu partout sur le corps, sur le bout des pieds, au dos des mains, ou sur une épée tenue entre les dents !


\begin{figure}[h]
\centering
\includegraphics[height=6cm,width=9cm,keepaspectratio]{p7295716.jpg}
\caption*{Combien d'heures de pratiques pour en arriver là ?}
\end{figure}


\begin{figure}[h]
\centering
\includegraphics[height=6cm,width=9cm,keepaspectratio]{p7295675.jpg}
\caption*{Les danseuses se préparent.}
\end{figure}


\begin{figure}[h]
\centering
\includegraphics[height=6cm,width=9cm,keepaspectratio]{p7295687.jpg}
\caption*{Concentrée...}
\end{figure}


\begin{figure}[h]
\centering
\includegraphics[height=6cm,width=9cm,keepaspectratio]{p7295708.jpg}
\caption*{Le mec à 4 pattes incarne un tigre. Il faut le savoir.}
\end{figure}


\begin{figure}[h]
\centering
\includegraphics[height=6cm,width=9cm,keepaspectratio]{p7295738.jpg}
\caption*{C'est beau, non ?}
\end{figure}


\begin{figure}[h]
\centering
\includegraphics[height=6cm,width=9cm,keepaspectratio]{p7295754.jpg}
\caption*{...}
\end{figure}

Nous arrivâmes enfin à Jodhpur. Là encore : un fort. Mais quel fort ! Il écrase la ville par sa présence ! Et ce fort est en plus très bien entretenu, ce qui ne gâche rien. Il n'a jamais été conquis, malgré les nombreux sièges. C'est étonnant de voir les astuces de défenses anti-éléphants (haha, des défenses ... contre les éléphants, c'est ironique !) : des couloirs à angle droit pour les empêcher de prendre de la vitesse, des gros piques à hauteur de leur tête.


\begin{figure}[h]
\centering
\includegraphics[height=6cm,width=9cm,keepaspectratio]{p8016273.jpg}
\caption*{Le fort de Jodhpur.}
\end{figure}


\begin{figure}[h]
\centering
\includegraphics[height=6cm,width=9cm,keepaspectratio]{p8016277.jpg}
\caption*{Le fort vu de l'intérieur.}
\end{figure}

Jodhpur est aussi la capitale des épices. Alors on a fait le stock ! A notre retour, ceux qui nous rendrons visite auront droit a un peu de cuisine s'ils ont de la chance. La cuisine indienne est franchement excellente, et majoritairement végétarienne. Nous n'avons quasiment pas mangé de viande pendant tout notre séjour sans que ça nous manque. Le restaurant végétarien est la norme, et même quand il y a de la viande, c'est poulet ou mouton. Les vaches sont sacrées, et ça veut dire quelque chose là-bas ! On a aussi vu de nombreux porc se balader dans les villes, mais personne ne les mange. En même temps, les Hindous, les Jaïnistes et les Sikhs sont végétariens, et il ne reste que les Musulmans qui mangent un peu de viande. Bref, on a l'impression que la seule utilité des porcs est de manger des déchets et de nettoyer les caniveaux. Revenons à la nourriture : c'est TRÈS épicé. Quand on demande "pas du tout épicé" dans un restaurant, on obtient habituellement quelque chose qui en France est considéré comme le maximum supportable. A force d'en manger tous les jours, on commence à s'habituer, et j'ai même mangé quelques fois un plat épicé normalement. Disons qu'on le sent bien passer. Deux fois. Heureusement, le lassi, une sorte de yaourt à boire aide à faire passer toutes ces brûlures, et on en fait une consommation intensive.


\begin{figure}[h]
\centering
\includegraphics[height=6cm,width=9cm,keepaspectratio]{p7275018.jpg}
\caption*{Un petit porc dans la rue.}
\end{figure}

Histoire de pousser l'expérience de la cuisine indienne jusqu'au bout, on s'inscrit à un cours. C'est une famille très sympathique qui s'en occupe. Et ce fut honnêtement le meilleur repas de toute l'Inde : du lassi au safran, du fromage frit, du curry de chou-fleur, des boules de coco, ... 9 plats au total, décorés avec des feuilles d'argent, et surtout carrément délicieux. On est impatient d'avoir à nouveau une cuisine rien que pour nous ! L'histoire de cette famille est aussi très intéressante : c'est un mariage d'amour, chose très rare en Inde ! Ils ne viennent pas de la même caste, et leurs deux familles (oui, même la famille de la caste la plus basse) se sont farouchement opposé au mariage. Ils ont du fuir leur village, et ont galéré quelques années avant de remonter la pente, d'économiser assez pour acheter un tuk-tuk, et ensuite de lancer ces cours de cuisine qui marchent très très bien. Ils ont un garçon, et une fille qui veut devenir scientifique, et à présent, certains membres de leur famille commencent à leur parler à nouveau...


\begin{figure}[h]
\centering
\includegraphics[height=6cm,width=9cm,keepaspectratio]{img_20160801_212403.jpg}
\caption*{Après le cours, le repas !}
\end{figure}


\begin{figure}[h]
\centering
\includegraphics[height=6cm,width=9cm,keepaspectratio]{p8036353.jpg}
\caption*{Un haveli (maison de riche marchand) dans Mandawa.}
\end{figure}







 Oui, je sais je n'ai rien raconté sur Mandawa, mais en même temps, on n'a rien fait à part se balader dans les rues. C'est la première ville un peu calme qu'on a visité en Inde.

\chapter{Varanasi, la ville sacrée au bord du Gange.}
Nous partîmes donc pour Varanasi. Enfin... nous tentâmes de partir pour Varanasi ! On est arrivé à la gare deux heures avant le départ, histoire d'être sûr, mais le train est parti avec 6h de retard, et est arrivé à 16h de l'après-midi, au lieu de 6h45 du matin... C'est officiellement notre record de retard de train. Varanasi est aussi une ville sainte. Elle est au bord du fleuve sacré qu'est le Gange. Tout le long de la rive droite est couvert de Gaths, des temples/escaliers au bord de l'eau, qui permettent aux fidèles de venir se purifier.


\begin{figure}[h]
\centering
\includegraphics[height=6cm,width=9cm,keepaspectratio]{p8066934.jpg}
\caption*{Faites un voeu !}
\end{figure}

Le gath le plus en amont est très particulier, car il est dédié à la crémation. Les hindous brûlent leurs morts ici, car le lieu est tellement saint, que l'âme monte directement au paradis et interrompt ainsi le cycle des réincarnations. C'est garanti. Alors forcément, il y a de la demande ! Et visiblement, la caste qui gère la crémation en profite et fait payer cher tout le monde, en fonction de leurs revenus. Les cendres sont ensuite répandues dans le fleuve.


\begin{figure}[h]
\centering
\includegraphics[height=6cm,width=9cm,keepaspectratio]{p8066669.jpg}
\caption*{Crémation en arrière plan.}
\end{figure}


\begin{figure}[h]
\centering
\includegraphics[height=6cm,width=9cm,keepaspectratio]{p8066671.jpg}
\caption*{Il faut compter environ 200kg de bois pour un corps.}
\end{figure}

Il y a 5 exceptions à la crémation, pour 5 cas où les Hindous estiment que la personne est morte en étant pure. Les trois premiers sont : Les prêtres, les enfants, les femmes enceintes, ça semble assez logique. Les deux derniers cas sont plus originaux à mon avis : les gens morts d'une morsure de serpent. C'est lié à la nature divine de Krishna, qui a résisté à une morsure de cobra (mais ça l'a rendu tout bleu). Et le dernier, ce sont les lépreux ! Ben oui, ils perdent leur peau, et avec, toutes leurs impuretés. Tous ces gens n'ont pas besoin d'être purifiés par le feu après leur mort. Ils sont donc simplement jetés dans Gange, tel quel. Quand on sait ça, et qu'on voit un truc tout gonflé qui descend le courant, on espère très fort que c'est juste le cadavre d'une vache...


\begin{figure}[h]
\centering
\includegraphics[height=6cm,width=9cm,keepaspectratio]{p8066696.jpg}
\caption*{Purification matinale}
\end{figure}


\begin{figure}[h]
\centering
\includegraphics[height=6cm,width=9cm,keepaspectratio]{p8066700.jpg}
\caption*{Presque toutes les marches sont inondées.}
\end{figure}


\begin{figure}[h]
\centering
\includegraphics[height=6cm,width=9cm,keepaspectratio]{p8066628.jpg}
\caption*{Il faut parfois se faire sa propre baignoire dans les plantes.}
\end{figure}

Continuons cette charmante découverte des traditions du Gange : il existe une tribu, un ordre de moines, qui, à des fins de quête spirituelle et de recherche d'illumination, collectent les cadavres, et les mangent. Ce sont les Aghori. Évidemment, ils vivent en marge de la société, et ils dégoutent tous les autres Indiens. Mais quand même, c'est un pays assez fou !


\begin{figure}[h]
\centering
\includegraphics[height=6cm,width=9cm,keepaspectratio]{p8066635.jpg}
\caption*{La cérémonie du lever du soleil.}
\end{figure}

Pour découvrir la vie du fleuve, il faut se lever tôt : la cérémonie du lever du soleil est pratiquée tous les jours à 5 heures du matin. Ça consiste globalement à agiter à bout de bras des gros fumoirs d'encens. Et quand c'est une rangée de prêtre en costume qui le fait de manière synchronisée, ça rend plutôt bien ! Ce matin nous avons eu de la chance, car la cérémonie était suivie d'un cours de yoga, et apparemment, le professeur qu'on a eu est une célébrité. On a fait quelques exercices, les plus notables étant sans conteste l'imitation du cri du tigre, puis l'éclat de rire forcé qui dégénère en hilarité générale. Le Yoga, c'est plus fun que ce que je pensais ! Ensuite nous avons enchainé sur un tour en bateau, afin de voir l'activité du matin sur les différents ghats. Comme c'est la mousson, le niveau du Gange empêche de se promener autrement qu'en bateau le long du fleuve, et seul le haut des escaliers est accessible. On voit les pèlerins qui viennent faire leur toilette, dans une des eaux les plus polluées du monde. Il y en a même qui se gargarisent, c'est sûrement pour renforcer le système immunitaire. Et avant que vous ne me posiez la question : oui, soyez rassurés, la crémation a bien lieu en aval de tous les gaths.


\begin{figure}[h]
\centering
\includegraphics[height=6cm,width=9cm,keepaspectratio]{p8066718.jpg}
\caption*{Une vendeur de bougies.}
\end{figure}

Pendant le tour en bateau, nous sympathisons avec notre guide, et il nous fait rencontrer un de ses amis astrologue, et nous avons ainsi droit à un thème astral gratuit ! Le soir, c'est le moment d'admirer le cérémonie du coucher du soleil. Même principe que le matin, mais avec des gros bougeoirs, ce qui est d'autant plus beau qu'il fait nuit. A la fin de la cérémonie, les fidèles, ainsi que les touristes vont mettre à l'eau une petite bougie.


\begin{figure}[h]
\centering
\includegraphics[height=6cm,width=9cm,keepaspectratio]{p8066973.jpg}
\caption*{Pyrotechnie, te voici !}
\end{figure}


\begin{figure}[h]
\centering
\includegraphics[height=6cm,width=9cm,keepaspectratio]{p8066988.jpg}
\caption*{Trois bougies, trois vœux !}
\end{figure}

Nous avons aussi visité la ville à pied, mais quelle galère ! Nous somme tombé le jour d'un festival, et les rues étaient bondées. Cela dit, c'était très intéressant de voir tous ces gens habillés en orange aller remplir les temples ! Nous ne sommes pas aller voir les crémations (sauf depuis le bateau). Même si c'est une attraction touristique, ça ne nous semble pas vraiment approprié. Je ne peux m'empêcher d'imaginer mes sentiments si une bande de touristes venaient visiter l'enterrement d'un de mes proches...


\begin{figure}[h]
\centering
\includegraphics[height=6cm,width=9cm,keepaspectratio]{p8077041.jpg}
\caption*{Les rues en plein festival.}
\end{figure}

Le temps passe, et c'est déjà l'heure du retour pour Sophie. Ce coup ci, pas de retard, le train arrive à l'heure, tout comme l'avion.

\emph{De Marion :} Merci ma Sophie d'être venue nous voir, d'avoir apporté avec toi de la bonne humeur, des mains magiques qui, je crois, ont réparé mon genou, et de l'énergie sans fin (sauf à l'heure de la sieste). Vivement les prochaines aventures ensemble !


\begin{figure}[h]
\centering
\includegraphics[height=6cm,width=9cm,keepaspectratio]{p8077043.jpg}
\caption*{La gare sous la pluie. Les quais sont immenses : jusqu'à 1km de long !}
\end{figure}


\begin{figure}[h]
\centering
\includegraphics[height=6cm,width=9cm,keepaspectratio]{p8097075.jpg}
\caption*{Déjà l'heure du départ...}
\end{figure}



\chapter{Un an déjà !}
Hé oui, comme le temps passe vite (première porte ouverte enfoncée) ! Mais quand on regarde derrière nous, que de chemin parcouru (hop, deuxième porte ouverte) !

Bref, ça fait un an, la date du retour approche, et le rythme de publication  ralenti. En même temps, ce n'est pas toujours facile de trouver du Wifi dans les montagnes au Népal. Mais ne vous inquiétez pas, quelques articles vont arriver bientôt.

En attendant voici une première petite photo du Népal, histoire de vous mettre en jambes pour ce qui arrive !


\begin{figure}[h]
\centering
\includegraphics[height=6cm,width=9cm,keepaspectratio]{p9139611.jpg}
\caption*{Il y a une faute sur le panneau : c'est 5160m. Vous allez dire que je chipote, mais 54m, ça compte mine de rien...}
\end{figure}

\chapter{Katmandou, premier contact avec le Népal}
Nous partîmes donc pour Katmandou. Après avoir franchi la frontière Inde-Népal, on prend un mini bus qui doit nous amener en 5 heures à destination. En fait, pas vraiment. Disons que 5 heures, c'est le temps de trajet donné par un GPS. Mais quand on rajoute :
\begin{itemize}
	\item la route défoncée
	\item des milliers de camions qui alimentent tous le Népal par le même chemin et qui ont tous des raisons d'être pressés et de vouloir dépasser les autres
	\item des passagers qui profitent de chaque pause pour se bourrer la gueule et ne veulent plus repartir
\end{itemize}
Eh bien on obtient un temps de trajet  de 12 heures. On est donc arrivé à 2h du matin au lieu de 7h du soir.


\begin{figure}[h]
\centering
\includegraphics[height=6cm,width=9cm,keepaspectratio]{p8297727.jpg}
\caption*{Technique de construction traditionnelle.}
\end{figure}

Malgré cette entrée en matière pas des plus reposante, Katmandou, après le choc de l'Inde, ça fait du bien. C'est plus calme, plus propre, et la circulation est moins stressante, même si tout autant chaotique. On en profite pour faire redescendre notre pression sanguine pendant quelques jours. Le seul point négatif, ce sont les gérants de l'hôtel qui nous demandent tout le temps ce qu'on va faire car ils veulent vraiment nous vendre des treks.


\begin{figure}[h]
\centering
\includegraphics[height=6cm,width=9cm,keepaspectratio]{p8287367.jpg}
\caption*{Katmandou c'est un peu moche, alors voici les ailes d'un papillon en échange.}
\end{figure}

Puis Marion m'abandonne 10 jours pour une retraite de méditation vipassana. C'est une forme extrême de méditation : interdiction totale de distraction. Pas de lecture, pas de téléphone, pas de conversation, pas de contact visuel. Les deux repas quotidiens se font face à un mur ! Ce fut dur, ce fut long, elle fut forte et elle revint changée, mais elle vous en parlera mieux que moi. De mon coté, après de longues hésitations, plutôt que de l'accompagner, je suis resté glander en ville. Enfin, quand je dis glander, j'ai surtout fait du blog ! C'est que ça ne s'écrit pas tout seul ces petits articles ! Je trouve aussi une salle d'escalade, et en profite pour me faire une tendinite dès la première séance... Je me contenterai donc ensuite de lever le coude avec mon partenaire de grimpe temporaire.


\begin{figure}[h]
\centering
\includegraphics[height=6cm,width=9cm,keepaspectratio]{img_20160825_072215.jpg}
\caption*{Le planning de la méditation.}
\end{figure}

Une fois réunis, on a commencé à visiter la vallée de Katmandou. Pour ça, le scooter est idéal, à condition d'être TRÈS prudent : c'est pas l'Inde, mais pas loin, alors on laisse la priorité à tout le monde (surtout aux plus gros), on klaxonne sans cesse (les gens ne regardent pas autour d'eux) et on ne dépasse jamais les 30km à l'heure. Notre première sortie nous amène à un petit parc national un peu au Nord de Katmandou. C'est une petite colline à gravir pour avoir une vue sur la ville, idéal pour une balade à la journée. Deux kilomètres avant l'entrée du parc, ça monte et notre scooter cale. On termine donc à pied et on arrive au guichet.


\begin{figure}[h]
\centering
\includegraphics[height=6cm,width=9cm,keepaspectratio]{p8267097.jpg}
\caption*{En se baladant, on tombe parfois sur d'étranges poulets ! Vous ne trouvez pas qu'il a un air à Donald Trump ?}
\end{figure}

C'est un parc national, donc il faut payer l'entrée. Et là, on nous explique qu'on doit aussi embaucher un guide. La loi date d'il y a quelques semaines, et c'est obligatoire. C'est pour notre "sécurité", et notre "information". Nous, on sortait de Katmandou pour être un peu tranquilles, et fuir les sollicitations constantes de gens qui veulent nous vendre des flûtes faites main, du baume du tigre, du haschisch, des informations sur les treks, des taxis, des rickshaws... Du coup, pas trop envie d'avoir une nounou sur le dos pour surveiller qu'on ne sort pas du sentier. Comme la plupart des autres visiteurs, on est allés se balader ailleurs...


\begin{figure}[h]
\centering
\includegraphics[height=6cm,width=9cm,keepaspectratio]{p8287387.jpg}
\caption*{Comme cette araignée, on a pris nos grandes pattes et on est allés voir ailleurs !}
\end{figure}

Notre deuxième sortie nous amène un peu plus loin dans la vallée, à Nagarkot. C'est une toute petite route en lacet qui nous y amène, quasiment vide de circulation. On arrive dans un petit village au sommet d'une colline qui surplombe la vallée. La vue est magnifique, et des petits sentiers de randonnée nous permettent de nous remettre un peu en forme. La vallée du Katmandou, c'est bien, mais on a du gros trek qui se profile à l'horizon...


\begin{figure}[h]
\centering
\includegraphics[height=6cm,width=9cm,keepaspectratio]{p8297732.jpg}
\caption*{Rizières dans la vallée de Katmandou.}
\end{figure}



\chapter{Oh Manaslu, mon beau Manaslu, où es-tu ?}
Nous partîmes donc pour une petite balade autour du Manaslu. Une petite balade de 22 jours quand même, autour d'un sommet qui culmine à plus de 8000m, et sans porteur ! Juste nous deux et un guide : Dipak, qui nous avait été chaudement recommandé quelques mois auparavant, par un couple qui nous avait donné envie en nous parlant justement de ce trek.


\begin{figure}[h]
\centering
\includegraphics[height=6cm,width=9cm,keepaspectratio]{p9139585.jpg}
\caption*{Est-ce le Manaslu ? Eh bien non !}
\end{figure}

Pourquoi ce trek, plutôt que le classique et renommé "Tour des Annapurnas" ? C'est simple, depuis quelques années, une route carrossable fait quasiment tout le tour des Annapurnas, alors que le Manaslu est uniquement accessible à pied, et de fait bien plus préservé de ces saloperies de touristes... C'est en bus local qu'on fait la première partie du voyage. Nous quittons rapidement l'autoroute (autoroute dans le sens népalais, c'est une route assez large pour que, la plupart du temps, deux véhicules puissent se croiser sans trop freiner) et nous attaquons les petites routes. A partir de là, on n'a plus l'impression de s'approcher de notre but, mais plutôt d'y "tendre" : plus on s'approche, plus on ralentit... La mousson tardive et les camions transforment les routes en champs labourés. Nous devons régulièrement descendre du bus pour le laisser faire des manœuvres incroyables destinées à traverser un champ de boue en montée. Mais toute la bonne volonté du chauffeur, ses coups de volants et d'accélérateur n'y pourront rien : le bus fini embourbé comme tout le monde et nous nous retrouvons piétons plus tôt que prévu. Heureusement, notre guide trouve un autre bus peu après, mais nous devons voyager sur le toit. Commencent alors 4 très, très longues heures. Le toit est blindé de gens, nous sommes assis sur une grille où il est impossible de trouver un position confortable, et les cahots de la route sont tels qu'on se cramponne à la grille au point d'en avoir mal aux mains, de peur d'être éjecté au prochain nid de poule.


\begin{figure}[h]
\centering
\includegraphics[height=6cm,width=9cm,keepaspectratio]{p9017749.jpg}
\caption*{Vingt kilomètres dans ces conditions, ça prend du temps.}
\end{figure}

Contre toute attente, nous finissons par arriver à bon port, avec à peine 4h de retard. Dernière ville, dernière douche chaude, et surtout, dernière connexion à internet, avant de pénétrer dans la montagne. La première journée nous met bien dans le rythme : plus de 20km, avec du dénivelé et des sacs. On est contents de se poser le soir. Contrairement à tout ce qu'on imaginait sur les treks au Népal, point de paysages minéraux ni de neige ni de vue incroyable sur des sommets enneigés : on marche dans la jungle. La végétation foisonne. Il fait chaud et humide. Et les nuages nous bloquent la vue même quand il ne pleut pas.


\begin{figure}[h]
\centering
\includegraphics[height=6cm,width=9cm,keepaspectratio]{p9047795.jpg}
\caption*{C'est beau, mais brumeux !}
\end{figure}

La pluie ne fait pas que retarder les randonneurs : elle façonne la montagne ! Depuis le terrible tremblement de terre de 2015, la montagne n'est toujours pas stabilisée. En ajoutant le poids et la lubrification de la pluie, on obtient des glissements de terrain quotidiens, pour ne pas dire horaires. Ces grondements sourds qui nous intriguaient régulièrement n'avaient rien à voir avec de l'orage ! Nous sommes passés plusieurs fois sur des glissements tout frais qui avaient emporté quelques dizaines de mètres de sentier, à peine quelques minutes avant notre arrivée. Nous avons vu un petit pan de forêt glisser subitement dans le torrent de la taille d'un fleuve, élargissant la vallée sous nos yeux, les temps géologiques se superposant à l'histoire humaine le temps de quelques secondes.


\begin{figure}[h]
\centering
\includegraphics[height=6cm,width=9cm,keepaspectratio]{p9179931.jpg}
\caption*{Marion traversant un glissement de terrain tout frais.}
\end{figure}


\begin{figure}[h]
\centering
\includegraphics[height=6cm,width=9cm,keepaspectratio]{p9098985.jpg}
\caption*{Ceci est un épouvantail... un épouvantail à singes !}
\end{figure}

En passant sur un pierrier, notre guide nous dit soudain qu'il y avait un village ici. On ne voit plus rien. Il a été enterré sous des milliers, des millions de tonnes de pierres. Pas de plaque, pas de stèle, pas de message, aucune trace de ce village si ce n'est dans les mémoires. L'Himalaya est un massif jeune, en pleine croissance, qui bouge tous les jours. Dans ces conditions, toute construction est forcément temporaire. Dipak nous précise que 20 personnes et 50 mules sont mortes. Évoquer en même temps les vies humaines et les mules me semble incongru sur le moment, mais ces mules sont le sang de ces vallées. En l'absence de routes praticables, ce sont des convois de mules qui transportent quasiment tout. Les choses fragiles comme les vitres sont transportées à dos d'hommes, et les plus riches se permettent hélicoptère, mais les récoltes, les bouteilles de gaz, le ciment, les packs de bière et les snickers à destination des touristes, c'est pour les mules !


\begin{figure}[h]
\centering
\includegraphics[height=6cm,width=9cm,keepaspectratio]{p9099252.jpg}
\caption*{Des mules au travail.}
\end{figure}


\begin{figure}[h]
\centering
\includegraphics[height=6cm,width=9cm,keepaspectratio]{p9078968.jpg}
\caption*{Ça pousse comme de la mauvaise herbe... alors que c'est de la bonne !}
\end{figure}

Quelques jours plus tard, notre guide a une mauvaise nouvelle : un glissement de terrain a rendu très dangereux l'accès à la vallée de Tsum. C'est une vallée qu'on peut visiter quand on fait le tour du Manaslu. Des exilés tibétains s'y sont installés, et on avait prévu d'y passer 5 jours, le temps de faire l'aller retour jusqu'au bout de la vallée. On décide de quand même d'aller voir a quoi ressemble le passage... Ok, donc c'est de l'escalade, sans corde bien évidemment. On a déjà grimpé du beaucoup plus dur, mais assuré. Là, si on tombe, c'est 30m de chute, directement dans le torrent en colère. Et ça fait longtemps qu'on n'a pas grimpé. Et on a des sacs. Et il pleut. Et on a quelques heures de marche dans les pattes. Et on n'est pas venu pour ça, alors tant pis, on fait demi-tour.


\begin{figure}[h]
\centering
\includegraphics[height=6cm,width=9cm,keepaspectratio]{p9098994.jpg}
\caption*{Petit village et rizières.}
\end{figure}

La pluie, ça n'entraine pas que des catastrophes ici. Ça permet aussi à tout un tas d'animaux mignons de sortir prendre l'air. Vous connaissez tous le proverbe "Il pleut, il mouille : c'est la fête à la .....  sangsue d'mes c**illes !". La premières fois, c'est comme ça qu'on les remarque : on enlève les chaussures et chaussettes à la fin de la journée, et on se demande d'où vient tout ce sang ! Elles ne se contentent pas d'en prélever quelques gouttes, elles laissent aussi une plaie pleine d'anticoagulant qui continue de saigner encore une à deux heures après le décès (brutal) de ladite sangsue. Mais comme une image valant 1000 injures, en voici deux :


\begin{figure}[h]
\centering
\includegraphics[height=6cm,width=9cm,keepaspectratio]{p9068470.jpg}
\caption*{Les petits points rouges, ce sont des morsures de sangsues.}
\end{figure}


\begin{figure}[h]
\centering
\includegraphics[height=6cm,width=9cm,keepaspectratio]{p9068473.jpg}
\caption*{Bien grasse et bien dodue... Hmmm !!!}
\end{figure}

Le plus surprenant, c'est avec quelle facilité elles arrivent à s'accrocher à la chaussure, remonter sous le pantalon, redescendre dans la chaussure, puis traverser la chaussette. On a tenté plusieurs stratégies, mais la seule qui marche un tant soit peu, c'est une pause nettoyage des sangsues toutes les 20 minutes. Malgré tout, on finira chacun avec une dizaine de cicatrices...


\begin{figure}[h]
\centering
\includegraphics[height=6cm,width=9cm,keepaspectratio]{p9098992.jpg}
\caption*{Les yeux de Bouddha surveillent la vallée.}
\end{figure}

Tout ceci ne nous empêche pas d'avancer. Nous partons tôt tous les matins : à 7h, nous sommes généralement déjà en train de marcher, et le plus souvent, on est arrivés avant 14h à destination. Toutes les guesthouses se ressemblent, et tous les menus sont identiques. C'est répétitif mais nous avons beaucoup plus de confort que ce à quoi on s'attendait. Nous avons à chaque fois notre chambre individuelle, et nous mangeons beaucoup de pâtes et de riz, parfois même des roestis ! Le plat traditionnel, c'est normalement le Dal Bhat : une grosse plâtrée de riz, avec une soupe de lentilles et un curry de légumes. Les Népalais en mangent tous les jours, et ce n'est pas une image : notre guide nous a dit, que les rares fois où il mange des pâtes ou une pizza, il mange aussi du Dal Bhat une heure plus tard. Il est toujours servi à volonté, et très rapidement car il est préparé à l'avance. Mais nous n'en mangeons presque pas car à notre grande surprise, c'est toujours le plat le plus cher ! Étonnant, alors que c'est le seul plat qui ne leur demande aucun boulot supplémentaire étant donné qu'il y en a toujours qui mijote dans la cuisine...


\begin{figure}[h]
\centering
\includegraphics[height=6cm,width=9cm,keepaspectratio]{p9099269.jpg}
\caption*{Au chaud dans la cuisine !}
\end{figure}

Ah... la cuisine... Alors que tous les jours nous montons un peu plus haut dans la vallée, la température baisse petit à petit, et il commence à faire froid. La cuisine étant la seule pièce chauffée, toute la famille s'y retrouve, ainsi que les guides et les porteurs de passage. Mais pas les touristes, relégués dans la grande salle à manger toute froide. Dès qu'on entre dans la cuisine, on est poliment invités à passer dans la pièce à coté. Seule une famille fera exception pendant tout le trek. Autre chose surprenante : notre guide ne mange jamais avec nous ! Il reste parfois debout à coté de la table pendant que nous mangeons, refuse de s'asseoir quand on l'invite, nous regarde, et parfois fouille dans nos affaires. Oui, vous avez bien lu. C'est qu'ils ont une autre conception de l'intimité par ici...


\begin{figure}[h]
\centering
\includegraphics[height=6cm,width=9cm,keepaspectratio]{p9078961.jpg}
\caption*{Confiance !}
\end{figure}

Nous traversons régulièrement des rivières sur des ponts suspendus. Le temps des ponts en lianes est révolu, ce sont des câbles en acier à présent. Je ne peux m'empêcher de remarquer que mêmes les ponts tout neufs présentent déjà des problèmes d'entretien : des écrous manquent un peu partout, et ceux qui restent semblent sérieusement déserrés. Renseignement pris : il semble que personne n'entretient ces ponts ! On trouve du budget pour les construire, il y a des petites stèles qui explique qui a financé, qui a réalisé quoi, et... plus rien ensuite. Dommage.


\begin{figure}[h]
\centering
\includegraphics[height=6cm,width=9cm,keepaspectratio]{p9078962.jpg}
\caption*{Pas confiance...}
\end{figure}

A partir de 3500m d'altitude, on commence enfin à sortir de la jungle, et ça commence à ressembler un peu plus à ce qu'on s'imaginait du Népal. Nous arrivons ainsi à Samagon, un des premiers endroit où on peut voir le sommet, si les nuages coopèrent, bien entendu. Et ils ont coopéré, un matin entre 5h45 et 6h. Oui, 15 minutes sur toute la journée, faut pas les louper ! Samagon est aussi le village de départ pour les expéditions vers le sommet. En fait, la quasi totalité des rares touristes que nous rencontrons font partie d'une expédition, et vont passer quelques semaines à naviguer entre 5000m et 8000m d'altitude et des poussières. A coté d'eux, on a vraiment l'impression d'être des promeneurs du dimanche...


\begin{figure}[h]
\centering
\includegraphics[height=6cm,width=9cm,keepaspectratio]{p9119323.jpg}
\caption*{Et en plus, il sourit pour la photo. Trop facile...}
\end{figure}


\begin{figure}[h]
\centering
\includegraphics[height=6cm,width=9cm,keepaspectratio]{p9098999.jpg}
\caption*{Les femmes aussi portent des trucs incroyables, comme des troncs d'arbre !}
\end{figure}

On voit aussi le travail incroyable des porteurs : les mules ne peuvent pas aller si haut, et il faut une quantité incroyable de bordel pour installer le camp de base et les différents camps temporaires des expéditions. On assiste à un ballet incessant d'énormes sacs, de bouteilles de gaz, de chaises, de casiers de poulets vivants, j'en passe et des meilleurs. Et si on regarde bien, on remarque, sous le gros tas de sac qui avance dans la montagne, oui, en effet, il y a un népalais en tong ! Même si la législation interdit normalement des charges supérieures à 25kg, il se dit que certains peuvent monter leur propre poids, à une altitude où un homme normalement constitué trouve déjà pénible de soulever simplement ses propres fesses. On a assisté à une scène improbable, où deux hommes étaient nécessaires pour mettre une charge sur le dos d'un troisième. Ce sont des surhommes.


\begin{figure}[h]
\centering
\includegraphics[height=6cm,width=9cm,keepaspectratio]{p9109284.jpg}
\caption*{Les gamins sont trognons.}
\end{figure}


\begin{figure}[h]
\centering
\includegraphics[height=6cm,width=9cm,keepaspectratio]{p9119362.jpg}
\caption*{Messe du dimanche dans un monastère. Il y avait une de ces ambiance... c'est simple, on se serait cru dans un monastère perdu au milieu de l'Himalaya !}
\end{figure}

Après une journée d'acclimatation à Samagon, occupée à ne pas aller au camp de base pour cause de brouillard, on attaque le point culminant du trek : le col de Larke. On commence par rejoindre la dernière guesthouse avant le col : une cahute en pierre et en terre, à 4400m d'altitude.


\begin{figure}[h]
\centering
\includegraphics[height=6cm,width=9cm,keepaspectratio]{p9129575.jpg}
\caption*{Je crois qu'on est encore un peu hors saison...}
\end{figure}


\begin{figure}[h]
\centering
\includegraphics[height=6cm,width=9cm,keepaspectratio]{p9129574.jpg}
\caption*{On se paie le luxe d'une chambre triple !}
\end{figure}

Le lendemain matin, on part à 6h pour la dernière ascension, avec comme objectif le col à 5160m d'altitude. La matinée commence bien : il fait beau, on voit quelques sommets, et plein de petites souris bourrues qui courent se réfugier sous des pierres à notre approche. On passe devant un lac tellement bleu qu'on se demande s'il n'a pas été photoshoppé.


\begin{figure}[h]
\centering
\includegraphics[height=6cm,width=9cm,keepaspectratio]{p9139597-panorama.jpg}
\caption*{Je vous jure, je n'ai pas triché !}
\end{figure}

Le silence n'est brisé que par les bruits de cailloux qui dévalent les pentes, et par notre respiration qui se fait courte. On commence à sentir les effets de l'altitude. A 5000m, il n'y a plus que la moitié de l'air, et donc de l'oxygène, qu'on trouve au niveau de la mer. Résultat : on a l'impression de se trainer comme des vieux pépés. On va mettre 4 heures pour faire 7km et et les derniers 800m de dénivelé. Deux heures avant d'arriver au col, il se met à neiger, et à venter. Nous sommes maintenant dans un désert de cailloux, où plus rien d'autre que du lichen ne pousse, et nous faisons des pauses toutes les 10 minutes. C'est une sensation étrange : ce n'est pas douloureux ni désagréable, mais chaque pas est épuisant, et on se demande ce qui nous arrive, à être mous comme ça...


\begin{figure}[h]
\centering
\includegraphics[height=6cm,width=9cm,keepaspectratio]{p91396111.jpg}
\caption*{Ça valait le coup de monter, non ?}
\end{figure}

Le col, c'est d'abord une tâche colorée au loin. Il y a des milliers de drapeaux, et un panneau qui nous félicite d'être arrivé là. Dommage, le panneau a une faute de frappe sur l'altitude. Il indique 5106m alors que c'est 5160m. On est très contents d'être arrivés, mais avec le brouillard, le vent et la neige, on ne voit rien et il fait froid, donc inutile de s'attarder, et on attaque les 1700m de descente dans les cailloux et la boue. La neige s'est transformée en pluie, le chemin est une pataugeoire, et nos goretex ont cessé de remplir leur rôle. Affamés, trempés et heureux, nous arrivons enfin au village, après 4 heures de descente !


\begin{figure}[h]
\centering
\includegraphics[height=6cm,width=9cm,keepaspectratio]{p9139578.jpg}
\caption*{Faut pas louper le créneau pour la photo.}
\end{figure}

Le lendemain matin, comme pour se faire pardonner du temps pourri de la veille, la montagne se dévoile enfin, et nous avons droit au lever de soleil sur le Manaslu entièrement dégagé. On ne sera pas venus pour rien !







\chapter{L'ABC des Alsaciens}
Quelle ne fut pas ma surprise, quand, au détour d'un mail, mon père m'annonce qu'il nous rejoindrait bien quelque part dans notre voyage ! Et moi qui croyait qu'avec son tout nouvel emploi du temps de retraité, ce serait impossible... On se met rapidement d'accord sur le Népal, et je lui suggère d'embarquer Kévin dans l'affaire. Une chose en entrainant une autre, la moitié de l'Alsace (environ) décide de se joindre à l'aventure.


\begin{figure}[h]
\centering
\includegraphics[height=6cm,width=9cm,keepaspectratio]{p9290699.jpg}
\caption*{La moitié de l'Alsace (environ) et un chien.}
\end{figure}

Par ordre alphabétique, Aurélien, Camille, Jean-Luc, Kévin et Pierre partirent donc pour Katmandou où nous les récupérâmes. Et là, tout s'enchaîne : on fait les permis de trek, on rentre à l'hôtel , et on boit une bière bien méritée, accompagnée d'un peu de saucisson et de fromage. Quel pied ! Le lendemain, une journée de bus nous amène à Pokhara, et le surlendemain, un taxi nous amène au début de la randonnée menant au camp de base des Annapurnas, en anglais : Annapurnas Base Camp, soit ABC.


\begin{figure}[h]
\centering
\includegraphics[height=6cm,width=9cm,keepaspectratio]{p9280330.jpg}
\caption*{C'est dur les escaliers. Même Pierre a l'air fatigué, c'est dire !}
\end{figure}


\begin{figure}[h]
\centering
\includegraphics[height=6cm,width=9cm,keepaspectratio]{p9280333.jpg}
\caption*{La vue une petite heure après le départ.}
\end{figure}

Le chemin commence par un escalier interminable. L'entrée en matière est rude ! Mais toute l'équipe s'est préparée physiquement, et tout le monde a bien optimisé le poids du sac, et ça passe sans problème. Il faut dire qu'on a fait le choix de faire ce trek sans guide ni porteur, juste entre nous :-)


\begin{figure}[h]
\centering
\includegraphics[height=6cm,width=9cm,keepaspectratio]{p9280344.jpg}
\caption*{Kévin "la gazelle de l'Himalaya", franchi d'un bond gracieux le petit ruisseau.}
\end{figure}

Les premiers jours, nous sommes quasiment seuls sur le chemin, mais rapidement, on se retrouve à croiser plus de Chinois que n'importe quelle autre nationalité. Les frontières sont ouvertes pour ce pays depuis peu, et quand un milliard et des poussières (poussières grandes comme l'Europe...) de Chinois rattrapent 50 ans de tourisme, ici, ça se remarque !


\begin{figure}[h]
\centering
\includegraphics[height=6cm,width=9cm,keepaspectratio]{p9280339.jpg}
\caption*{Il y a des ponts...}
\end{figure}


\begin{figure}[h]
\centering
\includegraphics[height=6cm,width=9cm,keepaspectratio]{p9290569.jpg}
\caption*{... et des rivières...}
\end{figure}


\begin{figure}[h]
\centering
\includegraphics[height=6cm,width=9cm,keepaspectratio]{p9290584.jpg}
\caption*{... et encore des ponts...}
\end{figure}


\begin{figure}[h]
\centering
\includegraphics[height=6cm,width=9cm,keepaspectratio]{p9290577.jpg}
\caption*{... et des troupeaux de chèvre, guidés par Kévin, le berger des Annapurnas.}
\end{figure}

Comme pour le Manaslu, il pleut tous les jours, vers 15h en général. Nous sommes donc ob-bli-gés de nous arrêter, et passons l'après-midi à boire de la bière (car l'hydratation c'est important), à manger du saucisson et du fromage (il faut alléger les sacs), et à jouer au tarot (non, là je n'ai pas de justification).


\begin{figure}[h]
\centering
\includegraphics[height=6cm,width=9cm,keepaspectratio]{p9280349.jpg}
\caption*{On boit aussi du rhum (merci Pierre !).}
\end{figure}


\begin{figure}[h]
\centering
\includegraphics[height=6cm,width=9cm,keepaspectratio]{p9280351.jpg}
\caption*{On mange même du lard fumé ! Rien que de l'écrire, j'en salive... Merci Papa !}
\end{figure}


\begin{figure}[h]
\centering
\includegraphics[height=6cm,width=9cm,keepaspectratio]{p9280356.jpg}
\caption*{On agrémente l'apéritif grâce à Kévin, le boucher des sommets.}
\end{figure}

La mousson a désormais 6 semaines de retard, mais on sent qu'on approche de la fin : les nuages sont moins présents, et on a souvent du soleil le matin, ce qui nous permet de voir de temps en temps quelques sommets, comme le fameux Machapuchare, qui culmine à quasiment 7000m. Sa forme particulière lui a valu son nom : ça veut dire \emph{queue de poisson} en népalais. C'est aussi un des rares sommet à n'avoir jamais été conquis. Il est interdit d'ascension depuis les années cinquante pour raison religieuse, et les quelques renégats irrespectueux de cette supposée demeure de Shiva ont tous échoué.


\begin{figure}[h]
\centering
\includegraphics[height=6cm,width=9cm,keepaspectratio]{pa0416381.jpg}
\caption*{La "queue de poisson".}
\end{figure}


\begin{figure}[h]
\centering
\includegraphics[height=6cm,width=9cm,keepaspectratio]{pa031601.jpg}
\caption*{L'Annapurna sud, lors d'une des rares nuits sans nuages. La neige du sommet est éclairée par un orage plus bas dans la vallée !}
\end{figure}

C'est arrivé au pied de ce sommet maudit (ou sacré, chacun son point de vue) qu'on sort enfin de la jungle. Les nuages continuent de s'accrocher aux cimes, mais quelle ambiance ! Nous sommes à 4000m d'altitude, et pourtant entourés de montagne presque encore deux fois plus hautes ! Encore une petite heure de montée, et nous sommes au ABC, au bord d'un reste de glacier. Le paysage est magnifique, mais le sommet continue de faire son timide, et reste caché sous un voile pudique.


\begin{figure}[h]
\centering
\includegraphics[height=6cm,width=9cm,keepaspectratio]{pa020944.jpg}
\caption*{OK, on ne voit pas le sommet, mais quand même, c'est beau !}
\end{figure}


\begin{figure}[h]
\centering
\includegraphics[height=6cm,width=9cm,keepaspectratio]{pa020966.jpg}
\caption*{Victoire !}
\end{figure}


\begin{figure}[h]
\centering
\includegraphics[height=6cm,width=9cm,keepaspectratio]{pa021000.jpg}
\caption*{Deux frères contents de voyager ensemble !}
\end{figure}


\begin{figure}[h]
\centering
\includegraphics[height=6cm,width=9cm,keepaspectratio]{pa021010.jpg}
\caption*{Deux frères qui... euh... ben qui voyagent ensemble !}
\end{figure}


\begin{figure}[h]
\centering
\includegraphics[height=6cm,width=9cm,keepaspectratio]{pa020978.jpg}
\caption*{Papa bien entouré !}
\end{figure}


\begin{figure}[h]
\centering
\includegraphics[height=6cm,width=9cm,keepaspectratio]{pa020982.jpg}
\caption*{Papa, sérieusement encadré !}
\end{figure}


\begin{figure}[h]
\centering
\includegraphics[height=6cm,width=9cm,keepaspectratio]{pa020954.jpg}
\caption*{Câlin ou casse-croûte ?}
\end{figure}

Pendant la redescente, certains souffrent, en particulier des genoux. La pudeur m'oblige à laisser leur identité cachée, la NSA n'en saura rien ! Par contre, c'est vraiment très pratique d'avoir un kiné dans l'équipe, et là, il ne m'en voudra pas de lui faire de la pub : Aurélien ralenti la chute des genoux, appliquez-en une couche généreuse deux fois par jour pour des résultats optimaux. Les marches n'ont qu'à bien se tenir !


\begin{figure}[h]
\centering
\includegraphics[height=6cm,width=9cm,keepaspectratio]{pa010734.jpg}
\caption*{Si on veut un sac léger ET sentir bon, faut laver ses affaires tous les jours !}
\end{figure}


\begin{figure}[h]
\centering
\includegraphics[height=6cm,width=9cm,keepaspectratio]{pa072275.jpg}
\caption*{J'aime quand les gens synchronisent leurs vêtements !}
\end{figure}

Le retour à Pokhara deux jours plus tôt que prévu nous laisse un peu de temps pour aller visiter la pagode de la paix. Ça commence de manière fourbe par un petit tour en bateau, et ensuite le piège : des marches qui font mal aux genoux ! Non, je déconne, tout s'est bien passé, tout le monde est monté, et tous les 10 genoux sont rentrés en état de marche en Alsace.


\begin{figure}[h]
\centering
\includegraphics[height=6cm,width=9cm,keepaspectratio]{pa072279.jpg}
\caption*{Kévin, le rebelle aux pieds nus, manifeste son mécontentement.}
\end{figure}


\begin{figure}[h]
\centering
\includegraphics[height=6cm,width=9cm,keepaspectratio]{p9200108.jpg}
\caption*{Le lac de Pokhara.}
\end{figure}

A la moitié de l'Alsace : Nous avons été ravis de tous vous voir le temps de ce trek, c'était une excellente idée de tous vous incruster avec Papa, et on espère que vous avez aimé ce bout de voyage autant que nous. Puisse cette expérience être la première d'une longue série !


\begin{figure}[h]
\centering
\includegraphics[height=6cm,width=9cm,keepaspectratio]{pa062231.jpg}
\caption*{La bière du retour !}
\end{figure}


\begin{figure}[h]
\centering
\includegraphics[height=6cm,width=9cm,keepaspectratio]{pa082283.jpg}
\caption*{Ça fatigue les voyages !}
\end{figure}



\chapter{Un dernier tour au Népal, et puis s'en vont...}
Les Alsaciens repartis, on avait encore quelques jours à passer à Katmandou avant de repartir nous-même. On n'a pas fait grand chose de vraiment digne d'être raconté, mais comme on a pris de jolies photos, voici quand même un petit article.


\begin{figure}[h]
\centering
\includegraphics[height=6cm,width=9cm,keepaspectratio]{pa122555.jpg}
\caption*{Je vooooooooole !}
\end{figure}

C'est ainsi que l'envie d'aller visiter un peu Katmandou nous a pris par surprise. Nous avons donc affronté encore un peu de circulation anarchique, et nous sommes arrivés au Durbar Square, la plus célèbre et principale concentration de temples de la ville. Et c'est aussi un endroit qui a beaucoup souffert du dernier tremblement de terre.


\begin{figure}[h]
\centering
\includegraphics[height=6cm,width=9cm,keepaspectratio]{pa092305.jpg}
\caption*{Je crois que j'ai trouvé le temple des pigeons, et son roi.}
\end{figure}


\begin{figure}[h]
\centering
\includegraphics[height=6cm,width=9cm,keepaspectratio]{pa092309.jpg}
\caption*{Un chien faisant honneur à ses ancêtres loups.}
\end{figure}

De nombreux temples ont été sérieusement endommagés, au point de n'être plus du tout visitables, et certains ont été totalement détruits. On trouve ainsi des enclos autour de l'espace vide ou se trouvait un temple, illustré par une photo montrant le temple avant le tremblement de terre, et une photo des gravats... Même si on a vu à de nombreux endroits les traces du tremblement de terre - de nombreux bâtiments sont encore en réparation, si ce n'est abandonnés - c'est ici qu'on a ressenti le plus fort l'étendu des dégâts.


\begin{figure}[h]
\centering
\includegraphics[height=6cm,width=9cm,keepaspectratio]{pa092292.jpg}
\caption*{Dégâts à Durbar Square.}
\end{figure}

C'était aussi une période de fête pour le Népal : quasiment 15 jours de vacances pour que tout le monde puisse fêter convenablement Dashain. Les temples sont pris d'assaut, les transports sont bondés car c'est le moment de rentrer dans sa famille, et la plupart des commerces réduisent leurs horaires voire ferment complètement. On voit aussi apparaître un peu partout des balançoires en bambou !


\begin{figure}[h]
\centering
\includegraphics[height=6cm,width=9cm,keepaspectratio]{pa122547.jpg}
\caption*{A deux, c'est mieux !}
\end{figure}


\begin{figure}[h]
\centering
\includegraphics[height=6cm,width=9cm,keepaspectratio]{pa092329.jpg}
\caption*{Un temple illuminé pour les fêtes.}
\end{figure}

Elles sont immenses, occupées du matin au soir, et les gamins vont haut, très haut, et les balançoires plient, jusqu'à ce que le ou les balanceurs atteignent ce point de rupture, où la corde est à l'horizontale et donc n'est plus tendue par la gravité : pendant une fraction de seconde, on flotte en l'air, sans appui, avant d'être sèchement appuyé à nouveau sur le siège par la gravité et la corde qui se tend. Ceux qui ont déjà fait cette expérience savent exactement ce que j'essaie de décrire... les autres, ben trouvez une balançoire ! A chaque fois, on les voit monter, monter, jusqu'à ce moment où le cœur semble louper un battement... et ils se laissent aller et laisse leur place au suivant.


\begin{figure}[h]
\centering
\includegraphics[height=6cm,width=9cm,keepaspectratio]{pa132717.jpg}
\caption*{Des étoiles en haut, et en bas !}
\end{figure}

C'est dans la vallée de Katmandou que nous passons nos derniers jours au Népal, à faire du scooter entre Dhulikel et Nagarkot, loin de la circulation anarchique de la ville. Nous avons eu la chambre avec probablement la plus belle vue de tout notre voyage : dans un angle avec deux grandes fenêtres donnant sur une vallée magnifique avec au loin la chaîne de l'Himalaya.


\begin{figure}[h]
\centering
\includegraphics[height=6cm,width=9cm,keepaspectratio]{pa132873.jpg}
\caption*{La vue depuis notre chambre à Dhulikel.}
\end{figure}

Je crois que c'est à ce moment là qu'on s'est le plus rendu compte de la taille de ces montagnes. On sent instinctivement où devrait être la ligne d'horizon, mais les montagnes sont plusieurs kilomètres au-dessus ! C'est très étrange d'en prendre conscience, et malheureusement, cet effet ne prend pas avec une simple photo : allez-y si vous le pouvez, et priez pour que les nuages vous laissent un peu de vue !


\begin{figure}[h]
\centering
\includegraphics[height=6cm,width=9cm,keepaspectratio]{pa132877.jpg}
\caption*{La chaîne de l'Himalaya au loin.}
\end{figure}


\begin{figure}[h]
\centering
\includegraphics[height=6cm,width=9cm,keepaspectratio]{pa223238.jpg}
\caption*{Les Annapurnas depuis l'avion. On reconnait la queue de poisson sur la droite.}
\end{figure}

 (attention, certaines ne sont pas vraiment pour tout public).

\chapter{Astana, la ville champignon dans le désert.}
Nous arrivâmes donc à Astana. En avion, malheureusement. Ben oui, si on voulait traverser la Chine depuis le Népal, nous aurions du le faire via un voyage organisé hors de prix, et ça, pas question ! C'est donc après une escale de 19h dans un aéroport trop froid malgré les 35\textdegree C à l'extérieur que nous sommes arrivés dans la nouvelle capitale du Kazakhstan.


\begin{figure}[h]
\centering
\includegraphics[height=6cm,width=9cm,keepaspectratio]{pa243328.jpg}
\caption*{Et en plus, il neige !}
\end{figure}

Nous commençons par attendre 2h à la douane le consul pour qu'il nous fasse nos visas. En tant que français, nous avons droit à 15 jours sans aucune formalité préalable ! Le consul arrive, et nous dit, tout étonné : "Vous êtes français, vous n'avez pas besoin de visa, vous passez la frontière, on vous met un tampon, et voilà !". Nous : "... euh ... merci !"
Eh oui, deux heures d'attente pour rien !


\begin{figure}[h]
\centering
\includegraphics[height=6cm,width=9cm,keepaspectratio]{pa243302.jpg}
\caption*{Cet immeuble ressemble à un briquet : c'est le siège du géant du gaz kazakh.}
\end{figure}

Et nous sortîmes de l'aéroport pour découvrir ce pays, et quel bonheur ! Il y a des trottoirs ! Pas de moto sur les trottoirs, ils sont réservés aux piétons ! Il n'y a pas de vaches sur la route ! Il n'y a pas de nids de poules ! Il y a des horaires dans les arrêts de bus ! Les gens sont gentils et nous aident naturellement sans essayer de nous vendre des souvenirs/hashish/hôtel ou autre ! Les voitures s'arrêtent pour nous laisser traverser ! Il y a une cuisine dans l'auberge de jeunesse ! Et surtout, et enfin, depuis le temps qu'on l'attendait : il fait FROID ! Oui, je sais, ça a l'air con dit comme ça, mais de pouvoir nous balader tranquillement dans la rue, emmitouflés dans nos manteaux, à regarder la neige tomber, et à se réjouir d'avance pour le thé chaud qu'on va boire en rentrant, ça nous a rendu ivres de bonheur.


\begin{figure}[h]
\centering
\includegraphics[height=6cm,width=9cm,keepaspectratio]{pa243347.jpg}
\caption*{Encore de la neige !}
\end{figure}


\begin{figure}[h]
\centering
\includegraphics[height=6cm,width=9cm,keepaspectratio]{pa233261.jpg}
\caption*{C'est le symbole de la ville.}
\end{figure}

La ville ne nous a pas semblé très vivante. Peut-être que le contraste avec les klaxons et la circulation de Katmandou a joué un rôle dans notre perception, mais quel plaisir, quel calme ! C'est une ville nouvelle, capitale récente, et couverte de nouveaux bâtiments modernes et audacieux qui sont la principale attraction de la ville. Donc après avoir visité tout ça, on s'est dit qu'on ferait bien un peu de cuisine.


\begin{figure}[h]
\centering
\includegraphics[height=6cm,width=9cm,keepaspectratio]{pa243326.jpg}
\caption*{Tim, qui a voulu nous faire gouter une spécialité kazakh : le lait de chamelle fermenté. Je vous laisse imaginer...}
\end{figure}


\begin{figure}[h]
\centering
\includegraphics[height=6cm,width=9cm,keepaspectratio]{pa243293.jpg}
\caption*{La prochaine exposition universelle aura lieu ici. Alors on en profite pour soutenir la France !}
\end{figure}

Mais petit dilemme : dans la guesthouse, la mémé fait de la cuisine pour tout le monde pour une somme dérisoires, c'est très bon, ça fait une grande tablée sympathique et on ne voulait pas faire bande à part, donc nous avons fait un dessert pour tout le monde ! La mémé était inquiète au début de nous voir cuisiner, car elle avait déjà prévu notre repas. On a donc tenté de lui faire comprendre qu'on ne faisait qu'un dessert, et on a appris au passage comment dire "cuisson au bain marie" en russe, et c'est plus facile que ce qu'on pense ! C'est ainsi qu'une bande de Kazakhs ont découvert le flan aux œufs !


\begin{figure}[h]
\centering
\includegraphics[height=6cm,width=9cm,keepaspectratio]{pa243309.jpg}
\caption*{Une yourte ? Non : un centre commercial !}
\end{figure}

Oui, c'est déjà la fin de l'article, mais mieux vaut ça que rien, non ? Plus de photos bientôt !

\chapter{La traversée du Kazakhstan et de la mer Caspienne.}
Nous prîmes ensuite le train pour Almaty. C'est avec beaucoup de plaisir que nous avons retrouvé ces fameux trains couchette qui nous avaient tant plus sur la traversée de la Russie. Une nuit plus tard, nous voilà dans l'ancienne capitale. L'ambiance est totalement différente : il y a plein de vieux bâtiments ! On dirait que c'est une ville qui a un peu plus d'histoire.


\begin{figure}[h]
\centering
\includegraphics[height=6cm,width=9cm,keepaspectratio]{pa283588.jpg}
\caption*{Mangez des pommes !}
\end{figure}

Cela dit, on n'a pas trouvé le centre ville passionnant. Une petite allée piétonne, un marché, une petite cathédrale, et voilà, on a fait le tour. Non, en vrai, ce qui est intéressant à Almaty, ce sont les montagnes aux alentours, donc on a pris nos chaussures de rando (oui bon, ce sont nos seules chaussures en même temps) et on est partis en direction du lac, un peu au hasard. Il faisait beau, on a marché dans la neige, et on n'a croisé qu'une seule personne. Hmmm...


\begin{figure}[h]
\centering
\includegraphics[height=6cm,width=9cm,keepaspectratio]{pa273564.jpg}
\caption*{Geronimooooooooooo !!!}
\end{figure}


\begin{figure}[h]
\centering
\includegraphics[height=6cm,width=9cm,keepaspectratio]{pa263376.jpg}
\caption*{Vous ne trouvez pas que ce pigeon a un air à John Snow ?}
\end{figure}

Non, en vrai, ce qui est intéressant à Almaty, ce sont les restaurants : conseillé par notre guesthouse, nous voilà dans un grand restaurant (grand = beaucoup de places ici) ambiance mardi soir de novembre à La Bourboule : les chaises sont recouvertes de moumoute, une table sur 10 est occupée, et il y a de la musique live pas trop forte ! Et on a mangé des super brochettes, notre premier gros repas de viande depuis ... depuis... oulà, bien tout ça oui ! Mouais...


\begin{figure}[h]
\centering
\includegraphics[height=6cm,width=9cm,keepaspectratio]{pa283581.jpg}
\caption*{Non, cette photo n'a aucun rapport avec le paragraphe précédent.}
\end{figure}

Non, en vrai, ce qui intéressant à Almaty, c'est, euh... attendez, je cherche... ah oui : les pommes ! Non, je vous jure, la légende dit que l'humanité a trouvé les pommes ici, et d'ailleurs le nom de la ville signifie "ville des pommes". C'est bien mis en valeur à l'entrée d'un grand parc avec quelques œuvres d'art. C'est vendeur, non ?


\begin{figure}[h]
\centering
\includegraphics[height=6cm,width=9cm,keepaspectratio]{pa283591.jpg}
\caption*{Le jour ou Apple entend parler de cette ville, je ne vous raconte pas le procès...}
\end{figure}

OK, une dernière tentative : non, en vrai, ce qui est intéressant à Almaty, c'est quand on prend le train pour s'en aller ! Et on n'a pas fait les choses à moitié : 61h, 3 nuits pour traverser tout le pays, et arriver au bord de la mer Caspienne à Aktau. On a donc bien eu le temps de papoter avec nos voisins, même si la barrière de la langue a rendu l'échange un peu laborieux. On a quand même compris que l'argent les intéressait vraiment : combien coute ci en France, et combien coute ça, et un appartement, et combien tu gagnes etc. On a un peu raconté notre voyage, et quand on a parlé de la Chine, notre voisin, un vrai kazakh, a rigolé et mimé un chinois en se bridant les yeux. On n'a pas su comment réagir, car il était déjà lui-même bridé naturellement... Comme quoi, tout est relatif !


\begin{figure}[h]
\centering
\includegraphics[height=6cm,width=9cm,keepaspectratio]{pa303622.jpg}
\caption*{Coucher de soleil depuis le train.}
\end{figure}

En arrivant à Aktau au petit matin, on file directement au bureau des ferrys pour acheter un billet pour traverser la mer Caspienne. On doit mettre toutes les chances de notre coté, car les ferrys sont rares et imprévisibles : certains voyageurs ont attendu plus d'une semaine, tout en étant prévenus à peine 2h à l'avance du départ. Mais pour nous, double coup de bol : on nous dit de revenir le soir même car il y aura probablement un ferry, et en plus, on rencontre un couple de Français qui font le même trajet !


\begin{figure}[h]
\centering
\includegraphics[height=6cm,width=9cm,keepaspectratio]{pa313634.jpg}
\caption*{On n'a pas réussi à se mettre d'accord sur la position des bras.}
\end{figure}

On passe donc la journée à papoter en bonne compagnie (de toute façon, il n'y a rien à faire à Aktau), et le soir, on se dirige vers le port. Il faut d'abord trouver un taxi. Ici, on agite le bras au bord de la route, et n'importe quelle voiture peut s'arrêter et s'improviser taxi. La première voiture à s'arrêter nous demande le double du prix normal, et on n'arrive pas à négocier. Ça attire un passant, qui nous demande si il peut nous aider, et après une courte explication, il se met au bord de la route, arrête une autre voiture, négocie, et... paie pour nous ! Il n'a rien voulu savoir quand on a voulu payer, et nous a souhaité un bon voyage avec un grand sourire !


\begin{figure}[h]
\centering
\includegraphics[height=6cm,width=9cm,keepaspectratio]{pb013639.jpg}
\caption*{La mer Caspienne.}
\end{figure}

Évidemment, une fois au port, il faut encore attendre. A 22h, on nous dit qu'on pourra probablement embarquer vers 1h du matin. On continue de papoter avec nos compagnons de voyage, et on parle du visa pour l’Azerbaïdjan, notre prochaine destination. Eux ont un visa, pas nous. Ils sont surpris quand on leur dit qu'il n'y a pas besoin de visa si on reste moins de trois jours. Ça nous met un doute, et après vérification, il y a un silence gêné. Hmmm. On n'a donc pas de visa. Et... il faut bien un visa... C'est ballot. On faisait les fiers avec nos un an de vagabondage en Asie, mais là, on a l'air bien con quand on ramasse nos affaires pour rentrer en ville au milieu de la nuit.


\begin{figure}[h]
\centering
\includegraphics[height=6cm,width=9cm,keepaspectratio]{pb023643.jpg}
\caption*{Allégorie de notre capacité à entrer en Azerbaïdjan à ce moment là.}
\end{figure}

Du coup, pas le choix, si on veut prendre le ferry, il faut faire le visa. Ça nous prend trois jours, et re-coup de bol : alors qu'on est en route pour le consulat azéri, le bureau des ferry nous appelle pour nous dire qu'un ferry part le soir-même ! On passe un très bon voyage. Marion étant la seule passagère, on bénéficie d'un traitement de luxe en se voyant attribuer la seule cabine privée avec salle de bain ! Hormis deux autres touristes, les passagers sont tous des chauffeurs de poids-lourds qui font des trajets du type "Almaty-Toulon". La mer est calme, la nourriture est bonne, et le seul évènement notable de la traversée, c'est la découverte d'une nuée d'oiseau morts sur le pont. On n'a toujours pas d'explication...


\begin{figure}[h]
\centering
\includegraphics[height=6cm,width=9cm,keepaspectratio]{pa313631.jpg}
\caption*{Allégorie de notre capacité à aller en Azerbaïdjan quelques jours plus tard.}
\end{figure}


\begin{figure}[h]
\centering
\includegraphics[height=6cm,width=9cm,keepaspectratio]{pb064278.jpg}
\caption*{Lui et ses cinquante potes sont aussi arrivés en Azerbaïdjan, mais bon, comment dire...}
\end{figure}

Enfin, on débarque à Baku. Nos visas sont acceptés, et nous voilà sur le sol azéri. L'Europe se rapproche !

.



\chapter{L'Azerbaïdjan, le pays des volcans de boue et des flammes de verre.}
Nous débarquâmes donc à Baku. Après quelques formalités (mais oui, on a bien nos visas, cf épisode précédent), nous voici dans le métro, à essayer de comprendre comment ça marche, quelle carte acheter etc. Les gens nous voient galérer, et plutôt que de nous aider à acheter la bonne carte, ils nous ont payé l'entrée, tout simplement ! Voilà des gens qui savent vendre leur pays !


\begin{figure}[h]
\centering
\includegraphics[height=6cm,width=9cm,keepaspectratio]{pb074849.jpg}
\caption*{Le centre de Baku.}
\end{figure}

Et nous voilà dans le centre de la plus grande ville du pays. Mais ce n'est pas la capitale, d'ailleurs, je crois que personne ne connait la capitale de ce pays. Par contre, c'est la capitale économique, et on sent bien l'argent apporté par les champs de pétrole et de gaz de la mer Caspienne : le centre a l'air tout neuf, tout est propre, même la magnifique vieille ville est toute neuve, à mi-chemin entre la Suisse et un parc d'attraction.


\begin{figure}[h]
\centering
\includegraphics[height=6cm,width=9cm,keepaspectratio]{pb074640.jpg}
\caption*{Le fameux "Carpet Museum": Subtile métaphore architecturale !}
\end{figure}

La ville est dominée par trois tours immanquables en forme de flamme. Évidemment, elle sont illuminées. Évidemment, les illuminations imitent des flammes. C'est classe !


\begin{figure}[h]
\centering
\includegraphics[height=6cm,width=9cm,keepaspectratio]{pb094983.jpg}
\caption*{C'est classe de loin !}
\end{figure}


\begin{figure}[h]
\centering
\includegraphics[height=6cm,width=9cm,keepaspectratio]{pb074829.jpg}
\caption*{De près aussi c'est pas mal !}
\end{figure}

Grâce à notre visa de transit, nous n'avons que 5 jours à passer dans le pays, ce qui nous laisse tout juste le temps de visiter la ville, mais aussi les volcans de boue situés à une heure de bus et quelques minutes de taxi hors de la ville. La partie bus n'a pas posé trop de problèmes, mais la partie taxi en revanche... Un taxi nous attendait à la descente du bus, probablement prévenu par le chauffeur de bus par ailleurs très sympathique. On savait à l'avance que le prix normal pour la balade était d'environ 30 Manats, et c'est armés de cette certitude que l'on s'approche du chauffeur souriant pour entamer les négociations.


\begin{figure}[h]
\centering
\includegraphics[height=6cm,width=9cm,keepaspectratio]{pb084896.jpg}
\caption*{Impressionnant, non ?}
\end{figure}

Alors que le chauffeur nous presse de monter dans son bolide, on lui demande "Combien ?". Il dit 40. C'est trop. On répond "Non, 30". Il nous dit Ok. Super, c'est ce qu'on voulait, et la négociation fut facile, à croire qu'on prend le coup de main ! Le chauffeur est sympa, et malgré son faible niveau d'anglais, l'ambiance dans la voiture est bonne. Il nous offre même des mandarines. Mais plus on avance sur le trajet, plus il insiste sur le fait que le trajet fait 15km (le GPS dit 5km, mais ne chipotons pas), et que la route est mauvaise (c'est vrai, c'est un chemin de terre plein de boue) et plus il fait des mimiques inquiètes à chaque nid de poule en nous faisant comprendre que sa voiture est vieille. On voit bien là où il veut en venir, mais ça ne prend pas. Puis il appelle quelqu'un et nous passe le téléphone : c'est sa fille (selon elle), et elle en revanche parle très bien anglais ! Et elle commence à nous dire que son père ne sait pas parler anglais, qu'il y a eu une incompréhension avec lui, que la voiture est vieille, que les cailloux sont trop durs, que la boue elle est trop molle, que la vie elle est pas facile et que le prix c'est 50 manats. Mouais. Bien sûr. Ça ne sent pas du tout la pièce de théâtre jouée tous les jours avec tous les touristes. On ne cède rien, on tient bon, on se réjouit de n'avoir rien payé en avance et on arrive aux volcans.


\begin{figure}[h]
\centering
\includegraphics[height=6cm,width=9cm,keepaspectratio]{pb084959.jpg}
\caption*{Heureusement, la boue, ça part au lavage.}
\end{figure}

C'est très rigolo. On dirait des projets de collégiens : ils font 30cm à 2m de haut maximum, et crachent une belle boue si lisse et douce qu'on a envie de rouler dedans. Étonnamment, la boue est froide ! Bref, c'est marrant, on prend des photos, on se fait éclabousser par les imprévisibles éruptions et on repart pour la suite. Il y a des pétroglyphes dans le coin, et ça fait partie de tour classique avec les volcans. On remonte dans la voiture, et là, re-coup de fil : maintenant, la fille joue la colère, les pleurs tant et si bien que d'épuisement, on finit par craquer et on augmente le tarif à 35 manats, mais ce n'est pas suffisant pour calmer ni la fille qui est toujours en colère, ni le père qui continue de geindre. D'ailleurs, il a désormais totalement oublié le peu d'anglais qu'il connaissait quelques minutes plus tôt, pour mieux coller à son rôle de victime incomprise...


\begin{figure}[h]
\centering
\includegraphics[height=6cm,width=9cm,keepaspectratio]{pb084905.jpg}
\caption*{:-)}
\end{figure}

Les pétroglyphes, c'est sympa aussi ! Il y a un musée tout neuf plutôt intéressant, qui explique l'histoire du coin. Ou plutôt la préhistoire : Les pétroglyphes, comme leur nom l'indique, sont des gravures sur des rochers, qui datent d'époques diverses allant de l'âge de la pierre aux tout récents Romains : un centurion arrivé dans le coin il y a quelque 2000 ans (à pied donc...), a vu ces gravures, et, tel un lycéen visitant Notre Dame, a gravé dans un coin "Caïus était là". Aucun respect ces envahisseurs...


\begin{figure}[h]
\centering
\includegraphics[height=6cm,width=9cm,keepaspectratio]{pb084980.jpg}
\caption*{Tout ces pétroglyphes, moi, ça me donne surtout envie de grimper !}
\end{figure}

Puis c'est le retour dans le taxi. On paie les 35 manats et personne n'est content : le chauffeur geint, et nous avons l'impression de nous être fait entuber de 5 manats. Un peu plus tard, en revoyant Flavie et Julien, nos presque compagnons de ferry, on compare nos expériences, et on se rend compte qu'on a eu le même chauffeur de taxi, qui a eu exactement la même stratégie. Eux en revanche n'ont payé que 23 manats.

Je crois que c'est définitif : je hais les taxis...



\chapter{La Géorgie, l'autre pays du vin et du fromage.}
Nous partîmes donc pour Tibliss... Tibil... Tsibil... Tbilissi ! Oui, nous aussi on a du s'entrainer à le dire, ne vous inquiétez pas. Et inquiétez-vous d'autant moins qu'une fois sur place, ça n'a plus vraiment d'importance : les locaux parlent tous de "Tifliss". Allez comprendre... Faut dire que niveau langue, le Géorgien, ça en impose : non seulement la langue n'a aucun lien avec aucune des langues des pays voisins, mais en plus l'alphabet est complètement différent. Pour nous, ça ressemblait au Thaïlandais, c'est dire ! Mais mis à part ces menus problèmes de communication, la Géorgie, ça a vraiment été une des bonnes surprises du voyage pour moi.


\begin{figure}[h]
\centering
\includegraphics[height=6cm,width=9cm,keepaspectratio]{pb145173.jpg}
\caption*{Un monastère sur une colline.}
\end{figure}

Déjà, la cuisine géorgienne est excellente ! Ils ont du bon pain, ils mettent du fromage partout, et ils ont une tradition vinicole vieille de près de 9000 ans. Oui, j'ai bien dit neuf mille ans ! Du pain, du vin, du fromage : ces gens savent vivre. Et je n'ai même pas encore parlé du fameux Katchpuri ! Le nom peu vendeur ne nous a pas freinés. C'est simple, on a fait un score parfait : on en a mangé tous les jours sans se lasser, en particulier le fameux parmi les fameux : Katchapuri Adjaruli. C'est un pain fourré au fromage, plié en forme de barque pour faire un contenant, qu'on rempli de fromage, avec un oeuf à cheval, et des morceaux de beurre par dessus. Voilà. Est-ce que j'ai vraiment besoin d'en rajouter ? Si vous avez pris 3kg juste en lisant ce passage, c'est une réaction normale !


\begin{figure}[h]
\centering
\includegraphics[height=6cm,width=9cm,keepaspectratio]{pb104999.jpg}
\caption*{Trouver une légende pour cette photo me fait saliver...}
\end{figure}

On trouve aussi des bains chauds dans un quartier bien précis de la ville, où les cinq établissements, pas un de plus, pas un de moins, sont collés les uns contre les autres. Ils sont vieux comme la route de la soie, et une citation d'Alexandre Pouchkine faisant la fierté de la ville est affichée partout : "Je n'ai jamais rien vu d'aussi beau que les bains de Tbilissi". On peut en déduire que ce pauvre monsieur n'a pas du beaucoup voyager dans sa vie... Faut être réaliste : ça reste des bêtes bâtiments en brique, avec un peu de mosaïque. Pas de quoi en écrire des romans ! Je pense plutôt que ce brave monsieur a un peu trop abusé de l'hospitalité et de la tradition vinicole des géorgiens, et, bourré comme un coing, a dit des trucs qui ne sont pas tombés dans l'oreille d'un sourd. Cela dit, c'est très agréable : pour une somme modique, on dispose d'un bain privé : une première salle pour se changer, ouvrant sur la salle ... euh, ben la salle de bain du coup, contenant une grande baignoire d'eau chaude alimentée en permanence par l'eau du sous sol. Et pour 4\euro de plus, on peut avoir un massage !


\begin{figure}[h]
\centering
\includegraphics[height=6cm,width=9cm,keepaspectratio]{pb1250311.jpg}
\caption*{Marion n'a pas voulu faire le modèle dans le bain. Je la remplace donc.}
\end{figure}

C'est aussi le pays où mon pote Pierre nous a rejoint. Comme il n'a eu que quelques jours à passer en notre compagnie, le rythme s'est sensiblement accéléré : plus question de glander par ci par là, il faut optimiser notre temps, profiter au maximum du pays !




\begin{figure}[h]
\centering
\includegraphics[height=6cm,width=9cm,keepaspectratio]{pb1250421.jpg}
\caption*{Et on a même bu du Cabernet Sauvignon !}
\end{figure}


\begin{figure}[h]
\centering
\includegraphics[height=6cm,width=9cm,keepaspectratio]{pb155638.jpg}
\caption*{En plein centre ville, un architecte inspiré s'est lâché... et le bâtiment semble abandonné !}
\end{figure}

On commence par une escapade à la frontière russe, à Stéphantsminda. Le nom ne vous dit probablement rien (moi, c'est à peine si je savais placer la Géorgie sur une carte...) mais vous avez peut-être déjà vu une photo de ce monastère au sommet d'une montagne.




\begin{figure}[h]
\centering
\includegraphics[height=6cm,width=9cm,keepaspectratio]{pb135078.jpg}
\caption*{C'est le même monastère que la première photo de l'article.}
\end{figure}

Le panorama est exceptionnel. Le mont Kazbek domine toute la vallée, du haut de ses 5000 et quelques mètres. Il fait la frontière avec la Russie. La météo étant clémente, nous sommes montés plus haut que le monastère, pour voir le glacier au pied du mont. Comme la plupart des glaciers, il a fondu, et on ne voit plus que des cailloux, mais ce n'est pas (trop) grave (enfin si, c'est très grave, mais là n'est pas le sujet), la vue est quand même magnifique !




\begin{figure}[h]
\centering
\includegraphics[height=6cm,width=9cm,keepaspectratio]{pb145183.jpg}
\caption*{Quelle vue !}
\end{figure}


\begin{figure}[h]
\centering
\includegraphics[height=6cm,width=9cm,keepaspectratio]{pb145615.jpg}
\caption*{Petit monument trouvé par hasard au bord de la route.}
\end{figure}

Notre seconde escapade nous amène à Signaghi, où nous retrouvons aussi Flavie et Julien. Mais oui, on avais failli faire la traversée de la mer Caspienne avec eux avant de se rendre compte que... hmmm, bref, passons. Ce village est connu pour son vin, et aussi pour le fait qu'il a presque été converti en un grand centre de vacances soviétique.




\begin{figure}[h]
\centering
\includegraphics[height=6cm,width=9cm,keepaspectratio]{pb155649.jpg}
\caption*{Détail de la statue "Mère de la Géorgie".}
\end{figure}


\begin{figure}[h]
\centering
\includegraphics[height=6cm,width=9cm,keepaspectratio]{pb105003.jpg}
\caption*{C'est une spécialité locale : des fruits secs enrobés d'une sorte de pâte de marc de raisin. C'est ... typique !}
\end{figure}

Nous logeons chez M. Shota : Un papy trapu gentil comme un nounours, qui après quelques minutes dans sa maison, nous amène dans son musée personnel : il n'est pas seulement trapu, mais aussi accessoirement champion d'haltérophilie ! Champion du monde même : 340kg en développé couché ! C'est là qu'on prend aussi conscience que dans nounours, il y a ours ! Son musée est rempli de coupes, de médailles et d'articles en tout genre ! Sûrement inquiet de nous voir si maigres, il nous offre sans cesse à manger et à boire : du raisin, des kakis, du vin. Tout vient de sa production personnelle. Le petit déjeuner fut un festin. En arrivant, la table était remplie : fromage, pain, muffin, khinkalis (j'y reviendrai), saucisses, œufs... et à peine une assiette se vidait, qu'il en apportait une nouvelle. A ma grande honte, on n'a pas pu finir. Même pas la moitié. Il nous a obligé à emporter le reste avec nous. Et on ne négocie pas avec un champion du monde d'haltérophilie. On dit oui monsieur, merci monsieur, le petit déjeuner était très bon monsieur !




\begin{figure}[h]
\centering
\includegraphics[height=6cm,width=9cm,keepaspectratio]{pb155663.jpg}
\caption*{Il sait accueillir !}
\end{figure}


\begin{figure}[h]
\centering
\includegraphics[height=6cm,width=9cm,keepaspectratio]{pb155661.jpg}
\caption*{On ne voit qu'un petit bout de son musée !}
\end{figure}

Près de Signaghi, sur la frontière azéri, se trouve un monastère troglodyte incontournable, que nous avons donc visité, si on peut vraiment appeler ça une visite, étant donné que l'on était quasiment privés de vue tellement le brouillard était épais. Le paysage est incroyable en théorie, les grottes ont une vue imprenable sur une grande vallée. Nous, on avait surtout du mal à voir nos pieds...




\begin{figure}[h]
\centering
\includegraphics[height=6cm,width=9cm,keepaspectratio]{pb165742.jpg}
\caption*{Comme il n'y a pas de paysage, on compense comme on peut...}
\end{figure}

Heureusement, le lendemain, nous sommes allés nous perdre dans d'autres grottes troglodytes à Uplistsikhe. Et là, la météo s'est rattrapée au niveau du brouillard ! Beau ciel bleu, quelques nuages dans le ciel, et surtout, un vent à décorner les bœufs ! On a du mal à tenir debout, mais ça fait des timelapses impressionnants.




\begin{figure}[h]
\centering
\includegraphics[height=6cm,width=9cm,keepaspectratio]{pb206333.jpg}
\caption*{Le chat fait tout pour avoir l'air mignon !}
\end{figure}

Juste à coté se trouve la ville de Gori. Ce nom éveillera peut-être de vieux souvenirs de cours d'histoire : c'est la ville natale de ce bon vieux Joseph Vissarionovitch Djougachvili, dit Staline ! On peut donc y visiter sa maison natale, le train dans lequel il cachait sa peur de l'avion, et un grand musée à sa gloire. Je ne vous cache pas qu'il y a quelques trous dans son hagiographie, et qu'on le voit surtout en chef de guerre victorieux et prenant la pose avec des enfants bien nourris.




\begin{figure}[h]
\centering
\includegraphics[height=6cm,width=9cm,keepaspectratio]{martinpecheur.jpg}
\caption*{Rassurez-vous, la plupart des Géorgiens ne sont pas des dictateurs, bien au contraire. Vous remarquerez aussi que j'ai été un peu lent pour la dernière photo...}
\end{figure}

Puis nous sommes allés à Mtskheta (aucun chat n'a marché sur le clavier, j'ai revérifié, c'est la bonne orthographe). Et nous avons visité, je vous le donne en mille : des monastères sur des collines, ainsi qu'une grande église au centre. La spécialité locale, c'est le \emph{lobio}, une soupe de haricot rouge. En vrai, les restaurants en ont rarement. J'imagine que ça fait moins rêver que le Katchapuri, plein de beurre et de fromage ?




\begin{figure}[h]
\centering
\includegraphics[height=6cm,width=9cm,keepaspectratio]{pb186278.jpg}
\caption*{En même temps, c'est beau les églises !}
\end{figure}


\begin{figure}[h]
\centering
\includegraphics[height=6cm,width=9cm,keepaspectratio]{pb186290.jpg}
\caption*{Avec Pierre sous la main, Marion a pu souffler un peu pour les photos.}
\end{figure}

Et puisque je parle de bouffe, revenons au khinkali : c'est l'autre grande spécialité de la Géorgie. De gros raviolis fourrés à un peu ce qu'on veut. Dans les restaurants, on en commande de grands plats à partager. Ensuite, attention à l'étiquette : il faut les saisir fermement par la queue, mais pas trop non plus, sinon on les casse, pour ensuite faire une petite ouverture délicate à la base, et tout de suite aspirer le jus de viande ! Peu importe de faire du bruit, mais il est primordial de ne pas laisser tomber une goutte dans l'assiette. Ensuite, on peut grignoter petit à petit le khinkali, ou bien le dévorer en une bouchée, mais seul un malandrin sans éducation le mangera jusqu'au bout : c'est faire son radin que de manger la queue, il faut la laisser sur le bord de l'assiette avec ses petits camarades, ce qui permet de compter les victimes, et une chose en entrainant une autre, ça amène à faire des concours où personne ne gagne, si ce n'est le restaurateur.




\begin{figure}[h]
\centering
\includegraphics[height=6cm,width=9cm,keepaspectratio]{pb135087.jpg}
\caption*{Tout au bout de la table, il y a une assiette de Khinkalis. Désolé, je n'ai pas mieux...}
\end{figure}

Comme à chaque fois, l'heure du départ arriva, et Pierre du retourner en France. Nous avons alors filé plein ouest, dans le train le plus confortable de tout notre voyage : tout neuf, avec de grands sièges, et surtout, du wifi très rapide et des prises pour l'ordi ! Dans ces conditions, le temps passe encore plus vite... Une petite nuit à Batumi, une ville pleine de casinos sur le bord de la Mer Noire, et nous voilà prêts à rentrer en Turquie !




\begin{figure}[h]
\centering
\includegraphics[height=6cm,width=9cm,keepaspectratio]{pb206746.jpg}
\caption*{Une œuvre d'art à Batumi. Et en plus, ça bouge !}
\end{figure}


\begin{figure}[h]
\centering
\includegraphics[height=6cm,width=9cm,keepaspectratio]{pb206333.jpg}
\caption*{Le chat fait tout pour avoir l'air mignon !}
\end{figure}





\chapter{La Turquie : on arrive en Europe !}
Nous arrivâmes donc en Turquie. Le passage de la frontière se fait sans encombre. Seul évènement notable : après un passage dans le poste de la douane, on me rend mon passeport avec une tache de nourriture. Mais pas n'importe quelle tache : une tache juste sur mon visage, autour de mon œil gauche, comme pour me rappeler un certain sauna russe... Ce passeport, c'est désormais mon portrait de Dorian Gray !


\begin{figure}[h]
\centering
\includegraphics[height=6cm,width=9cm,keepaspectratio]{pb236791.jpg}
\caption*{La mosquée bleue (bleue surtout à l'intérieur nous a-t-on dit).}
\end{figure}

Après un bref passage à Trabzon, sur la côte de la Mer Noire, nous poursuivons directement notre voyage en direction d'Istanbul. Après une nuit dans un bus très confortable et des autoroutes sans nids de poules, nous voilà en Europe et à deux pas de la mosquée bleue !


\begin{figure}[h]
\centering
\includegraphics[height=6cm,width=9cm,keepaspectratio]{pb236799-panorama.jpg}
\caption*{L'intérieur de la mosquée bleue. Oui, je vois ce que vous voulez dire, mais ce n'est pas moi qui l'ai baptisée (d'autant plus que je ne suis pas sûr que ça se fasse de baptiser une mosquée)}
\end{figure}

Cette dernière fait face à la cathédrale mosquée Sainte Sophie, et ensemble elles dominent complètement le quartier de Sultanamet. C'est ici la vieille ville, et donc c'est ici qu'étaient successivement Byzance puis Constantinople ! L'histoire est présente à chaque coin de rue.


\begin{figure}[h]
\centering
\includegraphics[height=6cm,width=9cm,keepaspectratio]{pb246853.jpg}
\caption*{Une jolie pièce dans le palais de Topkapi.}
\end{figure}

Nous visitons quelques mosquées, dont la mosquée bleue, mais Hagia Sophia (Sainte Sophie) est vraiment trop chère, nous l'admirons donc de l'extérieur. Par contre, la Citerne Basilique vaut vraiment l'entrée : il s'agit d'un ancien et énorme réservoir d'eau datant de Byzance ! Il a été oublié pendant des siècles, avant qu'un archéologue se demande d'où les habitants d'un quartier tiraient leur eau. En descendant dans un des puits, il découvre une salle énorme pleine de piliers, façon mines de la Moria inondée. Évidemment, la salle était presque entièrement envasée, mais après le nettoyage de quelques centaines de tonnes de boue, ça ressemblait à ça.


\begin{figure}[h]
\centering
\includegraphics[height=6cm,width=9cm,keepaspectratio]{pb256877.jpg}
\caption*{Ok, peut-être que l'éclairage n'est pas d'origine, mais quand même !}
\end{figure}

En prime, ils ont trouvé des têtes de méduse à la base de deux piliers ! Imaginez un peu l'émotion des découvreurs : trouver un bâtiment aussi beau et aussi grand, en plein centre d'une grande ville !


\begin{figure}[h]
\centering
\includegraphics[height=6cm,width=9cm,keepaspectratio]{pb256872.jpg}
\caption*{Une des deux têtes de méduse.}
\end{figure}

Nous visitons aussi le palais de Topkapi. Notre partie préférée fut la cuisine ! Alors que les autres pièces sont surtout des belles pièces dans lesquelles il ne se passe rien, la cuisine est un lieu toujours en ébullition ! Elle nourrissait entre dix et quinze mille personnes, alors je n'ose imaginer la corvée épluchage de patates. Les explications sont fournies sur l'importance de cet endroit, l'organisation des repas, comment les plats ont évolué au contact avec le monde occidental, avec par exemple l'apparition de la fourchette !


\begin{figure}[h]
\centering
\includegraphics[height=6cm,width=9cm,keepaspectratio]{pb246832.jpg}
\caption*{Casseroles pour une armée !}
\end{figure}


\begin{figure}[h]
\centering
\includegraphics[height=6cm,width=9cm,keepaspectratio]{pb246840.jpg}
\caption*{Un arbre ? Plus ? Vous voyez quelqu'un ? Regardez mieux !}
\end{figure}

Et la nourriture n'est pas importante que dans les musée ! De notre coté, on mange beaucoup de Kebab, qui bien entendu n'ont pas grand chose à voir avec ceux qu'on trouve en France. On mange des baclavas qui existent dans d'innombrables variations, certains salons de thé ont des menus de 10 ou 15 pages uniquement pour ça ! Il ne faut pas oublier non plus les loukoums. Au début, ce fut l'échec : les premiers achetés au hasard dans un attrape touriste se sont avérés tellement immangeables qu'on a jeté la boîte et même - c'est dire - recraché le demi loukoum. Celui à la menthe donnait l'impression de manger un gros bloc de dentifrice... Après quelques recherches, on trouve enfin des loukoums dignes de ce nom dans une boutique centenaire : fondants, goutus, et surtout... nourrissants !


\begin{figure}[h]
\centering
\includegraphics[height=6cm,width=9cm,keepaspectratio]{pb256883.jpg}
\caption*{Hmmmm !!!!}
\end{figure}

Nous retrouvons aussi Berfu, une Turque que nous avions rencontrée en Inde ! Elle et son amie nous ont fait découvrir les quartiers plébiscités de la jeunesse stambouliote (oui, ça fait bizarre, mais istamboulaise c'est pire...). Nous avons eu droit à un excellent café turc (à ne surtout pas remuer avant de boire), et ce qui est bien avec le café turc, c'est quand il est fini, car on peut attaquer la divination !


\begin{figure}[h]
\centering
\includegraphics[height=6cm,width=9cm,keepaspectratio]{pb256905.jpg}
\caption*{Je vois... je vois.... une girafe !}
\end{figure}

Elles nous ont aussi raconté comment la Turquie était en train de changer en ce moment : c'est devenu officiellement une dictature. Critiquer le gouvernement sur Internet, ou trop proches d'oreille indiscrètes peut vous amener directement en prison. Même un truc aussi anodin que courir dans la rue amène des policiers à vous arrêter car c'est comportement louche. En même temps, la montée de l'islamisme fait que de plus en plus, une fille est mal vue si elle se promène seule etc. Bref, le futur s’assombrit, alors qu'il n'y a pas si longtemps, la Turquie considérait de peut-être se rapprocher de l'UE. Quel changement de trajectoire !


\begin{figure}[h]
\centering
\includegraphics[height=6cm,width=9cm,keepaspectratio]{pb266919.jpg}
\caption*{En Turquie, mollesse interdite !}
\end{figure}

Notre trajectoire à nous, nous amène ensuite quelques jours à Budapest (quelle transition incroyablement souple, n'est-ce pas ?). Nous avons dormi dans une petite auberge de jeunesse sans jamais rencontrer le propriétaire : la porte s'ouvre par un code, la chambre est déjà prête et la clé est sur la porte, et quand on part, on laisse le paiement dans une boite au lettre rouge près de la porte ! Pourquoi s'embêter ? Nous avons mangé du goulasch servi dans une miche de pain, visité les marchés de Noël, et cuisiné dans l'auberge de jeunesse où nous étions seuls la plupart du temps. On ralentit un peu le rythme car quelques jours plus tard, ça va accélérer...


\begin{figure}[h]
\centering
\includegraphics[height=6cm,width=9cm,keepaspectratio]{pb256917.jpg}
\caption*{La mosquée bleue de nuit. Eh oui, encore loupé !}
\end{figure}





\chapter{Prague, la dernière pièce du puzzle}
Nous arrivâmes donc à Prague, après une traversée express de l'Europe ! Mais pourquoi tant de précipitation ? La réponse est simple : il manquait la dernière pièce du puzzle. Faites le compte (et relisez le blog), pendant le voyage, Marion avait reçu la visite de son ... et vous voyez maintenant là où je veux en venir ! C'est donc à Prague que ma mère nous a rejoint pour un dernier petit bout de voyage en notre compagnie, accompagnée par mon oncle et ma marraine !


\begin{figure}[h]
\centering
\includegraphics[height=6cm,width=9cm,keepaspectratio]{pc027679.jpg}
\caption*{C'est Noël !}
\end{figure}

Pour ceux qui ne le savaient pas, Prague est une des plus belle ville de l'Europe. Et même si l'exotisme de l'Asie est maintenant loin derrière nous, nous avons été vraiment surpris. On nous avait dit, et nous avions lu que cette ville était vraiment magnifique, mais malgré nos attentes forcément un plus élevées, on a trouvé que c'était encore plus beau que ce qu'on imaginait.


\begin{figure}[h]
\centering
\includegraphics[height=6cm,width=9cm,keepaspectratio]{pc037001.jpg}
\caption*{Ma maman !}
\end{figure}

Le fait d'être en décembre rend la ville encore plus intéressante, avec les décorations et marchés de Noël, et à part le week-end, les rues ne sont pas trop envahies de touristes. Mais un peu quand même. Bon, OK, j'avoue, c'est blindé de touristes, et on se croirait parfois pendant la fête des lumières à Lyon, mais par moment on peut se balader dans les rues sans slalomer entre les (autres) touristes.


\begin{figure}[h]
\centering
\includegraphics[height=6cm,width=9cm,keepaspectratio]{pc026960.jpg}
\caption*{Du cristal de Bohème.}
\end{figure}

Gérard était très curieux de visiter la ville, car c'est la deuxième fois qu'il vient. La première fois, c'était avant la chute du mur... Autant dire que la ville était méconnaissable. Aucun magasin, presque personne dans les rues, tous les murs couleur poussière de charbon... Et par chance, on a retrouvé l'hôtel (les trois autruches) dans lequel il avait dormi, juste à coté du fameux pont Charles !


\begin{figure}[h]
\centering
\includegraphics[height=6cm,width=9cm,keepaspectratio]{pc067174.jpg}
\caption*{Gérard (mon oncle) et les trois autruches.}
\end{figure}

Ce pont Charles est un très vieux pont, couvert de statues posées par les catholiques pour essayer de convertir les protestants qui traversaient le pont tous les jours. A présent, elles servent surtout de perchoirs à mouette et agrémentent les photos de coucher de soleil.




\begin{figure}[h]
\centering
\includegraphics[height=6cm,width=9cm,keepaspectratio]{pc087210.jpg}
\caption*{Aucun respect !}
\end{figure}


\begin{figure}[h]
\centering
\includegraphics[height=6cm,width=9cm,keepaspectratio]{pc067183.jpg}
\caption*{Coucher de soleil sur le pont.}
\end{figure}


\begin{figure}[h]
\centering
\includegraphics[height=6cm,width=9cm,keepaspectratio]{pc037028.jpg}
\caption*{Le pont Charles de nuit}
\end{figure}

Cela dit, on sent que la religion a été (est encore ?) très importante ici, ne serait-ce que par la profusion d'églises magnifiques ! Il y a du gothique comme on peut en voir pas mal en France, et surtout beaucoup de baroque : le gothique en met plein la vue avec des colonnes, des voutes et des arcs-boutant, et c'est aussi impressionnant à l'intérieur qu'à l'extérieur.


\begin{figure}[h]
\centering
\includegraphics[height=6cm,width=9cm,keepaspectratio]{pc057086.jpg}
\caption*{La cathédrale Saint Guy : c'est beau dehors ! (avec Gérard, Adrienne ma marraine, maman, et moi)}
\end{figure}


\begin{figure}[h]
\centering
\includegraphics[height=6cm,width=9cm,keepaspectratio]{pc057103.jpg}
\caption*{La cathédrale Saint Guy : c'est beau dedans !}
\end{figure}


\begin{figure}[h]
\centering
\includegraphics[height=6cm,width=9cm,keepaspectratio]{pc057100-panorama.jpg}
\caption*{Et il y a des magnifiques vitraux d'Alfons  Mucha, un des fers de lance de l'art nouveau.}
\end{figure}

Le baroque, de son coté, c'est surtout à l'intérieur que ça se joue. La plupart des façades ne laissent rien présager de l'intérieur, et sans un guide pour donner l'information, on passerait devant sans même tourner la tête. Mais quand on sait, et qu'on entre, on en prend plein les mirettes ! Après, faut aimer le style, car le moins qu'on puisse dire, c'est que c'est chargé ! Il y a des angelots et des dorures à foison, et plus c'est alambiqué, mieux c'est ! Les grandes plaques de marbre qui couvrent les colonnes de l'église Saint Nicolas nous ont vraiment impressionné, jusqu'à ce qu'on comprenne que c'est de l'imitation en stuc.


\begin{figure}[h]
\centering
\includegraphics[height=6cm,width=9cm,keepaspectratio]{pc067148-panorama.jpg}
\caption*{C'est chargé, non ?}
\end{figure}


\begin{figure}[h]
\centering
\includegraphics[height=6cm,width=9cm,keepaspectratio]{pc067162.jpg}
\caption*{Ce stuc est tellement réputé que l'artiste qui l'a réalisé a un plaque à son nom dans l'église Saint Nicolas.}
\end{figure}


\begin{figure}[h]
\centering
\includegraphics[height=6cm,width=9cm,keepaspectratio]{pc067165.jpg}
\caption*{Un visiteur dans l'église Saint Nicolas.}
\end{figure}

Et il n'y a pas que le marbre : les bougies aussi sont désormais des imitations. Il faut mettre une pièce dans une machine pour illuminer des bougies électriques pendant 15mn. Je comprends l'intérêt au niveau de la suie, des dangers d'incendie, et même des émissions de CO2, mais au fond de moi, mon petit cœur de pyromane est triste.


\begin{figure}[h]
\centering
\includegraphics[height=6cm,width=9cm,keepaspectratio]{pc057110.jpg}
\caption*{Ils auraient au moins pu mettre un peu de variété dans la forme des bougies.}
\end{figure}

Heureusement les trdelniks (ça se prononce "trdelniks", facile !) sont là pour nous remonter le moral. Ce sont des sortes de brioches cuites au barbecue : on peut donc se réchauffer de l'extérieur pendant la cuisson avant de se réchauffer aussi de l'intérieur en les mangeant !


\begin{figure}[h]
\centering
\includegraphics[height=6cm,width=9cm,keepaspectratio]{pc026955.jpg}
\caption*{Les trdelniks en pleine cuisson !}
\end{figure}

Évidemment, il n'y a pas que de la pâtisserie à manger : la cuisine tchèque est riche en choux, patate et viande ! C'est bien pour nous, car le choc du retour en Alsace sera moins rude :-) On mange donc de la choucroute, du canard, de la charcuterie, et aussi des travers de porc mémorables. On n'avait pas lu les petites lignes sur le menu qui disaient "1kg". Je me suis donc retrouvé avec 1kg de barbaque devant moi. Et bien à ma grande honte, et, si mes souvenirs sont justes, pour la première fois du voyage, je n'ai pas réussi à finir mon assiette... Pour vous achever, le prix de ce plat : 7\euro !


\begin{figure}[h]
\centering
\includegraphics[height=6cm,width=9cm,keepaspectratio]{pc036981.jpg}
\caption*{Bon appétit, et bon courage !}
\end{figure}

Nous avons aussi profité de la ville pour ses concerts de musique classique. On a commencé en douceur par un petit concert dans une jolie église pas chauffée, et ensuite, conseillé par un guide qui nous a fait découvrir la ville, nous avons trouvé le bureau qui vend les billets de concerts que les habitants de Prague vont voir. La qualité monte d'un cran : nous avons vu l'orchestre symphonique de Prague, dirigé par José Cura : une centaine de musicien dans la plus grande salle de spectacle de la ville ! En arrivant à la salle de concert le soir, on a quand même eu un moment de doute en voyant les autres spectateurs en tenue de gala, alors que nous étions en tenue de chaussure de marche et polaire portée tous les jours depuis plus d'un an... Le concert fut génial, et le chef d'orchestre s'est même permis une petite fantaisie en laissant un percussionniste diriger le Boléro de Ravel à la caisse claire, tandis que lui-même allait s'asseoir dans le public. Il nous a confié ensuite qu'on lui avait conseillé de ne pas faire ça, car on saurait alors qu'il ne servait à rien. Mais il s'en moque, il a trente ans de carrière et plus rien à prouver ! En tout cas, le percussionniste, dont c'était le dernier jour avant la retraite, était fier comme un paon !


\begin{figure}[h]
\centering
\includegraphics[height=6cm,width=9cm,keepaspectratio]{pc047080.jpg}
\caption*{Le premier concert.}
\end{figure}
\begin{figure}[h]
\centering
\includegraphics[height=6cm,width=9cm,keepaspectratio]{pc036986.jpg}
\caption*{PC036986.jpg}
\end{figure}La maison dansante de Frank Gehry


\begin{figure}[h]
\centering
\includegraphics[height=6cm,width=9cm,keepaspectratio]{pc037019.jpg}
\caption*{Encore une petite statue du pont Charles pour la route.}
\end{figure}


\begin{figure}[h]
\centering
\includegraphics[height=6cm,width=9cm,keepaspectratio]{pc047069.jpg}
\caption*{Le cimetière juif, où des milliers de personnes sont enterrées sur plusieurs couches.}
\end{figure}


\begin{figure}[h]
\centering
\includegraphics[height=6cm,width=9cm,keepaspectratio]{pc087269.jpg}
\caption*{Dédicace à Pierre !}
\end{figure}

Après ces quelques jours à Prague, nos visiteurs s'en retournent chez eux, et il ne nous reste alors plus que quelques jours à passer seuls tous les deux. Nous visitons Cesky Krumlov, un petit village médiéval dans le sud de la République Tchèque. On s'octroie ainsi nos derniers jours de repos, avant de prochainement rentrer pour de bon en France et d'enchainer les Noëls en famille. (Oui, pour ceux qui ne l'ont toujours pas compris, ce blog est en retard...)







\chapter{Retour}
\emph{Petit préambule : oui, j'ai traîné pour ce dernier article, mais votre patience va enfin être récompensée...}

Nous partîmes donc en direction... de la France ! Oui, pour de vrai ! Un direct Prague Mulhouse en bus. Nous sommes arrivés au milieu de la nuit, et mon frère Kévin nous attendait. C'était déjà lui, qui, un an deux mois et six jours plus tôt nous avait déjà amenés au milieu de la nuit à ce même arrêt de bus : la boucle est définitivement bouclée !


\begin{figure}[h]
\centering
\includegraphics[height=6cm,width=9cm,keepaspectratio]{pc157697.jpg}
\caption*{Dernière bière avant de rentrer en France !}
\end{figure}

On ne se sait pas trop ce qu'on ressent, tout se mélange... Nous sommes un peu tristes que ce soit fini, et en même temps contents d'être rentrés, surpris de se sentir chez soit aussi vite. Tout nous semble tellement normal et habituel, que nous avons presque l'impression de n'avoir jamais quitté la France. Et pourtant nous avons la tête tellement pleine de souvenirs que le cerveau déborde.

Nous avons lu que le retour après un grand voyage pouvait être difficile. Avoir vécu tellement de choses, avoir changé au point d'avoir l'impression d'être devenu quelqu'un d'autre, pourrait, ou plutôt aurait pu, rendre difficile la confrontation avec une France restée la même. Mais nous, pour l'instant, ça va.

Le tout premier matin, nous partons à la chasse au pain au chocolat. Ce qui nous frappe, c'est cette quantité de gens qui parlent français ! On comprend tout ! Notre cerveau qui a été à l’affût, voire en manque de notre langue maternelle, qui s'est montré capable d'identifier la moindre syllabe de français dans une foule, est soudain saturé. Nous prenons conscience de toutes les conversations de la rue avec une acuité jamais ressentie auparavant. Nous sommes déconcentrés au point d'avoir du mal à suivre notre propre conversation : Imaginez Doug, le chien de Là-haut (Pixar), soudainement plongé dans un élevage d'écureuils.


\begin{figure}[h]
\centering
\includegraphics[height=6cm,width=9cm,keepaspectratio]{doug.jpg}
\caption*{Pain au chocolat !}
\end{figure}

Petit moment de flottement au moment de commander les fameux pains au chocolat dans la boulangerie : "I would like...euh... Je voudrais deux pains au chocolats s'il vous plait". Je me sens handicapé dans ma propre langue. Je ne me moquerai plus jamais de Jean-Claude Van Damme et de ses anglicismes de snob/mec qui a passé trop de temps à l'étranger. Il nous faudra encore quelques jours pour arrêter de dire "sorry", plutôt que "pardon" aux gens dans la rue.

Nous passons quelques jours dans ma famille en Alsace. Nous fêtons Noël 3 ou 4 fois tout en tentant de limiter les excès... J'ai bien dit "tenté", pas "réussi". On parle quand même de NOËL en ALSACE, après 14 mois de riz et de nouilles ! Personnellement, j'ai eu le niveau de volonté d'un enfant de 3 ans dans un magasin de bonbons.


\begin{figure}[h]
\centering
\includegraphics[height=6cm,width=9cm,keepaspectratio]{pc277732.jpg}
\caption*{Ça, ça nous avait manqué...}
\end{figure}

Puis il est temps d'aller enfin retrouver la famille de Marion ! La première étape sera chez mamie Josette, la grand mère de Marion. Histoire de faire une petite surprise, on arrive un jour plus tôt que ce qu'on lui avait dit... Cette petite surprise de dernière minute amplifie les émotions, mais on survit aux embrassades, avec tout de même les yeux un peu humides...

Et c'est reparti pour Noël ! Et oui, la famille de Marion aussi nous attendait ! Là encore, on a tenté de limiter les dégâts, mais.... NOËL en ARDÈCHE ! Bref, aucune illustration n'est nécessaire à mon avis.

Nous avons aussi eu la joie/l'angoisse de retrouver toutes nos affaires. En se retrouvant au milieu de tous ces cartons, je me sens dépassé par les événements. Il va nous falloir des sacrés sacs dos ! Mais ça fait du bien d'enfiler autre chose que les deux mêmes sempiternels T-shirts. Quoiqu'avoir trop de choix fait parfois perdre plus que temps de nécessaire au moment de s'habiller : c'est ce qui s'appelle avoir des problèmes de riche !


\begin{figure}[h]
\centering
\includegraphics[height=6cm,width=9cm,keepaspectratio]{p1077966.jpg}
\caption*{Marion : Mon chapeau !!!!}
\end{figure}


\begin{figure}[h]
\centering
\includegraphics[height=6cm,width=9cm,keepaspectratio]{p1077968.jpg}
\caption*{Gérald : Ça ne rentrera jamais dans mon sac !}
\end{figure}

Bon, c'est pas tout de se faire gâter, après toutes ces retrouvailles, il est temps de recommencer une vie normale : \textbf{et demain, on se pose où ?}


\begin{figure}[h]
\centering
\includegraphics[height=6cm,width=9cm,keepaspectratio]{p2268335.jpg}
\caption*{Ceci est un indice pour nous trouver à présent.}
\end{figure}

\tableofcontents
\end{document}
