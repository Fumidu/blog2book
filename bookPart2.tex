
\documentclass{book}
% Chargement d'extensions
\usepackage[francais]{babel} % Pour la langue française
\usepackage[utf8]{inputenc}
\usepackage[T1]{fontenc}
\usepackage{float}
\usepackage{eurosym}
\usepackage{textcomp}
\usepackage{wrapfig} % to wrap figure in text
%\usepackage{caption} % to remove fig #
\usepackage[font=it]{caption}
\usepackage{titlesec} % to modify chapter header
\usepackage[normalem]{ulem}
\usepackage{soul}

\titleformat{\chapter}[display]
  {\normalfont\bfseries}{}{0pt}{\Large}

\usepackage[
  paperwidth=15.557cm,
  paperheight=23.495cm,
  %showframe,
  margin=15mm
  % other options
]{geometry}

\hyphenation{CRRR-RA-A-A-A-AK-K-K-K-K-KR-R-R-RB-B-B-B-BRA-A-A-A-A-A-AOUM-M-M-M-M-MR-R-R-RM-M-M-ML-L-L-LLLLL}

%\usepackage{layout} %TODO remove

\usepackage{graphicx} % Pour les images
\graphicspath{ {Images/petit/} }
%\graphicspath{ {Images/} }

\title{Et demain on va où ?}
\author{Gérald Schmitt \and Marion Abrial}

\begin{document}
%\layout
\maketitle


\chapter{Sapa, premier aperçu du Vietnam.}
Nous arrivâmes donc au Vietnam. Le passage de la frontière est assez rapide, et nous voilà dans la rue, à quelques kilomètres de la station de bus qu'on doit rejoindre si on veut pouvoir aller à Sapa, notre prochaine étape. Et on se sent tout de suite très oppressés : des vendeurs nous collent aux basques d'une manière qui serait vraiment très impolie en France. Ils nous suivent d'un distributeur à l'autre, nous attendent à la sortie du magasin où on va faire un peu de monnaie, et se penchent sans aucune gêne sur notre épaule pour tenter de voir ce qu'on tripote sur le téléphone. Et ceci couplé à tous les avertissements qu'on a lus à propos des arnaques au Vietnam nous met un peu sous pression. Ça ne s'arrange pas quand on essaie de négocier un taxi, et qu'ils nous proposent tous un prix au moins deux fois supérieur à ce qui devrait être le prix normal d'après de multiples sources concordantes sur internet. Alors oui, on parle de payer 1,5\euro au lieu de 75 centimes pour faire 3km, mais pour le principe, on refuse et on part à pied. On finit par trouver un taxi avec des tarifs normaux quelques centaines de mètres plus loin, et on monte rapidement dans le bus pour Sapa.


\begin{figure}[h]
\centering
\includegraphics[height=6cm,width=9cm,keepaspectratio]{p1185946.jpg}
\caption*{Vue depuis Sapa.}
\end{figure}

Dans ce bus, on assiste à ce qui nous semble être du racisme : un groupe de jeunes femmes d'une minorité ethnique entre dans le bus avec des gros paniers, et semblent revenir du marché. Pendant le trajet, un panier tombe contre la porte du bus un peu avant que le chauffeur n'ouvre la porte, ce qui coince le panier derrière la porte. Plutôt que de fermer la porte pour décoincer le panier, le chauffeur a plutôt regardé la jeune femme batailler, jusqu'à ce qu'un autre homme vienne le lui déchirer pour le décoincer, et même si on ne comprenait pas le viet, on a eu la sérieuse impression que ça faisait bien marrer les mecs dans le bus...


\begin{figure}[h]
\centering
\includegraphics[height=6cm,width=9cm,keepaspectratio]{p1175938.jpg}
\caption*{On l'a trouvé perdu dans la rue !}
\end{figure}

Et c'est sur ces bonnes impressions qu'on arrive à Sapa. C'est une ville construite par les Français dans la montagne, pour fuir la chaleur de Hanoi. A ce qu'il parait, la ville a eu un air de village alpin. Mais actuellement, c'est une usine à touristes. Les magasins de contrefaçons Northface succèdent aux restaurants qui proposent tous les mêmes menus. Les motos-taxis nous hèlent dès qu'on fait mine de vouloir aller quelque part à pied, pour laisser la place rapidement à des femmes des minorités qui viennent nous accoster avec un grand sourire et un excellent niveau d'anglais, voire de français, et qui essaient de nous vendre leur artisanat, ou de se vendre comme guide pour nous accompagner dans leur village. Car c'est ça en fin de compte le vrai intérêt de Sapa, c'est ce qu'il y a autour : la montagne, et les villages des minorités.



\begin{wrapfigure}{l}{0.4\textwidth}
\centering
\includegraphics[width=0.35\textwidth]{p1185986.jpg}
\caption*{Le genre de petite fille qu'on laisse tranquille à la récré !}
\end{wrapfigure}


D'ailleurs, c'est ce qu'on va essayer de faire rapidement. On achète une carte de rando à l'office du tourisme, on se prépare une petite boucle pour la journée, et on part plein d'entrain. La première embûche nous tombe dessus très vite : on doit payer pour traverser le premier village si on veut atteindre le fond de la vallée. Le village en question est de nouveau à mi-chemin entre Disneyland et un centre commercial.

\begin{figure}[h]
\centering
\includegraphics[height=6cm,width=9cm,keepaspectratio]{p1175937.jpg}
\caption*{Ça fait bizarre quand on tombe dessus en se baladant dans le marché... C'est une viande recommandée pour les femmes enceinte !}
\end{figure}

Hors de question de payer pour ça, on va bien trouver un moyen de le contourner. On prend donc un autre chemin, et un peu plus tard, on croise d'autres touristes qui remontent. Quand on leur demande d'où ils viennent, on apprend qu'ils voulaient faire la même chose que nous, et qu'ils se sont fait refouler une deuxième fois plus bas, et qu'il n'y a pas moyen de passer sans acheter le ticket d'entrée. On finit donc pas céder tous les quatre, on traverse ce village, et nous voilà désormais à 4 cerveaux à essayer de lire cette carte. Notre GPS et toute notre expérience de rando en France n'y fait rien, on ne comprend rien à la carte, et on finit par se balader au hasard en suivant des chemins qui nous inspirent. On apprendra un peu plus tard que la carte serait volontairement faussée pour encourager l'embauche de guides. On finit par faire une chouette balade dans la forêt en compagnie de ce couple de Français qui nous racontent leur expérience au tribunal international des Kmers Rouges au Cambodge où ils viennent de bosser quelques mois. C'est un petit avant-goût de notre prochaine destination !


Marion, ne supportant pas tout ces décors en cartons, trouve les coordonnées d'un guide français, Olivier, et on part trois jours avec lui et sa famille. La première nuit on dort chez lui : il est marié avec une Dao, une des minorité ethnique du Vietnam. Ils vivent à Ta Phin, un petit village Dao traditionnel, avec leurs enfants. C'est à 2 heures de marche de la route, il n'y a ni électricité ni eau courante, mais le téléphone mobile passe ! L'eau est fournie par un petit ruisseau, et l'électricité fournie par la petite turbine en aval du même ruisseau permet d'alimenter deux ampoules et de recharger le téléphone portable. La maison en elle même ressemble plutôt à une grange. Le sol est en terre battue et il y a des animaux partout. C'est d'ailleurs l'impression que cet endroit nous laissera : ça grouille de vie. Tous les animaux qu'on croise sont accompagnés de leurs petits : les buffles et leurs petits veaux/bufflons, les porcs et les porcelets, les chiens et leurs chiots, les poules et leur innombrable marmaille piaillante. Tout ce petit monde semble vivre en harmonie.


\begin{figure}[h]
\centering
\includegraphics[height=6cm,width=9cm,keepaspectratio]{p1216681.jpg}
\caption*{Pièce principale.}
\end{figure}


\begin{figure}[h]
\centering
\includegraphics[height=6cm,width=9cm,keepaspectratio]{p1206477.jpg}
\caption*{Olivier, sa femme, son fils.}
\end{figure}

Plus tard dans l'après midi, on continue tout seul l'ascension de la colline qui surplombe le village, car il y a une chance qu'on arrive à traverser la couche de nuage dans laquelle on baigne depuis le matin. Pleins d'espoir, on monte, et on monte, et on monte encore dans la brume (ça vous rappelle une histoire ?). On arrive au sommet, toujours dans le brouillard. Ce n'est pas la première fois que ça nous arrive, on ne se démonte pas pour autant et on commence à faire des photos à la con, genre selfie faussement épaté par le paysage inexistant. Là, le miracle se produit. La couche de nuage semble s'affiner, on commence à sentir la chaleur du soleil, et soudain, les nuages se déchirent pour laisser apparaitre quelques instants une mer de nuages, avant de nouveau nous recouvrir. Puis à nouveau le ciel se déchire, et ça recommence : nous étions en fait tout pile à la limite des nuages, et les vagues de brumes nous passaient dessus au ralenti. Magique !


\begin{figure}[h]
\centering
\includegraphics[height=6cm,width=9cm,keepaspectratio]{p1196007-panorama.jpg}
\caption*{D'un coup, on a vu ça !}
\end{figure}

De retour chez Olivier, un bain nous attendait. Mais alors il faut se rendre compte de la quantité de travail pour faire ce bain : il faut aller couper du bois pour alimenter le feu qui va chauffer l'eau. Et ce n'est pas un simple bain, c'est un bain aux herbes traditionnel dao. Donc il faut aussi ajouter la corvée de cueillette des herbes ! Puis il faut remplir la baignoire, qui est un tonneau en bois, avec une bassine, sans se brûler de préférence, et ajuster la température avec de l'eau froide. Et enfin on peut aller macérer dans l'eau chaude parfumée.


\begin{wrapfigure}{l}{0.55\textwidth}
\centering
\includegraphics[width=0.5\textwidth]{p1196463.jpg}
\caption*{L'eau du bain aux herbes en train de chauffer.}
\end{wrapfigure}

Olivier nous parle aussi de l'agriculture : ils dépendent de la culture du riz, et ici dans les montagnes, c'est dur ! Ils ne peuvent faire qu'une seule récolte par an, contre deux dans la vallée et même trois dans le delta du Mékong. Étant donné la configuration des rizières, pas de tracteur, tout est fait à la main, où à la rigueur avec un petit motoculteur qui commence à remplacer le buffle pour le labourage. Mais Montsanto est quand même présent : c'est cette entreprise qui fourni les graines et les produits chimiques de manière quasi obligatoire. Les paysans qui refusent et replantent leurs graines d'une année sur l'autre ont des amendes pour diverses raisons (corruption des dirigeants locaux, assimilation des gènes depuis les champs voisins qui sert de prétexte aux poursuites). C'est à tel point que les Dao ne savent plus, ou ne veulent plus faire sans la chimie. Olivier et sa femme ont chacun leur jardin potager : celui de sa femme, cultivé de manière "traditionnelle" avec des intrants, et celui d'Olivier, cultivé avec des engrais naturels, car il veut lui montrer que c'est possible de se passer en grande partie de chimie.


\begin{figure}[h]
\centering
\includegraphics[height=6cm,width=9cm,keepaspectratio]{p1196466.jpg}
\caption*{L'inévitable atelier nems !}
\end{figure}

Le lendemain, on passe la journée avec la belle-mère d'Olivier. Une toute petite Dao avec un sourire éclatant, et une pêche incroyable. J'aurais pu dire "une pêche incroyable pour son âge", car elle a déjà quelques années, elle est plusieurs fois grand-mère, mais non, elle a une pêche tout court : on avait rendez-vous à 9h en bas du village d'Olivier pour qu'elle nous amène chez elle, à trois heures de marche. Et elle était venue à pied : oui, 6h de marche dans la montagne à l'aube pour accompagner deux touristes ! Et arrivé chez elle, on est accueilli par le reste de la famille (fils, belles-filles, petits enfants) et l'odeur de l'alcool de riz en train d'être distillé dans la pièce principale. Les vapeurs nous enivrant tous, l'accueil a été très joyeux ! Pour ce qui est de la marche, ce n'était pas fini, on est allé dans la montagne avec elle pour faire la cueillette de plantes pour le bain du soir. Puis on est rentré, elle nous a dit de nous reposer pendant qu'elle repartait encore crapahuter car elle n'avait pas trouvé toutes les plantes qu'elle voulait ! Et pendant les temps morts, elle s'occupait de l'alambic. Nouvel an approchant, il fallait faire les réserves !


\begin{figure}[h]
\centering
\includegraphics[height=6cm,width=9cm,keepaspectratio]{p1206522.jpg}
\caption*{Parée pour la cueillette !}
\end{figure}


\begin{figure}[h]
\centering
\includegraphics[height=6cm,width=9cm,keepaspectratio]{p1206506.jpg}
\caption*{Rechargement de l'alambic avec du riz fermenté.}
\end{figure}

On a aussi pu visiter leur jardin : ils cultivent des orchidées spécialement pour le têt, le nouvel an vietnamien. Ces orchidées sont très populaires à cette période, et sur les routes, on voit beaucoup de motos transporter ces énormes pots de fleurs. Elle nous explique que ces fleurs mettent 3 ans à pousser, elles sont vendues pour le nouvel an, et elles meurent ensuite souvent rapidement.


\begin{figure}[h]
\centering
\includegraphics[height=6cm,width=9cm,keepaspectratio]{p1206509.jpg}
\caption*{Les toilettes au milieu des orchidées.}
\end{figure}

Les repas sont délicieux. La table est couverte de petits plats dans lesquels chacun se sert. Et pour la boisson, c'est aussi le partage qui prime : interdiction de siroter son alcool de riz dans son coin. Soit on trinque avec tout le monde, soit on ne boit pas ! Et leur alcool, il n'y a pas a dire, "\emph{ça reste une boisson d'homme"} : pur à la sortie de l'alambic, distillé le jour même, ça doit bien titrer dans les 70\textdegree . On l'a bien senti passer ! Alors qu'on venait de finir le repas du soir, et qu'on continuait frénétiquement à trinquer, on entend soudain une moto s'approcher. Et là, ce fut la panique. La belle mère nous fait signe de venir avec elle, et on part se cacher dans une chambre, en silence, tandis qu'on entend ses enfants parler avec quelqu'un. On finit par sortir un peu plus tard, pour apprendre qu'en fait, elle n'a pas encore officiellement le droit d'accueillir des touristes chez elle, et elle avait peur que ce visiteur inattendu puisse être la police ou un voisin curieux et dénonciateur. Mais non, c'était juste un voisin perdu. Apparemment, le suivi des touristes est très sérieux au Vietnam : seuls les endroits autorisés peuvent accueillir des touristes, et souvent, seulement des touristes. Il existe donc des hôtels pour Vietnamiens, et des hôtels pour étrangers, ce qui peut poser problème aux couples mixtes face à un employé d'hôtel bureaucratiquement zélé.


\begin{figure}[h]
\centering
\includegraphics[height=6cm,width=9cm,keepaspectratio]{p1216693.jpg}
\caption*{Petite séance photo !}
\end{figure}


\begin{wrapfigure}{l}{0.55\textwidth}
\centering
\includegraphics[width=0.5\textwidth]{p1216686.jpg}
\caption*{Le grand frère est convaincu !}
\end{wrapfigure}

On s'est fait un petit échange de cadeaux à la fin : avec les moyens du bord, on a fait un petit mobile en origami pour le dernier petit bébé de la maison, arrivé fraichement dans la semaine. En fin de compte, c'est son grand frère que ça a vraiment intéressé. Et Marion a eu droit à un vrai foulard Dao ! C'est le genre de truc qu'on ne voit pas habituellement sur les têtes des touristes, et beaucoup de Vietnamiens nous ont posés des questions sur l'origine du foulard une fois revenus en ville. On a l'impression d'avoir eu un cadeau plutôt rare !



\begin{figure}[h]
\centering
\includegraphics[height=9cm,width=12cm,keepaspectratio]{p1216709.jpg}
\caption*{Marion avec son nouveau foulard.}
\end{figure}

\chapter{Ba Be lake, le paradis, ou presque...}
Nous partîmes donc pour Ba Be Lake. En bus de nuit ! Notre premier bus de nuit en Asie. Les premières impressions ont été mitigées : un petit van avec des sièges confortables et légèrement inclinables, mais on était serrés dedans. Mais en fait non ! C'était juste une navette qui nous amenait au point de rendez-vous avec le vrai bus. Le point de rendez-vous, c'était sous un pont pas éclairé, et il faisait nuit. Le genre d'endroit où on attend plutôt son dealer, mais non, nous, on attendait notre bus ! Et le vrai bus était nettement mieux : de vraies couchettes dans lesquelles on est quasiment allongés. Bon, ce bus étant au format local, faut pas s'attendre à pouvoir déplier les jambes si on fait plus de 160cm, mais il est néanmoins possible de dormir pour de vrai.


\begin{figure}[H]
\centering
\includegraphics[height=9cm,width=12cm,keepaspectratio]{p1246786.jpg}
\caption*{Au bord du lac.}
\end{figure}

Et nous arrivons donc le matin dans une autre ville, on reprend un bus pour quelques heures, puis on fait 1h de taxi, et enfin 20mn de marche. Disons qu'on ne passe pas là-bas par hasard. Mais une fois sur place, ça présente de fortes similitudes avec ce que d'aucuns appelleraient "le paradis sur Terre". Un grand lac vert émeraude, bordés d'arbres immenses, les rares endroits plats sont des rizières qui alimentent un petit village paisible. Plus de klaxons, plus de rabatteurs, plus de bus. Le calme. Et en plus, il y fait bon toute l'année. Enfin... la plupart de l'année. Nous, on y est resté juste le temps de profiter de la super vague de froid descendue de Chine. Il a fait 4\textdegree C. Alors, vous allez me dire, pour des gars qui on fait du -35\textdegree C quelques semaines avant en Mongolie, 4\textdegree C, c'est du pipi de chat ? Non ? Et ben en fait, pas vraiment. Tout d'abord, l'humidité : 100\%, ou presque. Ça donne un froid qui rentre partout, sous les vêtements, et il n'y a pas de soleil, tout est toujours brumeux, rien ne sèche ! Ensuite, il faut savoir que les bâtiments sont conçus pour des températures oscillant entre 20\textdegree C et 35\textdegree C, donc il y a des trous partout, entre les planches du mur, sous la porte, entre les lattes du plancher, et il fait donc 4\textdegree C jusque dans la chambre, sous la couette. On avait de la chance d'avoir une chambre avec un climatiseur qui pouvait aussi faire plus ou moins radiateur, mais sinon, aucun chauffage n'est prévu ! De même, la salle à manger est dehors, les plats sont froids en 2mn, impossible de se réchauffer en mangeant ! Les locaux n'ont même pas de pull à mettre, on les voyait greloter toute la journée, alors que nous avions ressorti la doudoune ! Ils en ont été réduits à faire un feu dans une bassine en métal pour tenter de se réchauffer, ce qui a surtout eu pour effet d'enfumer notre chambre juste au-dessus (les trous entre les lattes, souvenez-vous). Croyez le ou non : on a eu plus froid ici qu'en  !



\begin{figure}[H]
\centering
\includegraphics[height=9cm,width=12cm,keepaspectratio]{p1246745.jpg}
\caption*{Petite balade en bateau.}
\end{figure}

\newpage


\begin{figure}[H]
\centering
\includegraphics[height=9cm,width=12cm,keepaspectratio]{p1246804-panorama.jpg}
\caption*{Pour ne rien gâcher, nous étions seuls dans la grotte !}
\end{figure}


\begin{wrapfigure}{l}{0.6\textwidth}
\centering
\includegraphics[width=0.55\textwidth]{p1236717.jpg}
\caption*{Enfin une feuille de la bonne taille !}
\end{wrapfigure}


Ça ne nous a pas empêché de nous balader un peu. On a vu quelques grottes absolument magnifiques. J'avais l'impression qu'à tout moment, Indiana Jones allait sortir en courant, pourchassé par une armée de sauvages, mais après un peu d'exploration, on se rend compte qu'il y a beaucoup plus de chauve-souris que d'archéologues kleptomanes dans ces grottes.


A certains endroits dans la forêt, on se serait crus dans chérie j'ai rétréci les gosses, tellement les plantes, les arbres, et parfois les insectes étaient gros ! Et à propos d'insecte, ou plutôt, d'arthropode, voici la photo d'un centipède (comme son nom l'indique, ça a beaucoup trop de pattes pour être un insecte) qui squattait juste à coté de l'interrupteur de la salle de bain. On a beau savoir que ça ne pique pas, ça ne mord pas, ça ne vole pas, et que ça bouge très lentement quand il fait froid, ça fait quand même bizarre de tomber dessus au milieu de la nuit.


\begin{figure}[H]
\centering
\includegraphics[width=0.7\textwidth]{p1236728.jpg}
\caption*{Il me regarde, non ?}
\end{figure}


\begin{wrapfigure}{l}{0.55\textwidth}
\centering
\includegraphics[width=0.5\textwidth]{p1256818.jpg}
\caption*{Là, c'est trop grand !}
\end{wrapfigure}

On a aussi fait une balade en bateau jusqu'à l'autre bout du lac, et on a pu visiter la fameuse île de la veuve. Il y a une légende qui dit qu'avant, ce lac n'existait pas, c'était une plaine. Un jour, alors que personne ne demandait rien, une vieille mendiante est apparue, et elle a commencé à faire ce que toutes les mendiantes font : demander de la bouffe ! Évidemment, la plupart des gens n'étaient pas dupes, et avaient bien vu que c'était du flan cette histoire de mendiante, et ne lui donnaient rien. Jusqu'à ce que la mendiante arrive à la maison de cette veuve, qui, probablement accablée de chagrin et surchargée de boulot, lui donna un peu à manger, histoire de s'en débarrasser rapidement. Et là, grosse surprise, on apprend que la mendiante est en fait une fée ! Oh, ben ça alors, tu parles d'une surprise ! Et que fait la fée pour remercier la veuve ? Elle lui dit : "Hop, monte dans ton panier que je viens de transformer en barque, et accroche-toi, je vais faire un rien de terrassement". Sur ces paroles, elle transforme la plaine en lac entouré de montagnes, et la maison de la veuve se retrouve isolée sur une minuscule île pleine de cailloux. Sympa la vieille !



\begin{figure}[H]
\centering
\includegraphics[height=7cm,width=12cm,keepaspectratio]{p1246792.jpg}
\caption*{Sur l'île de la veuve. Cosy, non ?}
\end{figure}

J'imagine que la morale de cette histoire c'est : il ne faut pas aider les vieilles un peu louches, sinon, on va perdre ses champs et se retrouver seul au monde ? Bref, je ne suis pas sûr d'avoir bien compris cette histoire, mais ils en ont l'air très fier là-bas. Si vous avez une meilleure idée sur ce qu'il y a à comprendre, n'hésitez pas !


\begin{figure}[H]
\centering
\includegraphics[height=7cm,width=12cm,keepaspectratio]{p1236723.jpg}
\caption*{Un nénufar ? (comment convertir un innocent article de voyage en article politiquement engagé en un seul mot)}
\end{figure}


\begin{figure}[h]
\centering
\includegraphics[height=12cm,width=12cm,keepaspectratio]{p1256823.jpg}
\caption*{S'il y a un biologiste qui me lit, je veux bien savoir ce que c'est !}
\end{figure}



\chapter{Hanoï}
Nous arrivâmes donc à Hanoï, la capitale du Vietnam. Et non, aucune trace de moines qui passent leur temps à empiler des disques sur des tours, c'est une légende inventée de toutes pièces ! C'est la deuxième plus grande ville du Vietnam, donc ça nous fait l'impression d'un village chinois en terme de taille. Et comme on reste dans le vieux centre, on a presque l'impression d'être en France ! Il y a des petites ruelles fleuries, des boulangeries avec des vraies baguettes et des petits pains, ce qui nous permet de nous faire un petit repas nostalgie avec des goûts qu'on n'avait plus eu depuis longtemps. En revanche, niveau circulation, pas de doute, on est bien au Vietnam.


\begin{figure}[h]
\centering
\includegraphics[height=6cm,width=9cm,keepaspectratio]{p1276839.jpg}
\caption*{Et pour ne rien arranger, on a l'impression que certains conducteurs ne voient vraiment rien.}
\end{figure}

On progresse encore d'un cran dans l'anarchie du trafic, avec un flux ininterrompu de scooters qui prend d'assaut toutes les rues. On doit re-apprendre à traverser en ignorant tous ces signaux d'avertissement que nos parents se sont échinés à mettre en place dès notre plus tendre enfance. Ici, si on attend que les véhicules s'arrêtent pour nous laisser traverser, on risque fort d'y passer la journée. Il faut avancer doucement, sans changer de vitesse ni de direction, et faire confiance aux scooters pour nous éviter et ça marche !


\begin{figure}[h]
\centering
\includegraphics[height=6cm,width=9cm,keepaspectratio]{p1276835.jpg}
\caption*{Et si on en a marre des scooters, il y a cette rue...}
\end{figure}

Une particularité des bâtiments du Vietnam commence à nous sauter franchement aux yeux : tous les bâtiments sont très étroits. Notre hôtel par exemple, ne doit faire qu'environ 4m de large, et c'est le cas de la plus grande partie des bâtiments à Hanoï et dans le Vietnam. Après enquête, il s'avère qu'il y a une taxe qui dépend de la largeur. Ça me fait penser au Pérou, ou la plupart des maisons ont un étage en construction, car les impôts commencent uniquement quand les travaux sont finis. C'est fou les conséquences d'une loi mal faite...


\begin{wrapfigure}{l}{0.55\textwidth}
\centering
\includegraphics[width=0.5\textwidth]{p1286874.jpg}
\caption*{Exemple typique d'optimisation fiscale.}
\end{wrapfigure}

On va visiter comme il se doit le mémorial d'Ho Chi Minh, ou, comme ils l'appellent là-bas : "Oncle Ho". C'est un peu le Genghis Khan vietnamien : il a bouté les américains hors du pays, réunifié le Nord et le Sud, il apparait sur tous les billets, toutes les villes ont baptisé leur plus belle avenue en son honneur, et chaque ville se doit également d'avoir un mémorial qui lui est dédié. Celui de Hanoï est bien sûr le plus grand : en plus d'être construit autour de sa dernière résidence (il s'est fait construire une maison traditionnelle de pêcheur, sur pilotis, pour bien montrer son origine populaire), on peut aussi voir le mausolée qui, dans la tradition communiste la plus pure, abrite son corps momifié, contre sa volonté bien entendu. Évidemment, on s'est levé trop tard pour voir son corps, qui n'est visible que le matin. Quel dommage...


\begin{figure}[h]
\centering
\includegraphics[height=6cm,width=9cm,keepaspectratio]{p1276860.jpg}
\caption*{PEUGEOT 404 : PRESIDENT NOT FOUND}
\end{figure}

En fait, ce qui nous plait le plus à Hanoï, ce sont les restaurants. On commence à découvrir vraiment la richesse de la gastronomie vietnamienne, et vu le prix des restaurants, on ne se prive pas ! Il y a les classiques : les nems de toutes sortes, frits ou frais, le pho (à prononcer feu, comme pot au feu, héritage des Français), et tout plein de plats avec plein d'épices : citronnelle, gingembre, coriandre, piment... Plein de saveurs, beaucoup de produits frais, des mélanges surprenants pour nous, bref, on s'éclate !


\begin{figure}[h]
\centering
\includegraphics[height=6cm,width=9cm,keepaspectratio]{p1286867.jpg}
\caption*{Je ne sais plus ce que c'était, mais qu'est-ce que c'était bon !}
\end{figure}


\begin{figure}[h]
\centering
\includegraphics[height=9cm,width=12cm,keepaspectratio]{p1286888.jpg}
\caption*{Hanoï de nuit.}
\end{figure}



\chapter{La baie d'Halong, des cailloux, de l'eau et des calamars.}
Nous partîmes donc pour la baie d'Halong, un des incontournables du Vietnam, et ce d'autant plus que depuis quelques années, elle fait partie du patrimoine mondiale de l'UNESCO. Comme la grande majorité des gens, on prend un tour organisé au départ de Hanoï, pour deux jours et une nuit sur un bateau.


\begin{figure}[h]
\centering
\includegraphics[height=6cm,width=9cm,keepaspectratio]{p1297269-panorama.jpg}
\caption*{La baie d'Halong.}
\end{figure}


\begin{wrapfigure}{l}{0.45\textwidth}
\centering
\includegraphics[width=0.4\textwidth]{p1297785.jpg}
\caption*{Un calamar curieux.}
\end{wrapfigure}

Même le Lonely Planet, chantre du voyage individuel, recommande de prendre un tour plutôt que de tenter d'y aller tout seul. On avait le choix entre 3 gammes de tarifs : pour 45\euro, les repas sont à peine mangeables, les activités promises ne sont pas faites, et les guides sont désagréables. Pour 200\euro, c'est repas gastronomique, cours de Tai-Chi le matin, chambres climatisées, et chaises longues avec un matelas sur le toit du bateau. Nous on a pris le forfait à 130\euro, (en fait 100\euro, merci Marion), et on a eu des repas très bons, une chambre avec salle de bain privée, et toutes les activités promises ont été faites : on a visité une caverne, on a fait du kayak même si il a fallut se lever tôt.



A la fin du repas du soir, un des cuisinier est venu nous montrer les calamars qu'il venait de pêcher, c'était le signal pour aller tenter ça nous-même : on met un grosse lumière à l'avant du bateau, et on agite un appât en plastique vert garni d'hameçons à l'aplomb de la lumière. Le calamar, animal curieux et naïf s'il en est, s'en approche alors  histoire d'en savoir un peu plus. Le pêcheur attentif, aux réflexes intactes malgré la bière, peut alors tirer sur sa ligne d'un coup sec, espérant ainsi embrocher le calamars qui verra sa curiosité récompensée et découvrira un monde nouveau, à base de citronnelle et d'huile chaude. Enfin, ça c'est la théorie. Mais quand même, que d'émotions quand un calamar s'approche de l'appât, puis s’enfuit à toute vitesse  en lâchant un nuage d'encre, alors que, les yeux rivés sur le calamar fuyard, on prie pour que les hameçons retombent dans l'eau plutôt que sur nos têtes...


\begin{figure}[h]
\centering
\includegraphics[height=6cm,width=9cm,keepaspectratio]{p1297791.jpg}
\caption*{Un Gérald patient.}
\end{figure}


\begin{figure}[h]
\centering
\includegraphics[height=6cm,width=9cm,keepaspectratio]{p1297260.jpg}
\caption*{Le guide a insisté : c'est un doigt ! Mais un doigt sacrément drôle car il continue de pouffer comme un ado à chaque visite quotidienne.}
\end{figure}


On a aussi visité une ferme de perle, dont la partie la plus intéressante était probablement l'apparition d'une énorme méduse, puis sa capture par des gens du cru. Les perles, c'était intéressant aussi hein ! On a vu toutes les étapes, de l'insertion du noyau à l'ouverture d'une huitre pour y trouver, oh surprise, une perle, en passant par la boutique de souvenirs ou je n'ai jamais vu autant de zéros sur une étiquette de prix ! Mais ça, comparé à un mec en chaussures de ville, en équilibre sur une latte, en train passer une corde autour d'une méduse de 30kg, j'avoue, j'hésite, et je ne sais plus très bien si je dois continuer la visite guidée ou rester pour voir qui va manger qui !


\begin{figure}[h]
\centering
\includegraphics[height=6cm,width=9cm,keepaspectratio]{p1307808.jpg}
\caption*{A table !}
\end{figure}




\begin{figure}[h]
\centering
\includegraphics[height=6cm,width=9cm,keepaspectratio]{p1307806.jpg}
\caption*{Oh, une perle !}
\end{figure}

Je parle des activités mais on en oublierai presque le plus important ! La baie en elle-même ! On n'est quand même pas venus là juste pour pêcher ! Alors : c'est beau. Et on a eu de la chance, car les précédents jours, les sorties en bateau étaient interdites à cause du mauvais temps. Nous, on a bénéficié d'un temps potable : très nuageux, pas un seul rayon de soleil, mais pas de pluie. Et malgré ces conditions moyennes, c'est magnifique ! On arrive même à faire abstraction du coté usine à touristes, et à oublier les déchets qui flottent un peu partout.


\begin{figure}[h]
\centering
\includegraphics[height=9cm,width=12cm,keepaspectratio]{p1297574.jpg}
\caption*{Notre bateau.}
\end{figure}

\chapter{Ninh Binh, là où les pelleteuses chassent les barques.}
Nous arrivâmes donc à Ninh Binh. Il faisait nuit, la gare était déserte, et un mec nous attendait avec un panneau à nos noms. C'était notre taxi, grande classe non (enfin, pour des backpackers) ? On avait fait une exception à nos habitudes, et réservé le trajet jusqu'à notre auberge de jeunesse. Et on a bien fait, elle est perdue au milieu des rizières, sur une route en terre sans éclairage. On a découvert les paysages le lendemain matin : c'est magnifique. Le coin ne vole pas son nom de Baie d'Halong terrestre : des pics karstiques entourés de rivières et de rizières. C'est calme, beau, et entièrement couvert de brouillard...


\begin{figure}[h]
\centering
\includegraphics[height=9cm,width=12cm,keepaspectratio]{p1318117.jpg}
\caption*{C'est le paradis des oiseaux !}
\end{figure}




\begin{figure}[h]
\centering
\includegraphics[height=8cm,width=12cm,keepaspectratio]{p2029405.jpg}
\caption*{Le bâtiment à droite caché dans les arbres, c'est notre guesthouse.}
\end{figure}

On loue des vélos sur place, et on part visiter le coin. Faut voir les vélos : ce sont des vélos de ville au format Vietnamien, donc on peut ne peut déplier les jambes qu'à moitié. Heureusement, tout est quasiment plat, et presque sans s'en rendre compte, on a fait 50km dans la journée. Vers midi, en passant le long d'une digue, deux hommes nous font signe de venir les rejoindre. Ils étaient en train de mettre une toute petite barque à l'eau, et nous invitent à monter avec eux. On leur demande combien ils veulent, et non, ils ne veulent pas d'argent. Deux inconnus dont vous ne parlez pas la langue, vous invitent à les accompagner vers une destination inconnue au milieu d'un marécage : vous faites quoi ? La bonne réponse est, "On dit gentiment non, et on se barre". Nous, confiants, on y est allés ! A quatre dans une barque pour deux, l'eau passait doucement par dessus bord, et la barque restait souvent empêtrée dans les algues, mais on a quand même fini par arriver à la pelleteuse flottante des mecs ! On a réussi à comprendre deux mots : Hollywood, et King Kong 2. Et on est allé vérifier, et oui, King Kong 2 est bien en tournage au Vietnam, et les mecs étaient super fiers à l'idée d'y contribuer. Ils sont montés sur leur pelleteuse, et nous ont donné les clés de la barque, histoire qu'on aille se balader pendant qu'ils bossent. On s'est retrouvé à ramer dans un marécage, poursuivi par une pelleteuse qui élargissait le passage et nous rattrapait petit à petit. On se serait cru dans un film d'action, même si je n'ai jamais vu une poursuite entre une pelleteuse flottante et une barque.


\begin{figure}[h]
\centering
\includegraphics[height=6cm,width=9cm,keepaspectratio]{p1318127.jpg}
\caption*{Fuyez, pauvres fous !}
\end{figure}


\begin{figure}[h]
\centering
\includegraphics[height=9cm,width=12cm,keepaspectratio]{p1318130.jpg}
\caption*{Comme on était trop mauvais pour ramer, il nous a ramené, mais en cassant une rame au passage.}
\end{figure}

On est aussi allé faire un tour en bateau plus classique, dans un circuit bien organisé, avec achat de billets, et portique d'entrée. Juste après le portique d'entrée, on arrive sur le quai, où on voit 20 bateaux et leurs rameuses qui attendent sagement, mais on nous fait signe d'attendre car ce sont des bateaux de 4 places. Après avoir attendu 30mn et vu plusieurs groupes de 4 déjà formés nous passer devant, nous finissons par aller demander si on ne pourrait pas partir quand même, et ils acceptent ! La balade est plutôt sympa, nous sommes deux dans une barque pour 4 personnes, et comme pour tous les bateaux dans le coin, c'est une femme qui rame. Elle ne parle pas anglais, mais elle semble quand même contente et souris chaque fois qu'on est ébahi par un canard, une fleur ou un caillou. Mais ce qu'on préfère, c'est traverser les grottes creusées par la rivière à travers les pics. Notre rameuse se transforme en pilote pour éviter les nombreuses stalactites, et on doit souvent se coucher dans le bateau quand le plafond baisse un peu trop. Ravis de la balade, on sort du bateau, et on lui donne le pourboire recommandé. Et là, elle fait la gueule, et commence à parler en viet avec les autres gens, l'air visiblement énervée. On s'est renseigné un peu plus tard, et on apprend qu'on aurait du donner un double pourboire vu qu'on était que deux dans le bateau... Ce fonctionnement m'énerve sur tellement de niveaux, que je ne sais pas par où commencer ! La prochaine fois, s'ils veulent des trucs, ils n'ont qu'à l'expliquer. C'est dommage, à part ça, la balade était sympa.


\begin{figure}[h]
\centering
\includegraphics[height=7cm,width=12cm,keepaspectratio]{p2018478.jpg}
\caption*{Un échantillon des rameuses désœuvrées. La file de bateaux continue, il y en a au moins 500 (cinq cents, ce n'est pas une faute de frappe).}
\end{figure}


\begin{figure}[h]
\centering
\includegraphics[height=7cm,width=12cm,keepaspectratio]{p2019391.jpg}
\caption*{Petit temple au milieu de l'eau.}
\end{figure}

Le soir, on discute un peu avec le propriétaire de l'auberge et on apprend qu'il a commencé à mettre en place des cours d'anglais gratuits pour les gamins du village. Il fournit un local, et cherche des volontaires parmi ses clients pour donner un cours un soir. Les élèves qui viennent sont volontaires (ou en tout cas, quelqu'un les encourage fortement à être volontaire sinon tu vas devoir aller désherber la rizière de nuit), et ils sont quand même accompagnés par leur prof d'anglais de l'école, histoire de ne pas laisser un pauvre gars sans formation ni expérience seul face à un groupe de gamins (on a l’Éducation Nationale pour ça...). Le jour où ce fut mon tour (et ne me dites pas que vous ne le sentiez pas venir), j'ai eu de la chance : on a dépassé le record de fréquentation : 21 gamins, enfin, 19 gamines et 2 gamins, deux fois plus que d'habitude ! Le principale apport du voyageur, c'est l'accent. Et oui, même un français peut aider. J'avais même du mal à comprendre la prof d'anglais officielle par moment. Par contre, niveau motivation, rien à voir avec mes souvenirs du collège... Elles se battent presque pour poser des questions. Elles parlent déjà toutes anglais suffisamment bien pour avoir une petite conversation (oui, il y avait aussi deux garçons, mais ils semblaient nettement moins motivés).


\begin{figure}[h]
\centering
\includegraphics[height=7cm,width=12cm,keepaspectratio]{p2019393.jpg}
\caption*{Les filles.}
\end{figure}


\begin{figure}[h]
\centering
\includegraphics[height=7cm,width=9cm,keepaspectratio]{p2019395.jpg}
\caption*{Les garçons.}
\end{figure}


\begin{figure}[h]
\centering
\includegraphics[height=9cm,width=12cm,keepaspectratio]{p1318116.jpg}
\caption*{J'adore cette ambiance post-apocalyptique !}
\end{figure}



\chapter{Hué, là où on a appris ce que survivre veut dire.}
Nous partîmes donc pour Hué. Après nos expériences de train de nuit, on s'est dit qu'on se devait de comparer aussi le train de nuit du Vietnam. On a pris la 1ère classe. Et bien en terme de confort, ça ne vaut pas la 3ème classe russe. Le lit est ok, même si on n'est pas sûr d'être les premiers à dormir dedans. Les toilettes ne sont vraiment pas terribles (toilettes occidentales, rarement lavées, dans un train qui bouge, je vous laisse imaginer), les lavabos sont utilisés pour stocker des plantes, et on se fait réveiller au milieu de la nuit par des gens pas foutus de voir que le téléphone qu'ils cherchent est sur leur lit. Sans oublier les contrôleurs qui se mettent devant notre compartiment avec des grosses lampes de poche pour se raconter des blagues au milieu de la nuit. Ah oui, il y a aussi des blattes... Le tout à une vitesse moyenne de 40km/h. Faut dire que ce sont les français qui ont construit tout le réseau, et que le matériel n'a pas changé depuis, donc on voit partout les bricolages, les rustines et les tentatives de modernisation du genre "et si on rajoutait une télé ?". Bref, on prendra désormais le bus pour les longs trajets.


\begin{figure}[h]
\centering
\includegraphics[height=6cm,width=9cm,keepaspectratio]{p2039409.jpg}
\caption*{Vous croyez que ce sont des couleurs naturelles ?}
\end{figure}

On finit par arriver à Hué, l'ancienne capitale du Vietnam. Il a fait mauvais tout notre séjour, et on n'avait pas envie de visiter encore des monuments, donc on a fait l'impasse sur la cité interdite de Hué, qui serait presque aussi bien que la cité interdite de Pékin. On les a crus sur parole. Par contre, on a pris un tour d'une journée pour aller visiter la DMZ (celle du Vietnam bien sûr, rien à voir avec les Corées). Cette Zone Démilitarisée (DMZ) séparait le Vietnam du Nord et du Sud, et est rapidement devenue une des zones les plus militarisée du monde. Le bus nous emmène à plein d'endroits ou ne voit plus rien. Les bombes, ça n'aide pas à la conservation : On voit le début du sentier de Ho Chi Minh, qui s'avère être un simple sentier qui disparait dans la forêt, et une colline où il y avait une base d'observation américaine : une colline toute bête. Très instructif...


\begin{figure}[h]
\centering
\includegraphics[height=6cm,width=9cm,keepaspectratio]{p2059424.jpg}
\caption*{En effet, c'est une colline.}
\end{figure}



\begin{wrapfigure}{l}{0.45\textwidth}
\centering
\includegraphics[width=0.4\textwidth]{p2059587.jpg}
\caption*{La lumière et les marches, c'est un ajout postérieur pour les touristes.}
\end{wrapfigure}

On visite un musée de la guerre, avec des reconstitutions de tranchées, quelques engins exposés mais guère plus. Et enfin, on arrive à la partie vraiment intéressante du tour : les tunnels de Vinh Moc. Dans la DMZ, les gens qui voulaient survivre n'ont eu qu'un seul choix : vivre sous terre. Plus de 120 villages dans la zone ont créé un grand réseau de sous-terrains pour abriter la population et les soldats. Le village qu'on a visité comptait 600 personnes sous terre, vivant dans des conditions incroyables. Dans le noir complet et l'humidité toute la journée, ils ne sortaient de temps en temps que tôt le matin ou tard le soir. Il y a même 17 bébés qui sont nés dans ces conditions. En visitant les tunnels, on a vu la "maternité" : une cavité d'un mètre sur deux, sur 1,6m de hauteur.



Dans les rues de la ville, c'est l'effervescence : tout le monde est en train de préparer têt qui arrive dans quelques jours. Il y a des marchés aux fleurs de partout, et des scooters qui transportent des énormes pots de fleurs, voire des orangers ! L'oranger, c'est un peu leur arbre de Noël : tout le monde se doit d'en avoir un, il est posé au milieu du salon et décoré avec des guirlandes brillantes.
Un soir, le proprio de la guesthouse nous invite, ainsi que tous les autres clients, pour un repas de pre-têt, deux jours avant la vraie date. Le dîner fut somptueux, avec beaucoup plus à manger que ce qui était possible, même avec 20 convives ! Et il a fait tout son possible pour vider sa grande réserve de bière. Il surveillait les niveaux dans les cannettes, et gare à celui qui en avait une vide !


\begin{figure}[h]
\centering
\includegraphics[height=9cm,width=9cm,keepaspectratio]{p2049422.jpg}
\caption*{Tout doit disparaitre !}
\end{figure}

Mais ce n'est pas à Hué qu'on a passé le têt, mais dans une ville proche, à Hoi An. La veille du dernier jour de leur année, on est parti en train, et on a mis 4h pour faire 100km. Réellement, je vous jure. Outre la vétusté du matériel, il n'y a qu'une seule ligne, donc les trains ne peuvent se croiser qu'en de rares endroits, et on peut se retrouver à attendre longtemps à l'arrêt dans une gare abandonnée qu'un train passe dans l'autre sens.

C'est dans ce train que, intrigué par les cris d'un gamin capricieux, nous avons découvert non pas un gamin relou, mais un petit singe en laisse !

\emph{Imaginez ici une conclusion appropriée, rigolote mais néanmoins profonde, car là tout de suite, je n'en ai pas. Une chose est sûre en revanche, c'est la fin de l'article.}


\begin{figure}[h]
\centering
\includegraphics[height=9cm,width=12cm,keepaspectratio]{p2069610.jpg}
\caption*{Le singe, pendant un des rares moments où il ne courrait pas dans tous les sens}
\end{figure}



\chapter{Hoi An, la ville aux 1000 tailleurs.}

\begin{wrapfigure}{l}{0.55\textwidth}
\centering
\includegraphics[width=0.5\textwidth]{p2099955.jpg}
\caption*{Boutique de lanternes dans Hoi An.}
\end{wrapfigure}

Nous arrivâmes donc à Hoi An, la ville que nous avons choisie pour passer la semaine de Têt. Il faut savoir que la grosse fête de fin d'année est suivie d'une semaine de vacance pour tout le pays. La majeure partie des commerces sont fermés, les transports sont blindés de gens qui rentrent dans leur famille, et globalement, tout est plus cher. C'est pourquoi nous avons décidé de nous offrir une semaine à la plage, dans une ville assez touristique pour que certains restaurants et hôtels restent ouverts.





Une autre caractéristique de Hoi An, ce sont les tailleurs : apparemment, dans les années 2000, il y avait une vingtaine de tailleurs à Hoi An, capables de faire des vêtements sur-mesure de bonne qualité pour un prix défiant toute concurrence. Pour une raison ou une autre, ça a grandi en popularité et d'autres tailleurs sont arrivés. En 2016, il y aurait plus de 1000 boutiques de tailleurs se battant pour attirer les touristes. On n'a pas compté les boutiques, mais on a pu constater que des rues entières, longues de plusieurs kilomètres, alignent les tailleurs les uns à coté des autres. Il y a des ateliers un peu partout pour réaliser les commandes, et ils peuvent faire un costume en quelques heures en s'y mettant à 5 ou 6 si besoin : un ouvrier sur une manche, un sur une jambe etc. Évidemment, il n'y a pas assez de touristes pour faire vivre toutes ces boutiques, ce qui tire les prix vers le bas, et les commissions vers le haut (commissions qui vont sans problème jusqu'à 40\%) ! On a bien essayé nous-même de faire un devis dans une boutique, mais pendant la seule semaine de congé nationale, disons que les prix étaient un tout petit peu moins intéressants. Au regret de Marion, je ne continuerai pas le voyage en costume sur mesure.




\begin{figure}[h]
\centering
\includegraphics[height=6cm,width=9cm,keepaspectratio]{p2079648.jpg}
\caption*{C'est bientôt Têt, donc les gens transportent des fleurs. Au fait, vous avez vu le deuxième gars ?}
\end{figure}

Le soir de têt, on demande à l'accueil de l'hôtel s'il se passe quelque chose dans les rues, et on a une comme réponse "ça va être amusant". Têt a la réputation d'être une fête plutôt familiale, donc on ne s'attendait pas forcément à voir des trucs incroyables. Mais finalement, le centre ville était très animé, avec beaucoup de stands de nourriture, une sorte de fête foraine à l'ancienne, et plein d'illuminations. Ils aiment aussi beaucoup les loteries façon bingo, mais c'est aussi un spectacle : plusieurs présentateurs se relaient pour chanter en permanence. Sans comprendre le vietnamien, on avait l'impression que c'était une sorte d'improvisation sur le thème des symboles tirés au sort. Le chant ne s'arrête pas une seule fois, les enchainements sont parfaits ce qui a un coté hypnotisant.




\begin{figure}[h]
\centering
\includegraphics[height=6cm,width=9cm,keepaspectratio]{p2079706.jpg}
\caption*{Illuminations le soir de Têt.}
\end{figure}




\begin{figure}[h]
\centering
\includegraphics[height=6cm,width=9cm,keepaspectratio]{p2089730.jpg}
\caption*{De l'autre coté de la rivière.}
\end{figure}


\begin{figure}[h]
\centering
\includegraphics[height=6cm,width=9cm,keepaspectratio]{p2079726.jpg}
\caption*{Les vendeuses de lanternes s'ennuient...}
\end{figure}

Vers 22h, un peu fatigués, on fini par rentrer après avoir tout vu. Là, on vide nos poches de tous les prospectus reçus, et on voit une référence à un feu d'artifice à minuit ! Ben on est ressortis aussi sec, on a trouvé la place idéale, et on a regardé le feu d'artifice le plus impressionnant qu'on a jamais vu, juste au dessus de nos têtes. On a cru plusieurs fois voir le bouquet final tellement ça pétait dans tous les sens, mais ça recommençait encore plus fort, à tel point qu'on s'est même senti "menacés" et qu'on a eu un mouvement de recul. Je vous assure, d'habitude, quand des trucs brillent, explosent, ou brûlent, j'ai plutôt tendance à me sentir attiré...




\begin{figure}[h]
\centering
\includegraphics[height=6cm,width=9cm,keepaspectratio]{p2089738.jpg}
\caption*{C'est beau !}
\end{figure}

Hué est aussi connu pour ses grandes plages, et comme il faisait enfin de nouveau un peu chaud, et qu'on avait une semaine à passer dans le coin, on a pas mal larvé sur le sable, une fois la bonne plage trouvée : la première était plutôt dans un sale état, et on aurait dit qu'un banc de baleines s'était échoué.




\begin{figure}[h]
\centering
\includegraphics[height=6cm,width=9cm,keepaspectratio]{p2079641.jpg}
\caption*{C'est vendeur comme plage, non ?}
\end{figure}


\begin{figure}[h]
\centering
\includegraphics[height=6cm,width=9cm,keepaspectratio]{p2079644.jpg}
\caption*{La plage se fait grignoter.}
\end{figure}

On a pu admirer les pêcheurs locaux en train de manœuvrer leurs grosses bassines qu'ils appellent des bateaux. J'ai du mal à comprendre comment, parmi toutes les formes possibles pour faire un bateau, ils ont choisi la bassine... Mais ça marche, et ils reviennent sur la plage avec quelques poissons, vendus quelques instant après aux vacanciers.




\begin{figure}[h]
\centering
\includegraphics[height=6cm,width=9cm,keepaspectratio]{p2130011.jpg}
\caption*{Je vous jure, ça flotte !}
\end{figure}

Il y a aussi la même vieille qui passe toutes les 20 min pour essayer de nous vendre des bières de sa glacière. A chaque fois, elle s'assoit à un mètre de nous, et nous répète "Beer ? Beer ?" puis repart en maugréant des trucs qui ne nous sonnaient vraiment pas gentil à l'oreille. Et elle nous a fait ça toute la semaine, et a dû nous proposer sa bière une bonne cinquantaine de fois, passant rapidement de la catégorie "je fais mon boulot", à "je suis reloue" !


\begin{figure}[h]
\centering
\includegraphics[height=9cm,width=12cm,keepaspectratio]{p20796391.jpg}
\caption*{Elle par contre, elle était tellement sympa et commerciale, qu'elle a réussi à nous vendre des souvenirs...}
\end{figure}

Par contre, dans la catégorie, "je suis hyper sympa", il y a cette famille (étendue) vietnamienne qui s'était installée pas loin de nous pour pique-niquer. Quand ils nous ont vus, ils ont insisté pour nous inviter à partager leur pique nique. On a bu quantité de bière, et mangé du poisson séché. Comme quoi mon degré de sympathie envers les inconnus dépend fortement du prix de leur bière. Après avoir fait les désormais traditionnels selfies avec tous les membres de la famille, ils ont même tenus à me donner de l'argent ! Je me suis donc retrouvé avec un dollar dans la main ! Visiblement, ça porte bonheur chez eux de donner de l'argent à des étrangers pendant Têt. Mais pourquoi donc avoir choisi un riche occidental ?


\begin{figure}[h]
\centering
\includegraphics[height=9cm,width=12cm,keepaspectratio]{p2079645.jpg}
\caption*{Nouveau sport extrême : la cueillette de noix de coco !}
\end{figure}

En sa baladant dans la rue, soudainement, mon regard est interpellé par une silhouette qui me semble familière. L'intérêt de la silhouette en question semble réciproque, et on reste tous les deux plantés un quart de seconde avant le cerveau accepte la conclusion de l'analyse visuelle après l'avoir re-vérifiée, mais oui, c'est bien Cyril, ami d'enfance, qui se balade dans la rue, et qui n'en revient pas lui non plus de me trouver là ! Quand je pense à toutes les façons possibles qu'on aurait eu de se louper, je n'en reviens pas qu'on ait réussi à boire un coup ensemble à l'improviste à l'autre bout du monde.


\begin{figure}[h]
\centering
\includegraphics[height=9cm,width=9cm,keepaspectratio]{p2079636.jpg}
\caption*{Alsace power !}
\end{figure}

C'est aussi à Hoi An que j'ai fait du scooter pour la première fois de ma vie ! On a commencé par en louer un, car on voulait visiter les ruines de My Son (non, rien à voir avec mon fils, c'est le nom du lieu en vietnamien). Marion a fait les premiers kilomètres puis on a trouvé par hasard un parking abandonné (je dis par hasard pour ne pas dire qu'on s'était planté de rue), ce qui m'a permis d'apprivoiser l'engin à l'abri de la circulation anarchique propre au Vietnam. Et ce mode nous a tellement plu à tout les deux que  depuis ce moment, c'est moi le chauffeur officiel ! Nous avons donc foncé à la vitesse ébouriffante de 35km/h, avec des pointes à 45 par moment, vers le site de My Son. Il y avait des vieux cailloux, des vieilles briques et des grosses araignées : c'était plutôt sympa, mais la découverte de la journée, ça reste quand même le scooter !


\begin{figure}[h]
\centering
\includegraphics[height=6cm,width=9cm,keepaspectratio]{p2099907.jpg}
\caption*{Faut aimer les briques...}
\end{figure}

On avait bien anticipé la semaine de Têt, et pour la première fois depuis le voyage, on avait tout réservé deux semaines en avance. Par contre, on n'avait pas anticipé que les gens allaient aussi repartir à la fin de la semaine... Résultat, tout est complet. Faisant contre mauvaise fortune bon cœur (Oh mince, encore de la plage et des supers bons restaurants...), nous avons rallongé notre séjour et réservé un bus un peu plus tard, pour seulement trois fois plus cher que le reste de l'année. C'était encore un bus de nuit, tellement plein que les gens n'arrivaient même plus à s'allonger dans les couloirs. Il semble que la stratégie ici est de d'abord remplir les vraies places avec des occidentaux qui paient cher, puis remplir les espaces restants avec des locaux pour maximiser la marge. Au petit matin, on a une correspondance, et on monte dans un mini-bus pour la fin du trajet. J'hésite à raconter cette dernière partie car nos mères nous lisent, mais tant pis, je me lance : (mères, sachez que tout s'est bien passé) le chauffeur, pensant sans doute que conduire au Vietnam manquait de piment, s'est mis à regarder un film sur l'écran du tableau de bord ! Les mobylettes n'ont qu'à faire gaffe...


\begin{figure}[h]
\centering
\includegraphics[height=6cm,width=9cm,keepaspectratio]{p2129989.jpg}
\caption*{Et ça, c'est un spectacle de marionnettes dans l'eau. En vietnamien s'il vous plait !}
\end{figure}


\begin{figure}[h]
\centering
\includegraphics[height=6cm,width=12cm,keepaspectratio]{p2120007.jpg}
\caption*{Encore une vendeuse de lanternes.}
\end{figure}



\chapter{Mui Ne, sable rouge et grosses crevettes.}
Nous arrivâmes donc à Mui Ne, une des stations balnéaires les plus connues du Vietnam. Au programme : plage, piscine, fruits de mer, et dunes de sable.


\begin{figure}[h]
\centering
\includegraphics[height=6cm,width=9cm,keepaspectratio]{p2170086.jpg}
\caption*{Je n'avais jamais vu de si gros bernard-l'hermite (oui, c'est invariable).}
\end{figure}


Niveau plage, on est mitigés : la plage a disparu sur la moitié de la côte, le sable a été emporté par les vagues, et il ne reste plus qu'une digue en béton qui protège la rangée d'hôtels. On doit marcher quelques km pour trouver une plage digne de ce nom.


\begin{figure}[h]
\centering
\includegraphics[height=6cm,width=9cm,keepaspectratio]{p2170105.jpg}
\caption*{L'horizon est rempli de voiles.}
\end{figure}

Mais cette plage est le royaume du kite-surf, ce qui implique deux choses. Premièrement, il faut faire attention de ne pas se prendre un kite-surfeur dans les gencives - non qu'ils soient mal-intentionnés, mais une proportion non négligeable des pratiquants est en phase d'apprentissage, et leurs trajectoires peuvent, si le vent est suffisamment blagueur, ponctuellement coïncider avec la trajectoire des gencives d'un baigneur trop confiant. Donc aller dans l'eau, c'est un coup de poker. Deuxièmement, si les kite-surfeurs aiment ce coin, c'est parce qu'il y a un vent très régulier, et fort. Qui soulève le sable. Et le redépose partout, comme par exemple sur une mangue fraichement découpée, ou bien sur l'objectif de l'appareil photo qui se met à faire un son louche ! Bref, rester sur la plage, c'est nul. C'est pourquoi nous avons été ravis de profiter de la magnifique piscine de notre auberge de jeunesse, qui, outre le fait d'être exempte de kite-surfeurs, de sable et de vent, comporte un bar adjacent proposant des mojitos très convenables à 1\euro, sirotables dans une chaise longue ombragée !



\begin{figure}[h]
\centering
\includegraphics[height=6cm,width=9cm,keepaspectratio]{p2170094.jpg}
\caption*{Le retour de la pêche.}
\end{figure}


Pour ce qui est des fruits de mer, on est apparemment dans un des meilleurs endroits du Vietnam. De nombreux restaurants exposent la pêche du jour sur le trottoir, il suffit de pointer les victimes du doigt puis d'aller s'asseoir (en vrai faut aussi dire quelle cuisson, et s'enquérir un minimum du prix). On s'est ainsi fait une langouste entière, mais la révélation, ce furent les petites coquilles saint-jacques à l'échalote, cuites au barbecue ! Des étoiles s'allument dans les yeux de Marion à la simple évocation de ce plat.


\begin{figure}[h]
\centering
\includegraphics[height=6cm,width=9cm,keepaspectratio]{p2170082.jpg}
\caption*{Le tri de la pêche.}
\end{figure}


\begin{figure}[h]
\centering
\includegraphics[height=6cm,width=9cm,keepaspectratio]{p2170067.jpg}
\caption*{Un bol de poulpes pour le petit déjeuner ?}
\end{figure}


\begin{figure}[h]
\centering
\includegraphics[height=9cm,width=12cm,keepaspectratio]{p2170069.jpg}
\caption*{Les langoustes le matin.}
\end{figure}


\begin{figure}[h]
\centering
\includegraphics[height=9cm,width=12cm,keepaspectratio]{p2170409.jpg}
\caption*{Les langoustes le soir.}
\end{figure}

Enfin les dunes de sable, autre attraction fameuse de Mui Ne. Le circuit classique que tous les touristes font commence par le "Fairy stream", ou ruisseau des fées. C'est un ruisseau qui a creusé des paysages similaires au grand canyon dans du sable, et c'est une balade sympathique et rafraichissante de remonter le cours d'eau pieds nus.


\begin{figure}[h]
\centering
\includegraphics[height=9cm,width=12cm,keepaspectratio]{p2160028.jpg}
\caption*{Le ruisseau et ses étranges sculptures.}
\end{figure}

Ensuite on attaque les choses sérieuses : les dunes blanches ! C'est simple, on se croit dans le désert. Sur un ou deux kilomètres, rien d'autre que du sable, des dunes et des dunes de sable. Ça pourrait être un endroit fantastique, propre à l'évasion, à la méditation, face à l'immensité désertique ! Mais certaines personnes ont jugé pertinent de proposer aux touristes la location de quads. On est donc entourés de touristes stupides sur des quads poussifs et bruyants, dont les émissions ajoutent une couche de poussière noire sur le sable blanc. En plus, ces quads ne sont même pas amusants, leur vitesse de pointe plafonne à 10km/h, et ils n'ont que deux roues motrices, donc s'ensablent à la première montée. Il y a sûrement quelque chose qui nous échappe.


\begin{figure}[h]
\centering
\includegraphics[height=9cm,width=12cm,keepaspectratio]{p2160048.jpg}
\caption*{Je vous jure, c'est au Vietnam !}
\end{figure}

La fin de la journée nous amène aux dunes rouges, pour voir le coucher de soleil, et là, avec la couleur du sable accentuée par le soleil couchant, on en prend plein les yeux !


\begin{figure}[h]
\centering
\includegraphics[height=9cm,width=12cm,keepaspectratio]{p2160059.jpg}
\caption*{On en prend plein les yeux : de la lumière, mais aussi du sable !}
\end{figure}


\begin{figure}[h]
\centering
\includegraphics[height=9cm,width=12cm,keepaspectratio]{p2170084.jpg}
\caption*{J'adore cet animal !}
\end{figure}



\chapter{Ho Chi Minh Ville (Saigon), petite pause urbaine.}
Nous arrivâmes donc à Ho Chi Minh Ville, anciennement Saigon. C'est la plus grande ville du Vietnam, et sa capitale économique, et niveau circulation, ça se ressent ! On progresse encore d'un cran dans le chaos, avec des trois voies saturées de motos. On est content d'avoir eu de l'entrainement dans des villes plus faciles, sinon, je pense qu'on aurait du prendre un tuk-tuk pour traverser la rue !


\begin{figure}[h]
\centering
\includegraphics[height=6cm,width=9cm,keepaspectratio]{p2190668.jpg}
\caption*{Ho Chi Minh Ville de nuit.}
\end{figure}

Pour ce séjour, on fait une pause dans le tourisme/voyage. Tout d'abord, on va se faire un cinéma ! On était très frustrés d'avoir loupé Star Wars à cause des décalages de dates de sorties (on était trop tôt en Chine, puis trop tard au Vietnam), et on s'est vengé sur Zootopia que l'on vous recommande chaudement. Ce fut la première fois qu'on regardait un film en Anglais sous-titré vietnamien, et étant donné qu'on riait tous les deux de concert avec le reste de la salle, on se dit que pas grand chose du film n'a du nous échapper.




\begin{figure}[h]
\centering
\includegraphics[height=6cm,width=9cm,keepaspectratio]{p2190683.jpg}
\caption*{Est-ce que vous savez pourquoi les crocodiles ont le ventre bleu ? Une carte postale au plus rapide !}
\end{figure}

On a aussi fait un peu de ménage dans les affaires : les vêtements chauds sont dans un paquet pour la France, et on s'est équipé un peu mieux pour le chaud, et Marion s'est fait faire une nouvelle paire de lunettes de soleil à sa vue : arrivé chez l'opticien le matin à 9h avec l'ordonnance, quelques minutes pour choisir la monture, et on a récupéré les lunettes finies le soir à 8h. Le tout pour la somme modique de 26\euro. Va falloir que j'en parle à mon opticien de frangin pour comprendre...




\begin{figure}[h]
\centering
\includegraphics[height=9cm,width=12cm,keepaspectratio]{p2190672.jpg}
\caption*{Une fontaine pile au moment où la lumière change de couleur.}
\end{figure}

Je ne sais pas si on vous avait déjà parlé du café Vietnamien, mais dans le doute, je vais en remettre une couche : c'est le meilleur café du monde (selon moi). Il a beaucoup de gout, mais sans être amer, et, plutôt que de sentir le brulé, a un gout très rond. Bref, il est bon. Et pour ne rien gâcher, c'est facile à faire : il faut juste avoir le petit filtre métallique à poser sur la tasse, et faire un petit café filtre individuel. Libre ensuite d'ajouter de la glace et/ou du lait concentré sucré, toutes les combinaisons sont bonnes ! Histoire de pousser l'expérience du bon café vietnamien jusqu'au bout, nous sommes allés dans le meilleur café de la ville, goûter le meilleur café de la carte, fait avec le café le plus cher du monde (je commence à trouver pénible le fait qu'en français, café puisse signifier "le lieu", "les grains", "la poudre" ou "la boisson"). J'ai nommé : le "café belette".

Petite leçon d'histoire coloniale : des colons français ont apporté la culture du café au Vietnam, mais évidemment, le café avait beaucoup trop de valeur pour qu'on laisse les locaux en consommer, il était réservé à l'exportation. Les locaux ont alors remarqué qu'une sorte de belette mangeait les fruits, dont les graines étaient ensuite retrouvées intactes dans ... bon, je ne vais pas vous faire un dessin ! Il faillait bien laver les grains évidemment, mais ça marchait ! Et ça marchait même drôlement bien, les sucs digestifs de la belette font une torréfaction beaucoup plus douce, et ajoutent même des arômes de vanille. Évidemment, on est rapidement passé de "je cherche des crottes dans la forêt" à "j'élève mes belettes", puis à "je synthétise les sucs digestifs de la belette" et enfin pour certains audacieux "je colle une photo de belette sur mon pot de café à 3\euro le kg, ces c**s de touristes n'y verront que du feu". En effet, si d'un coté vous avez des vendeurs qui vendent le café à ce prix, et que de l'autre une tasse du supposé même café est à 8\euro, il y a forcément une arnaque quelque part. Nous, on a fait le pari que l'arnaque était du coté café pas cher, et on est allé claquer le prix de deux repas dans une tasse de café belette, accompagnée, à titre de café témoin, par une tasse de bon café. Mes papilles en tremblent encore...


\begin{figure}[h]
\centering
\includegraphics[height=6cm,width=9cm,keepaspectratio]{p2190410.jpg}
\caption*{On a religieusement suivi les conseils du guide en allant visiter la poste. Voilà, c'est fait. Bon, ben on va rentrer !}
\end{figure}

Il y a quelques buildings qui poussent à Saigon, et ils sont idéaux pour admirer un coucher de soleil. En faisant un peu de recherche, on se rend compte que c'est moins cher d'aller boire une bière au sommet du Sheraton (*****) que de payer un billet pour accéder à la plateforme d'observation à peine plus haute de l'immeuble d'à coté. Tant qu'à faire, autant boire de la bière ! Nous voici donc dans le bar du Sheraton, et deuxième bonne nouvelle : en happy hour, pour un cocktail acheté, le deuxième à 1\$ ! Dans ce cas, autant boire des mojitos plutôt que de la bière pas bonne ! Le bar est quasiment vide, on a une table face au soleil couchant, et le cocktail est servi format saladier. Tout est parfait ! Le soleil fini par se coucher, et on se dit qu'on va faire de même. L'addition arrive. Ah, petit problème ! L'offre happy hour n'a pas été prise en compte, ils veulent nous faire payer les deux boissons au plein tarif. On appelle la serveuse, puis son manager, qui nous explique que l'offre est valable "par personne", que c'est uniquement la deuxième boisson d'un seul consommateur qui est à 1\$. Si on veut profiter de l'offre, libre à nous de commander chacun une deuxième boisson. L'offre étant ambiguë, ce n'était pas ce qu'on avait compris, et étant de bonne fois, dans un établissement de ce standing, alors que nous sommes les seuls consommateurs, on se dit qu'ils vont faire un geste ! Eh bien non, le manager ne voulait rien savoir. Toujours en gardant les formes, en s'excusant à chaque phrase, le mieux qu'il nous offrait était un bon pour deux consos à 1\$ à utiliser un autre jour, mais il voulait quand même qu'on paye les deux premières conso (22\$ quand même). Et il me disait qu'il était coincé, qu'ils avaient ouvert une table pour deux clients sur l'ordinateur, et que le logiciel n'autorisait pas cette promo dans ces conditions. Coincé par un ordinateur. Me dire ça... à moi... Après 20mn de pourparlers, et seulement après que j'ai demandé à parler à son supérieur, il a enfin consenti à un geste. Mais cette lutte de marchand de tapis nous a semblé malvenue ici, même les bouibouis dans la rue proposent un meilleur service !



\begin{figure}[H]
\centering
\includegraphics[height=6cm,width=9cm,keepaspectratio]{p2190680.jpg}
\caption*{Je ne sais pas ce que c'est, mais c'est joli !}
\end{figure}


Il y a quand même un endroit de Ho Chi Minh Ville qu'on était obligé de visiter en tant que touriste : le musée de la guerre. Il est rempli de photos et d'histoires qui décrivent les horreurs de la guerre du Vietnam. Et les horreurs n'ont pas seulement eu lieu pendant la guerre, mai continuent maintenant encore : le pays est encore loin d'être déminé (vous saviez que les Américains avaient plus balancé de bombes sur le Vietnam que l'ensemble des bombes balancées pendant toute la seconde guerre mondiale par l'ensemble des belligérants ?), et l'agent orange est encore présent dans l'eau, et continue de provoquer des malformations chez les nouveaux-nés, 40 ans plus tard...


\chapter{Vinh Long, le calme du delta du Mékong.}
Nous arrivâmes donc à Vinh Long, plus précisément sur l'île de An Binh. Après avoir pris trois bus successifs puis un ferry, et enfin une moto qui a suivi une route, puis un chemin bétonné, puis un chemin en gravier, nous arrivâmes à notre guesthouse. On peut dire qu'elle était un rien isolée : au milieu d'une île du détroit du Mékong, perdue entre une plantation de bananier et un petit canal de navigation. Et pourtant, dans la guesthouse, que des Français... Et que font des Français qui se rencontrent à l'étranger ? Eh bien ils parlent de bouffe ! Donc on a passé les quelques jours sur place à se raconter à quel point le fromage nous manquait...


\begin{figure}[h]
\centering
\includegraphics[height=6cm,width=9cm,keepaspectratio]{p2230709.jpg}
\caption*{Coucher de soleil sur le Mékong.}
\end{figure}




\begin{figure}[h]
\centering
\includegraphics[height=6cm,width=9cm,keepaspectratio]{p2230729.jpg}
\caption*{Bon appétit !}
\end{figure}

Le delta du Mékong est énorme : le fleuve se divise en 9 bras, chacun bien plus gros que le Rhône. C'est un endroit parfait pour cultiver du riz : avec 3 récoltes par an, la zone produit plus de riz que le Japon et la Corée réunis, ce qui explique pourquoi on a vu autant de bateaux remplis de riz pendant le tour en barque qu'on a fait sur le Mékong. Mais cette zone est en danger : les différents barrages en amont du fleuve régulent le débit et limitent les inondations nécessaires pour fertiliser les terres. Les terres qui ne sont plus inondées deviennent impropre à la culture. L'autre danger, est la baisse du débit : la Chine, puis le Laos, la Thaïlande et le Cambodge puisent dedans pour l'irrigation et l'industrie, ce qui laisse l'eau de mer remonter plus haut dans le delta, salinisant les terres par la même occasion. Nous étions à plus de 100km de la cote, et la marée faisait quand même varier le niveau d'eau de presque deux mètres !


\begin{figure}[h]
\centering
\includegraphics[height=6cm,width=9cm,keepaspectratio]{p2230722.jpg}
\caption*{Une cargaison de riz.}
\end{figure}

Le tour en barque, c'est le truc à faire dans le coin, et ils vendent ça surtout comme une visite au marché flottant. Ah, le fameux marché flottant, tellement typique avec ses gens qui se baladent de barque en barque, tellement à l'aise qu'ils nous font oublier que le sol ondule sous leurs pas... et bien depuis l'arrivée de la route, ce marché a été réduit à la portion congrue. Il ne reste que quelques bateaux qui font de la vente en gros de tubercules, le reste du marché flottant est maintenant bien ancré sur la terre ferme. C'est principalement un marché alimentaire, et on y voit des choses vraiment étranges, voire choquantes pour un occidental au cœur sensible : un premier stand de grenouilles les vendait entières, mortes et dépecées. Contrairement aux Français, ici, ils mangent les grenouilles entières et pas uniquement les cuisses. Un deuxième stand les vendait vivantes, intactes, attachées par paquets de 5. Vous sentez venir le troisième stand ? Je conseille aux âmes sensibles de passer directement au paragraphe suivant, et de ne pas essayer d'analyser la photo qui suit. Vous êtes sûrs de continuer ? Bon, vous ne pourrez pas dire que je ne vous ai pas prévenus... Le troisième stand, donc, vend les grenouilles vivantes, mais dépecées, sans la peau. Elles sont dans une grande assiette métallique, baignant dans leur sang, et essaient de temps en temps de sauter par dessus le rebord de 3 cm, mais leurs moignons glissent, elles retombent dans l'assiette, et restent là, la bouche ouverte dans un grand cri silencieux.


\begin{figure}[h]
\centering
\includegraphics[height=6cm,width=9cm,keepaspectratio]{p2230758.jpg}
\caption*{...}
\end{figure}

Et nous continuons cette visite, et cet article, sur une note plus joyeuse avec la visite d'une manufacture de bonbons !  On voit la fabrication de caramel à la noix de coco, de feuilles de riz, et une démonstration de popcorn de riz, ou poprice, avant de passer à la dégustation de diverses infusions d'alcool de riz. D'après le guide, le classique alcool fermenté au cobra/scorpion est un puissant aphrodisiaque, à réserver aux hommes donc, parce que sinon, oulala, je ne vous raconte pas... Pour les femmes, le guide recommande l'alcool gout banane, idéal pour bien dormir. Marion fait alors justement remarquer que c'est quand même stupide de vouloir faire dormir sa femme à ce moment là, c'est un coup à dormir sur la béquille... Ils n'ont rien compris ces Vietnamiens !


\begin{figure}[h]
\centering
\includegraphics[height=5.5cm,width=9cm,keepaspectratio]{p2230830.jpg}
\caption*{Ça fait quand même moins peur sans les dents, et dans une bouteille, et mort...}
\end{figure}

Après ces derniers jours au calme, c'est le moment pour nous de dire au revoir au Vietnam, et nous allons passer une dernière nuit à la ville frontière de Chau Doc. Pas grand chose à raconter, mis à part les 100 moustiques qu'on a tué dans la chambre, avant de monter la moustiquaire de la tente sur le lit. Le lendemain, passage de frontière en bateau, et rien, aucune tentative de corruption, malgré tous les avertissements qu'on avait lus à propos des compagnies de transports et des douaniers, personne ne nous a demandé plus d'argent que ce qui était prévu, et c'est sur ces bonnes impressions que nous entrons au Cambodge !


\begin{figure}[h]
\centering
\includegraphics[height=6cm,width=9cm,keepaspectratio]{p2250901.jpg}
\caption*{La brigade fluviale est impressionnante !}
\end{figure}


\begin{figure}[h]
\centering
\includegraphics[height=6cm,width=9cm,keepaspectratio]{p2210699.jpg}
\caption*{Ces petites filles ont tenu à jouer les modèles pour nous.}
\end{figure}



\chapter{Phnom Penh, chute et reconstruction du Cambodge.}
Nous arrivâmes donc à Phnom Penh, la capitale du Cambodge. Et on progresse encore d'un cran dans l'anarchie circulatoire : les scooter y roulent aussi mal qu'au Vietnam, mais en plus, il y a masse de 4x4, ce qui est nettement plus compliqué à esquiver. Les rares trottoirs sont utilisés pour le commerce ou le stationnement. Hormis dans quelques rues touristiques, c'est extrêmement désagréable, voire dangereux de s'y promener. D'ailleurs, personne ne le fait, mis à part les mendiants et quelques touristes. Il est inconcevable pour un Cambodgien d'aller quelque part à pied, il perdrait la face, et il se déplace donc dans le plus gros véhicule qu'il peut se permettre. Avec la progression du tourisme, et le développement urbain qui l'accompagne, de nombreux paysans se sont retrouvés très riches quand leur rizières ont été déclarées constructibles. Et que fait un Cambodgien riche à votre avis ? D'après des sources bien informées, il se contente d'acheter un plus gros 4x4 que son voisin.


\begin{figure}[h]
\centering
\includegraphics[height=6cm,width=9cm,keepaspectratio]{p2261229.jpg}
\caption*{L'entrée du Palais Royal.}
\end{figure}




\begin{figure}[h]
\centering
\includegraphics[height=6cm,width=9cm,keepaspectratio]{p2261226.jpg}
\caption*{Le palais est plein de moines. Le plus dur, c'est d'en trouver qui ne sont pas entourés de touristes !}
\end{figure}

Nous visitons le palais royal, en fait un parc rempli de palais, de pagodes, et de temples de toute tailles. Le clou du spectacle est la pagode d'argent. Elle est appelée ainsi car son sol est recouvert de dalles en argent, il y en a pour plusieurs tonnes, une vraie fortune. Comme dans toutes les pagodes, il faut enlever ses chaussures avant d'entrer, et c'est pieds nus que nous entrons. Et nous voyons... des tapis ! Plein de tapis qui recouvrent la quasi intégralité du sol. Seul un petit coin protégé par des barrières est visible. On voit donc quelques dalles ternes, dont les joints ont été protégés il y a déjà quelque temps par du gros scotch marron qui part à présent en lambeaux...


\begin{figure}[h]
\centering
\includegraphics[height=6cm,width=9cm,keepaspectratio]{p2261242.jpg}
\caption*{Heureusement, l'extérieur du palais est bien plus beau !}
\end{figure}

L'autre endroit à visiter, c'est la prison S21. C'est une ancienne école que les Khmers Rouges ont, comme beaucoup d'autres écoles, transformé en prison pour les opposants au régime. Et cette prison était leur prison VIP, là où ils envoyaient leurs ennemis les plus puissants, les plus éduqués. Nous avons passé 4h ici, à apprendre l'histoire de cet endroit et du régime Khmer Rouge, à faire connaissance avec les victimes et les bourreaux (rôles qui se confondent parfois), à visiter les cellules et voir les instruments de torture, et finir par voir les restes des victimes dont les ossements portent des stigmates visibles. On apprend comment le directeur de la prison, un professeur de mathématique adoré de ses élève, a pris des jeunes des campagnes pour les former correctement à la torture. Il n'était pas question de torturer pour le plaisir, il fallait soutirer des aveux : si "l'organisation" arrêtait quelqu'un, c'est qu'il était coupable et il ne restait plus qu'à trouver de quoi. On a ainsi découvert qu'un grand nombre de Cambodgiens étaient des espions de la CIA. Si un bourreau faisait mourir un prisonnier avant qu'il ait signé ses aveux, il risquait fortement de passer lui aussi à la question, donc tout était mis en place pour garder les prisonniers en "bonne santé" le plus longtemps possible. Le but des Khmers Rouges était de créer une utopie communiste basée sur une société agricole. Quelques jours après leur victoire, ils ont vidé toutes les villes, et envoyé leurs habitants travailler dans les champs. Ils ont aboli la monnaie, la propriété et l'éducation. Ils se sont débarrassés des intellectuels, notions très vaste pour eux : avoir des lunettes, un stylo, savoir faire du vélo, savoir écrire son nom, ou avoir un parent dans ce cas était un motif amplement suffisant. En quelques année, ils ont détruit toute l'élite du pays, et toute sa culture. Le pays panse encore aujourd'hui ses blessures.


\begin{figure}[h]
\centering
\includegraphics[height=6cm,width=9cm,keepaspectratio]{p2271252.jpg}
\caption*{Les barbelés, c'est pour empêcher les prisonniers de se suicider.}
\end{figure}

Et puis, d'un coup, comme ça, pouf, sans prévenir, une envie subite de vélo nous a saisi ! Et nous avons acheté des vélos. De jolis vélos chinois, bien adaptés à la ville, pour une somme très raisonnable. Vous devinez la suite ? Oui ? Non ? Quoi qu'il en soit, ce sera dans le prochain article !


\begin{figure}[h]
\centering
\includegraphics[height=5.5cm,width=9cm,keepaspectratio]{p2271253.jpg}
\caption*{Mais qu'allons nous donc bien pouvoir faire avec ces bolides ?}
\end{figure}


\begin{figure}[h]
\centering
\includegraphics[height=5.5cm,width=12cm,keepaspectratio]{p2250906.jpg}
\caption*{Les moines sont censés renoncer à leurs possessions, mais le téléphone, c'est dur...}
\end{figure}


\begin{figure}[h]
\centering
\includegraphics[height=5.5cm,width=12cm,keepaspectratio]{p2261245.jpg}
\caption*{De l’encens.}
\end{figure}



\chapter{A vélo vers les temples d'Angkor.}
Et nous partîmes donc en direction de Siem Reap. A vélo ! 350km sur un vélo de ville chinois, avec des températures prévues entre 28\textdegree C et 40\textdegree C, ça vend du rêve, non ?


\begin{wrapfigure}{l}{0.65\textwidth}
\centering
\includegraphics[width=0.6\textwidth]{p3021286.jpg}
\caption*{Lever de soleil sur la route.}
\end{wrapfigure}

Mais qu'est-ce qui a bien pu se passer dans nos têtes pour qu'on en arrive là ? En fait, depuis quelques semaines déjà, on commençait à trouver qu'on s'était installé dans une sorte de routine. Oui, je sais, c'est bizarre de parler de routine quand on change d'endroit en permanence. Mais la routine, c'était trouver notre prochaine étape, puis trouver un moyen de transport, puis trouver un logement. Toujours les mêmes sites d'information sur le web, toujours les mêmes bus, les mêmes guesthouses... Avec Internet, toutes ces étapes sont faciles à faire, toutes les informations disponibles en avance, et tout est prévu pour le touriste. Sérieusement, c'est très (trop?) facile de voyager dans ces conditions. Alors on a eu envie de quelque chose de différent, et vu que le pays est plat, pourquoi ne pas faire du vélo ? En plus, les travaux de la route entre Phnom Penh et Siem Reap sont quasiment finis, et la route est comme neuve !


\begin{figure}[h]
\centering
\includegraphics[height=5.5cm,width=9cm,keepaspectratio]{p3011268.jpg}
\caption*{Nos bolides !}
\end{figure}

N'empêche que ça nous a fait un peu sortir de notre zone de confort : les jours précédents, nous étions tour à tour un peu tendus à l'idée du trip qui nous attendait, ce qui ne nous était plus arrivé depuis un petit bout de temps.


\begin{figure}[h]
\centering
\includegraphics[height=5.5cm,width=9cm,keepaspectratio]{p3021282.jpg}
\caption*{Parfois, c'est de la piste !}
\end{figure}

Le matin du départ, on est sur le pied de guerre à 5h. Le temps de ranger nos affaires, de réveiller le veilleur de l'hôtel, puis de trouver comment sangler le sac à dos sur le porte bagage, il est 6h au moment où on commence à pédaler, pile aux premières lueurs de l'aube. Vu la chaleur qu'il fait en ce moment, on s'est dit qu'il fallait vraiment qu'on profite au maximum de la fraicheur du matin.


\begin{figure}[h]
\centering
\includegraphics[height=5.5cm,width=9cm,keepaspectratio]{p3021288.jpg}
\caption*{Quand on sort de la route principale, on se perd...}
\end{figure}

Et ce fut génial. En quelques minutes, on est déjà dans un endroit où les seuls touristes habituellement visibles le sont subrepticement à travers la vitre d'un bus. Alors quand les gens voient deux couillons tout blanc rouge passer à 15km/h, ça interpelle ! On va passer tout le trajet à répondre aux "hello" tous les 50m. Ce n'est pas grand chose, mais tous ces gens qui nous saluent, ça encourage sérieusement ! Résultat, on a la banane. Et c'est encore plus marrant pendant les entrées et sorties des écoles : on se retrouve à pédaler au milieu de centaines d'enfants à vélo, qui ont un moment de doute quand ils réalisent que, par le plus grand des hasards, je suis habillé comme eux : pantalon noir et chemise blanche.


\begin{figure}[h]
\centering
\includegraphics[height=5.5cm,width=9cm,keepaspectratio]{p3011270.jpg}
\caption*{C'est quand j'ai crevé qu'on a compris que les gens ne mettaient pas des piles de pneus sur le bord de la route pour faire joli, mais pour indiquer qu'ils réparaient les pneus crevés !}
\end{figure}

On trouve des petits stands de nourriture un peu partout, les noix de coco et les jus de canne à sucre nous aident à combattre la déshydratation. Il faut dire que le soleil n'est pas tendre. A partir de 9h du matin, il fait déjà trop chaud (genre, 35\textdegree C à l'ombre), alors à midi, je vous laisse imaginer... Donc l'après-midi, c'est sieste dans la première guesthouse qu'on trouve. La réaction des tenanciers  est généralement une surprise mêlée d'incrédulité : ils nous demandent si on travaille dans le coin, semblent parfois douter qu'on veuille vraiment dormir ici, et il y en a même un qui a cru comprendre qu'on voulait la chambre à l'heure... Mais ils se détendent rapidement une fois qu'on leur montre nos vélos.


\begin{figure}[h]
\centering
\includegraphics[height=5.5cm,width=9cm,keepaspectratio]{p2291262.jpg}
\caption*{Hello, hellooo, HELLOOOOO !}
\end{figure}

Pour ne pas perdre de temps et profiter au maximum du frais du matin, on se contente de quelques petites bananes au petit déjeuner. Le midi et le soir, on mange dans toutes sortes de cantines locales : habituellement, ce sont quelques casseroles de plats mijotés au choix, et du riz à volonté. On a même mangé un excellent poulet aux fourmis (bien que la cuisinière ait tenté de secouer le poulet) et Marion vous parlera encore des courgettes à la viande hachée. Même en sachant que la faim est la meilleure des épices, on a très bien mangé tout le long du trajet !


\begin{figure}[h]
\centering
\includegraphics[height=5.5cm,width=9cm,keepaspectratio]{p30212831.jpg}
\caption*{La photo est trompeuse, ces vaches sont très lentes...}
\end{figure}

Sur le trajet, il y a quand même quelques trucs à voir. Le premier fut une montagne avec un temple envahi de singes. Et oui, dans un pays plat comme le Cambodge, un caillou de 50m de haut s'appelle une montagne !


\begin{figure}[h]
\centering
\includegraphics[height=5.5cm,width=9cm,keepaspectratio]{p3021341.jpg}
\caption*{Oui, appuie toi sur la tête, l'autre main sur la hanche, et maintenant, fais moi ton regard langoureux... parfait, c'est dans la boite !}
\end{figure}


\begin{figure}[h]
\centering
\includegraphics[height=5.5cm,width=9cm,keepaspectratio]{p3021354.jpg}
\caption*{Il y a aussi des femmes-moines.}
\end{figure}


\begin{figure}[h]
\centering
\includegraphics[height=5.5cm,width=9cm,keepaspectratio]{p3021362.jpg}
\caption*{Si seulement les humains savaient poser aussi bien que ces singes...}
\end{figure}

Après quelques aller-retour, on fini aussi par trouver la ferme de ver à soie dont tout le monde parle, mais que personne n'est capable de placer sur une carte. C'est en fait un atelier de tissage plus qu'une ferme. Un vétéran du Vietnam, ancien employé d'ONG, s'est mis à son compte pour créer cet atelier. Il a formé des dizaines de jeunes femmes au tissage, et vend leur production aux touristes. On peut voir les jeunes femmes travailler, et se faire prendre en photo avec une de leur production. La partie "ferme" est toute petite. C'est juste pour la démonstration : on voit quelques vers et quelques cocons, et on a une démo de tout le processus, depuis l’œuf, jusqu'au sacrifice de la larve et au filage des cocons. La soie est en fait importée de Chine, ici, il n'y aurait pas assez d'eau pour faire pousser les muriers qui nourrissent les vers.


\begin{figure}[h]
\centering
\includegraphics[height=5.5cm,width=9cm,keepaspectratio]{p3021302.jpg}
\caption*{Le fondateur, Bud.}
\end{figure}


\begin{figure}[h]
\centering
\includegraphics[height=5.5cm,width=9cm,keepaspectratio]{p3021365.jpg}
\caption*{Sachant qu'il faut au moins 20 cocons pour faire un fil, au moins ouate-mille fils pour une robe,  et qu'un ver à soie mange 1,345 feuilles par jour, combien de kilos de murier faut-il pour avoir la classe en soirée ?}
\end{figure}

Enfin, histoire de se mettre en jambes pour tous les temples qui nous attendent à Angkor, on visite le site de Sambor Prey Kuk, un des sites de temples pré-angkoriens les plus connus. Nous improvisons un groupe avec Laura et Woussi, rencontrés sur le site, et nous embauchons un guide. En plus de nous fournir quelques données historiques, il évite aussi qu'on se perdre : dans une forêt toute plate, remplie de ruines toutes semblables, c'est vite arrivé ! On voit plein de petits temples en brique, certains bâtiments sont envahis par les arbres, et d'autres par des ... cobras ! Non sérieusement, le guide nous a fortement déconseillé de rentrer dans certains endroits. Vous imaginez bien qu'on est allé voir !


Au détour d'un sentier, on remarque un tronc calciné. Et d'ailleurs, à y regarder de plus près, c'est un tronc non seulement calciné, mais aussi calcinant ! Il fume et il y a des braises ! Et on est au milieu de la saison sèche, dans une forêt pleine de feuilles mortes ! La conversation qui suit est surréaliste, mais néanmoins véridique :

\emph{Nous, inquiets :} "Euh, il y a un arbre qui brûle là !"

\emph{Le guide, calme :} "Ah oui, j'ai vu, les habitants font parfois ça pour déloger des oiseaux dans le tronc."

\emph{Nous, anxieux :} "Mais il faut faire quelque chose !"

\emph{Le guide, mou :} "Oui, il faudrait."

\emph{Nous, limite paniqués :} "La forêt risque de prendre feu !"

\emph{Le guide, amorphe :} "Oui, ça arrive de temps en temps."

\emph{Nous, incrédules :} "Alors... on appelle les pompiers ? Quelqu'un ?"

\emph{Le guide, fatigué :} "Oh bah, il y a bien quelqu'un qui va le faire."

\emph{Nous, résignés :} "C'est pas nous qui travaillons ici, dans une heure, on sera loin..."

\emph{Le guide, plein d'entrain :} "On continue la visite ?"


Je dois aussi vous raconter ce qui peut arriver quand on ne prend pas soin de ses affaires : le matin de notre journée de pause pédalage au milieu du trajet, Marion, prête avant moi comme souvent, part prendre des photos de l'aube. Elle revient avec un masque soucieux sur le visage, et me dit d'une petite voix que la mise au point ne marche plus, tout est flou, et il y a une barre noire dans un coin. Un rapide coup d’œil à l'appareil me permet de constater que l'angle de l'objectif par rapport au boitier est passé d'un standard 90\textdegree  à un inquiétant 75\textdegree /80\textdegree ... C'est pas un angle naturel, et ça ne s'était pas fait sans dégâts :


\begin{figure}[h]
\centering
\includegraphics[height=5cm,width=9cm,keepaspectratio]{p3031370.jpg}
\caption*{Je ne sais pas vous, mais moi, ça me brise le cœur.}
\end{figure}

Pourtant, impossible de se souvenir d'un quelconque évènement traumatisant, ni choc ni chute. Juste le souvenir d'avoir peut-être bourré une bouteille d'eau un peu fort dans le sac à dos la veille au soir. Quoi qu'il en soit, le résultat est là : un objectif inutilisable. Je monte l'objectif à portrait pour tester le boitier, et au moins ça, ça marche. Et on part pour la journée en essayant de penser à autre chose. Mais c'est dur de penser à autre chose quand on a l'appareil avec pas le bon objectif toute la journée dans la main...


\begin{figure}[h]
\centering
\includegraphics[height=5cm,width=9cm,keepaspectratio]{p3031401.jpg}
\caption*{Ne pas trembler...}
\end{figure}

Doutant de pouvoir trouver un réparateur d'objectif, n'ayant plus rien à perdre, on achète un set de mini-tournevis. Je vais tenter la réparation moi-même ! On prépare l'atelier dans la chambre d’hôtel : ventilateur coupé pour éviter de faire circuler la poussière, mains propres, masque anti-postillons, outils prêts. Je commence par dévisser proprement la monture qui a été arrachée, et là, c'est le drame : je vois qu'une nappe électronique a été déchirée, et ce n'est pas avec un mini tournevis que je vais pouvoir réparer ça... Sans plus y croire, je rebranche quand même ce qu'il reste de la nappe, remets les 25 rondelles un peu au hasard, et j'arrive à refermer tout ça proprement. Je remonte l'objectif, et là, il y a ça qui se passe :


\begin{figure}[h]
\centering
\includegraphics[height=5cm,width=9cm,keepaspectratio]{p3031412.jpg}
\caption*{Rare exemplaire de Marion ne souriant pas, ce qui montre le niveau de tension à ce moment là. Mais le fait est que l'objectif remarche !}
\end{figure}

Ça marche. Contre toute attente, alors j'étais prêt à parier que non, l'objectif marche ! Il est toujours un peu tordu (genre 89.5\textdegree ) car j'ai remis les rondelles n'importe comment, problème que je corrigerai un peu plus tard, mais l'autofocus, l'ouverture, tout marche ! C'est un miracle ! On va pouvoir prendre des photos !

Temples d'Angkor, nous voici !


\begin{figure}[h]
\centering
\includegraphics[height=5cm,width=9cm,keepaspectratio]{p3031425.jpg}
\caption*{Ça marche !}
\end{figure}



\chapter{Siem Reap et les temples d'Angkor.}
Nous arrivâmes donc à Siem Reap, après une semaine de vélo. C'était un samedi matin, et après quelques jours sur une route plutôt calme, on retrouve l'agitation d'une grande ville. On se pose dans un bar avec du wifi, et on décide de se récompenser de cette semaine d'efforts en réservant un chouette hôtel, avec piscine, et bien oui,  parce que merde quoi !


\begin{figure}[h]
\centering
\includegraphics[height=6cm,width=9cm,keepaspectratio]{p3061684.jpg}
\caption*{Angkor Vat, le symbole du Cambodge.}
\end{figure}

Les temples d'Angkor, c'est, comme on va le découvrir bientôt, bien plus que l'Angkor Vat. Il y a des centaines de monuments, répartis autour de Siem Reap. Plusieurs jours ne sont pas de trop pour visiter tout ça, et ça tombe bien, ils vendent des tickets de plusieurs jours, et confort ultime : les jours ne sont pas à utiliser d'affilé, on peut par exemple répartir 3 jours sur une semaine, ce qui est bien pour nous, car on a une piscine à rentabiliser !


\begin{figure}[h]
\centering
\includegraphics[height=6cm,width=9cm,keepaspectratio]{p3061694.jpg}
\caption*{Le dernier étage d'Angkor Vat.}
\end{figure}

Un bonheur n'arrivant jamais seul, un couple de français nous aborde dans la rue, et nous demandent : vous voulez un livre sur les temples ? Et nous tendent le livre dont je parlais à Marion le matin même. Et ils s'en vont, nous laissant à peine le temps de leur dire merci. Nous sommes donc fin prêt pour aller nous perdre dans ces vieux cailloux !


\begin{figure}[h]
\centering
\includegraphics[height=6cm,width=9cm,keepaspectratio]{p3061431.jpg}
\caption*{Lever de soleil. Ça ne valait vraiment pas le coup de mettre le réveil...}
\end{figure}

On commence par le plus connu : Angkor Vat, tôt le matin, car le lever de soleil est fameux. On est toujours sur le rythme du vélo, donc pas de souci pour se lever à 5h. A 6h, on est avec la foule, devant le temple le plus connu du Cambodge, et ... rien. Il y a du brouillard, on ne voit pas le soleil, les couleurs restent blafardes, aucun intérêt... On visite le temple jusqu'à ce que la chaleur devienne insupportable, on rentre se rafraichir dans la piscine, et on ressort le soir pour aller voir le tout aussi fameux coucher de soleil. Mais le temple ayant le meilleur point de vue est limité en places, et évidemment, on le découvre seulement une fois au sommet de la colline où se trouve le temple. Et la queue est longue, et les gens qu'on voit au sommet du temple donnent vraiment l'impression de ne pas bouger, les places ne risquent pas de se libérer. On est déçus dans un premier temps, mais quand le soleil disparait dans la brume comme le matin, on se dit qu'on n'a pas loupé grand chose.


\begin{figure}[h]
\centering
\includegraphics[height=5.5cm,width=9cm,keepaspectratio]{p3061689.jpg}
\caption*{Le temple n'a jamais cessé d'être en activité.}
\end{figure}

Et ce sera ça toute la semaine qu'on va passer à Siem Reap :  pas de lever ni de coucher de soleil, mais un soleil de plomb le reste du temps. Je n'ai pas pu faire les photos dont je rêvais, tant pis...

Malgré cette (petite) déception niveau éclairage, il faut admettre que les temples sont vraiment impressionnants. Et j'avoue, là, je ne sais pas par où commencer pour raconter tout ce qu'on a vu. Des gens très compétents ont passé des années à étudier ces ruines, à écrire des gros bouquins détaillants le moindre bas-relief (et il y a des kilomètres de bas reliefs !), à compiler des listes de noms de rois terminant en -avarman, et à tenter de retrouver qui a construit quoi.


\begin{wrapfigure}{l}{0.45\textwidth}
\centering
\includegraphics[width=0.4\textwidth]{p3081883.jpg}
\caption*{Les restaurateurs ont décidé de laisser certains temples dans un état de restauration partielle. C'est probablement une des meilleure idées qu'ils aient jamais eu !}
\end{wrapfigure}

Comme à beaucoup d'autres endroits dans le monde, ce fut la course à qui a la plus grosse. Et ici, c'est la plus grosse citadelle. Chaque roi, ou presque, a donc construit une ville plus grande que ses prédécesseurs : une grande muraille carrée en pierre pour protéger la ville, puis une plus petite muraille pour protéger le temple, et enfin le temple. Seuls ces éléments ont survécus, car ce sont les seuls construits en pierre : Même les palais royaux étaient en bois. Les temples qui restent, toujours au centre de la ville, étaient considérés comme de véritable demeures divines. Ce ne sont pas des lieux où les fidèles se rassemblent comme nos églises, seuls quelques moines peuvent y entrer. Pas besoin de faire des grandes salles, les temples sont très grands, mais il n'y a que des petites salles et des couloirs étroits. Ils n'ont jamais utilisé la voûte, et on peut se poser la question : Ils ne connaissaient pas la voûte, donc ils n'ont jamais pu construire de grandes salles, ou bien, ils n'ont jamais voulu construire de grande salle, donc ils n'ont jamais eu besoin de voûtes ?

Bref, trêve de bavardages inutiles, place aux photos !


\begin{figure}[h]
\centering
\includegraphics[height=9cm,width=12cm,keepaspectratio]{p3081868.jpg}
\caption*{Ce temple a servi de décor à Tomb Raider.}
\end{figure}



\begin{figure}[h]
\centering
\includegraphics[height=9cm,width=12cm,keepaspectratio]{p3081902.jpg}
\caption*{Certains murs ne tiendraient plus sans les arbres.}
\end{figure}


\begin{figure}[h]
\centering
\includegraphics[height=8cm,width=12cm,keepaspectratio]{p3081971.jpg}
\caption*{Le temple du Bayon, couvert de visages géants.}
\end{figure}


\begin{figure}[h]
\centering
\includegraphics[height=9cm,width=9cm,keepaspectratio]{p3081977.jpg}
\caption*{Ces danseuses peu habillées sont des Apsaras, des nymphes célestes de la mythologie bouddhiste. A mon avis, c'est surtout un prétexte pour décorer le palais avec des femmes à poil.}
\end{figure}

On fera aussi une balade à vélo loin des temples, sur les conseils avisés de Jean-Louis, le patron de l'hôtel. On voit des champs de lotus, des laotiens très pauvres et très souriants, une ferme de crocodile et un énorme centre de vacance qui semble abandonné.


\begin{figure}[h]
\centering
\includegraphics[height=6cm,width=9cm,keepaspectratio]{p3092037.jpg}
\caption*{Il y avait des canards à vendre, pour donner à manger aux crocos. On n'a pas osé...}
\end{figure}


\begin{figure}[h]
\centering
\includegraphics[height=6cm,width=9cm,keepaspectratio]{p3091993.jpg}
\caption*{Un lodge abandonné.}
\end{figure}


\begin{figure}[h]
\centering
\includegraphics[height=6cm,width=9cm,keepaspectratio]{p3081920.jpg}
\caption*{Encore quelques racines pour la route, mais avec des gens pour donner l'échelle.}
\end{figure}



\chapter{Kratie et les dauphins du Mékong.}
Nous partîmes donc pour Kratie, dans l'est du Cambodge. Et on prend un bus local. Ce bus, c'est le sang du pays : il sert à tout transporter. Les soutes sont pleines de planches de bois, le toit est chargé de tuyaux, et le chauffeur sert aussi de facteur, il s'arrête partout pour charger ou décharger des paquets, et bien sûr, il transporte aussi des gens. Et des poules, bien entendu. Pour les gens qui n'ont jamais ouvert un guide de voyage, sachez-le : ce n'est pas un bus local authentique si il n'y a pas de poules. Cela fait partie des figures imposées avec le vieux édenté souriant, la mémé qui persiste à te parler dans sa langue même si tu ne comprends rien, et les gamins cul-nus. Bref, on a fait 350 bornes en 9h...


\begin{figure}[h]
\centering
\includegraphics[height=6cm,width=9cm,keepaspectratio]{p3132174.jpg}
\caption*{En attendant le ferry.}
\end{figure}




\begin{figure}[h]
\centering
\includegraphics[height=6cm,width=9cm,keepaspectratio]{p3152195.jpg}
\caption*{Elle doit être habituée, elle n'a pas fait un bruit du voyage.}
\end{figure}

En arrivant à Kratie, le transport n'est pas fini, on veut traverser le Mékong pour rejoindre notre homestay sur l'île située en face de la ville. L'endroit est encore à peu près épargné par le monde moderne, et il y règne une ambiance très paisible. La seule route de l'île n'est bétonnée que sur 500m, et permet à peine à deux motos de se croiser. Il n'y a pas encore l'électricité, mais c'est en train de changer : nous avons vu des ouvriers planter les premiers poteaux électriques. Notre homestay est assez moderne : afin d'accueillir des touristes, ils ont investi dans un groupe électrogène qui tourne quelques heures le soir, et ils ont même installé une vraie douche ! Mais la pression est tellement faible qu'on a hésité à leur demander si on pouvait se laver comme eux, dans un bac dehors à l'arrière de la maison, mais je pense qu'ils n'auraient pas compris. Les gens là-bas ont parfois des idées bien arrêtées sur les besoins des touristes...


\begin{figure}[h]
\centering
\includegraphics[height=6cm,width=9cm,keepaspectratio]{p3142187.jpg}
\caption*{Quand il fait chaud, il faut que les vaches se baignent.}
\end{figure}

Kratie est connue surtout pour les dauphins du Mékong. Nous partons donc en scooter pour quelques kilomètres, et arrivons vers la zone de la rivière où ils vivent. Après moult hésitations et discussions, nous décidons de payer un bateau pour nous amener plus près des dauphins, en se disant que si le bateau commence à chasser/embêter les dauphins, on fera demi-tour. Mais tout c'est très bien passé : la bateau a à peine fait 200m avant de couper le moteur et continuer à la rame, et nous étions soudain au milieu des dauphins ! Ce n'est pas non plus Aqualand hein ! Il faut bien tendre l'oreille pour localiser les dauphins quand ils respirent, et espérer qu'ils restent dans le même coin jusqu'au prochain souffle. Les apparitions sont fugaces, et les dauphins ne font pas de saltos ! Mais ils ont l'air de tolérer les bateaux de touristes et restent autour de nous pendant plus de 45 min. On aime à croire que le tourisme a contribué à empêcher l'extinction de ces rares dauphins d'eau douce.


\begin{figure}[h]
\centering
\includegraphics[height=6cm,width=9cm,keepaspectratio]{p3132112.jpg}
\caption*{La petite tâche sombre, c'est un dauphin !}
\end{figure}

En rentrant, toujours en scooter, nous décidons de faire un petit détour, histoire de sortir de la route principale et de voir la campagne. On trouve sur le GPS une petite boucle qui a l'air inoffensive... Disons que j'ai gagné quelques points de compétence en scootercross. On a mis une heure pour faire 4km. Les nids de poules faisaient parfois, sans mentir, 50cm de profondeur sur toute la largeur de la route. Ça ne nous a pas empêché de croiser plusieurs autres mobylettes, confirmant que nous n'étions pas perdus !


\begin{figure}[h]
\centering
\includegraphics[height=6cm,width=9cm,keepaspectratio]{p3132170.jpg}
\caption*{Nid-de-poulesque, non ?}
\end{figure}


\begin{figure}[h]
\centering
\includegraphics[height=9cm,width=12cm,keepaspectratio]{p3142192.jpg}
\caption*{Un lézard s'imaginant que je ne l'avais pas vu !}
\end{figure}



\chapter{Mondolkiri, kiri !}
(J'aurais aussi pu faire Mondolkiri, Mondolkipleure, alors ne vous plaignez pas trop...)

Nous partîmes donc pour Sen Monorom, ville principale de la région de Mondolkiri. Nous nous étions levés très tôt pour être sûrs de ne pas louper le bac nous permettant de traverser la rivière : il y en a un par heure, mais on ne sait jamais quand dans l'heure : pas très pratique quand on a des horaires à tenir ! A 7h30, le bus part comme prévu, fait des détours, récupère des gens et des paquets, et... revient à la station de bus ! D'autres gens montent, on attend encore un peu, et on finit par partir vraiment à 8h30... On a vraiment l'impression de s'être fait voler une heure de sommeil !


\begin{figure}[h]
\centering
\includegraphics[height=6cm,width=9cm,keepaspectratio]{p3162213.jpg}
\caption*{Appel aux biologistes/fleuristes : c'est quoi ?}
\end{figure}

La province du Mondolkiri est un des rares endroits du Cambodge à avoir de la montagne. Enfin, on voit un peu de relief, et de la forêt tropicale, même si le coté luxuriant n'est pas au top pendant la saison sèche. Les températures redeviennent acceptables, il ne fait plus que 30\textdegree C et on peut dormir sans ventilo ! Nous logeons dans un bungalow à quelques mètres de la forêt, avec tout ce que ça implique en terme d'araignées, de geckos et autres bestioles curieuses qui n'ont pas compris que c'est notre bungalow. Il y a aussi les cigales aux hormones. C'est un peu comme une cigale normale, mais qui a fait de la muscu, a probablement pris des produits dopants, afin de pouvoir GUEULER plus fort que ses 30000 voisines. Elles font tellement de bruit que les conversations tournent à la version dialogue de sourd en boite de nuit genre :

\emph{"Bonne ambiance, non ?"}

\emph{"Quoi ?"}

\emph{"Je disais : BONNE AMBIANCE !"}

\emph{"Hein, ah non, pas du tout, c'est juste un gecko"}

\emph{"Ah ouais, moi pareil, je l'adore"}


\begin{figure}[h]
\centering
\includegraphics[height=6cm,width=9cm,keepaspectratio]{p3172429.jpg}
\caption*{Un gecko plutôt maousse !}
\end{figure}




\begin{wrapfigure}{l}{0.45\textwidth}
\centering
\includegraphics[width=0.4\textwidth]{p3172235.jpg}
\caption*{A moi, la banane !}
\end{wrapfigure}

Bon, pour être honnête, on n'est pas venu par ici pour le frais, ni pour les cigales, mais pour les éléphants ! C'est devenu le truc touristique du coin, et il y a plusieurs agences/sanctuaires/fondations dédiées au financement de la sauvegarde des éléphants par le tourisme. Après un peu de recherche, on choisit une agence qui caresse notre conscience dans le bon sens du poil, en nous garantissant que chez eux, c'est vachement mieux, les éléphants sont libres, ne travaillent plus, ne portent pas de gens, et ne viennent voir les touristes que s'ils le veulent bien, et que les autres agences, c'est le diable, ils mangent même des éléphanteaux crus.

On va donc joyeusement rendre visite aux éléphants dans la forêt, sans oublier de prendre au passage un bon stock de bananes. Faut pas rêver, les éléphants ne vont pas venir pour nos beaux yeux. Et soudain, au milieu de la forêt, on voit deux éléphants qui se dirigent nonchalamment vers nous. Pas de barrière, pas de fossé, pas de vitre, juste deux éléphants qui se dirigent vraiment dans notre direction, ou tout du moins en direction des bananes qui sont dans nos mains. C'est impressionnant, ça fait une boule au ventre, on les entend respirer, on sent leur pas, on les entend comme "ronronner", ils font des sons tellement graves qu'on les ressent plus qu'on ne les entend.


\begin{wrapfigure}{l}{0.5\textwidth}
\centering
\includegraphics[width=0.45\textwidth]{p3172288.jpg}
\caption*{Voici "Princesse", une éléphante tellement classe qu'elle ne va pas s'abaisser à mettre elle même de la nourriture dans sa bouche : elle a des employés pour ça. Mieux : des gens paient pour avoir ce privilège !}
\end{wrapfigure}

Quelques instants plus tard, nous voilà avec des bananes cachées dans le dos, à nourrir les éléphants une banane après l'autre. Tous n'ont pas la même méthode : il y a celle (il n'y a que des éléphantes) qui prend la banane délicatement avec sa trompe, celle qui ne mange que ce qu'on lui met directement dans la bouche, celle qui essaie de chopper directement toutes les bananes dans ton dos, et la boss qui bouscule sans ménagement les autres éléphantes afin d'accéder aux bananes en premier. Petit à petit, on s’enhardit, on s'approche, on les touche, on sent la force impressionnante de leur trompe et la rugosité de leur peau. Les éléphants sont très doux dans leurs gestes, et certains semblent même un peu craintifs, et reculent si l'on s'approche d'eux.


Ce sont tous des éléphants qui ont été dressés pour travailler. Et il ne faut pas y aller de main morte pour dresser un animal qui peut tuer un homme d'un seul coup ! Ils ont été rachetés par le sanctuaire pour leur permettre de vieillir tranquillement et leur rendre un peu de liberté dans la forêt que le sanctuaire a aussi rachetée. Nous n'avions aucune idée du prix d'un éléphant : autour de 50000\$ ! Nous ne savions pas non plus que la dernière naissance d'éléphant au Cambodge remonte à une vingtaine d'année, car personne ne peut se permettre l'investissement en temps d'éléphant : deux ans de gestation puis deux ans d'allaitement, ça fait 4 années sans travail pour une éléphante, sans compter l'année nécessaire pour que les deux éléphants tombent amoureux, dixit le guide. C'est pourquoi ce sanctuaire est en train de se lancer dans la reproduction d'éléphant et cherche activement un mâle. Si vous connaissez un éléphant célibataire, faites passer le mot !


\begin{figure}[h]
\centering
\includegraphics[height=6cm,width=12cm,keepaspectratio]{p3172275.jpg}
\caption*{Alors, elle sont où les bananes, hein, elle sont où ? Elles sont parties les bananes ?}
\end{figure}


\begin{figure}[h]
\centering
\includegraphics[height=6cm,width=12cm,keepaspectratio]{p3172353.jpg}
\caption*{"Mes nouvelles copines !"}
\end{figure}

Ensuite, c'est l'heure du bain : on se jette dans la rivière, et on attend les deux éléphants les plus sympas, les seuls assez doux et attentifs pour pouvoir se baigner avec des touristes. On nous donne des balais-brosse et des seaux, et c'est parti pour frotter. Et il faut y aller, ils aiment qu'on frotte fort, et si on ne frotte pas assez fort à leur gout, ils se détournent de nous et cherchent quelqu'un de plus vigoureux !


\begin{figure}[h]
\centering
\includegraphics[height=9cm,width=12cm,keepaspectratio]{p3172365.jpg}
\caption*{Et que ça brille !}
\end{figure}

Et pour finir notre séjour à Mondolkiri, un petit trek dans la jungle ! Notre guide était super et s'arrêtait tous les 5m pour nous montrer des champignons bizarres, des fourmilières, des noix de cajou... on a même deviné un toucan au loin ! Et son secret pour faire des kilomètres dans la jungle sans trop souffrir de la chaleur ? C'est simple : il suffit de connaitre les rivières, les cascades et se baigner dès qu'on peut ! 3 baignades en 20km, ça aide !


\begin{figure}[h]
\centering
\includegraphics[height=8cm,width=9cm,keepaspectratio]{p3182452.jpg}
\caption*{Une cascade perdue au milieu de la jungle.}
\end{figure}


\begin{figure}[h]
\centering
\includegraphics[height=9cm,width=12cm,keepaspectratio]{p3172216.jpg}
\caption*{Ils font beaucoup de culture sur brûlis.}
\end{figure}


\begin{figure}[h]
\centering
\includegraphics[height=9cm,width=12cm,keepaspectratio]{p3162211.jpg}
\caption*{Ce qui est bien avec les libellules, c'est que c'est assez facile à prendre en photo.}
\end{figure}







\chapter{Arrivée au Laos : calme et chaos.}
\subsubsection{Le Chaos}
Nous partîmes donc en direction du Laos. Après une nuit à Stung Treng, dont le seul intérêt pour le touriste est ... d'être à coté du Laos, nous prenons un bus pour les 4000 îles, au Laos. Nous avions lu que la frontière était gangrénée par la corruption et, bien en avance, nous avons décidé d'essayer d'y résister. C'est de la petite corruption omniprésente : chaque coup de tampon est soumis à une taxe de 1 ou 2\$, et ce, des deux cotés de la frontière. C'est bien entendu complètement illégal, mais la plupart des touristes ne se posent aucune question. Dans le bus qui nous amène aux 4000 îles, nous sommes sept. Nous discutons bien sûr de cette histoire, et trouvons une autre fille qui a envie de lutter un peu contre ce système, les autres ne sont pas au courant ou ne veulent pas prendre de risques avec la douane.


\begin{figure}[H]
\centering
\includegraphics[height=6cm,width=9cm,keepaspectratio]{p3212508.jpg}
\caption*{Allégorie (subtile) de la corruption.}
\end{figure}

Première étape : tampon de sortie du Cambodge. Il faut tout d'abord trouver où obtenir ce tampon, car rien n'est indiqué. Il y a bien une cabane encerclée de touristes, mais quand on leur demande, ils nous disent qu'ils rentrent au Cambodge, alors que nous, on en sort ! Bon, en fait, c'est au même endroit. Il n'y a pas de queue, les gens arrivent de partout, et on joue des coudes pour arriver à passer le passeport à travers un guichet. On entend "two dollars". Évidemment, on ne voit rien ni personne : les douaniers ont ajouté des caches derrière les vitres...


\begin{figure}[H]
\centering
\includegraphics[height=6cm,width=9cm,keepaspectratio]{p3212466.jpg}
\caption*{Allégorie du douanier qui se cache.}
\end{figure}

On essaie de parlementer, de dire que non, mais c'est difficile quand on est entouré de gens qui crachent la thune. Malgré tout, à force d'insister, on ne paye que 1\$ pour deux passeports.

Deuxième étape : obtention du visa Laotien. Il y a une affiche très claire qui indique le prix par pays, et indique qu'un dollar en plus est demandé pour frais de services. On paie sans discuter et on obtient le visa.


\begin{figure}[H]
\centering
\includegraphics[height=6cm,width=9cm,keepaspectratio]{p3232578.jpg}
\caption*{Allégorie du touriste plein de principes qui se prend pour un prédateur alors qu'il n'est qu'un tout petit gecko de rien du tout.}
\end{figure}

Troisième étape : le tampon d'entrée au Laos. C'est un autre bureau juste à coté du bureau des visas, et ils sont en possession de nos passeports. Ils nous appellent un par un, et demandent 2\$ à chaque fois. Manque de bol, ce sont les 4 qui ne veulent pas négocier qui sont appelés les premiers. Quand arrive notre tour, nous tentons de dire que nous n'allons pas payer. Ça les fait doucement sourire, ils ne répondent pas et mettent ostensiblement nos passeports dans un tiroir. Marion s'occupe de la négociation, et décide de les faire un chier un peu, et leur parle encore et encore, sans réponse. On essaie de négocier un dollar pour les deux passeports, mais non. Pendant ce temps, le bus nous attend. Une lueur d'espoir apparait quand une nouvelle douanière apparait : peut-être a-t-elle une position différente vis-à-vis de la corruption ? Nos espoirs sont rapidement douchés quand elle commence à nous expliquer que si on persiste à ne pas vouloir payer, le bus va partir sans nous. Ah ! A d'autres ! Il est clair qu'elle bluffe, on ne va pas se laisser avoir si facilement. Et en effet, au bout de bien une heure de négociation, elle nous explique qu'elle accepte, à condition que ceux qui ont déjà payé aillent en direction du bus pour ne pas être témoins de notre arrangement. On fait donc ça, et on rejoint les 4 autres un instant plus tard, avec nos passeports tamponnés, tout content d'avoir réussi à un peu résister. Et nous avons économisé deux dollars ! Ce n'est pas grand chose, mais c'est pour le principe !


\begin{figure}[H]
\centering
\includegraphics[height=8cm,width=11cm,keepaspectratio]{p3212469.jpg}
\caption*{Allégorie du piège à fric qu'est la douane.}
\end{figure}

Et là, c'est le drame : pas de bus. Il est vraiment parti sans nous ! Il y a vraiment peu de circulation, et les quelques véhicules qui passent refusent de nous amener à la prochaine ville 10km plus loin. Après 4h d'attente, et après avoir vu plusieurs fois les douaniers passer devant nous, l'air satisfait, on finit par prendre un taxi. Comme on se sentait coupables pour les autres qui étaient aussi coincés là par notre faute, on a payé leur part. Résultat : quinze dollars et une après-midi de perdue. Le seul point positif : pendant l'attente, on a rencontré Stéphane, un photographe cyclotouriste très sympa.


\begin{figure}[H]
\centering
\includegraphics[height=8cm,width=10cm,keepaspectratio]{p3222529.jpg}
\caption*{Allégorie du soleil couchant de nos illusions déçues...}
\end{figure}


\subsubsection{Le calme}


\begin{wrapfigure}{l}{0.6\textwidth}
\centering
\includegraphics[width=0.55\textwidth]{p3222530.jpg}
\caption*{On tente de se fondre dans le paysage.}
\end{wrapfigure}


Nous arrivâmes donc aux 4000 îles, en essayant d'oublier le passage de la frontière, histoire de profiter un peu du Laos. On passe sans s'arrêter à travers l'île la plus connue, Don Det, non sans profiter des odeurs disons... illégales, émanant des party-hostels, et on se dirige vers Don Khon, une île bien plus calme. Et quand on dit calme, au Laos, comment dire, c'est TRÈS calme ! Le sport national, c'est la sieste ! Bon, ça c'est une blague. En réalité, c'est plutôt la pétanque le sport national, ce qui vous donne aussi une idée du niveau d'énergie du pays. On a l'impression de passer notre temps à réveiller des gens : il y a des hamacs partout, sous leur maisons sur pilotis, à l'arrière des tuk-tuks, derrière les comptoirs des magasins...


Avec la chaleur qui règne, c'est facile de se mettre au même rythme. On passe quelques jours sur place à boire des fruitshakes glacés, et à faire un peu de vélo pour aller visiter les quelques chutes d'eau autour des îles. Nous sommes rejoints par Stéphane, le photographe rencontré à la frontière Cambodge/Laos. Quand on en a marre du vélo et que la lumière est mauvaise pour les photos, on barbote dans le Mékong, en analysant bien les tourbillons afin de ne pas de faire emporter.


\begin{figure}[h]
\centering
\includegraphics[height=9cm,width=12cm,keepaspectratio]{p3222546.jpg}
\caption*{Stéphane en plein travail.}
\end{figure}

En découvrant le Mékong, les premiers explorateurs l'imaginaient comme le futur axe de transport principal de l'Asie. C'est en arrivant aux 4000 îles que leurs espoirs ont été anéantis : les rapides sont impossibles à traverser en bateau. Quelques années plus tard, les Français se sont mis en tête de construire une ligne de chemin de fer à la place. Le projet n'a jamais abouti, mais il reste encore quelques ponts destinés à faire partie du projet, dont celui qui lie l'île de Don Det à Don Khon. Du coup, un mec a tenté de mettre un péage pour les touristes sur ce pont, mais vu l’énergie qu'il met à sortir de son hamac, c'est plus facile de continuer son chemin en l'ignorant...


\begin{figure}[h]
\centering
\includegraphics[height=9cm,width=12cm,keepaspectratio]{p3222569.jpg}
\caption*{Du coton et du soleil.}
\end{figure}



\chapter{Tad Lo, le temps ralenti et ... s'arrête ?}
Nous arrivâmes donc à Tad Lo, petit village du plateau de Bolaven. Après le rythme effréné des 4000 îles, ça nous fait du bien de nous poser un peu. Oui, non, je sais, vous allez me dire que vous avez lu le précédent article, et qu'aux 4000 îles, ce n'est certainement pas un rythme effréné qui règne, alors faudrait voir à pas prendre ses lecteurs pour des truffes ! Et vous n'avez pas tort. Mais le fait est que le niveau moyen d'énergie baisse encore un peu.


\begin{figure}[h]
\centering
\includegraphics[height=6cm,width=9cm,keepaspectratio]{p3262808.jpg}
\caption*{Se baigner, ou ne pas se baigner ? Les avis divergent...}
\end{figure}

Par exemple, la guesthouse qui nous héberge ferme le portail d'entrée sur les coups de 20h30... Ou encore la fameuse cascade située à quelques kilomètres du village qu'on est allé visiter à pied, et bien cette cascade ne s'est même pas donné la peine de faire couler un peu d'eau pour nous : nous avons admiré à la place une grande falaise toute sèche !


\begin{figure}[h]
\centering
\includegraphics[height=6cm,width=9cm,keepaspectratio]{p3242602.jpg}
\caption*{Regardez bien : à droite, on voit un petit filet d'eau !}
\end{figure}


\begin{figure}[h]
\centering
\includegraphics[height=6cm,width=9cm,keepaspectratio]{p3242606.jpg}
\caption*{Une maison à l'ancienne.}
\end{figure}


\begin{figure}[h]
\centering
\includegraphics[height=6cm,width=9cm,keepaspectratio]{p3242608.jpg}
\caption*{Ils n'ont pas l'eau courante, mais ils ont la télévision satellite !}
\end{figure}

Vu la température ambiante, une des activités favorites des touristes aussi bien que des locaux, c'est la baignade dans la rivière. Ça change des 4000 îles non ? Mais pour le moment, on n'est pas lassé. D'autant plus qu'ici, il y a aussi les éléphants qui se baignent.


\begin{figure}[h]
\centering
\includegraphics[height=7cm,width=9cm,keepaspectratio]{p3252682.jpg}
\caption*{"Bonne nouvelle : mes vieux m'ont lâché les clés de l'éléphant pour la soirée. Je passe te prendre vers 20h ?"}
\end{figure}

Ce n'est pas vraiment la même ambiance qu'à Mondolkiri en revanche. Un hôtel de luxe propriétaire de deux éléphants les fait se baigner tous les jours. Cet hôtel, on y avait fait un petit tour à notre arrivée alors qu'on cherchait une chambre : on en est vite ressorti en essayant de garder un air digne après avoir vu les tarifs, et on est allé se cacher dans un bungalow tout à fait confortable et surtout 25 fois moins cher. Et cet hôtel donc, possède deux éléphants que les touristes peuvent louer le temps d'une balade. La plupart du temps, ils sont attachés à une chaine de 10m de long, et 10m, pour un éléphant, c'est pas énorme.


\begin{figure}[h]
\centering
\includegraphics[height=8cm,width=12cm,keepaspectratio]{p3242610.jpg}
\caption*{On t'a vu derrière ton arbre ! (J'essaie de blaguer, mais ça fait mal au cœur en vrai...)}
\end{figure}

Leur seul autre moment de liberté, c'est la baignade : ils viennent l'un après l'autre avec leur mahout sur le cou, qui les dirige en appuyant sur leurs oreilles avec les pieds, vers un petit endroit de la rivière où l'eau est un peu plus profonde. Et là, désormais debout sur l'éléphant dans l'eau, le mahout les fait s'immerger en entier et les frotte un peu. Puis les éléphants retournent sur la berge où ils sont récompensés avec de la canne à sucre et des bananes.

Le propriétaire, un expat américain, est dans les parages et raconte avec une verve toute ... américaine, à quel point ses éléphants sont bien traités et heureux, tout en démontrant qu'ils sont gentils et dociles en leur attrapant la trompe. Dans le doute, je lui fais part d'une observation : j'ai remarqué qu'un des mahouts avait dans la main une pointe fichée dans un petit morceau de bois. Se pourrait-il que ce soit la version portable/discrète du fameux instrument de dressage tant décrié ? Sûr de lui, il me répond que j'ai sûrement vu autre chose, probablement une brindille que le mahout a enlevé du dos de l'éléphant, ces grosses bêtes ayant tendance à se jeter sur le dos poussière, herbes sèches et autre bidules pour se protéger du soleil. N'étant pas là pour polémiquer, et après tout, je n'ai pas vu le mahout utiliser le pic, je lâche l'affaire, quand bien même je suis sûr de savoir faire la différence entre un pic et une brindille à 2m de distance. De plus, je n'y connais pas grand chose en éléphant, peut-être est-ce une mesure de sécurité : un éléphant pourrait si facilement blesser un touriste...


\begin{figure}[h]
\centering
\includegraphics[height=5.5cm,width=9cm,keepaspectratio]{p3242619.jpg}
\caption*{Un éléphant dans une flaque.}
\end{figure}

Le lendemain, à notre grande surprise, le propriétaire nous hèle en nous voyant nous promener, et vient nous présenter ses excuses ! Après notre conversation, il est allé voir son mahout qui utilisait effectivement un pic pour forcer l'éléphant à s'immerger en entier dans l'eau, ce qui les a amenés à avoir une petite discussion sur ce qu'on peut faire ou pas avec un éléphant devant des touristes ! J'avoue, pendant un petit moment, j'ai eu l'impression d'être Brigitte Bardot sauvant des bébés phoques ! Mais rassurez vous, ça m'a passé.


En remontant un peu la rivière, on voit à quel point c'est une partie intégrante de la vie du village. Pour beaucoup de monde, c'est leur salle de bain : ils s'y lavent et font aussi leur lessive. C'est aussi le domaine des pêcheurs. Chacun sa méthode : il y a ceux qui choppent les poissons à la main sous les rochers, en s'aidant d'un masque de plongeur, il y a les pièges permanents, ceux qui tendent des filets, ceux qui jettent des filets, et ceux qui s'assoient en haut d'une cascade pour pêcher à la ligne. On a frissonné de nombreuses fois en voyant des petits gamins traverser le courant un mètre en amont de la cascade pour aller rejoindre un meilleur spot de pêche !


\begin{figure}[h]
\centering
\includegraphics[height=5.5cm,width=9cm,keepaspectratio]{p3262807.jpg}
\caption*{Il y a dix mètres de chute d'eau, juste là, à quelques pas...}
\end{figure}


\begin{figure}[h]
\centering
\includegraphics[height=5.5cm,width=9cm,keepaspectratio]{p3262812.jpg}
\caption*{Ici, certains gamins font leur lessive à la main, dans la rivière.}
\end{figure}


\begin{figure}[h]
\centering
\includegraphics[height=6cm,width=9cm,keepaspectratio]{p3252703.jpg}
\caption*{Le fameux  lancer de filet.}
\end{figure}


\begin{figure}[h]
\centering
\includegraphics[height=6cm,width=9cm,keepaspectratio]{p3262796.jpg}
\caption*{Ici, un appareil photo est une machine à faire sourire les gens :-)}
\end{figure}


\begin{figure}[h]
\centering
\includegraphics[height=6cm,width=9cm,keepaspectratio]{p3262766.jpg}
\caption*{Un bout de bambou, un fil, un hameçon, et un short. What else ?}
\end{figure}



\chapter{Thakhek et "La Boucle"}
Nous arrivâmes donc à Thakhek, ville dont le seul intérêt est d'être le point de départ de la fameuse "boucle". Cette boucle est un circuit classique parmi les backpackers : entre 3 et 5 jours à scooter dans les montagnes et les karsts, avec comme point d'orgue la visite de la fameuse grotte de Kong Lore !


\begin{figure}[h]
\centering
\includegraphics[height=6cm,width=9cm,keepaspectratio]{p33028371.jpg}
\caption*{Grotte de Kong Lore.}
\end{figure}

On se prend un scooter un peu classe ce coup-ci. Si on a des bornes à faire, autant ne pas les faire avec la première bouse chinoise venue. On hérite donc d'un Honda 125cm3 rutilant ! Eh ouais, fini de rire maintenant ! Et on part pour la boucle. Les premiers kilomètres se font dans un paysage de karsts qui nous rappellent étrangement le Vietnam. Bon, pas si étrangement que ça en fin de compte, à y regarder de plus près, on n'est qu'à quelques kilomètres des karsts vietnamiens.


\begin{figure}[h]
\centering
\includegraphics[height=6cm,width=9cm,keepaspectratio]{p33028661.jpg}
\caption*{A la sortie de la grotte de Kong Lore.}
\end{figure}

Il y a des tas de cavernes à visiter, toutes payantes, et on fait notre première pause à la caverne du Bouddha, un peu au hasard. On se retrouve dans une toute petite caverne, de peut-être 20m\textsuperscript{2}, remplie de statues de Bouddha. On dirait une brocante, c'est bordélique, moche, et on essaie d'oublier qu'on a payé et fait 1h de piste poussiéreuse pour ça...


\begin{figure}[h]
\centering
\includegraphics[height=6cm,width=9cm,keepaspectratio]{p32928131.jpg}
\caption*{Voilà ce qui se passe quand on inonde une forêt !}
\end{figure}

Heureusement, les paysages sont vraiment magnifiques, et on se retrouve rapidement sur des portions de routes plutôt tranquilles. Après les karsts, ça commence à monter, et on passe à coté d'un barrage tout neuf. Comme tout ce qui est neuf au Laos, c'est construit par des étrangers, ici, des Chinois. Des informateurs, qui préfèrent rester anonymes, nous ont confié que les Français avaient fait une proposition pour le barrage, mais qu'elle avait été refusée car elle impliquait d'apprendre aux Laotiens à exploiter le barrage, alors que les Chinois s'occupaient de tout. Difficile de savoir quelle est la part de vérité dans cette explication, mais l'idée colle bien à la façon dont on perçoit le Laos (et c'est un peu triste). Le fait est que la Chine, la Thaïlande et le Vietnam se battent pour exploiter au mieux les ressources du Laos, pour le seul bénéfice d'une petite élite corrompue (et ça c'est encore plus triste).


\begin{figure}[h]
\centering
\includegraphics[height=6cm,width=9cm,keepaspectratio]{p32928161.jpg}
\caption*{Ambiance !}
\end{figure}

La route serpente ensuite sur un dédale de digues, entre les nouvelles zones inondées. Des milliers de squelettes d'arbres pointent leur troncs hors de l'eau de part et d'autre de la route. L'ambiance est halloweenesque, il ne manque qu'un voile de brume sur l'eau et des cris de corbeaux pour compléter l'ambiance. (C'est mon blog, j'invente des mots si je veux)


\begin{figure}[h]
\centering
\includegraphics[height=6cm,width=9cm,keepaspectratio]{p33028321.jpg}
\caption*{A pied dans la grotte.}
\end{figure}

Enfin, nous arrivons à Kong Lore. La grotte se traverse en bateau. Elle n'est pas éclairée, donc tout le monde a sa petite lampe frontale. J'ai bien essayé de prendre des photos, mais dans un bateau qui bouge dans une grotte, une frontale, ça ne fait pas assez de lumière. Croyez moi cependant : c'est classe ! Le seul moment où j'ai pu prendre quelques photos, c'est à la pause à mi-parcours. Le bateau nous débarque, et on suit un petit chemin parmi des concrétions, et c'est le seul endroit éclairé. Arrivés de l'autre coté, on fait une petite pause, et ... on repart dans l'autre sens. La grotte est pour le moment le seul passage pour rejoindre cette partie du monde.


\begin{figure}[h]
\centering
\includegraphics[height=9cm,width=12cm,keepaspectratio]{p33028741.jpg}
\caption*{On se croirait presque en boite de nuit.}
\end{figure}

Après trois jours et 500 km de scooter, nos fesses en compote sont contentes de prendre le bus de nuit pour Vientiane. Un siège de bus, même de bus local, c'est quand même plus confortable qu'un scooter...


\begin{figure}[h]
\centering
\includegraphics[height=9cm,width=12cm,keepaspectratio]{p3302831.jpg}
\caption*{Notre guide/capitaine de bateau.}
\end{figure}



\chapter{Vientiane, une capitale à l'image de son pays}
Nous arrivâmes donc à Vientiane, la capital du Laos. D'après la légende, c'est la capitale la plus relax du monde. Déjà, elle ne fait "que" 200 000 habitants, et en plus, elle est pleine de Laotiens. Le centre ville se traverse en une petite demi-heure à pied, et la circulation est plutôt calme. Au programme : famille, paperasse et shopping.


\begin{figure}[h]
\centering
\includegraphics[height=6cm,width=9cm,keepaspectratio]{p4032917.jpg}
\caption*{Le frangipanier, un des symboles nationaux du Laos.}
\end{figure}

Nous sommes restés quelques jours ici pour diverses raisons techniques. Tout d'abord, refaire le plein de pain au chocolat. Ensuite, faire le visa thaïlandais. D'ailleurs, l'ambassade de Thaïlande à Vientiane est probablement le lieu le plus actif de tout le Laos. Alors que les rues alentours étaient désertes, on a trouvé un lieu grouillant de vie : c'est un des endroits favoris des touristes pour faire un "visa run" depuis la Thaïlande, la ville n'est qu'à 20km de la frontière. C'est donc rempli de gens qui ne veulent qu'une chose : rentrer le plus vite possible en Thaïlande !


\begin{figure}[h]
\centering
\includegraphics[height=6cm,width=9cm,keepaspectratio]{p4012896.jpg}
\caption*{Coucher de soleil sur la Thaïlande, de l'autre coté du Mékong.}
\end{figure}

On a aussi cherché une sacoche d'appareil photo : la sangle a lâché quelques jours avant, et je ne peux plus le porter en bandoulière. Cela c'est avéré plus difficile que prévu : malgré tous nos efforts, impossible de trouver une boutique photo. Des téléphones, des laves-linges et des voiture autant qu'on veut, mais d'appareil photo point. Soit. Une sacoche toute bête fera aussi l'affaire.


\begin{figure}[h]
\centering
\includegraphics[height=6cm,width=9cm,keepaspectratio]{p4042924.jpg}
\caption*{Pha That Luang, la stupa la plus célèbre du Laos.}
\end{figure}

On se confronte une fois de plus à l'esprit Laotien quand on essaie d'aller voir un film. Il n'y a qu'un seul cinéma au Laos, dans un centre commercial construit par des chinois. On arrive une heure avant la séance, et on nous dit que c'est trop tôt. Bon, on attend. 20mn avant la séance, on retente, et là, le mec semble un peu gêné. On ne comprend pas trop au début, mais à force d'insister, il finit par nous dire en passant par plusieurs chemins détourné qu'il n'y a pas de séance, mais qu'on peut revenir pour la prochaine dans 4h... (Soit disant qu'il n'avait pas la clé, parce que la belle-sœur de son chat a eu une attaque de lait sur le feu en prison, ou un truc dans le genre). C'est un peu le problème de la culture asiatique, où le fait de dire non est très difficile, et de notre coté, en gros lourdauds d'européens, on a du mal à comprendre leurs messages subtils. Résultat, on insiste encore et encore jusqu'à les acculer et les obliger à dire un gros "non" bien franc.


\begin{figure}[h]
\centering
\includegraphics[height=6cm,width=9cm,keepaspectratio]{p4012913.jpg}
\caption*{C'est la saison des fourmis volantes !}
\end{figure}

La grosse raison qui nous a fait rester à Vientiane, c'est aussi l'arrivée du papa de Marion. On l'a retrouvé à l'aéroport de Vientiane et il nous a suivi pendant quelques jours dans le nord du Laos. Mais on a commencé par un petit tour en scooter autour de Vientiane, en direction du Bouddha Parc : une sorte de palais idéal du facteur cheval, mais rempli de sculptures en béton sur de thème de la mythologie Bouddhiste. C'est kitch à souhait, il y en a dans tous les coins, et on n'a pas assez d'yeux pour remarquer tous les détails.


\begin{figure}[h]
\centering
\includegraphics[height=6cm,width=9cm,keepaspectratio]{img_20160405_135553091.jpg}
\caption*{A fond dans la campagne !}
\end{figure}


\begin{figure}[h]
\centering
\includegraphics[height=9cm,width=9cm,keepaspectratio]{p4052933.jpg}
\caption*{Échantillon du Bouddha Parc.}
\end{figure}

Sur le trajet, on croise aussi l'arc de triomphe de Vientiane. Il a été copié ostensiblement sur celui de Paris, mais en plus grand (prend ça dans tes prérogatives post-coloniale, France !), et en béton.


\begin{figure}[h]
\centering
\includegraphics[height=9cm,width=12cm,keepaspectratio]{p4042930.jpg}
\caption*{Sans les palmiers, on se croirait à Paris, non ?}
\end{figure}

Et le soir venu, ce fut un peu Noël pour nous : le papa de Marion a ouvert sa hotte, et nous a sorti fromage, saucisson et chocolat. Depuis le temps qu'on en rêvait !





\chapter{Luang Prabang, des moines, et des cascades.}
Nous arrivâmes donc à Luang Prabang. En tant qu'ancienne capitale du Laos, c'est une ville riche en histoire, de plus, les paysages alentours sont magnifiques et il y a quelques chutes d'eau qui valent vraiment le voyage. Et croyez-nous, depuis qu'on est parti, on en a vu quelques unes des chutes d'eau... Tout ceci fait de la ville un incontournable du Laos.


\begin{figure}[h]
\centering
\includegraphics[height=6cm,width=9cm,keepaspectratio]{p4073069.jpg}
\caption*{C'est biquet, non ?}
\end{figure}




\begin{figure}[h]
\centering
\includegraphics[height=5.5cm,width=9cm,keepaspectratio]{p4062972.jpg}
\caption*{A vélo dans la jungle !}
\end{figure}

Nous avons fait le voyage en bus de nuit. Un première pour le papa de Marion ! Nous voilà à 6h du matin dans les rues de la ville à chercher notre hôtel. Comme d'habitude, nous l'avions réservé en avance et soigneusement noté l'emplacement sur le GPS. Mais ce coup ci, impossible de le trouver. Nous qui comptions impressionner mon beau père avec notre expérience de voyageur aguerri, c'est l'échec... Rassurez-vous, on n'a pas dormi dans la rue pour autant. Dans ce coin, il suffit de secouer un lampadaire et dix hôtels en tombent.


\begin{figure}[h]
\centering
\includegraphics[height=6cm,width=9cm,keepaspectratio]{p4062960.jpg}
\caption*{Je n'avais pas de banane pour donner l'échelle, mais je vous jure, c'est un p****n de gros papillon !}
\end{figure}

Nous louons des vélos, et nous voilà partis vers une petite chute d'eau dans la jungle. Bon, en vrai, c'est la saison sèche, et la chute d'eau... voilà quoi. L'avantage, c'est qu'on était les seuls sur le sentier, et que la jungle est magnifique malgré tout, surtout une fois que le leader de la troupe a enlevé les toiles d'araignées qui barrent le chemin.


\begin{figure}[h]
\centering
\includegraphics[height=5.5cm,width=9cm,keepaspectratio]{p4062966.jpg}
\caption*{Tout ça pour une cascade à sec...}
\end{figure}

Le soir, c'est l'heure pour un petit sandwich avec un milkshake au marché de nuit. Il y a des dizaines de stands alignés, mais ils font tous les mêmes sandwiches et milkshakes, alors on choisi le stand ou on a les plus grands sourires :-)
Nous avions presque fini le repas quand soudain, un petit coup de vent ! Mais un petit coup de vent un peu louche, genre, il a une odeur de pluie, et il y a une rumeur qui agite encore l'air après son passage, les arbres bruissent plus fort que d'ordinaire. On sent un changement d'ambiance très clair, et certain marchands ayant plus de nez que leurs voisins commencent à ranger leur stand. Ils font bien, car quelques instants plus tard, le grand frère du coup de vent débarque et ce coup-ci, retourne quelques chapiteaux ! Et là, ce fut une mini panique, et deux stratégies s'opposaient :
\emph{"On va tous mourir, donc je range mes affaires avant que tout s'envole en poussant des petits couinements"}
contre
\emph{"On va tous mourir mais je n'ai pas encore fait mon chiffre de la soirée alors je m'accroche à mon chapiteau pour l'empêcher de s'envoler en poussant des petits couinements".}

Je vous laisse admirer la stratégie de notre stand sur la photo suivante.


\begin{figure}[h]
\centering
\includegraphics[height=5.5cm,width=9cm,keepaspectratio]{p4062975.jpg}
\caption*{C'est le moment d'être lourd !}
\end{figure}

Le pic de couinements a été atteint quand la lumière a été coupée pendant quelques minutes. Quand on ne voit plus rien, qu'on est accroché à 20m\textsuperscript{2} de toile et que le vent fait s'envoler les chapiteaux adjacents, j'avoue, on ne fait pas trop le malin. Finalement, tout s'est calmé, et je ne crois pas qu'il y ai eu de dégâts autres que matériels.


\begin{figure}[h]
\centering
\includegraphics[height=6cm,width=9cm,keepaspectratio]{p4072983.jpg}
\caption*{L'attente.}
\end{figure}

Le lendemain, nous nous sommes levés à l'aube pour aller assister à l'offrande aux moines. Tous les matins, les moines font le tour de Luang Prabang à pied, et de nombreux bouddhistes les attendent agenouillés sur les trottoirs pour leur donner de la nourriture et recevoir une bénédiction. C'est devenu très connu et fait désormais partie du circuit touristique classique de la ville. Et comme vous le savez bien : il est impossible d'observer un système sans le modifier. Ici, la modification vient de gens qui vont vendre des offrandes aux touristes désireux de faire plus que simplement observer. Il y a eu plusieurs histoires de moines rendus malades par de la mauvaise nourriture obtenue par ce biais. Les moines s'en sont plaint et ont menacé d'aller ailleurs. Le gouvernement leur a dit que s'il faisaient ça, ils embaucheraient des acteurs pour les remplacer, tellement c'est important pour le tourisme. Je ne sais pas quelle a été le résultat de la négociation, tout ce que je peux vous dire, c'est qu'on a vu des mecs rasés en robe orange sortir en file d'un temple, et recevoir de la nourriture de la part de Laotiens, mais aussi de quelques touristes qui avaient plus l'air hippies que bouddhistes. Donc je vous le dis : à moins d'être bouddhiste, contentez vous d'observer les moines de loin !


\begin{figure}[h]
\centering
\includegraphics[height=5.5cm,width=9cm,keepaspectratio]{p4072997.jpg}
\caption*{Les moines sont sortis.}
\end{figure}

Les moines, c'est bien, mais selon moi, les cascades de Luang Prabang, c'est mieux. Et je crois que ça se passe de mots :


\begin{figure}[h]
\centering
\includegraphics[height=5.5cm,width=9cm,keepaspectratio]{p4073074.jpg}
\caption*{C'est fou, non ?}
\end{figure}


\begin{figure}[h]
\centering
\includegraphics[height=5.5cm,width=9cm,keepaspectratio]{p4073043.jpg}
\caption*{On peut aussi s'y baigner, mais c'est un peu frais !}
\end{figure}


\begin{figure}[h]
\centering
\includegraphics[height=5.5cm,width=9cm,keepaspectratio]{p4073057.jpg}
\caption*{:-)}
\end{figure}


\begin{figure}[h]
\centering
\includegraphics[height=5.5cm,width=9cm,keepaspectratio]{p4073087.jpg}
\caption*{En route vers les sources de la cascade.}
\end{figure}


\begin{figure}[h]
\centering
\includegraphics[height=5.5cm,width=9cm,keepaspectratio]{p4073106.jpg}
\caption*{Ambiance Tahiti douche !}
\end{figure}

Seul regret : à cause de deux chinoises qui partageaient notre tuktuk, on n'a pas pu y rester aussi longtemps qu'on voulait. Ils ne comprennent rien ces chinois : un selfie, et ça repart !


Petite dernière surprise sur le chemin du retour : un parc aux papillons ! On y a découvert l'existence de chrysalides dorées ! C'est étonnant, car il me semble qu'une chrysalide devrait se camoufler plutôt que de porter des dorures façons rappeur américain, non ? Une hypothèse suggérée par le guide est que ça camoufle peut-être en imitant les gouttes d'eau qui brillent elles aussi au soleil ? Chers lecteurs, si vous avez des connaissances sur ce sujet, je prends !


\begin{figure}[h]
\centering
\includegraphics[height=5.5cm,width=9cm,keepaspectratio]{p4073120.jpg}
\caption*{Dans le genre sobre et discret, on a vu mieux !}
\end{figure}



\chapter{Muang Ngoy, le village où il pleut des cendres.}
Nous partîmes donc pour Muang Ngoy. Si vous n'en avez jamais entendu parler, c'est normal ! Il n'y a même pas de route qui y va : quatre heures de bus pour aller tout au nord du Laos, jusqu'au bout de la route, puis une heure de bateau pour aller encore un peu plus au nord, et on arrive dans un tout petit village en train de s'ouvrir au tourisme !


\begin{figure}[h]
\centering
\includegraphics[height=6cm,width=9cm,keepaspectratio]{p4093239.jpg}
\caption*{Bain de boue pour tous !}
\end{figure}

Pendant le trajet, on croise plusieurs bateaux couverts d'une armée de rameurs : dans quelques jours, chaque village sur le fleuve va envoyer une équipe de ses meilleurs rameurs pour la compétition annuelle. La rivière est alors réservée aux compétitions. On a eu de la chance de passer à coté d'un bateau à l'entrainement. Pendant la compétition, on aurait du rester sur la rive, mais là, ils sont passés si près qu'on les entendait même respirer à l'unisson !


\begin{figure}[h]
\centering
\includegraphics[height=6cm,width=9cm,keepaspectratio]{p4083164.jpg}
\caption*{26 personnes sur un frêle esquif.}
\end{figure}

A peine débarqués, un charmant jeune homme, au teint qu'on devine très scandinave malgré le soleil laotien, nous accoste. Ou plutôt nous aborde, étant donné qu'on avait encore un pied dans le bateau. Il propose de nous loger dans des bungalows avec vue sur la rivière. En chemin, on apprend qu'il est suédois, et qu'il est arrivé ici en voyageant il y a quelques années déjà. Et le voilà désormais marié, père, et propriétaire d'une auberge dans le village ! En visitant un peu le village, on comprend en partie ce qui l'a retenu ici. L'ambiance est paisible au possible, et les paysages semblent magnifiques. Un peu plus tard, on comprend l'autre partie qui l'a fait rester ici en rencontrant sa femme, puis sa fille.


\begin{figure}[h]
\centering
\includegraphics[height=6cm,width=9cm,keepaspectratio]{p4083181.jpg}
\caption*{La rue principale.}
\end{figure}

Dans le paragraphe précédent, vous l'avez sûrement remarqué, j'ai écrit : "Les paysages semblent  magnifiques". Mais pourquoi donc "semblent" ? Et bien parce que nous n'avons fait que les deviner : avril, c'est le dernier mois avant la mousson, c'est le mois le plus chaud et le plus sec de l'année, et c'est logiquement la saison des brûlis ! Le ciel est rempli de cendres qui nous tombent dessus, et la visibilité est réduite à quelques centaines de mètres. L'ambiance est vaporeuse, et les montagnes disparaissent très vite dans la brume.


\begin{figure}[h]
\centering
\includegraphics[height=6cm,width=9cm,keepaspectratio]{p4093241-modifier1.jpg}
\caption*{A ce qu'il parait, il y aurait encore d'autres montagnes derrière ces collines.}
\end{figure}

Nous faisons une petite balade dans les rizières derrière le village, ce qui nous permet de nous éloigner encore un peu de la civilisation (occidentale). On dirait que le temps s'est arrêté, et que rien ne va bouger avant l'arrivée de la pluie : les rizières sont craquelées, les gens passent leurs journées à l'ombre, à bouger le moins possible, tant la chaleur est écrasante. On dépasse les 40\textdegree C. Les buffles passent le temps en tentant de se rafraichir, entassés dans la dernière mare de boue du village. J'ai l'impression que si la pluie n'arrive pas rapidement, ils vont se transformer en statue de terre cuite...




\begin{figure}[h]
\centering
\includegraphics[height=6cm,width=9cm,keepaspectratio]{p4093240.jpg}
\caption*{On n'est pas bien, là ?}
\end{figure}


\begin{figure}[h]
\centering
\includegraphics[height=6cm,width=12cm,keepaspectratio]{p4093213.jpg}
\caption*{Dans les rizières.}
\end{figure}


\begin{figure}[h]
\centering
\includegraphics[height=6cm,width=12cm,keepaspectratio]{p4093222.jpg}
\caption*{Un artisan en plein travail.}
\end{figure}


\begin{figure}[h]
\centering
\includegraphics[height=5.5cm,width=9cm,keepaspectratio]{p4093231.jpg}
\caption*{C'est donc une boite pour cuire du riz collant.}
\end{figure}

Nous serions volontiers restés ici quelques jours de plus, mais la vraie vie se rappela à nous. Ne dit-on pas : "Chassez la vie réelle, elle revient au galop" ? Nous galopâmes donc en direction de l'aéroport de Vientiane, non sans nous arrêter quelques jours à mi-chemin, dans l'éminente ville de Vang Vieng, Eldorado de la fête en Asie pour les backpackers de tout poils jusqu'à ces dernières années.


\begin{figure}[h]
\centering
\includegraphics[height=5.5cm,width=9cm,keepaspectratio]{p41132631.jpg}
\caption*{Par 40\textdegree C, c'est une activité conseillée !}
\end{figure}

Il n'y a pas si longtemps, c'était un passage obligatoire pour tout backpacker qui se respecte. Voici les ingrédients : une belle rivière, des chambres à air, des bars pas chers, quelques dealers (à prononcer "dilair" sinon ça ne rime pas).
On ajoute quelques jeunes, et ça donne des descentes de rivières sur des grosses bouées, le tout avec un esprit qui a du mal à évaluer les risques. Entre les noyades et les sauts dans les rochers (notez bien, j'ai dit "dans les rochers", et non "depuis les rochers"), l'endroit a commencé à avoir mauvaise réputation, au point d'attirer l'attention des médias internationaux. Il y a eu 20 morts au cours de l'année 2011, et plus de 3000 fêtards sont passés par l’hôpital. Et avant que CNN aie eu le temps d'envoyer un reporter sur place, quasiment tous les bars avaient été fermés par le gouvernement. L'ambiance est désormais plus calme. Il est toujours possible de descendre la rivière en bouée, mais le Coréen prudent en gilet de sauvetage a remplacé le jeune Australien en quête de nouvelles expériences.


\begin{figure}[h]
\centering
\includegraphics[height=5.5cm,width=9cm,keepaspectratio]{p4113253.jpg}
\caption*{Je suis le John Travolta des grottes !}
\end{figure}

Nous avions tellement lus de choses négatives sur cette rivière, qu'on n'y est même pas allé, préférant chercher les grottes loin des sentiers battus. Voici le résultat :


\begin{figure}[h]
\centering
\includegraphics[height=5.5cm,width=9cm,keepaspectratio]{p4113254.jpg}
\caption*{Même pas besoin de LSD pour planer !}
\end{figure}




\begin{figure}[h]
\centering
\includegraphics[height=5.5cm,width=9cm,keepaspectratio]{p4113252.jpg}
\caption*{Même le papa de Marion se met au light painting.}
\end{figure}


\begin{figure}[h]
\centering
\includegraphics[height=5.5cm,width=9cm,keepaspectratio]{p4123270.jpg}
\caption*{La faune locale très rassurante.}
\end{figure}




\chapter{Vientiane, le retour : la fête de l'eau du nouvel an !}
Nous retournâmes donc à Vientiane, juste avant le début de Pimai, aussi appelé "fête de l'eau", aussi appelé nouvel an Lao. Rassurez-vous, au Laos, on fête aussi le nouvel an chinois, ainsi que le nouvel an occidental. Pas con !


\begin{figure}[h]
\centering
\includegraphics[height=6cm,width=9cm,keepaspectratio]{p1060335.jpg}
\caption*{A l'assaut !}
\end{figure}




\begin{figure}[h]
\centering
\includegraphics[height=6cm,width=9cm,keepaspectratio]{p1060322.jpg}
\caption*{Bouddha aussi a chaud.}
\end{figure}

A l'origine, la fête de l'eau est une fête bouddhiste : on arrose les mains et la tête des bouddhas avec de l'eau parfumée aux fleurs. On arrose aussi les grand-mères et grand-pères en signe de respect, et en fin de compte on arrose un peu tout le monde, parce qu'après tout, c'est la fin de la saison chaude, il fait une chaleur à crever, et c'est bien agréable un peu d'eau fraîche !


\begin{figure}[h]
\centering
\includegraphics[height=6cm,width=9cm,keepaspectratio]{p1060310.jpg}
\caption*{Oh oh, on dirait que c'est moi le prochain...}
\end{figure}

Bref, de proche en proche, c'est devenu une grosse bataille d'eau à l'échelle du pays. Les gens installent des piscine gonflables partout sur les trottoirs, qui vont servir de réserves d'eau. Et vers 12h, ça commence : les pick-up chargés eux aussi de bassins circulent dans la ville, et les passagers arrosent les piétons, les piétons ripostent, et ce sera à celui qui aura le plus gros débit/bassin/pistolet à eau, le tout dans une ambiance bon enfant ! Un peu partout dans la ville, il y a des concerts live, avec arroseurs automatiques au dessus de la foule, voire une machine à mousse. On a même vu un camion citerne équipé d'une lance à incendie circuler dans les rues : c'était vraiment le roi de la fête ! On s'est fabriqué des pistolets à eau avec des bouteilles percées, mais niveau puissance de feu, on était vraiment ridicules...


\begin{figure}[h]
\centering
\includegraphics[height=6cm,width=9cm,keepaspectratio]{p1060304.jpg}
\caption*{C'est moi qui ai la plus grosse (réserve d'eau) !}
\end{figure}

Il est très difficile de rester au sec, et de toute façon, personne n'en a vraiment envie. Mais il est quand même fortement recommandé d'avoir toutes ses affaires dans des sacs étanches pendant ces quelques jours. Oui, j'ai bien dit quelques jours : ça dure en théorie 4 jours, mais en vrai, il y a toujours quelques gamins qui commencent un peu avant. Il faut d'ailleurs être prudent en scooter à cette période de l'année, car un bac d'eau et de glaçons attend peut-être au prochain virage... Et si en plus le pilote est bourré (mais non, qui donc boit à nouvel an ?), je vous laisse imaginer les conséquences ! (Oui, un pilote trempé faisant la fête avec un arroseur, c'est aussi une conséquence possible.)


\begin{figure}[h]
\centering
\includegraphics[height=6cm,width=9cm,keepaspectratio]{p10603081.jpg}
\caption*{Il peut y avoir un tireur embusqué dans la moindre poubelle !}
\end{figure}

Cela dit, la plupart des gens sont très respectueux, et sur un simple geste, ils comprennent si on ne veut pas être arrosé, parce que, par exemple, on a un appareil photo pas étanche en main. Dans ce cas, ils nous arrosent seulement un peu les mains ou le dos, en nous expliquant que c'est une tradition, que c'est pour porter bonheur pour la nouvelle année. Comment dire non à une bénédiction d'eau fraiche quand il fait 40\textdegree C à l'ombre ?


\begin{figure}[h]
\centering
\includegraphics[height=8cm,width=12cm,keepaspectratio]{p10603311.jpg}
\caption*{Trois hypothèses : 1- il n'avait pas vu que j'avais un appareil photo, 2- il l'avait vu mais il s'en foutait, 3- il avait vu que l'appareil était étanche.}
\end{figure}

Le petit problème avec la fête du nouvel an, c'est que la plupart des restaurants et boutiques sont fermés, la majorité des chauffeurs de tuktuks et taxis complètement démotivés pour travailler (démotivé est probablement en dessous de la réalité). La dernier journée avec mon papa fut un peu difficile : la fatigue des transports, les crises de nerfs d'incompréhension avec les laotiens qui ne comprennent pas la nécessité de respecter les horaires quand on a un avion à prendre, et le pincement au cœur de dire au revoir à mon papa.




\begin{figure}[h]
\centering
\includegraphics[height=9cm,width=12cm,keepaspectratio]{p1060280.jpg}
\caption*{Dernière journée avec Jean-Michel à Vientiane.}
\end{figure}

Et voilà nous sommes à l'aéroport. Nous sommes tristes de te voir partir et tellement heureux d'avoir pu partager un peu de ce voyage avec toi papa !


\tableofcontents
\end{document}
