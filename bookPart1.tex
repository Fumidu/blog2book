
\documentclass{book}
% Chargement d'extensions
\usepackage[francais]{babel} % Pour la langue française
\usepackage[utf8]{inputenc}
\usepackage[T1]{fontenc}
\usepackage{float}
\usepackage{eurosym}
\usepackage{textcomp}
\usepackage{wrapfig} % to wrap figure in text
%\usepackage{caption} % to remove fig #
\usepackage[font=it]{caption}
\usepackage{titlesec} % to modify chapter header
\usepackage[normalem]{ulem}
\usepackage{soul}

\titleformat{\chapter}[display]
  {\normalfont\bfseries}{}{0pt}{\Large}

\usepackage[
  paperwidth=15.557cm,
  paperheight=23.495cm,
  %showframe,
  margin=15mm
  % other options
]{geometry}

\hyphenation{CRRR-RA-A-A-A-AK-K-K-K-K-KR-R-R-RB-B-B-B-BRA-A-A-A-A-A-AOUM-M-M-M-M-MR-R-R-RM-M-M-ML-L-L-LLLLL}

%\usepackage{layout} %TODO remove

\usepackage{graphicx} % Pour les images
\graphicspath{ {Images/petit/} }
%\graphicspath{ {Images/} }

\title{Et demain on va où ?}
\author{Gérald Schmitt \and Marion Abrial}

\begin{document}
%\layout
\maketitle

\chapter{Lancement du blog !}
Voilà, c'est officiellement notre premier article de blog consacré à notre voyage à venir !

Et c'est même mon premier article de blog tout court. Je ne sais pas encore très bien comment ça marche, ni si je vais réussir à raconter des trucs intéressants... mais c'est un début !

On a envoyé le préavis pour rendre l'appartement il y a quelques jours. Ça rend d'un coup le voyage très proche et très réel ! Ça fait envie (surtout à Marion) et aussi un peu peur (surtout à Gérald). Mais à deux, ça va l'faire ! :-)

Bon, et avant de se demander où on va demain, la prochaine question sera plutôt, et demain, on fait quoi pour préparer tout ça ?

\chapter{Le symbole de la liberté !}
Voici mon porte-clé :


\begin{figure}[h]
\centering
\includegraphics[height=9cm,width=12cm,keepaspectratio]{p9280216.jpg}
\caption*{Mon porte clé}
\end{figure}

Eh oui, il est passé par une sérieuse cure d'amaigrissement cette semaine. Plus d'appartement, plus de boulot, plus de voiture.
Reste ce réflexe :

Je tâte ma poche gauche. Téléphone présent.

Je tâte ma poche droite... vide ! Mon coeur s'arrête une fraction de seconde. Je suis coincé dehors !

Soupir de soulagement. Gérald, t'es con. Pour ne rien arranger, c'est la troisième fois en dix minutes que tu paniques...
Et oui, nous sommes bien coincés dehors. Et le monde entier est potentiellement notre maison.

\chapter{Du bordel plein le dos}
Voilà le bordel qu'on va avoir sur le dos pour l'année à venir :


\begin{figure}[h]
\centering
\includegraphics[height=9cm,width=12cm,keepaspectratio]{p9300220.jpg}
\caption*{ Un an de bordel}
\end{figure}

Un minuscule extrait de l'amoncellement de bordel qu'on a réussi à accumuler dans 70m\textsuperscript{2} à Lyon. Et c'est là dessus qu'on va compter pour tout :
\begin{itemize}
	\item Manger
	\item Dormir
	\item Se laver
	\item Se soigner
	\item S'habiller
	\item Cuisiner
	\item Communiquer
	\item Photographier
	\item Glander
\end{itemize}
On est parti d'un appartement plein, qu'on a vidé en quasi totalité une semaine avant de le rendre. Mais on a quand même gardé un peu plus que nos affaires du tour d'Asie. Ensuite, on est allés squatter chez les parents, mais on n'a pas encore renoncé totalement à tout ce qui ne va pas partir en Asie...
C'est comme si on repoussait l'inéluctable, ce moment ou on n'aura plus que notre sac à dos, et vraiment plus rien d'autre.
Je pense à ce sketch de Roland Magdane sur le merdier... Si je trouve un lien, je le mettrai dans l'article, car ça correspond assez bien à ce qu'on vit en ce moment.

Tout ça pour en arriver à un petit sac de bordel de 12kg environ. On est plutôt content ! On visait les 10kg, mais il aurait fallu faire l'impasse sur la tente ou l'appareil photo.

Et maintenant, on se demande dans quel sens le poids du sac va évoluer au cours du voyage.

Vous faites quel pronostic ?

\chapter{Le sac est prêt !}
Juste une petite photo de nous avec le sac à dos complet. On est prêt à prendre le bus.
\begin{figure}[h]
\centering
\includegraphics[height=9cm,width=12cm,keepaspectratio]{pa100286.jpg}
\caption*{}
\end{figure}

Au programme : Mulhouse - Berlin - Varsovie - Brest (en Biélorussie) - Saint Pertersburg. Arrivée prévue à Saint-Petersburg jeudi 15 octobre.

On vous tient au courant dès qu'on a un peu de temps. A bientôt :-)

\chapter{Le bus de Babel}
Donc on a pris le premier bus de notre voyage. On part de France (Mulhouse) pour aller en Allemagne (Berlin). Le chauffeur du bus ne parle qu'une seule langue : quoi de plus logique que l'espagnol ? Le bus est rempli en majorité de... Polonais !

Et là, on nous met un film dans le bus. Un film anglais (avec Hugues Grant mais aucun rapport). Doublé en Russe. Ben oui ! Vous savez, ce fameux doublage russe d'une qualité comment dire... disons d'une qualité remarquable. Un gars lit son texte sans aucune intonation. Et le même gars fait toutes les voix ! Et la bande son du doubleur est juste superposée à la bande originale du film. Je ne sais pas si les Russophones comprennent quelque chose au film, mais pour les Anglophones, aucune chance !

Et tout ça, alors qu'on était même pas encore vraiment sorti d'Alsace. Tu parles d'un départ sur les chapeaux de roues :-)


\begin{figure}[h]
\centering
\includegraphics[height=8cm,width=8cm,keepaspectratio]{pa120299.jpg}
\caption*{On est à Berlin}
\end{figure}

On a quand même finit par arriver à Berlin.

\chapter{Des conditions aux bornes}
Donc nous voilà parti de Berlin. Après l'expérience du bus Mulhouse-Berlin, on ne savait pas trop à quoi s'attendre pour le Berlin-Varsovie.
Eh bien on a été pour le moins surpris. Vous saviez, vous, qu'il existait des bus avec prise 220V à chaque siège, petit écran de divertissement individuel avec plein de films récents comme dans les avions, un petit bouton pour appeler l’hôtesse et lui demander un thé, et, oh douce surprise : du wifi ! Du wifi qui marche pour de vrai ! Dans un bus en mouvement entre l'Allemagne et la Pologne !
Et l’hôtesse parlait même anglais.

On arrive à Varsovie à l'heure prévue, à 6h du matin. Notre bus suivant pour Brest ne part qu'à 9h. Donc on attend. Et on cherche des toilettes. Et oui, il y en a. elles sont payantes. Et en zloti. Et pas moyen de négocier avec la dame pipi, se sont des zloti, ou bien rien. Dans notre cas, ce sera rien...

Puis, Varsovie-Brest. On arrive à la frontière vers 13h30 heure locale. Notre destination est 4km derrière la frontière. On se dit qu'on va arriver bien avant les 15h00 prévues. Eh bien non ! Le passage de la frontière dure, et dure, et dure encore. La sortie de Pologne est assez rapide. Mais l'entrée en Biélorussie... On remplit le questionnaire d'immigration. On attend. Un garde passe prendre nos passeports. On attend. Longtemps. Personne dans le bus n'ose rien dire. On a l'impression qu'au premier éternuement, c'est direction le goulag. Des gens sortent nos sacs du bus. On attend. On sort, on prends nos sacs, on rentre dans les bâtiments de la douane pour une fouille, et on peut enfin reprendre le bus.
Il est 15h45...

A 16h, le bus nous dépose dans le centre de Brest, et on se met en route pour notre auberge de jeunesse. On la trouve au sous-sol d'un grand ensemble d'immeubles. Nous sommes les seuls voyageurs de toute l'auberge. Tout est neuf. On pourrait croire que nous sommes les premiers clients tout court ! Après deux jours à dormir dans des bus. Pas envie de partir crapahuter dans Brest. Donc on fait des courses dans un supermarché tout ce qu'il y a de plus classique, sauf qu'on ne comprend rien aux étiquettes. On a acheté un peu par hasard des petits gâteaux qui ressemblait à des pains d'épices. En fait, ils étaient à la menthe. C'est marrant, on a vraiment l'impression de manger du dentifrice !

\begin{figure}[h]
\centering
\includegraphics[height=9cm,width=12cm,keepaspectratio]{pa130322.jpg}
\caption*{Eglise dans Brest}
\end{figure}
Et le lendemain, enfin on prend le train pour Saint-Petersburg.
Dans le train, on commence à réfléchir au passage de frontière et aux visas. Et on prend conscience d'une chose terrible : on a bien un visa Biélorusse jusqu'au 14 octobre, et un visa Russe à partir du 15 octobre, et nous sommes dans la nuit du 14 au 15 octobre, mais nous ne sommes pas dans un avion, bien dans un train. Si on passe la frontière avant minuit, peut-être que les Russes ne nous laisseront pas rentrer ? Si on passe la frontière après minuit, peut-être que les Biélorusses vont nous dire que notre visa a expiré et nous mettre une prune ? On essaie de se rassurer en se disant que quand même, on ne fait rien de mal, que les douaniers ne sont pas si méchants etc. Mais quand même, on suit anxieusement l'approche de la frontière sur le gps.


\begin{figure}[h]
\centering
\includegraphics[height=8cm,width=11cm,keepaspectratio]{pa140330.jpg}
\caption*{Notre train vers la Russie}
\end{figure}
Heureusement, nous sympathisons avec nos voisins de compartiment. Entre l'allemand et un russe balbutiant, nous arrivons à comprendre qu'ils sont un couple d'ingénieurs russes à la retraite, de retour d'un sanatorium en Biélorussie. Il faut savoir qu'il n'y a rien d'autre à faire en Biélorussie que la chasse, cueillir des champignons, et globalement se promener dans la nature. C'est un pays complètement plat, dont le point culminant ne dépasse pas 400m.
Et ce couple sympathique s’évertue à nous parler de la Russie, à nous montrer des photos, nous dire quoi faire absolument à Saint Petersburg. Et bien entendu, impossible de boire un thé sans qu'ils tentent de nous l'améliorer avec de la gnôle à base de plantes :-)


\begin{figure}[h]
\centering
\includegraphics[height=10cm,width=9cm,keepaspectratio]{pa140331.jpg}
\caption*{Notre compartiment}
\end{figure}
Et on continue de s'approcher de la frontière. Vers 22h, tout le monde monte dans sa couchette, et la frontière approche. Je me demande si on va réveiller les gens. Bref, c'est étrange. Un peu avant minuit, on passe la frontière, le train s'arrête et (suspens insoutenable)... ben il repart, et c'est tout. En fait, et on l'a appris ensuite : il n'y a pas de frontière. C'est un espace économique commun comme l'Europe.
Du coup, on arrive sans encombre à Saint Petersburg.

\chapter{Quelques jours à Saint-Petersbourg}
Alors on est arrivé à Saint Petersbourg mercredi 15 octobre à 7h00 du matin, encore tout surpris de n'avoir pas franchi de douane. Et à cette heure, la ville est encore assez calme.
On a trouvé assez facilement notre auberge de jeunesse, on a laissé nos bagages, et on est partis en ville en suivant la fameuse Nevky Prospekt. Disons que c'est l'équivalent des Champs-Elysées, mais en Russe. Tous les bâtiments sont incroyables. A chaque coin de rue, on est obligé de s'arrêter devant une nouvelle cathédrale, une statue monumentale, ou un bâtiment couvert de moulures dorées.


\begin{figure}[h]
\centering
\includegraphics[height=6cm,width=9cm,keepaspectratio]{pa150337.jpg}
\caption*{Cathédrale du sang versé}
\end{figure}
On ne fait qu'une seule visite ce jour-là. Mais alors quelle visite ! La cathédrale du sang versé et du Christ ressuscité. Pour nous qui sommes habitués aux cathédrales gothiques, voir tant de couleurs et de dorures est stupéfiant dans une église, que ce soit à l'intérieur ou à l'extérieur.


\begin{figure}[h]
\centering
\includegraphics[height=6cm,width=9cm,keepaspectratio]{pa150351.jpg}
\caption*{ L'intérieur de la cathédrale est entièrement recouvert de mosaïques.}
\end{figure}

Nous avons aussi fait le tour de notre dame de Kazan, qui tient plus du temple romain que de l'église catholique pour ce qui est des colonnes.


\begin{figure}[h]
\centering
\includegraphics[height=6cm,width=9cm,keepaspectratio]{pa150333.jpg}
\caption*{ Oui, c'est une église, en effet !}
\end{figure}

Et on s'est aussi arrêté dans un restaurant un peu au hasard. Un endroit qui avait l'air un peu moins attrape touriste que les autres, mais bon, ne nous faisons pas trop d'illusions, il y avait quand même des menus en anglais. Et à peine installés, on entend d'autres français dans le restau un peu plus loin. Mais ils changent de table et viennent s'installer sur la table voisine. Ils sont trois, un couple et une amie. La conversation commence :
L'amie : "Alors vous faites quoi ?"
Le couple : "On s'est pris un an pour faire le tour de l'Asie, et on a commencé la semaine dernière par la Russie".
Ben oui, ce genre de coïncidence arrive vraiment. 5mn plus tard, on avait fait table commune et on comparait nos trajets ! (Coucou à Aurélie, Sophie et Gauthier si vous nous lisez).

Le lendemain, nous avons passé une journée entière consacrée à l'Ermitage, le musée emblématique de Saint Petersbourg. Une journée, ce n'est pas de trop. Les livres nous encourageaient à prendre les billets en avance tellement il y a de monde. Et c'est ça qu'il y a de bien en voyageant hors saison : pas de souci dans le genre, on peut tout organiser à la dernière minute, il y a de la place !


\begin{figure}[h]
\centering
\includegraphics[height=6cm,width=9cm,keepaspectratio]{pa150370.jpg}
\caption*{ Et encore, en n'en voit pas le quart sur cette photo.}
\end{figure}

On s'attendait à visiter un musée classique, genre une enfilade de pièces remplies de tableaux et d’œuvres diverses, mais que nenni ! Un des intérêts de l'Ermitage, c'est que c'était un palais. Et un palais possédé par des gens qui voulaient en mettre plein la vue des visiteurs. Je ne vais pas essayer de décrire ces pièces avec des mots. Je n'ai ni le talent, ni le temps. Et même les photos ne rendent pas justice. Il faut y aller et le voir de ses propres yeux. Je dirais juste que l'audioguide a qualifié de "petit" un lustre grand comme notre salle de bain. Je vous laisse imaginer la taille du grand juste à coté.
Et dans toutes les pièces, une mama russe en uniforme qui surveille son coin de salle depuis sa petite chaise. Il y a celle qui fusille du regard si on se penche trop vers un tableau, celle qui ne dors pas-si-si-je-vous-jure, celle qui tient absolument à te donner une double page d'un texte dense en Russe sur l'art Iranien, et celle qui passe des trèèèès longues journées à surveiller une salle vide.


\begin{figure}[h]
\centering
\includegraphics[height=6cm,width=9cm,keepaspectratio]{pa160417.jpg}
\caption*{ Pour une idée de la taille de la salle, cherchez la chaise.}
\end{figure}

On nous a beaucoup vanté les fontaines de Peterhof, donc nous y sommes allés. C'était presque une expédition : on a pris le métro puis le bus en vérifiant toutes les 30 secondes qu'on avait bien pris le bon bus, et on est arrivé à Peterhof. Et alors ce qu'il y a de bien quand on voyage hors saison, c'est que parfois c'est gratuit parce que les fontaines sont fermées... C'est con, on s'était dépêché pour être à l'heure pour l'ouverture des robinets dont on nous avait vanté la cérémonie, et... ben rien. Pas d'eau. Il fait déjà trop froid ! On a donc passé la journée dans un grand parc plein de fontaines éteintes. Et on a jeté une pièce de 10 kopecks dans la botte d'une statue. Les Russes passent leur temps à jeter des pièces dans des endroits improbables, à tel point qu'on a vu des panneaux l'interdisant. En même temps, quand on sait ce que ça vaut 10 kopecks (1,4 cents pour les curieux, pas sûr que ça vaille son poids en métal).
Mais bon, c'était quand même un très joli parc.


\begin{figure}[h]
\centering
\includegraphics[height=6cm,width=9cm,keepaspectratio]{pa170473.jpg}
\caption*{ Les jardins de Peterhof.}
\end{figure}


\begin{wrapfigure}{l}{0.45\textwidth}
\centering
\includegraphics[width=0.4\textwidth]{pa180535.jpg}
\caption*{ La légende dit que 200kg d'or ont été nécessaires pour toutes les dorures.}
\end{wrapfigure}


Enfin, on ne pouvait pas partir de Saint Petersbourg sans visiter le palais de Catherine dans la ville de Pouchkine. Mais alors, quelle expédition ! Car on a décidé d'y aller en train local. Et on a réussi à acheter des billets de train en parlant seulement Russe ! Là, je peux vous dire qu'on était fiers comme des paons. On avait vraiment l'impression d'être des aventuriers.


Et ce palais, comment dire... Non, vraiment, je n'ai pas les mots. Et je n'ai pas toutes les photos non plus. Une enfilade de salles toutes plus magnifiques les unes que les autres, avec la salle d'ambre en point d'orgue (photos interdites, désolés, faudra y aller vous même).
C'est ... pffff... non mais allez-y, vraiment. Ça vaut le voyage !



\chapter{Moscou}
On a fait le trajet Saint-Pétersbourg Moscou dans le SAPSAN, une sorte de TGV russe. En plus lent. Mais mieux sur tous les autres points ! 200km/h de vitesse max, pas de quoi défriser un mouton, mais quel confort ! La deuxième classe est largement aussi confortable que la première en France. Pour vous donner une idée de la largeur : les toilettes sont accessibles en fauteuil roulant. Et ils passent même des films, écouteurs fournis.


\begin{wrapfigure}{l}{0.55\textwidth}
\centering
\includegraphics[width=0.5\textwidth]{pa200641.jpg}
\caption*{ La cathédrale Saint Basile, construite en l'honneur d'un fou qui se baladait à poil.}
\end{wrapfigure}

Quand on rentre sur la Place Rouge la première fois, on a vraiment l'impression de rentrer dans l'histoire. C'est grand, c'est solennel, il fait froid. La cathédrale de Saint Basil est plus petite que ce qu'on imaginait, et le Kremlin plus grand.
Et de part et d'autre de la place se font face :
A droite : Le mausolée de Lénine, ainsi qu'un cimetière des grands communistes, dont Staline.
A gauche : Le GUM. Un immense et magnifique centre commercial rempli de boutiques de luxe. Toutes les grandes marques sont présentes.
Vous la sentez aussi, la grosse ironie de l'Histoire ?


\begin{figure}[h]
\centering
\includegraphics[height=6cm,width=9cm,keepaspectratio]{pa200640.jpg}
\caption*{ L'intérieur du GUM}
\end{figure}

Un peu plus loin, au bord de la Moskva, on a eu le souffle coupé en rentrant dans le monastère du Christ Ressuscité. Encore une fois, pas mots, pas de photos. C'est immense, doré, rempli de bougies, de marbre et de mosaïques. Et vraiment, c'est immense. Allez-y !


\begin{figure}[h]
\centering
\includegraphics[height=6cm,width=9cm,keepaspectratio]{pa220685.jpg}
\caption*{ Oui, on fait aussi des photos de "touristes".}
\end{figure}

Le lendemain, le Kremlin : inévitable. Donc le Kremlin (qui à la base veut fortifications, donc il y a des kremlins un peu partout en Russie) c'est une grande muraille de quelques kilomètres qui entoure des bâtiments officiels du gouvernement et des cathédrales ainsi qu'un musée. Bien entendu, il y a un militaire à chaque coin de rue qui surveille ces monceaux de touristes circulant le nez en l'air sans remarquer qu'ils sont entrain de traverser hors des clous. Un coup de sifflet, voir deux ou trois, et tout rentre dans l'ordre.


\begin{figure}[h]
\centering
\includegraphics[height=6cm,width=9cm,keepaspectratio]{pa210673.jpg}
\caption*{ La tour d'Yvan, dans le Kremlin.}
\end{figure}

Et on se fait fouiller à l'entrée bien entendu, et on a du se séparer de nos couteaux suisses. On s'est rendu compte à cette occasion que le saucisson, ça peut aussi se découper à coups de dents (merci Nico Villedieu, c'est ton saucisson qu'on a apporté).
Dans l'Armurerie, un musée comportant des armes, mais pas que, on a vu assez de richesses pour lancer un programme d'exploration spatiale. Des bibles recouvertes de diamants, tous les objets de la vie courante, mais en or, et toutes les décorations les plus kitsch possibles aussi en or. Cela dit, difficile de qualifier tout cela de kitsch une fois qu'on sait que la moindre babiole vaut plus que le PIB d'un petit état africain (estimation à la louche par moi).
Beaucoup d'objets sont des cadeaux offerts par des diplomates en visite, et on sent la conccurence : ce sera à celui qui apportera le cadeau le plus doré et le plus extravagant. J'ai noté un brule-parfum en or en forme de montagne surmontée de châteaux, dont les fenêtres laissent échapper la fumée.
Les ancêtres des Kinder Surprises sont aussi exposés, les fameux œufs de Fabergé. Dont un fantastique qui contient une locomotive et des wagons, en platine et en or cela va sans dire, et dont la locomotive peut être remontée avec une clé ! Un petit cadeau pour fêter l'aboutissement du transsibérien.
On a terminé la journée avec nos nouveaux amis Belges (cf la rencontre improbable à Saint-Pétersbourg) dans un restaurant Géorgien ou on a bu notre première bière depuis notre arrivée en Russie !


\begin{figure}[h]
\centering
\includegraphics[height=6cm,width=9cm,keepaspectratio]{pa210678.jpg}
\caption*{ Nous avec Gauthier et Sophie.}
\end{figure}

Suite au désistement d'un couchsurfeur qui devait nous héberger à Moscou jusqu'à dimanche, on décide de partir vendredi à Ekaterinbourg. Et comme on est jeudi, il faut acheter les billets de trains. On l'a déjà fait en ligne, les billets électroniques, ça marche très bien. Mais là, non, rien ne marche, le site mouline. On a perdu toute la matinée avant de réussir à les acheter en utilisant la version Russe du site grâce à la réceptionniste de l'auberge de jeunesse. A croire que certains jours, seuls ceux qui parlent russe peuvent acheter des billets de train en ligne.
Ensuite, on s'est dit qu'on irait bien visiter le mausolée de Lénine. On regarde les horaires : ca ferme à 13h, et il est 12h15. Oups. Bon, ben demain alors ? Demain, c'est fermé. Ah. Il y a eu un blanc de deux secondes, on s'est regardé, puis on s'est mis à courir. Dans tous les sens dans un premier temps, puis en direction de la place rouge dans un deuxième temps. Et on a vu le corps de Lénine. Mais vraiment le corps de Lénine, momifié et entretenu depuis sa mort il y a plus de 90 ans ! C'est silencieux, en sous-sol, tout de marbre noir recouvert, et on voit d'autres touristes s'arrêter le temps d'une révérence. On en oublie notre course effrénée à travers la ville pour arriver à l'heure.


\begin{figure}[h]
\centering
\includegraphics[height=12cm,width=12cm,keepaspectratio]{pa220694.jpg}
\caption*{ Moi, et Pierre le Grand.}
\end{figure}

Puis le monument dédié à Pierre le grand : une énorme statue d'une pile de bateaux empilés à travers une gerbe d'écume, surmontée d'un navire piloté par un Pierre plus grand encore que le navire qu'il pilote !
Et on fini par manger dans un restaurant sur pilotis avec Frédéric. Il nous a raconté qu'en hiver, l'étang gèle autour du restaurant et se transforme en patinoire. Les gens bien emmitouflés s'installent sur la terrasse, chauffée par des braseros, et regardent les patineurs. A imaginer cette scène, on se croyait dans un conte russe...

\chapter{Ekaterinbourg}
Enfin le jour du départ pour Ekaterinbourg.
C'est la première "vraie" étape du transsibérien pour nous. Départ à 13h de Moscou, arrivée à Ekaterinbourg le lendemain à 16h heure de Moscou, mais 18h, heure locale. Eh oui, pour plus de "simplicité", tous les trains, toutes les gares, tous les billets sont à l'heure de Moscou. Il faut toujours avoir en tête le décalage horaire par rapport à Moscou quand on prend le train.


\begin{wrapfigure}{l}{0.55\textwidth}
\centering
\includegraphics[width=0.5\textwidth]{pa240716.jpg}
\caption*{ Grosse ambiance dans le train.}
\end{wrapfigure}

Dans notre compartiment, on fait la connaissance d'un  médecin russe parti seul faire de la randonnée en Crimée. Notre langue commune : l'allemand, encore une fois. On apprend qu'il est père de 5 enfants, et qu'il adore la nature. Et il insiste pour parler politique, à défendre avec beaucoup de nuances les décisions du gouvernement. Et il nous fait aussi regretter de ne pas nous être arrêtés à Kazan, sa ville natale.
A Kazan, il part, et est remplacé par un autre gars, qui ne parle que Russe. On arrive à comprendre qu'il est dans le train pour 3 jours entiers. On comprend la taille de son stock de provisions ! Provisions avec lesquelles il est très généreux, et il insiste pour qu'on goute à tout. On partage à notre tour, mais sans succès, à chaque fois, il dit non en montrant son énorme sac de provision. Mais à regarder nos voisins, on a vraiment l'impression que le train est un endroit très convivial, et beaucoup d'échanges ont lieu !
Globalement, le wagon est très calme, et nous sommes surpris de ne voir personne boire. Autant pour les préjugés !


\begin{figure}[h]
\centering
\includegraphics[height=6cm,width=9cm,keepaspectratio]{pa240717.jpg}
\caption*{ Notre voisin très généreux avec sa nourriture.}
\end{figure}

En fait, on l'aura appris un peu plus tard : il est interdit de consommer de l'alcool dans le train (sauf celui du wagon restaurant), donc ceux qui se mettent des mines essaient de le faire discrètement.

A Ekaterinbourg, on dort dans une auberge de jeunesse tenue par Vladimir. Il parle anglais, est très sympa et fait tout pour nous faciliter la vie. Il nous trouve un chauffeur pas trop cher pour aller visiter les deux attractions majeures du coin, accompagné de Daniel, un chef cuisiner Italien qui loge également dans l'auberge :
D'abord, la frontière Europe-Asie un peu à l'extérieur de la ville. On s'y arrête quelques minutes le temps de faire les même photos que tout le monde, et voir les rubans laissés sur les arbres par les jeunes mariés. C'est le long de l'autoroute, sur une ligne de séparation des eaux. Le nom du blog commence enfin à avoir du sens !


\begin{figure}[h]
\centering
\includegraphics[height=6cm,width=9cm,keepaspectratio]{pa250726.jpg}
\caption*{ Un pied en Asie !}
\end{figure}

Ensuite Ganina Yama : un monastère fondé à l'endroit où les corps de la famille Romanov ont été retrouvés. Afin de respecter les coutumes locales, les visiteuses se doivent de porter un foulard ET une jupe, le tout gracieusement fourni à l'entrée par un moine barbu.


\begin{figure}[h]
\centering
\includegraphics[height=6cm,width=9cm,keepaspectratio]{pa250744.jpg}
\caption*{ Jupe obligatoire !}
\end{figure}

C'est tout joli et tout bucolique, plein de petites et grandes chapelles réparties au milieu de la forêt de bouleaux (note : en Russie, deux arbres sur trois sont des bouleaux, au bas mot). Mais bon, quand on ne parle pas couramment russe, on ne comprend pas grand chose...


\begin{figure}[h]
\centering
\includegraphics[height=6cm,width=9cm,keepaspectratio]{pa250740.jpg}
\caption*{ Ganina Yama}
\end{figure}

Et voilà en gros pour Ekaterinbourg. On a aussi fait le tour du centre ville, qui est équipé d'une ligne rouge peinte sur le trottoir pour que les touristes sachent ou aller. On va dire que ça occupe...


\begin{figure}[h]
\centering
\includegraphics[height=6cm,width=9cm,keepaspectratio]{pa250751.jpg}
\caption*{ La ligne rouge (et du boulot de fainéasse).}
\end{figure}

Et la boue. On ne vous pas encore parlé de la boue ? Alors il faut imaginer que les petites chutes de neiges fondent à cette époque de l'année, et la neige fondue se mélange avec la poussière, la pollution et forme une couche uniforme d'une belle boue bien lisse qui sert à repeindre tous les véhicules en marron.


\begin{figure}[h]
\centering
\includegraphics[height=6cm,width=9cm,keepaspectratio]{pa270782.jpg}
\caption*{ Jaune devant, marron derrière ! Une photo de Lénine au premier qui chope la référence.}
\end{figure}

Et les chaussures aussi. A moins d'imiter le pas des locaux, qui arrivent à marcher là-dedans assez précautionneusement pour ne pas éclabousser.


\begin{figure}[h]
\centering
\includegraphics[height=9cm,width=12cm,keepaspectratio]{pa270784.jpg}
\caption*{ La boue !}
\end{figure}

On est aussi sortis un peu du centre, pour aller marcher une journée sur une colline voisine. Première sortie nature pour nous depuis deux semaines, on en avait besoin !


\begin{figure}[h]
\centering
\includegraphics[height=9cm,width=9cm,keepaspectratio]{pa260773.jpg}
\caption*{ Oh ! Là-bas, un bouleau !}
\end{figure}

Et la dernière journée, rien de prévu, sauf prendre le train le soir, à 17h30 heure de Moscou. Donc 19h30 heure locale. Alors on glande, on traine, on va faire des courses, et à 17h, on récupère nos bagages. Marion vérifie son billet, puis demande d'une petite voix : "On a bien 4h de décalage avec Moscou ?"
"Non, deux heures."
"Tu es vraiment sûr ?"
"Oui."
"Alors notre train part dans 30mn."
"..."
Oui, notre train était à 17h30 heure locale, pas heure de Moscou. Il était 17h, et on était dans le centre ville. On ne sait toujours pas comment on a fait pour se planter comme ça, mais devant le fait accompli, on ne prend pas le temps de chercher un coupable.
Plein d'espoir, on est parti en courant, des sacs plein le dos. Évidement, c'est pile ce jour là que la police juge judicieux de nous arrêter pour passer nos bagages aux rayons X en entrant dans le métro...
Eh bien, croyez le ou non, mais on a eu notre train, à 1mn. On a presque eu le temps de s'asseoir avant que le train ne parte. Rien de tel qu'une bonne suée avant deux jours de train sans douche !



\chapter{Krasnoyarsk}
Nous partîmes donc pour 39 heures de train, entre Ekaterinbourg et Krasnoyarsk. Deux nuits et une journée entière dans le train. Et ce coup-ci, on a deux couchettes l'une au dessus de l'autre, au lieu d'avoir deux couchettes hautes comme les autres fois. Au moins on peut décider nous même quand on dort et quand on mange. Eh oui, quand on est sur la couchette du haut, on a un accès à la couchette du bas uniquement si son occupant est d'accord. Si l'occupant du bas veut dormir, celui du haut n'a pas le choix, il doit s'allonger lui aussi, et non, on ne tient pas assis en haut !

\begin{wrapfigure}{l}{0.55\textwidth}
\centering
\includegraphics[width=0.5\textwidth]{pa280790.jpg}
\caption*{ Tout le wagon profite de l'odeur !}
\end{wrapfigure}


Et plus on va vers l'est, plus les gens sont curieux. Saint-Pétersbourg est blindé de touristes, même hors saison. Mais dans ce train, tous les regards convergent vers nous. Du moins, dès qu'ils nous entendent parler. Et ils sont tous très sympas et très accueillants, et veulent savoir d'où on vient, où on va, et nous donnent à manger : On a ainsi reçu un demi poisson fumé de la part de nos voisins de compartiment, et ils ont vraiment insisté ! Et ce poisson était vraiment très bon, mais contribuait pour une fraction non négligeable au fumet du wagon. On l'a mangé très vite. Ce coup-ci, personne ne parle anglais, ni même allemand. La conversation est plus difficile, mais on sent qu'on progresse en russe et en langue des signes.


On arrive à Krasnoyarsk le matin, et on se balade au hasard dans la ville pour la découvrir. Un pêcheur nous alpague en anglais, et nous raconte qu'il adore Paris, et qu'il faut qu'il pêche un poisson pour ses deux chats. Une Russe nous dit un petit bonjour timide en français, et s'étonne de voir des touristes en Sibérie en hiver. Et on mange dans un restaurant allemand ! Bon, disons germano-russe. Mais quand même, ils avaient des vraies bières allemandes, et même des belges !


\begin{figure}[h]
\centering
\includegraphics[height=6cm,width=9cm,keepaspectratio]{pa290795.jpg}
\caption*{ Valentin, qui pêche pour ses chats.}
\end{figure}

Bien que le propriétaire de notre auberge propose des tours organisés des montagnes environnantes, on décide d'y aller par nous même, accompagné de Cathy, une allemande voisine de chambrée.


\begin{figure}[h]
\centering
\includegraphics[height=6cm,width=9cm,keepaspectratio]{pa300852.jpg}
\caption*{ Ça ne vous donne pas envie de grimper ?}
\end{figure}

Ces montagnes s'appellent "Stolby", qui veut dire "piliers", car il y a des éperons rocheux au sommet de certaines buttes. Montagnes, c'est un bien grand mot pour une colline de 700m de haut. Mais quand on vient de faire 4000km en train dans une grande plaine, ça fait du bien ! Il y a un peu de neige, la balade est sympa, et de temps en temps, on croise un "stolb", qui s'avère être un bête caillou de quelques mètres de haut, et sur lequel on ne peut plus grimper car il est couvert de glace. Mais c'est joli quand même. Les écureuils et les oiseaux ont quasiment été dressés à sauter dans les bras des promeneurs. Un groupe de jeune s'évertue à nous soutirer des informations du genre "est-ce qu'il fait beau chez vous" à grand renforts de dessins dans la neige. C'est comme jouer au pictionnary, mais sans savoir si on a gagné.


\begin{figure}[h]
\centering
\includegraphics[height=6cm,width=9cm,keepaspectratio]{pa300829.jpg}
\caption*{ Nos compagnons de pictionnary.}
\end{figure}

On y refait même un tour le lendemain, ce coup-ci en montant par le télésiège de la station de ski locale. Et on remarque un couple de vieux venus eux aussi marcher un peu. On les dépasse. On s'arrête pour voir un truc. Ils nous rattrapent. On fait ça encore une fois, et ce coup ci, erreur tragique, on s'arrête devant un panneau. Et là, ils commencent à nous expliquer les chemins, combien de temps, quels cailloux, et pleins d'autres trucs qui ont l'air très importants, à voir l’énergie que ce vieux met dans ses explications. Du coup, on part dans un chemin de traverse voir encore un caillou. Et au retour, ils nous avaient encore suivis... (oui bon, ce n'est pas facile de rendre passionnant une balade dans la forêt, alors on fait avec ce qu'on a).


\begin{figure}[h]
\centering
\includegraphics[height=6cm,width=9cm,keepaspectratio]{pa310880-panorama.jpg}
\caption*{ Panorama de Stolby.}
\end{figure}

Et on apprend quelques trucs sur les populations autochtone dans le musée régional, qui ressemble de l'extérieur à un temple égyptien. Le musée est pas mal, mais on n'a pas vu le rapport avec les égyptiens...

Et le soir, on a fait un excellent restaurant Ukrainien. On était fatigué, et on cherchait un restaurant pas loin de l'auberge. On sort, on part dans la mauvaise direction, on fait le tour du pâté de maison, et on fini par trouver l'entrée... juste en face de l'auberge, il n'y avait qu'à traverser la rue ! Et l'Ukraine, gastronomiquement parlant, c'est quand même assez proche de l'Alsace ! Ils ont même du vrai lard. Pas du lard pollué par de la viande, non, juste le gras, coupé en fines en tranches. Tout le monde n'est pas fan...


\begin{figure}[h]
\centering
\includegraphics[height=6cm,width=9cm,keepaspectratio]{pa310921.jpg}
\caption*{ Une assiette de charcuterie (à peine entamée).}
\end{figure}

Le lendemain... rien. Enfin si, on a rencontré des français qui avaient déjà entendu parlé de nous, via Cathy. Faut dire qu'à cette époque, les touristes sont rares. Et on a réussi à acheter nos billets pour la Mongolie. Et ça a pris une heure à la guichetière, littéralement. Une heure à taper des trucs sur l'ordinateur, à imprimer des bidules, à remplir d'autres bidules à la main, à nous faire signer des papiers en russe. J'espère juste que je ne me suis pas vendu corps et âme à l'administration russe en signant ces papiers !



\chapter{Irkoutsk}
Direction Irkoutsk. Petit trajet en train. 19h, une nuit. On a à peine eu le temps de s'installer qu'on était déjà arrivés. C'est qu'on y prendrait goût à voyager dans ces conditions !
Irkoutsk donc, une des plus grande ville de Sibérie. Connue pour sa proximité avec le lac Baïkal, mais c'est une proximité toute relative : 70km. Même si ici, c'est une paille, difficile d'y aller à pied. On ira donc un peu plus tard.


\begin{figure}[h]
\centering
\includegraphics[height=6cm,width=9cm,keepaspectratio]{pb041101.jpg}
\caption*{ De la neige !}
\end{figure}

Durant notre première nuit sur place a eu lieu la première vraie chute de neige : 20cm d'un coup. On se balade donc toute la journée dans la neige et la ville. Je ne sais pas vous, mais nous, lors des premières neiges de l'année, on est un peu comme des gamins. Et là, on a pleins de nouveaux trucs à voir ! Comment, les gens roulent (comme s'il n'y avait pas de neige), comment les gens s'habillent (ben, comme s'il n'y avait pas de neige, on dirait que les femmes russes ont des chaussures à talons crantés), comment ils font pour déblayer la neige (ben... avec des pelles et des chasse-neige, on espérait des tanks et des lance-flammes, on est un peu déçus).


\begin{figure}[h]
\centering
\includegraphics[height=6cm,width=9cm,keepaspectratio]{pb030948.jpg}
\caption*{ Ok, les chasse-neige sont classiques, mais ils se baladent en troupeaux !}
\end{figure}

Et ce coup-ci, ça ne fond pas ! On dirait que le froid s'est installé pour l'hiver, les prévisions de températures pour les jours à venir oscillent entre -1\textdegree C et -18\textdegree C. On a fait péter les doudounes, les collants, les gros bonnets... et on se demande comment on fera avec les potentiels -30\textdegree C voire -40\textdegree C de la Mongolie.
On passe la journée, puis la soirée, avec Evgéniya, une couch-surfeuse photographe native d'Irkoutsk. Elle nous apprend quelques trucs sur Irkoutsk, mais aussi et surtout sur l'Inde, dont elle est fan !


\begin{figure}[h]
\centering
\includegraphics[height=6cm,width=9cm,keepaspectratio]{pb030987.jpg}
\caption*{ Enfin une vraie bière !}
\end{figure}

Elle nous avait donné rendez-vous devant le "Babr". Au début, j'ai cru à une faute de frappe en lisant babr. Mais non, c'est bien ça, le Babr, animal mythique de Sibérie, ressemblant à un gros tigre, assoiffé de sang. C'est le symbole de la ville, sur le blason de laquelle il apparait. Et non, je n'invente pas le "assoiffé de sang", ça fait partie de la description officielle héraldique.


\begin{figure}[h]
\centering
\includegraphics[height=6cm,width=9cm,keepaspectratio]{pb030961.jpg}
\caption*{ Promis, sous la neige, il y a vraiment un Babr.}
\end{figure}

Les prix baissent avec l'éloignement de l'Europe, donc on mange de plus en plus dehors. On a quand même réussi à manger dans un petit restaurant Bouriate dont la serveuse, Bouriate elle aussi, avait fait des études à Strasbourg !
Alors comme j'en entends au fond dire "C'est où la Bouriatie ?", voici quelques précisions : Les Bouriates sont une ethnie de la Sibérie, vivants principalement autour du lac Baïkal, et qui font de très bons raviolis à la viande.

Plutôt que de faire un énième musée, on opte pour un aquarium spécialisé dans les phoques de Nerpa. Se sont des phoques d'eau douce qu'on ne trouve que dans le lac Baïkal. Et on se retrouve en fait devant un véritable numéro de cirque ! Avec des phoques, mignons, mais alors mignons ! Des grands yeux noirs humides, une petite tête, bien gras, vifs comme tout ! Et les voilà à chanter, enfin disons renifler, peindre des tableaux, jouer de la trompette, faire du foot et du basket, se faire la cour l'un à l'autre, j'en passe et des meilleures.
Bon, ce sont des animaux sauvages maintenus en captivité et dressés. Clairement, on peut douter du bien-être de ces animaux. Mais là, honnêtement, on a envie d'y croire. Je ne sais pas si leurs grands yeux y sont pour quelque chose, mais on a l'impression qu'ils sont contents de nous amuser.


\begin{figure}[h]
\centering
\includegraphics[height=6cm,width=9cm,keepaspectratio]{pb030982.jpg}
\caption*{ Si ça se trouve, c'est même le deuxième étage...}
\end{figure}

En se baladant dans la ville, on voit aussi de nombreuses vieilles maisons en bois. Ces maisons datent du 19ième siècle, à l'époque où la ville s'est fortement développée sous l'action conjointe d'une ruée vers l'or et d'une série d'exil d'opposants au tsar. Et une chose marrante se passe avec ces maisons : le niveau des rues monte, mais pas les maisons, qui se retrouvent petit à petit noyées par les couches successives de gravats, pavés et autres enrobés.

Ensuite, petite variation sur le thème "Je visite la plus belle église de la ville". Ce coup-ci, on la visite ... pendant la messe ! Bon, honnêtement, en entrant, on ne savait pas encore, il n'y a pas de lumière rouge au dessus de la porte avec un panneau "Attention, messe en cours". Donc on entre, et il y a plein de gens debout. Hé oui, ici, la messe, faut la mériter, voire, l'endurer... Bref, on se fait petit, on se met dans un coin au fond et on profite des chœurs qui étaient déjà à l’œuvre. La messe a un coté hypnotisant ici, le prêtre psalmodie quelque chose en rythme, puis la foule des fidèles psalmodie une réponse elle aussi en rythme. Et tout hypnotisés qu'on était, on se rend à peine compte que le prêtre est entrain de faire le tour de la salle en balançant son encensoir dans tous les sens. Et on dirait bien qu'il compte aussi enfumer le fond de l'église, ou nous étions pourtant bien cachés. Et il nous a effectivement enfumés en nous regardant bien dans les yeux. Je sais pas vous, mais moi, un gros barbu qui balance un truc brûlant à 20cm de mon nez tout en me regardant très... sérieusement, ben ça a tendance à ne pas me mettre à l'aise.


\begin{figure}[h]
\centering
\includegraphics[height=6cm,width=9cm,keepaspectratio]{pb030978.jpg}
\caption*{ On n'a pas osé prendre de photos de la messe, donc voici juste l'extérieur, imaginez le reste.}
\end{figure}

Et comme la ville, ça commence à bien faire, on décide d'aller visiter sérieusement le lac Baïkal en allant passer quelques jours sur l'ile d'Olkhon. Promis, on vous raconte tout ça très vite !



\chapter{Lac Baïkal - Île d'Olkhon}
Nous partîmes donc un matin vers l'île d'Olkhon (prononcer "olrrrone", avec un "r" espagnol, et non pas "oh l'con"). Le trajet dure environ 7h en minibus, 300km de piste en terre. Deux choses notables durant le trajet :
Lors de la première pause, il ne faisait "que" -14\textdegree C, et ça nous a semblé vraiment, mais alors vraiment très très froid sur le moment.


\begin{figure}[h]
\centering
\includegraphics[height=6cm,width=9cm,keepaspectratio]{pb051128.jpg}
\caption*{ Le ferry pour l'île.}
\end{figure}

Ensuite, on a pris un petit ferry pour rejoindre l'île. Et ce sont les dernières semaines pendant lesquels ce ferry circule. Quand le lac est gelé, le minibus passe directement sur la glace. Et pendant quelques semaines à l'intersaison, quand les glaces sont en formation ou entrain de fondre, l'île est isolée du reste du monde, à moins de pouvoir se payer un petit coucou. Et pour vous donner une meilleure idée de l'isolement de l'île, sachez que l'électricité n'y est arrivée qu'en 2006, soit 37 ans après le premier homme sur la lune...


\begin{figure}[h]
\centering
\includegraphics[height=6cm,width=9cm,keepaspectratio]{pb102254.jpg}
\caption*{ Isolement -> pas de pollution lumineuse -> photo de la voie lactée :-)}
\end{figure}

On arrive dans une auberge ouverte par un ancien champion de tennis de table soviétique : Nikita homestead. C'est sous l'impulsion de ce type que le tourisme s'est développé sur l'île. Il n'y a que 1500 habitants permanents sur cette île de 70 sur 15km, mais on nous a dit que la population peut passer à 10000 en haute saison.


\begin{figure}[h]
\centering
\includegraphics[height=6cm,width=9cm,keepaspectratio]{pb112291.jpg}
\caption*{ J'aime comme ils recyclent les chaudières.}
\end{figure}

En novembre, je ne vous cache pas qu'on est loin de la haute saison. Nous sommes tranquilles pour profiter de la magnifique auberge, toute en bois sculpté, ainsi que de l'île bien entendu (et si vous avez compris que c'était une île toute en bois sculpté, relisez doucement la phrase).

Le lac n'a pas encore commencé a geler. Il est tellement grand qu'il adoucit la région un peu comme la mer le ferait. Du coup, l'hiver est retardé de quelques semaines, et il fait 5 à 10\textdegree C de plus qu'à Irkoutsk. Il n'empêche que les plages sont complètement gelées, le sable est dur comme du béton, et localement très glissant.


\begin{figure}[h]
\centering
\includegraphics[height=6cm,width=9cm,keepaspectratio]{pb071591.jpg}
\caption*{ Ça commence à geler !}
\end{figure}

La nourriture dans cette auberge était vraiment excellente : menu unique pour tout le monde, et c'était comme manger à la maison, voire, dans le cas de certain plats, comme chez ma grand-mère. Il y a beaucoup de soupes, de ragouts, toujours des crudités, et tous les jours, c'est viande ET poisson. Et le petit-dej... Leur spécialité, ce sont les "blinis", qui sont en fait des crêpes tout ce qu'il a de plus classiques, donc blinis à tous les petits dej, et les Russes les mangent avec du lait concentré sucré. On va apporter cette pratique en France ! Et comme les crêpes ne suffisent pas, on a du porridge tous les matins. Et on a eu un porridge différent pour tous les 7 petits déjeuners pris sur place.


\begin{figure}[h]
\centering
\includegraphics[height=6cm,width=9cm,keepaspectratio]{pb122293.jpg}
\caption*{ Le cuistot et le buffet du petit-déjeuner.}
\end{figure}

On est allé visité le nord de l'île dans une sorte de vieux mini-bus 4x4 soviétique très commun dans le coin. Ça tombe tout le temps en panne, mais c'est assez simple à réparer, et visiblement, ça résiste mieux au froid que le reste.


\begin{figure}[h]
\centering
\includegraphics[height=6cm,width=9cm,keepaspectratio]{pb071541.jpg}
\caption*{ Le 4x4 !}
\end{figure}

Ça nous prend 7h pour faire 70km dans la neige. Parfois, on a l'impression que Sergueï, le chauffeur, se balade vraiment au hasard, et ne tient absolument pas compte des routes, ou du moins des traces, déjà existantes. On se demande si la voiture va tenir jusqu'au bout tellement elle grince à chaque cahot, et des cahots, il y en a ! Mais le seul truc qui arrêtera la voiture sera une congère qui nous obligera à pelleter. Enfin, nous, surtout Sergueï, vu qu'il n'y avait qu'une seule pelle...


\begin{figure}[h]
\centering
\includegraphics[height=6cm,width=9cm,keepaspectratio]{pb071625.jpg}
\caption*{ Au boulot !}
\end{figure}

Depuis la pointe Nord de l'île, on peut enfin apprécier, ou tenter d'apprécier les dimensions du lac. Si face à l'auberge, on voit la rive opposée à 15 km de distance, là, c'est plutôt 300km. Comme l'air est très pur et très sec, on voit bien qu'on n'en voit pas bout. C'est majestueux. Nous restons là ce qui semble une éternité à apprécier le paysage.


\begin{figure}[h]
\centering
\includegraphics[height=6cm,width=9cm,keepaspectratio]{pb071544.jpg}
\caption*{ On se sent tout petit face à ce lac.}
\end{figure}

Et nous ne sommes probablement pas les premiers à sentir cette ambiance particulière : De nombreux arbres sont couverts de rubans multicolores. Mais pourquoi donc ? La réponse est dans le chamanisme ! Pour les shamans bouriates (cf Irkoutsk), ce sont des endroits sacrés, donc il convient d'accrocher des rubans aux branches des arbres en faisant un vœux. Quand la branche en grossissant déchire le ruban, le vœux est réalisé. Facile ! Du coup, là où on comprend moins, c'est quand on voit des arbres morts ou des poteaux plein de rubans tout neufs. Enfin bon, on nous a aussi dit que la plupart de gens ne savent plus pourquoi ils font ça.


\begin{figure}[h]
\centering
\includegraphics[height=6cm,width=9cm,keepaspectratio]{pb061510.jpg}
\caption*{ Tout le monde ne sait peut-être pas à quoi ça sert, mais en tous cas, c'est beau.}
\end{figure}

A propos de ne pas savoir quoi faire, on a aussi été bien embêté quand une shamane nous a donné un verre de lait un jour qu'on se baladait devant le site le plus sacré d'Olkhon : quand on est arrivé, une shamane était en train d'asperger le sol de lait avec une grande cuillère. Très intrigués, nous nous arrêtons pour observer cet étrange rituel. Et là, à notre grande surprise, une autre shamane, toute jeune, vient nous parler quelques instants dans un anglais impeccable. On apprend qu'elle vient d'Oulan-Bator, et qu'on assiste à une offrande aux esprits. Et avant qu'on ait le temps de poser plus de questions, elle nous file un verre de lait dans un gobelet en plastique, et s'enfuit en nous criant qu'elle va prendre des photos comme n'importe quel touriste en voyage. Et nous voilà seuls avec nos questions : on fait partie des esprits auxquels on fait une offrande, où bien sommes nous supposés faire nous aussi une offrande aux esprits, mais sans la cuillère ? Dans le doute, et ne voulant prendre aucun risque, on a bu la moitié, et jeté le reste devant les poteaux sacrés. Vous pensez qu'on a bien fait ?


\begin{figure}[h]
\centering
\includegraphics[height=6cm,width=9cm,keepaspectratio]{pb061344.jpg}
\caption*{ Bon appétit, les esprits !}
\end{figure}

Une nuit, on a été réveillés par des grignotements. Hé oui, les maisons tout en bois, ça offre plein de passages pour tout un tas de rongeurs attirés par notre réserve de nourriture. Si au moins la souris était silencieuse. Mais vous avez déjà vu quelqu'un réussir à tripoter, déchirer, ou marcher dans un sachet plastique silencieusement ? Bref, on change plusieurs fois de suite le sachet de bouffe d'endroit, mais à chaque fois la souris trouve un chemin pour y accéder et nous réveille... Le matin, on signale donc le problème à l'auberge. Leur réponse... comment dire, était à la fois tellement surprenante, et pourtant d'une logique implacable : "Vous êtes allergiques aux chats ?". On ne savait pas trop de quelle façon prendre cette réponse. Et la bonne façon était : sérieusement. Quand on est rentré le soir même, il y avait un petit chat blanc tout paniqué enfermé dans la chambre ! Évidemment, les chats de nos jours, ce n'est plus ce que c'était. Ils passent leur journée à fumer des pétards et à jouer au baby-foot, et la souris était encore là pour nous réveiller. Pour avoir la paix, on s'est résolu à laisser un bout de fromage facilement accessible et rongeable par une souris sans bruit. Avec le recul, j'ai vraiment l'impression qu'on s'est fait manipuler par la souris...


\begin{figure}[h]
\centering
\includegraphics[height=6cm,width=9cm,keepaspectratio]{pb081666.jpg}
\caption*{ Je n'ai pas de photo de souris, alors voici un peu de glace.}
\end{figure}

Une des principales activité à faire sur place reste quand même la balade. Mais cette année, on a eu droit à des avertissement inhabituels pour le coin : attention aux ours ! Normalement, l'île est trop petite pour nourrir un ours. Mais des incendies de forêt juste en face de l'île ont privé les ours de leur nourriture. Du coup, ils ont fait 15km à la nage pour venir sur l'île, et les locaux n'étaient pas encore sûrs que les ours avaient tous commencé à hiberner. Donc il ne fallait pas trop s'aventurer dans la forêt. Je savais déjà que les ours couraient plus vite et grimpaient mieux aux arbres que nous, et maintenant, ils nagent bien mieux que nous...


\begin{figure}[h]
\centering
\includegraphics[height=6cm,width=9cm,keepaspectratio]{pb071523.jpg}
\caption*{ Oui, c'est ça que les ours ont traversé...}
\end{figure}

On a aussi profité du froid pour faire du patin à glace au bord de la plage : de grandes flaques d'eau sur l'herbe ont gelé, et on a été les premiers à faire du patin dessus :-)


\begin{figure}[h]
\centering
\includegraphics[height=6cm,width=9cm,keepaspectratio]{pb081672.jpg}
\caption*{ Sympa comme patinoire.}
\end{figure}

C'est sûr, ce n'est pas pareil que de faire du patin sur le lac Baïkal lui-même, mais nous on avait un chien errant pour nous suivre ! Il a commencé à nous suivre dans le village, et ne nous a pas lâché jusqu'à qu'on soit de retour, et il a même tenté de nous suivre sur la glace. Et s'il y a bien un truc qui fait rire à chaque fois, c'est un animal innocent qui se pète la gueule tout seul !

Enfin, le dernier jour sur place, alors qu'on en parlait depuis notre arrivée, on a trouvé le courage de mettre notre plan à exécution : on a trouvé une plage isolée et... on s'est baigné ! On avait lu quelque part que quiconque se baignerait dans le lac se verrait accorder une bonne santé toute sa vie. Personne sur l'île n'avait l'air au courant de ça, mais ça ne nous a pas dissuadé pour autant !


\begin{figure}[h]
\centering
\includegraphics[height=6cm,width=9cm,keepaspectratio]{pb112279.jpg}
\caption*{ Le rocher du shaman.}
\end{figure}

C'est avec regret, et après avoir prolongé notre séjour le plus possible, que nous avons quitté cette île et la Russie, notre visa arrivant à expiration. 30 jours, ce n'est pas tant que ça en fin de compte...


\begin{figure}[h]
\centering
\includegraphics[height=6cm,width=9cm,keepaspectratio]{pb112288.jpg}
\caption*{ Dernier coucher de soleil sur le lac.}
\end{figure}



\chapter{La curiosité est un vilain défaut, parfois.}
Avertissement: ceci n'est pas pour les enfants ou personnes sensibles.
Mais tout se termine bien.

Après une journée froide, rien de mieux qu'un peu de temps au chaud dans un banya. Particulièrement après que tous ceux que l'on a rencontré nous l'aient conseillé.
C'est une sorte de sauna à la Russe : une petite cabane dont une pièce est chauffée au feu de bois, avec une cuve d'eau froide et un robinet d'eau chaude ainsi que des casseroles pour se rincer.
Tout se passait bien lorsque un petit événement inattendu arriva:
J'étais en train de me rincer lorsque j'entends Gérald (qui se trouvait dans le sauna) crier des noms d'oiseaux.
"Mince il s'est fait mal."
Puis plus un bruit.
"Bon ben ça va c'était pas grave."
Et là, Gérald sort en trombe du sauna et plonge sa tête dans la cuve d'eau froide. Lorsqu'il la ressort... Il avait une partie de la peau du visage qui c'était volatilisée et une autre qui n'était plus tout à fait accrochée comme ça aurait du l'être.
"Oh p***** , oh p***** !!!" (j'étais légèrement paniquée) Je lui dis de vite remettre la tête dans l'eau froide !

Il est resté la tête dans la cuve en attendant que je l'aide à se rhabiller, et grâce à une casserole d'eau froide on a pu retourner à notre chambre.
Après une heure dans l'eau, constatation des dégâts : le nez et la joue gauche n'ont plus d'épiderme, et l’œil gauche est enflé mais pas endommagé.

Pour que vous ayez quelques explications sur ce qu'il c'est passé: et ben Gérald a fait son Gérald. La curiosité le pousse à savoir comment les choses marchent même si cela implique de soulever le couvercle d'un récipient d'eau bouillante et de braises fumantes, alors qu'il est nu... Et pas de bol le couvercle est lourd et lui échappe des mains.

Je lui tartine le visage de Flamazine, lui donne un doliprane pour la douleur, et cours chercher un antalgique un peu plus fort (dont il n'aura même pas l'utilité, quel homme!)

On commence à avoir le ventre qui gargouille : si la faim domine la douleur c'est que ce n'est pas si grave, non ? Dans le restaurant Gérald attire les regards, la crème s'écaille lui donnant un aspect de peau de crocodile à l’œil rouge.

Mes craintes étaient infondées, il dort très bien, mais le lendemain repos et passage à la pharmacie. Ils nous on donné de la bépanthène russe, c'est à dire que ce n'était pas de la crème mais de la mousse ! Un peu surprenant et pénible à étaler, mais efficace.

Dans les jours qui suivirent, la peau autour des brûlures s'est mise à peler, et de la belle peau rose est apparue. Deux semaines plus tard on ne voit presque rien.

Marie-Christine, Jean-Luc, tout va bien. Votre fils est toujours aussi beau !


\begin{figure}[h]
\centering
\includegraphics[height=9cm,width=12cm,keepaspectratio]{pb091702.jpg}
\caption*{ Le lendemain de l'accident.}
\end{figure}


\begin{figure}[h]
\centering
\includegraphics[height=9cm,width=12cm,keepaspectratio]{pb232683.jpg}
\caption*{ Quelques jours plus tard, au frais.}
\end{figure}

\chapter{Mongolie}
Nous partîmes donc pour la Mongolie. En train vu qu'on y a pris goût. Et ce coup ci, pas le choix, 2e classe obligatoire, donc on se retrouve dans des petits compartiments de 4 personnes. C'est moins facile de rencontrer des gens, mais c'est un peu plus confortable que la 3ème classe : on peut enfin s'allonger de tout son long. Nous n'avons qu'une seule voisine de compartiment, une Mongole qui parle aussi Russe. On en profite donc et on prend un petit cours de mongol, et c'est très compliqué à prononcer. Deux semaines plus tard, on ne saura toujours pas exactement comment on prononce merci !
La frontière prend environ 10h à passer, a faire des aller retour en train, remplir des papiers, faire tamponner les passeports, assister à la fouille du train... Et enfin, on entre en Mongolie !


\begin{figure}[h]
\centering
\includegraphics[height=6cm,width=9cm,keepaspectratio]{pb142303.jpg}
\caption*{ Et derrière les buildings, on voit les montagnes !}
\end{figure}

On arrive à 6h du matin dans une ville déserte. On trouve un café avec du wifi, histoire de dire à notre hôte couchsurfeur qu'on est arrivés. Et on se reconnecte aussi à l'actualité, qui n'est pas très glorieuse en ce vendredi 13 novembre (les attentats de Paris). On encaisse le coup. Ça fait bizarre de se sentir plus en sécurité à 10000 km de chez soi...

A 9h, on est chez Naraa et Tunga pour le petit-déjeuner. Et on se sent tout de suite à la maison. Ils viennent d'avoir un bébé, et vivent aussi avec la grand-mère de Tunga. Tunga est une excellente cuisinière qui nous apprendra à cuisiner une spécialité mongole : des beignets au mouton. On lui apprendra à faire de la mousse au chocolat, après avoir bien galéré pour trouver une bête tablette de chocolat.


\begin{figure}[h]
\centering
\includegraphics[height=6cm,width=9cm,keepaspectratio]{pb142312.jpg}
\caption*{ C'est nous qui l'avons fait, enfin, surtout Tunga.}
\end{figure}

Comme à chaque fois qu'on arrive dans une grande ville, on commence par se balader à pied. Il fait froid, mais beau... mais froid quand même ! Le centre-ville est plein de gens et de voitures. Il y a des bouchons partout, ce qui est assez ironique quand on sait que la Mongolie est le pays avec la plus faible densité de population du monde : 1,4 habitants au km\textsuperscript{2}. Mais la moitié de la population du pays est concentrée sur une toute petite surface, alors forcément, ça pose des problèmes...

Tout le pays voue un culte autour de Gengis Khan (ou Chinggis Khan). Sa statue en souverain trône devant le bâtiment du gouvernement, sur la place qui porte son nom, il a sa tête sur tous les gros billets (et les petits billets, ce sont des chevaux), et on lui a construit une autre statue monumentale de lui à cheval, dans un grand parc touristique en devenir en banlieue d'Oulan-Bator. Je ne sais pas vous, mais ce que j'ai retenu de mes cours d'histoire sur Gengis Khan, c'est surtout le coté envahisseur sans pitié. Pas le coté "J'ai conquis et unifié le plus grand royaume de l'Histoire, mis en place des routes commerciales à travers tout le continent, garanti la liberté religieuse, unifié le système de taxe et mis en place une zone de libre échange, et inventé l'immunité diplomatique à une époque où il était plutôt de coutume de torturer sur la place publique les messagers de l'ennemi". La vérité est probablement quelque part entre les deux...


\begin{figure}[h]
\centering
\includegraphics[height=6cm,width=9cm,keepaspectratio]{pb142304.jpg}
\caption*{ Le bâtiment du gouvernement.}
\end{figure}


\begin{figure}[h]
\centering
\includegraphics[height=6cm,width=9cm,keepaspectratio]{pb152387.jpg}
\caption*{ Devinez qui c'est !}
\end{figure}

Naraa et Tunga nous emmènent une journée dans le parc national de Terelj, pour une première petite rando. Mais finalement, Tunga reste dans la voiture avec le bébé, car il ne fait que -14\textdegree C au milieu de l'après midi. C'est beau, nous sommes seuls, et il y a des chevaux sauvages-mais-pas-tant-que-ça partout sur le trajet. Naraa nous explique que les nomades laissent leurs chevaux se débrouiller seuls pendant l'hiver. Les chevaux restent naturellement en troupeaux, et les nomades se contentent de les surveiller de loin, histoire de savoir vaguement dans quels coins ils sont. C'est au printemps qu'ils doivent de nouveau les ré-apprivoiser.


\begin{figure}[h]
\centering
\includegraphics[height=6cm,width=9cm,keepaspectratio]{pb152366.jpg}
\caption*{ Un cheval à moitié sauvage.}
\end{figure}

On passe aussi une journée à faire de la paperasserie pour obtenir le visa chinois. Ce jour là, alors qu'on se baladait entre cyber-cafés, photomatons manuels, ambassades et agences de voyage qui comprennent les besoins en faux billets (d'avions) des voyageurs, un truc incroyable nous est arrivé : soudain, une jeune fille mongole nous rattrape, l'air paniquée, et me tend ma liseuse électronique qui aurait vraiment du se trouver alors dans mon sac. Elle nous explique qu'on doit faire attention, qu'il y a des pickpockets partout : Nos deux sacs à dos étaient grand ouverts, ils avaient pris la liseuse dans le mien. Elle les avait alors visiblement agressés en retour pour la reprendre, non sans se prendre des menaces de morts au passage. Et on n'avait rien remarqué ! Entre l'empressement de la jeune fille à aller quelque part (sûrement un job de super héro ou un truc dans le genre) et notre propre incrédulité quant aux 5 dernières minutes, on n'a même pas eu le temps de la remercier convenablement. Je le fais donc ici : Jeune fille héroïque, nous te sommes éternellement reconnaissant. Le monde a besoin de gens comme toi !


Et à peine quelques heures plus tard, c'est au tour de Marion d'engueuler un mec qui avait lui aussi réussi à ouvrir son sac ! Je commence à croire que les filles sont le cauchemar des pickpockets mongols.

\begin{wrapfigure}{l}{0.55\textwidth}
\centering
\includegraphics[width=0.5\textwidth]{pb152319.jpg}
\caption*{ Un vautour pas du tout sauvage.}
\end{wrapfigure}


Au cas où, voici leur tactique : un groupe de deux ou trois grand gars font une masse compacte autour d'un gringalet collé aux basques de sa victime. Protégé des regards, le gringalet ouvre petit à petit le sac et se sert. En cas de pépin, la petite troupe se disperse aussitôt. De plus, il y a ce que j'espère être des légendes urbaines sur la supposée violence des pickpockets : ils se baladeraient avec des petites lames avec lesquelles ils tranchent les yeux des témoins qui l'ouvrent un peu trop, tout en leur disant "Comme tu as des grands yeux !" (S'il vous plait, dites moi que les Mongols ont juste mal compris le petit chaperon rouge).

On a quand même eu beaucoup de chance. On n'a rien perdu, mais au contraire gagné en expérience !

Tout ceci ne nous empêche pas de perdre de vue notre objectif immédiat : organiser avec Naraa un road-trip d'une semaine dans la campagne. Il compte se lancer dans le tourisme à terme, et nous l'engageons pour sa première mission : nous faire découvrir la Mongolie en vivant et voyageant comme un mongol. Coup de bol, on arrive même à trouver un troisième compagnon en la personne d’Élodie. On a alors besoin d'une plus grosse voiture, ce qui rajoute Mandul et son 4x4 Toyota à l'aventure.


\begin{figure}[h]
\centering
\includegraphics[height=6cm,width=9cm,keepaspectratio]{pb182412.jpg}
\caption*{ Une autoroute en bon état.}
\end{figure}

Alors, la Mongolie, c'est grand. Et leurs autoroutes sont plus ou moins l'équivalent chez nous d'une route communale du bout du village, vous savez bien, les derniers 100m où il n'y a plus de maisons, juste avant d'arriver à la forêt. Non, je n'exagère même pas, il y a vraiment des nids de poules sur leurs autoroutes. On y roule entre 40 et 80km/h selon l'état. Et c'est payant, car c'est tout de même beaucoup plus rapide que les pistes. Et parfois, en particulier quand on cherche une yourte d'un nomade, la route c'est simplement : je vais tout droit vers cette montagne en évitant les plus gros cailloux.


\begin{figure}[h]
\centering
\includegraphics[height=6cm,width=9cm,keepaspectratio]{pb182427.jpg}
\caption*{ Oui, c'est une route.}
\end{figure}

Ainsi, après une journée entière de voiture, nous arrivons à la tombée de la nuit au campement d'un cousin éloigné de Naraa.


\begin{figure}[h]
\centering
\includegraphics[height=6cm,width=9cm,keepaspectratio]{pb182436.jpg}
\caption*{ La yourte sous les étoiles.}
\end{figure}

Traditionnellement, l'hospitalité en Mongolie n'est pas tant une faveur qu'une condition de survie dans un pays où le prochain campement est "quelque part" 10 ou 20km plus loin. Mais il est quand même bien vu de faire des cadeaux à l'hôte : pain, bonbons, vodka, pâtes sont toujours les bienvenus (c'est bien entendu différent dans un camp de yourtes pour touristes). Bref, nous voilà assis dans la yourte avec la femme du cousin, à attendre le retour dudit cousin en sirotant un thé. Enfin, je dis thé parce que c'est comme ça qu'on nous l'a présenté, mais c'est véritablement un bouillon de viande salé, mélangé à du lait entier, infusé au thé. Les amateurs de thé seront surpris, mais pour une soupe légère, c'est plutôt pas mal.


\begin{wrapfigure}{l}{0.5\textwidth}
\centering
\includegraphics[width=0.45\textwidth]{pb182450.jpg}
\caption*{ Belles bottes, non ?}
\end{wrapfigure}


Enfin arrive le cousin. Quel moment ! La porte s'ouvre, et une forme bleue entre par la toute petite porte, et se déplie en un robuste gaillard souriant qui prend le temps d'observer tout ce petit monde. Il n'y a pas à dire, on sent que c'est chez lui ! Il porte les habits traditionnels : un long manteau qui descend sous les genoux, un chapeau de fourrure, et ses énormes bottes sont une œuvre d'art.

Arrive alors l'heure du repas. On nous présente un grand plat d'os bouillis. Mais quand on regarde bien, on remarque qu'il reste de la viande sur les os. Alors, avec les couteaux fournis, on commence à attaquer les os. Il y a aussi ce qui semble être de l'estomac (ou tablier de sapeur pour les amateurs) et une sorte de saucisse, mais probablement faite avec le colon, et remplie de gros morceaux. Mais... malgré toute la bonne volonté du monde, c'est dur. Je parle bien de la viande qui est très dure. On coupe des tous petits bouts, et on mâche, on mâche, et on remâche encore. Et pas question de recracher, ce serait impoli devant nos hôtes qui nous logent et nous nourrissent si généreusement. Et ce serait d'autant plus gênant que nous sommes les seuls à manger, et que le chef de famille, sa femme, et leurs 4 autres invités nous regardent avec attention !


Donc on avale tout rond. Devant notre inexpérience manifeste, notre hôte se décide à nous donner un coup de main pour trouver les derniers bons morceaux. On a l'impression d'être des petits enfants nourris par les adultes, mais ça nous permet de manger un bout de moelle qu'on a presque pas besoin de mâcher ! Et alors qu'on pensait que le repas était fini, voilà qu'arrive une soupe à la viande et aux nouilles ! Nous sommes sauvés ! Nous qui croyions que nous allions passer la nuit avec en tout et pour tout 50g de viande dans l'estomac, nous voilà devant un vrai repas !

Après le repas, c'est l'heure du digestif. Pour le prendre, faut aller casser la glace ! Ben oui, il est congelé dans un tonneau dehors. Et ce n'est rien de moins que du lait de jument fermenté ! Une fois l'airag décongelé, notre hôte en rempli une grande tasse, la passe à un de ses invités - toujours de la main droite à la main droite - qui doit ensuite la boire avant de la rendre notre l'hôte. Et on tourne comme ça dans le sens horaire. Pour les hommes, impossible de refuser, c'est dans leur culture. Un homme, ça boit, sinon, c'est une tapette (ceux qui parlent anglais utilisent le mot "pussy"). Heureusement, l'airag passe plutôt bien. J'y trouve une ressemblance avec les bières IPA pour le coté acide/vinaigré, mais adouci par le lait. Pour vous faire une idée, c'est une sorte de mélange de bière, de lait et de jus de cornichon.
Après 3 tournées d'airag, on décide de sortir la poire de la grand-mère de Marion. Elle remporte un grand succès, et on fini la bouteille dans la foulée et la bonne humeur. Leur petit garçon met aussi beaucoup d'ambiance, et alors qu'il n'a que 3 ou 4 ans, et toujours pas de pantalon ni de slip, il s'amuse à attraper tout et n'importe quoi avec un petit lasso. Dont mon bras. Je me prête au jeu jusqu'au moment ou il a voulu me chevaucher. Mais il n'avait toujours pas de slip...


\begin{figure}[h]
\centering
\includegraphics[height=6cm,width=9cm,keepaspectratio]{pb182441.jpg}
\caption*{ L'heure du digestif, ou pas...}
\end{figure}

Nous sommes donc 12 à dormir dans cette yourte. On s'installe serrés les uns contre les autres et on s'endort. On voit même les étoiles à travers le trou au sommet de la yourte !


\begin{figure}[h]
\centering
\includegraphics[height=6cm,width=9cm,keepaspectratio]{pb182454.jpg}
\caption*{ Dodo !}
\end{figure}

Le lendemain matin, c'est dur pour moi et Marion. On est exténués, et impossible d'avaler quoi que ce soit : l'airag a frappé ! On va mettre deux à trois jours avant de reprendre un peu d'appétit, pendant lesquels l'odeur de lait ou de mouton bouilli nous révulsera. Ça ne nous empêche pas d'aller faire un peu de cheval dans la neige et d'assister à une exécution de mouton surprenante : (Âmes sensibles, passez au paragraphe suivant) Ils mettent le mouton sur le dos, et incisent 10cm juste en dessous de la cage thoracique. Puis, ils rentrent une main dans le ventre du mouton, et vont bloquer une artère à la main pendant une 20aine de seconde, jusqu'à ce que le mouton soit pris de tremblements puis s'affaisse, mort. Le tout dure moins d'une minute, presque sans bruit, et sans qu'une seule goutte de sang soit versée.


\begin{figure}[h]
\centering
\includegraphics[height=6cm,width=9cm,keepaspectratio]{pb192465.jpg}
\caption*{ Balade du matin.}
\end{figure}

On assiste aussi à la traite des vaches. Ils ont des vaches normales, qui font 4 à 6L par jour, on ne voit presque pas leurs pis ! Ça change de nos usines sur pattes... On aurait aussi voulu voir la traite des juments, mais ce n'est possible qu'en été. En hiver, les chevaux sont en liberté dans la montagne, et c'est d'ailleurs une de leurs principales activité hivernale que de surveiller la position du troupeau.


\begin{figure}[h]
\centering
\includegraphics[height=6cm,width=9cm,keepaspectratio]{pb192456.jpg}
\caption*{ Il est pas mignon ?}
\end{figure}

On passe le reste de la journée dans la voiture. En fait, on passera beaucoup de temps dans la voiture : la Mongolie, c'est immense, et Naraa nous dira plus tard qu'il un peu sous-estimé les temps de trajet en hiver. On arrive devant un bâtiment qui s'avère être une église chrétienne, et qui nous servira d'abri pour deux nuits, le temps de faire l'aller retour vers un lac magnifique dans le Nord.


\begin{figure}[h]
\centering
\includegraphics[height=6cm,width=9cm,keepaspectratio]{pb202537.jpg}
\caption*{ Enfin au sommet !}
\end{figure}

Après encore une journée entière dans la voiture, on arrive dans un hôtel-restaurant. En fait, c'est une toute petite maison avec trois pièces : le restaurant (une table et 10 chaises), l'hôtel (deux grand lits) et la cuisine. Les toilettes sont à 100m de l'autre coté de la route : une cabane autour d'un trou, comme partout à la campagne. Vous devez vous dire que par -20\textdegree C, ça doit être l'horreur, mais on se dit qu'en plein été, ça doit être bien pire. En hiver au moins, le contenu est gelé !


\begin{figure}[h]
\centering
\includegraphics[height=6cm,width=9cm,keepaspectratio]{pb212584.jpg}
\caption*{ La chambre d'hôtel.}
\end{figure}

On visite le lendemain un immense volcan qui surplombe la plaine. Et signe de l'activité du volcan, on trouve de nombreuses "bouches" qui exhalent un air chaud et légèrement soufré. On les remarque facilement, car une magnifique cheminée de givre se forme autour.


\begin{figure}[h]
\centering
\includegraphics[height=6cm,width=9cm,keepaspectratio]{pb222587-panorama.jpg}
\caption*{ Panorama du volcan.}
\end{figure}


\begin{figure}[h]
\centering
\includegraphics[height=6cm,width=9cm,keepaspectratio]{pb222624.jpg}
\caption*{ Du givre partout.}
\end{figure}

On passe la nuit dans une vraie maison en dur, chez les parents d'un ami à Naraa. C'était la première fois que Naraa les rencontrait, mais ça marche comme ça en Mongolie : le petit frère de l'ami de la belle sœur de ton cousin débarque chez toi un soir avec 4 potes, pas de souci, tu les loges et les nourris sans poser de question ! On découvre alors la tradition du tabac à sniffer. Pour se souhaiter la bienvenue, point de poignée de main ni d'embrassade ni de bise. Non, on s'assoie et on échange ses fioles de tabac en signe de bienvenue. Ce sont des objets précieux, taillés d'un seul tenant dans des pierres fines, avec un bouchon en corail, or et argent. Les premiers prix commencent autour de 1000\euro et peuvent monter très haut ! Les fioles sont effectivement des œuvres d'art. On ouvre alors légèrement le bouchon et on sniffe un coup avec chaque narine. Ça sent un peu les herbes. Voilà, on s'est dit bonjour !


\begin{figure}[h]
\centering
\includegraphics[height=6cm,width=9cm,keepaspectratio]{pb222655.jpg}
\caption*{ C'est après l'avoir tripotée qu'on a appris sa vraie valeur...}
\end{figure}

On visite aussi des sources chaudes. Au milieu de nulle part (bon, OK, en Mongolie, on est toujours au milieu de nulle part), une station thermale s'est construite autour des sources chaudes. Et c'est là qu'on ne comprend pas : c'est fermé en hiver ! Il n'est ouvert qu'en été, quand la température peut dépasser les 35\textdegree C, mais en hiver, quand il fait -20\textdegree C, qui voudrait donc se baigner dans des sources chaudes, voyons... Tout ce qui est accessible, c'est un bassin à l'air libre dont l'eau est assez chaude pour cuire un œuf, donc on ne se trempe pas... Perso, j'ai eu ma dose d'eau bouillante !


\begin{figure}[h]
\centering
\includegraphics[height=6cm,width=9cm,keepaspectratio]{pb222646.jpg}
\caption*{ Voilà ce qui arrive quand on se balade par -20\textdegree C dans de la vapeur d'eau.}
\end{figure}

La nuit suivante se passe dans une toute petite yourte pour touriste. Pas grand chose à dire, si ce n'est qu'un mec est passé pour demander un fusil car il avait vu des loups à moins de 200m des yourtes. La routine on vous dit...

On nous avait promis la plus grande cascade de glace de toute la Mongolie. Perso, j'attendais un truc exceptionnel, pas un truc de 15m de haut. Je pense que le Saut du Gier en Ardèche est plus haut. Mais bon, ça leur suffit pour organiser une compétition tous les ans (elle aura lieu cette année une semaine après notre passage). Les compétiteurs sont assurés en moulinette, et les meilleurs mettent 15s. On devrait appeler du Ice Speed Climbing à mon avis ! La cascade était magnifique quand on l'a vue, car intacte. Après la compétition, elle ne doit plus ressembler à grand chose...


\begin{figure}[h]
\centering
\includegraphics[height=6cm,width=9cm,keepaspectratio]{pb242808.jpg}
\caption*{ La cascade de glace.}
\end{figure}

La vraie découverte se trouvait 100m plus loin en aval de la cascade : quelques chevaux paissaient tranquillement au bord de la rivière. Avec cette chaude lumière, la neige, les mélèzes dorés, la rivière... c'était enchanteur.


\begin{figure}[h]
\centering
\includegraphics[height=6cm,width=9cm,keepaspectratio]{pb242826.jpg}
\caption*{ Sûrement un fan de The Cure.}
\end{figure}

On a aussi fait un petit tour dans l'ancienne capitale de la Mongolie, Karakorum, dont la principale attraction est un ancien monastère bouddhiste. Ça nous a donné un avant-goût du Népal !


\begin{figure}[h]
\centering
\includegraphics[height=6cm,width=9cm,keepaspectratio]{pb242840.jpg}
\caption*{ Un mur et des moutons.}
\end{figure}

Puis, le moment que Naraa et Mandul attendaient depuis le début : la pêche sous glace ! On fait une première tentative de nuit, ils veulent chopper une race de poisson nocturne. Mandul et Naraa étaient un peu réticents à l'idée de nous emmener, car il faisait vraiment très froid. Sur le chemin, on regardait avec attention le thermomètre de la voiture, en espérant le voir passer sous les -30\textdegree C, quand Naraa nous a expliqué que c'était le seuil bas du thermomètre, et qu'en réalité, il faisait probablement entre -35\textdegree C et -40\textdegree C... Et bien une fois arrivés, malgré l'intégralité nos vêtements sur le dos, le fait qu'on bougeait et qu'on allait de temps en temps de réfugier dans la voiture pour tenter de se réchauffer, je peux vous dire qu'on a eu froid aux pieds ! Heureusement, avec le bout du nez, ce sont les seuls endroits qui ont souffert. Et tout ça pour rien, on n'a choppé aucun poisson.


\begin{figure}[h]
\centering
\includegraphics[height=6cm,width=9cm,keepaspectratio]{pb242859.jpg}
\caption*{ Pêche sous glace et sous les étoiles.}
\end{figure}

Après ça, quand on se réveille le lendemain matin dans la yourte, et qu'il fait -20\textdegree C (le feu s'est éteint et il y a un trou au sommet de la yourte je vous rappelle), on trouve ça plutôt agréable !


\begin{figure}[h]
\centering
\includegraphics[height=6cm,width=9cm,keepaspectratio]{pb252932.jpg}
\caption*{ De gauche à droite : une yourte, un bidon de bordel, les toilettes.}
\end{figure}

A propos du froid, quelques effets marrants : Tout gèle. Très vite ! Une heure de balade dehors, et la gourde est gelée. C'est thermos obligatoire si on ne veut pas se contenter de sucer des glaçons ! Les fruits gèlent bien évidemment, et font de la purée quand on les réchauffe. Les briquets à gaz ne marchent plus ! Les bières gèlent aussi (canette obligatoire : ça n'explose pas), mais pas la vodka. Le diesel gèle aussi, mais pas l'essence. Et les orteils de Marion gèlent aussi un peu : elle a eu une petite gelure sans conséquences.

La pêche sous-glace qu'on a faite le jour suivant s'est donc beaucoup mieux passée. On a choppé des poissons ! Et on a aussi eu un petit moment d'incertitude quand on a entendu la glace craquer. Je savais déjà que ça existait, les guides du lac Baïkal en parlent. Alors je m'attendais à entendre "Crac", voire "Craaaac", mais pas "CRRRRAAAAAKKKKKKRRRRBBBBBRAAAAAAAOUMMMMMMRRRRMMMMLLLLLLLL"... Le son, les vibrations viennent de tout le lac en même temps, la glace et tout notre corps vibrent pendant pas loin d'une minute, on a l'impression d'être sur le point d'être avalé par un animal tellement gigantesque qu'il ne remarquera même pas qu'il a humain coincé entre les dents...
Mais ça n'inquiète pas Naraa, donc nous non plus. Ce qui l'inquiète en revanche, c'est cette idiote de voiture qui roule en notre direction. Sur la glace. Si les 20cm d'épaisseur sont largement suffisants pour supporter quelques piétons, pour la voiture, c'est une autre histoire. On décide alors de plier les gaules (c'est la première fois que j'utilise cette expression littéralement !), car il y a un vrai risque que la voiture passe à travers la glace, et si elle le fait proche de nous, ça pourrait mal se passer... Tant qu'on est sur la glace, on espère vraiment que rien ne se passe, mais une fois à l'abri, cette fameuse curiosité macabre reprend le dessus. La voiture poursuit son chemin sous notre regard anxieux en provoquant craquement après craquement jusqu'à disparaitre... à l'horizon.


\begin{figure}[h]
\centering
\includegraphics[height=6cm,width=9cm,keepaspectratio]{pb252937.jpg}
\caption*{ D'abord on fait un trou.}
\end{figure}


\begin{figure}[h]
\centering
\includegraphics[height=6cm,width=9cm,keepaspectratio]{pb252943.jpg}
\caption*{ Puis on agite un ver en espérant que le poisson n'a pas vu qu'il était congelé.}
\end{figure}


\begin{figure}[h]
\centering
\includegraphics[height=9cm,width=12cm,keepaspectratio]{pb252945.jpg}
\caption*{ Enfin, on fait le fier !}
\end{figure}

Ah oui, et aussi, sur les derniers km, on a vu des chameaux se balader tranquilou dans la neige par -20\textdegree C. Et ce sont des chameaux indigènes, complètement adaptés au froid. On vend même des couettes très chaudes en laine de chameau ! Ça change de l'image d’Épinal du chameau dans le Sahara.


\begin{figure}[h]
\centering
\includegraphics[height=9cm,width=12cm,keepaspectratio]{pb252948.jpg}
\caption*{ Un chameau sous la neige, quoi de plus logique ?}
\end{figure}



\chapter{Beijing}
Nous arrivâmes donc à Beijing. Le choc fut rude. Pour mémoire, on venait de Mongolie, un pays complètement vide, avec des paysages à perte de vue, un ciel bleu quasi permanent. Et on arrive dans une agglomération grande comme la Belgique, et contenant l'équivalent de la moitié de la population de la France, le tout en plein pic de pollution.


\begin{figure}[h]
\centering
\includegraphics[height=6cm,width=9cm,keepaspectratio]{pb302971.jpg}
\caption*{ La place Tian'anmen. La moitié visible en fait.}
\end{figure}

C'est difficile à se représenter cette pollution, mais on va essayer quand même : je crois que l'OMS dit que quand le taux de PM2,5 dépasse 25, c'est pas bien. En France, on fait des plans pollution à partir de 80 (footing déconseillé, circulation alternée et tout le tsoin-tsoin). Les mesures de pollution en Chine vont jusqu'à 600, et ce jour là, les appareils de mesure étaient saturés et le bruit courait qu'on était à 750. Mais quand on est arrivés, on ne savait pas tout ça. Tout ce qu'on savait, c'était qu'il y avait vraiment beaucoup de Chinois en Chine, qu'à Pékin, le haut des immeubles disparaissait dans une brume jaune/maronnasse, et que l'air n'avait pas seulement une odeur, mais aussi comme qui dirait... de la texture, oui, une sorte de texture épaisse et rapeuse. Beaucoup de gens portent des masques. Différents types de masques : le vrai masque anti-pollution, qui épouse bien le visage, avec filtre à particules et valves d'expiration. Il y a aussi le masque en laine fantaisie, avec des animaux tricotés, dont l'efficacité nous semble incertaine, mais pourquoi pas, il y a peut-être un filtre sous la laine... Et enfin, il y a le masque que je qualifierais de "psychologique" dans ces conditions : un bête masque anti-postillon, celui que les soignants et les malades mettent pour éviter de refiler leurs germes aux autres, mais qui n'a certainement aucun effet pour éviter à la pollution d'être inspirée. Et on s'est renseigné : les pharmacies vendent effectivement ces masque en disant que ça protège de la pollution (dont les constituants les plus dangereux sont des particules de moins de 2,5\textmu{}m) tout en ayant des espaces de 1cm de chaque coté. Je ne cesse de m'étonner de la propension d'une partie de l'humanité à profiter de la crédulité de l'autre. Promis, un jour, j'ouvrirai les yeux et j'arrêterai de vivre dans un monde de Bisounours.


\begin{figure}[h]
\centering
\includegraphics[height=6cm,width=9cm,keepaspectratio]{pc013043.jpg}
\caption*{ Des tas de Français !}
\end{figure}

On rencontre Vaï et Guillaume, deux Français qui dorment aussi dans notre dortoir. Ils sont à la fin de leur voyage en Chine, ce qui fait d'eux une source de renseignements précieuse ! On tente un premier restaurant chinois (oui, bon, à partir de maintenant, tous les restaurants sont des restaurants chinois). Évidemment, pas de carte en anglais, pas d'images sur le menu, et le serveur ne parle pas anglais. La solution qu'on a trouvée sur le moment, c'est notre voisin de table qui parle un peu anglais : on a réussi à lui faire comprendre qu'on lui faisait entièrement confiance pour commander pour nous 4. Et ce fut une réussite ! On a mangé super bon pour pas cher : 15\euro pour 4, bières comprises. La bouffe en Chine, on va le découvrir, est très variée, aussi bien en terme de contenu de l'assiette que de prix. Histoire de bien nous mettre dans le bain, on passe rapidement devant une série de petit stands qui offrent tout un tas de brochettes plus délirantes les unes que les autres : poulpe, araignée, scorpion, scolopendre, hippocampe, étoile de mer... Après quelques renseignements, on teste le poulpe, qui est plutôt bon. Les autres, visiblement, sont surtout des attrape-touristes hors de prix : c'est rigolo/exotique de manger une tarentule, mais niveau gastronomique, aucun intérêt.


\begin{figure}[h]
\centering
\includegraphics[height=6cm,width=9cm,keepaspectratio]{pb302975.jpg}
\caption*{ Bon appétit !}
\end{figure}

Un deuxième restau avec Vaï et Guillaume nous amène dans un restaurant islamique (mais quand même chinois). Et ce coup-ci, il y a des images sur le menu, nous sommes sauvés ! On ne remarque pas de grande différence avec un restaurant non-islamique, sauf pour deux choses : il n'y a pas de porc, et les serveuses portent le voile. Par contre, aucun problème avec l'alcool.


\begin{figure}[h]
\centering
\includegraphics[height=6cm,width=9cm,keepaspectratio]{pc012992.jpg}
\caption*{ Je me fond dans la population en adoptant leurs mœurs.}
\end{figure}

Un des passages obligé de Beijing, c'est la place Tian'anmen, juste en face de la cité interdite. Alors un jour de pic de pollution, c'est dommage : on n'arrive pas à voir à l'autre bout de la place, ça gâche un peu... Pareil pour la cité interdite : l'immensité des lieux est gâchée par la visibilité. Mais quand même, c'est grand ! Et pour le coup, on a vraiment l'impression de rentrer dans ces films chinois, ou au moins d'inspiration chinoise genre tigre et dragon ou kung-fu panda.


\begin{figure}[h]
\centering
\includegraphics[height=6cm,width=9cm,keepaspectratio]{pc013004.jpg}
\caption*{ La citée interdite.}
\end{figure}


\begin{figure}[h]
\centering
\includegraphics[height=6cm,width=9cm,keepaspectratio]{pc013026.jpg}
\caption*{ Le degré d'importance des bâtiments est indiqué par le nombre d'animaux sur les arrêtes.}
\end{figure}

On découvre aussi qu'on est pas les seuls touristes, mais que la plupart des touristes sont chinois. Et que beaucoup des touristes chinois viennent de coins de la Chine où très peu d'Occidentaux se rendent. Et que quand ils visitent Beijing pour la première fois, non seulement ils voient la Cité Interdite pour la première fois, mais ils voient aussi des blancs pour la première fois : bref, on se fait dévisager, et mitrailler sans cesse, nous volons la vedettes aux autres attractions... Le plus polis/courageux, viennent nous demander s'ils peuvent nous prendre en photo avec eux. Ils ne veulent rien de plus, prennent leur photo, et repartent sans savoir nos prénoms ou d'où on vient. Et visiblement, pour le Chinois moyen (le pékin moyen, mouarfarfahah !) c'est la classe de montrer une photo de lui avec des blancs ! On accepte toutes les demandes, mais on veut une photo aussi, donc on fera un album de tous les Chinois qui nous ont demandé une photo !


\begin{figure}[h]
\centering
\includegraphics[height=6cm,width=9cm,keepaspectratio]{pc013001.jpg}
\caption*{ Ce vase est le summum de la technique de porcelaine chinoise. Une copie est à vendre pour 10000\euro.}
\end{figure}

Notre troisième jour à Beijing a été surprenant : le vent s'était levé dans la nuit et avait balayé toute la pollution : on voyait le ciel bleu et le soleil ! On est donc allé visiter le palais d'été. C'est un ensemble de palais dans un parc un peu à l'extérieur de la ville (enfin, c'était à l'extérieur de la ville il y a 200 ans, maintenant, c'est accessible en métro). Et tout est artificiel : Bien entendu, les bâtiments sont artificiels, mais aussi le lac et la montagne. Ils ont creusé un lac immense, et fait une montagne avec les débris, puis construit des palais sur la montagne. C'est démesuré, beau, et on remercie le ciel d'avoir chassé la pollution pour nous ce jour là !


\begin{figure}[h]
\centering
\includegraphics[height=6cm,width=9cm,keepaspectratio]{pc023055.jpg}
\caption*{ Le palais d'été... et du ciel bleu !}
\end{figure}


\begin{figure}[h]
\centering
\includegraphics[height=6cm,width=9cm,keepaspectratio]{pc023094.jpg}
\caption*{ La palais est désormais en pleine ville.}
\end{figure}


\begin{figure}[h]
\centering
\includegraphics[height=6cm,width=9cm,keepaspectratio]{pc023111.jpg}
\caption*{ Oui, c'est un bateau en marbre.}
\end{figure}


\begin{figure}[h]
\centering
\includegraphics[height=6cm,width=9cm,keepaspectratio]{pc023307.jpg}
\caption*{ C'est beau hein ?}
\end{figure}


\begin{figure}[h]
\centering
\includegraphics[height=6cm,width=9cm,keepaspectratio]{pc023308-panorama.jpg}
\caption*{ Encore une, parce que vraiment, c'est beau !}
\end{figure}

On découvre aussi les patates cuites dans la rue dans des petits poêles au charbon. On pensait que c'étaient des patates standards, mais en fait non, ce sont des patates douces super bonnes !


\begin{figure}[h]
\centering
\includegraphics[height=6cm,width=9cm,keepaspectratio]{pc023316.jpg}
\caption*{ Manger !}
\end{figure}

Le métro à Beijing est aussi une surprise pour nous : tout neuf, tout moderne, on a presque l'impression de rentrer dans un hôpital tellement tout est carrelé et lumineux. Le seul problème est qu'aucun des automates de vente de ticket n'accepte nos billets. Soit il y a un problème de calibration, soit on a un gros stock de faux billets à écouler ! Autre surprise : tous les bagages, tous les sacs, sont passés au scanner à chaque entrée dans le métro, sans exception. Le tout est opéré par des agents qui ne pourraient mieux incarner l'ennui. Non, sérieusement, vous imaginez ? Assis toute la journée à regarder une émission vraiment chiante, avec pas un seul retournement de situation, et comme point culminant de la journée, un mec qui n'a pas vu qu'il avait un couteau dans son sac... Au moins, dans les aéroports, le contenu des sacs est un peu plus varié, mais là, ils voient toute la journée des porte-feuilles et des parapluies.


\begin{figure}[h]
\centering
\includegraphics[height=6cm,width=9cm,keepaspectratio]{pc023320.jpg}
\caption*{ Il y a aussi des gens qui dansent dans le métro.}
\end{figure}

Sur les conseils de Vaï et Guillaume, on part visiter un pan isolé de la fameuse grande muraille. Plutôt que de payer une fortune pour accéder à un bout de mur tellement envahi par les touristes qu'on ne voit plus le sol, on prend des bus locaux pour aller dans un petit village qui a accès à une portion officiellement non ouverte au public. Le mur n'est pas restauré, en accès libre, le ciel est bleu et nous sommes seuls !


\begin{figure}[h]
\centering
\includegraphics[height=6cm,width=9cm,keepaspectratio]{pc033351.jpg}
\caption*{ Le mur... vide !}
\end{figure}

On n'aurait pas pu rêver de meilleures conditions. Par contre, le mur est abimé par endroit, c'est escarpé au point de devoir progresser avec les mains par moment (hein, quelqu'un a dit escalade ?), et la neige tombée quelques jours auparavant forme une couche de glace par endroit. Bref, on met deux heures pour faire deux km entre les difficultés du mur et les pauses paysages (et Kyle qui s'est rajouté à l'aventure avec des petites chaussures de ville pas vraiment adaptées).


\begin{figure}[h]
\centering
\includegraphics[height=6cm,width=9cm,keepaspectratio]{pc033363.jpg}
\caption*{ Ça glisse.}
\end{figure}

On a du mal à croire ce qu'on voit : ce mur rejoint l'horizon des deux cotés, en suivant un chemin compliqué qui descend dans les vallées et remonte sur la crête des montagnes. Il n'y a des créneaux que du coté Mongol, et perso, si j'avais été Mongol, je n'aurais pas su comment attaquer ce mur. La légende dit que le mur n'a jamais servi, mais peut-être était-ce justement parce qu'il était tellement impressionnant que les attaquants ont été dissuadés avant même de commencer ? Retour brutal à la réalité de la Chine pragmatique : a la fin de cette section, le mur a été abattu pour laisser la place à une route. 10m après être descendu du mur, un grand panneau bilingue nous rappelle que cette section est fermée au public. 10m après ce panneau, un petit panneau en bois gravé à la main posé sur le chemin indique selon nos meilleurs experts sinophones : "bla-bla-bla-en-chinois-3-yuans". 10m après ce panneau, une femme nous demande 3 yuans chacun. Oui, c'est aussi ça la Chine !


\begin{figure}[h]
\centering
\includegraphics[height=6cm,width=9cm,keepaspectratio]{pc043420.jpg}
\caption*{ Un moine.}
\end{figure}

Notre dernier jour à Beijing est occupé par le Lama Temple, un des plus grand temple bouddhiste du monde. La plupart des gens qui le visitent (et paient donc l'entrée comme tout le monde) ont l'air d'être des vrais bouddhistes, étant donné qu'ils ont l'air bien au courant des us et coutumes : ils savent comment et où se prosterner, quoi faire avec ces millions de bâtons d’encens.


\begin{figure}[h]
\centering
\includegraphics[height=6cm,width=9cm,keepaspectratio]{pc043400.jpg}
\caption*{ Ils font tous ça.}
\end{figure}


\begin{figure}[h]
\centering
\includegraphics[height=6cm,width=9cm,keepaspectratio]{pc043402.jpg}
\caption*{ Puis ça.}
\end{figure}


\begin{figure}[h]
\centering
\includegraphics[height=6cm,width=9cm,keepaspectratio]{pc043410.jpg}
\caption*{ Et enfin ça.}
\end{figure}

Du coup, j'ai des sentiments contradictoires. D'un coté, je me sens un intrus, comme si j'étais entré par erreur dans une église orthodoxe au milieu de la messe (exemple pris totalement au hasard), mais d'un autre coté, bordel, j'ai payé mon entrée comme tout le monde ! Pendant un moment, je trouve cocasse d'imaginer devoir acheter un ticket pour assister à la messe à Waldighoffen, puis je me rend compte qu'à Waldighoffen, ce n'est pas l'entrée mais la sortie qu'on paye... Une autre chose m'intrigue : devant les brûloirs d’encens, un panneau dit que c'est interdit d'en brûler les jours de pollution et les jours de vents. Étant donné que c'est le vent qui chasse la pollution, je me dis qu'ils ne doivent pas brûler de l’encens tous les jours... Cela dit, le temple est magnifique. L’encens, les moines, l'architecture, les fidèles perdus dans leurs prières, tout ça fait qu'on oublie qu'on est toujours dans le centre de Beijing. Le clou du spectacle est atteint avec le plus grand Bouddha du monde taillé dans un seul morceau de bois : 3 étages, 9m de haut !


\begin{figure}[h]
\centering
\includegraphics[height=6cm,width=9cm,keepaspectratio]{pc043415-panorama.jpg}
\caption*{ Beau morceau !}
\end{figure}

D'ailleurs : je commence à me rendre compte qu'il y a un truc avec le plus grand Bouddha du monde. Il y en a des tas, mais à chaque fois avec une condition : le plus grand debout en intérieur, le plus grand en pierre à l'extérieur, le plus grand couché, le plus grand en lotus... Je me demande combien il y a de Bouddhas le plus grand du monde.
Sur le retour, on fait un saut à la cité des antiquaires, remplies de merveilles en Jade. Hors de prix bien entendu. Je me demande du coup qui est capable de claquer plusieurs millions pour un service à thé, et surtout, est-ce que le domestique (oui, je pense que le mec qui achète cette théière a un domestique dédié au thé) a une assurance responsabilité civile suffisante en cas de bris.


\begin{figure}[h]
\centering
\includegraphics[height=6cm,width=9cm,keepaspectratio]{pc043424.jpg}
\caption*{ Je parie que la théière ne verse même pas bien.}
\end{figure}

Puis, c'est l'heure de prendre le train de nuit ! On nous avait dit d'être au moins 1h, voire plus, en avance à la gare, même en ayant déjà les billets. Ça nous a semblé un peu exagéré, mais on a suivi les conseils. Nous avons bien fait, une heure, ce n'est pas de trop ! La gare est conçue un peu comme un aéroport : il faut tout d'abord faire la queue à l'entrée de la gare, où le ticket et le passeport sont vérifiés, et les bagages scannés. Les accompagnants ne rentrent pas dans la gare. Les couteaux sont interdits en théorie, en pratique, si c'est un petit couteau suisse caché au fin fond du sac, ça peut passer. Mais on a vu des gens se les faire confisquer, et on n'a pas réussi à trouver de règle officielle. On a l'impression que les grands couteaux à lame non repliables sont interdits, et que les petits... ça dépend. Pour le moment, on est passé entre les gouttes. Une fois le sac vérifié, on doit trouver notre salle d'attente, qui est en fait plutôt une salle d'embarquement. Le moment venu, un portail s'ouvre au fond de la salle, et les passagers du train ont quelques minutes pour embarquer, et c'est seulement à ce moment là qu'on a accès au quai. Et je peux vous dire qu'il y a du monde qui rentre dans un train chinois !

Les trains chinois de nuit ont trois classes, presque comme les trains russes. Soft-sleeper : compartiments fermés de 4 personnes, hard-sleeper : compartiments ouverts de 6 personnes, hard-seat : des sièges tout bêtes. Vu les bons souvenirs des trains russes, on a pris du hard-sleeper, en se disant que ce serait similaire à la 3ième classe russe. Bon, en vrai, c'est un peu moins bien. C'est très jouable, mais quand même, c'est moins bien. Les couchettes sont empilées par groupe de 3, donc il y a moins de place pour les bagages. Il n'y a pas de draps propres pour chaque passager, les draps ne sont changés qu'une fois au départ du train, donc si on arrive en cours de route, il est possible qu'on ne soit pas le premier à dormir dedans. Les Chinois sont bruyants. Ça ne les dérange pas d'être à plusieurs avec leur petite enceinte à fond, en plus de la musique passée dans le train. Un nombre certain jettent leur ordures par terre comme ça, pouf, normal quoi. En particulier, ceux qui mangent des graines de tournesol laissent des gros tas de coquilles partout. Les Chinois ont une façon bizarre d'exprimer leur intérêt envers les occidentaux : ils les fixent, l’œil vide, la bouche ouverte, et ne bougent pas ni ne parlent, y compris si on les fixe en retour, jusqu'au moment où on leur fait un petit signe de la main. Là, deux réactions possibles : ils sont contents, sourient, et nous font aussi un petit signe, ou alors ils détournent les yeux, prennent un air détaché et font genre "hein, non, c'était pas moi, très intéressant cette tâche sur la peinture". Il est vrai que la barrière de la langue n'aide pas. Ils ne parlent ni anglais ni allemand ni rien, et nous parlons très exactement 2 mots de chinois. Ça limite les échanges...



\chapter{Pingyao}

\begin{wrapfigure}{l}{0.55\textwidth}
\centering
\includegraphics[width=0.5\textwidth]{pc063507.jpg}
\caption*{Les remparts, de jour.}
\end{wrapfigure}


Nous arrivâmes donc à Pingyao. A 5h du matin. Et à 5h du matin à Pingyao, il y a des hordes de tuk-tuk ! Ils nous assaillent comme il se doit, et nous les repoussons car d'une part, on n'aime pas être assaillis, et d'autre part, l'hôtel est à deux pas de la gare. Mais on se rend vite compte qu'à 5h du matin à Pingyao, c'est calme, sombre et désert. L'éclairage urbain est éteint, et personne ne se promène dans les rues, hormis dans un rayon de 5m autour de la sortie de la gare. A l'intérieur de la ville, on ne voit rien. Oui, l'intérieur, c'est l'intérieur de la muraille. Pingyao est une ancienne ville fortifiée. C'est en gros un carré de 2,5km de coté entouré d'une muraille de 10m de haut. Ce serait une des plus belle et mieux conservée des villes fortifiées du monde.



Et à l'ombre de cette muraille, de nuit, on y voit comme dans un vieux four à pain éteint. Donc on attend l'aube sur un banc un peu à l'extérieur, ce qui nous permet d'assister au réveil de la ville. On voit petit à petit apparaitre des gens qui fond leur footing, des gens qui vont au boulot en chantant, on voit un mec marcher à reculons, et on a l'impression d'entendre des "réveilleurs publiques" ? Aux premières lueurs de l'aube, on s'engage dans la ville déserte et on retrouve notre hôtel.


\begin{figure}[h]
\centering
\includegraphics[height=6cm,width=9cm,keepaspectratio]{pc053430.jpg}
\caption*{Pingyao, à 7h du matin.}
\end{figure}

Au passage, on prend un petit-déjeuner vraiment pas bon (c'est suffisamment rare en Chine pour que ça mérite d'être souligné) et beaucoup trop cher. Heureusement, notre hôtel rattrape le coup. Certes, il fait 7\textdegree C dans la chambre, mais on a, pour la première fois depuis le début du voyage, notre propre salle de bain ! Et la température de la chambre va rapidement remonter grâce au climatiseur, ne vous inquiétez pas.


\begin{figure}[h]
\centering
\includegraphics[height=6cm,width=9cm,keepaspectratio]{pc053463.jpg}
\caption*{On s'y croirait, non ?}
\end{figure}

La ville est vraiment magnifique, on dirait un musée habité. Pas de pub lumineuses, pas de néons, pas de buildings, et pas de voitures ! Le temps s'est arrêté avant la révolution industrielle et la ville a été petit à petit transformée en une sorte de grand parc d'attractions. Tout est en bois et en pierre sculpté. Le sol est pavé et les rues touristiques sont bardées d'échoppes d'artisanat, de restaurants à touristes (chinois) et de musées. L'entrée dans la ville est gratuite, et on peut acheter un billet d'entrée pour l'ensemble des musées de la ville, valable trois jours.


\begin{figure}[h]
\centering
\includegraphics[height=6cm,width=9cm,keepaspectratio]{pc063515.jpg}
\caption*{Pas de doute, c'est bien son signe !}
\end{figure}

Les musées sont en fait des ensembles de maison d'époque (courtyard en anglais). On peut en particulier visiter la première banque du monde ayant inventé le principe du chèque. La ville est en effet devenue très riche grâce à quelques marchants, qui ont profité de leur fortune pour créer leurs propres banques. Et dans cette ville, berceau des banques modernes, cruelle ironie, impossible de trouver un distributeur compatible Visa ou Mastercard...


\begin{figure}[H]
\centering
\includegraphics[height=6cm,width=9cm,keepaspectratio]{pc063531.jpg}
\caption*{Un vieux tout mignon qui joue un peu de guitare quand personne ne lui achète ses crêpes.}
\end{figure}

Nous sommes quasiment les seuls touristes occidentaux. En trois jours, on croisera moins de 10 blancs, et probablement 8 milliards de Chinois. Du coup, ils ne parlent pas un seul mot d'anglais, ni à l'hôtel, ni dans les restaurants. Il y a quand même quelques panneaux avec des explications en anglais, et même quelques fois en français ! Mais il y a tant de choses à dire sur ces traductions que ça fera l'objet d'un article dédié.
On verra aussi le fameux "Mickey Chinois". Qui fait un peu peur. Il a même essayé de me soutirer de l'argent pour avoir pris cette photo :


\begin{figure}[H]
\centering
\includegraphics[height=6cm,width=9cm,keepaspectratio]{pc053434.jpg}
\caption*{Mais quel rapport avec une ville fortifiée chinoise ?}
\end{figure}

J'ai hésité à lui parler de propriété intellectuelle, mais quelque chose me dit qu'en Chine, ils s'en contrefoutent (sans oublier que même moi, je n'y crois qu'à moitié).
On voit aussi que le problème de la pollution n'est pas cantonné aux grandes villes. A l'arrivée, le ciel était bleu. On pensait que c'était normal, on était quand même assez loin des grandes villes. Mais en fait, c'était à cause du vent des jours précédents. Durant nos trois jours à Pingyao, on a vu le ciel s’opacifier petit à petit, pollué par le chauffage et la cuisine au charbon omniprésents. Et l'armée de véhicules électriques n'y change rien (surtout si les centrales électriques tournent au charbon).


\begin{figure}[H]
\centering
\includegraphics[height=6cm,width=9cm,keepaspectratio]{pc063534.jpg}
\caption*{Il y en a un comme ça tous les 10m. Forcément, ça n'aide pas.}
\end{figure}

On a aussi eu un petit aperçu des conditions de travail en Chine. Mieux vaut illustrer par une photo :


\begin{figure}[H]
\centering
\includegraphics[height=6cm,width=9cm,keepaspectratio]{pc063504.jpg}
\caption*{Que fait le CHSCT ?}
\end{figure}

Perso, je trouve déjà que c'est un peu limite. Mais quand on sait que de l'autre coté, c'est juste un chinois tout maigre qui tient la corde à la main, ça devient carrément suicidaire...
Niveau nourriture, on a vite compris qu'il fallait sortir des remparts (enfin vite, le temps d'un petit déjeuner et d'un repas de midi). Et le fait de voir un restaurant rempli de chinois n'est pas un gage de qualité du restaurant : ce sont tout simplement des touristes chinois qui n'y connaissent pas plus que nous. Avec le temps, on apprend à faire la différence entre le touriste chinois et le local chinois, grâce à des indices subtils du genre sac à dos et gros appareil photo. Bref, toujours est-il que hors des remparts, nous trouvons beaucoup de bouffe de rue : des nouilles sautées, des soupes aux nouilles, des sortes de crêpes, des brochettes, du tofu grillé etc. Et aussi des nouilles en soupe : on commence par choisir des brochettes de viande/légumes/tofu/autres trucs bizarre, la cuisinière les cuit au wok en quelques secondes, et rajoute les nouilles et du bouillon. C'est très bon, on n'a pas besoin de parler la langue, et ça coute environ 1,5 \euro par personne.


\begin{figure}[H]
\centering
\includegraphics[height=6cm,width=9cm,keepaspectratio]{pc053478.jpg}
\caption*{Vous êtes sûre de ce que vous faites ?}
\end{figure}

On trouve enfin des trucs un peu sucrés : des sortes de beignets genre churros, qu'on peut saupoudrer au choix de piment ou du sucre (devinez ce qu'on a choisi). Tellement fan qu'on y retournera tous les jours.


\begin{figure}[H]
\centering
\includegraphics[height=5cm,width=9cm,keepaspectratio]{pc053475.jpg}
\caption*{Enfin du sucre (et du gras) !}
\end{figure}


\begin{wrapfigure}{l}{0.55\textwidth}
\centering
\includegraphics[width=0.5\textwidth]{pc0735391.jpg}
\caption*{Cet homme est une mitraillette !}
\end{wrapfigure}

En se baladant un peu plus loin dans les petites rues hors des murailles, un soir, on est absorbé par un cuisinier qui fait des nouilles à la baguette. Il a un plat de pâte à nouille dans un main, une grande marmite d'eau bouillante devant lui, et avec son autre main, il utilise une baguette pour, en un seul mouvement, détacher un long bout de pâte et le jeter dans la marmite. Et il fait ça si vite qu'on a du mal à suivre à l’œil nu. Sans même se concerter avec Marion, on décide de s'asseoir dans ce restaurant. Et on n'a pas regretté. On a vraiment très bien mangé, et quelle surprise lors de l'addition. On demande le prix, et avec les mains, la serveuse nous fait "11", c'est à dire 1,5\euro. Content de payer si peu cher, on lui donne 22, pensant que c'était le prix par personne. Mais non, c'était pour les deux !



Puis, c'est l'heure d'aller à Xi'an (prononcer chie-anne), et le tout, en train à grande vitesse, où plus communément "bullet train", à prononcer "belette trai-inne", et le train belette, ça nous fait beaucoup marrer. Blague à part, c'est un train très confortable, plus encore que nos TGV. Il s'arrête sur le quai pile à l'endroit prévu, pas besoin de faire la moitié du quai en courant pour rentrer dans le bon wagon. Le seul point négatif, c'est la télé. Oui, il y a une télé tous les 5m dans le train, qui passe des pubs en chinois, avec le son, pendant tout le trajet...



\chapter{Xi'an}
Nous arrivâmes donc à Xi'an. Et pour ceux qui n'ont pas lu le précédent article, ça se dit "chie-anne", avec probablement des tons en plus. Elle fait partie des 10 plus grandes villes chinoise avec 8 millions d'habitants, a été pendant longtemps la capitale de la Chine, et abrite un des candidats au titre de 8ème merveille du monde : le mausolée de l'empereur Qin. Et je n'en avais jamais entendu parler avant de commencer ce voyage ! Alors qu'à coté de ça, je connais, bien malgré moi, le prénom du bébé de Kayne West... (Aucun rapport, je cherchais juste l'information la plus inutile possible que je connaisse) (et pour ceux qui ne savent pas, c'est un prénom complètement con, quand on porte le nom West : North...)


\begin{figure}[h]
\centering
\includegraphics[height=6cm,width=9cm,keepaspectratio]{pc083590.jpg}
\caption*{Circulation dans Xi'an}
\end{figure}

 Petit tour rapide des trucs qu'on a vus dans la ville :
\textbf{La muraille autour de la ville.} On peut payer pour aller faire du vélo dessus, mais bon, on avait déjà vu la grande muraille de Pékin et la muraille de Pingyao, donc niveau muraille, on était plutôt bien, pas besoin d'en remettre une dose, on l'a vue vite fait d'en bas.
\textbf{ La Bell Tower.} C'est une tour au centre de Xi'an. Et elle a des cloches, comme son nom l'indique. On peut payer pour aller dedans, mais elle est plus jolie vue de l'extérieur.


\begin{figure}[h]
\centering
\includegraphics[height=6cm,width=9cm,keepaspectratio]{pc103650.jpg}
\caption*{PC103650.jpg}
\end{figure}La Drum Tower de nuit

\textbf{La Drum Tower.} C'est une tour presque juste au centre de Xi'an (la place au centre était déjà prise). Et elle a plein de gros tambours comme son nom l'indique. A notre grande déception, on ne les entend jamais, ce qui nous amène à nous poser la question suivante : "Mais quel intérêt ?". Nous n'avons malheureusement pas de réponse fiable, mais c'est probablement une raison du genre "conservation de reliques, pas les user, blablabla". On peut payer pour aller voir les tambours de près. On a craqué, on aurait pu s'en passer.


\begin{figure}[h]
\centering
\includegraphics[height=6cm,width=9cm,keepaspectratio]{pc1136611.jpg}
\caption*{Aucune oie en vue...}
\end{figure}

\textbf{La grande pagode de l'oie sauvage}. C'est une grande pagode, et comme son nom ne l'indique pas, aucune oie sauvage dans les parages. On peut payer pour rentrer dans le monastère construit autour il y a moins de 10 ans (gros intérêt historique), puis repayer pour monter au sommet et admirer d'un peu plus haut le nuage de pollution jaunâtre qui couvre Xi'an. Heureusement, nous faisons cette visite en compagnie d'Eve, une jeune chinoise ravie de pratiquer son français et de nous présenter un peu sa ville.


\begin{figure}[h]
\centering
\includegraphics[height=6cm,width=9cm,keepaspectratio]{pc113678.jpg}
\caption*{Merci pour la visite, Eve !}
\end{figure}

Il y a aussi la petite pagode de l'oie sauvage, mais on ne s'est pas donné la peine d'aller vérifier la présence d'oies dans le coin.

Sinon, se balader dans la ville est finalement plus intéressant que les monuments. On assiste à une compétition féroce entre plusieurs marques de téléphone et/ou d'opérateurs mobiles : les rues sont envahies de gros bonhommes en plastique contenant des petits bonhommes bien vivants. Ils font des processions devant les boutiques des concurrents, des haies d'honneur pour les consommateurs, et ralentissent beaucoup la circulation sur les trottoirs. Il y a aussi les grands ballons et les scènes temporaires. Les photos ne rendent pas compte du bruit : des haut parleurs poussés au maximum en permanence complètent le tableau.


\begin{figure}[h]
\centering
\includegraphics[height=6cm,width=9cm,keepaspectratio]{pc123694.jpg}
\caption*{Les armées sont en place !}
\end{figure}


\begin{figure}[h]
\centering
\includegraphics[height=6cm,width=9cm,keepaspectratio]{pc123695.jpg}
\caption*{Regardez bien, derrière la publicité, on remarque le symbole de Xi'an, la Drum Tower !}
\end{figure}

La rue musulmane fait aussi partie des sorties obligatoires. C'est une rue, en fait un petit quartier, pleine de street food, et de touristes. Chaque échoppe essaie de se démarquer, souvent en mettant l'atelier directement sur le trottoir, ce qui nous permet de voir des techniques de confections plutôt surprenantes.


\begin{figure}[h]
\centering
\includegraphics[height=6cm,width=9cm,keepaspectratio]{pc083558.jpg}
\caption*{Quoi, tu n'as jamais vu un vrai mec faire de la pâtisserie ?}
\end{figure}


\begin{figure}[h]
\centering
\includegraphics[height=6cm,width=9cm,keepaspectratio]{pc083560.jpg}
\caption*{Ça chauffe !}
\end{figure}

Il y a un monde fou, et pour couvrir le bruit et se faire entendre, les haut-parleurs sont inévitables ! On essaie d'acheter des bananes frites, car ça semble être un des trucs à tester ici. On demande le prix, le mec met d'abord les bananes à frire, puis nous dit "20 yuans". Que n'avait-il pas dit... Marion l'engueule comme il se doit, il ne comprend rien, et reste avec ses bananes sur les bras. 3\euro pour deux bananes, faudrait pas nous confondre avec des lapins de six semaines ! On fait la queue comme tout le monde pour avoir le privilège de manger le fameux hamburger chinois (de la viande d'agneau hachée dans un pain frit). La technique est bien rodée : dès qu'on commence à faire la queue, on doit acheter le bon pour le burger. Ce qui nous incite fortement à rester dans la file même si on attend longtemps. La file est donc toujours bien pleine, ce qui intrigue les autres passants qui se disent que ce burger doit être fantastique pour que la file soit si longue... bref, vous avez compris l'idée. Et oui, le burger était pas mal, bonne quantité de viande, et sans os ! Par la suite, on s'est rendu compte qu'il y en avait dans tous les restaurants, et que ce n'était pas la peine de faire la queue pour en avoir...


\begin{figure}[h]
\centering
\includegraphics[height=6cm,width=9cm,keepaspectratio]{pc123698.jpg}
\caption*{Mais puisque je vous dis que ça passe !}
\end{figure}

C'est aussi à Xi'an qu'on essaie d'envoyer pour la première fois un paquet en France. Trois semaines avant Noël, c'est le bon moment, et on va se débarrasser de quelques vêtements chauds puisqu'on a pris 20\textdegree C en deux semaines. A la poste chinoise, la première étape, c'est l'emballage : impensable d'arriver avec son paquet déjà fait ! Sinon, comment pourrait-ils vérifier le contenu ? C'est donc ce qu'ils font. On est assez confiant, que vont-ils trouver à redire à une paire de collant, des chaussettes en poils de chameau, quelques cartes postales et des petits jouets ? Rien. Ou si peu... Quoi ? Du bois ? Quelle insulte, quel affront fait à la sécurité de leur pays ! Oser envoyer un cerf-volant qui contient facilement de quoi faire 3 cure-dents, non, vraiment, ce n'est pas négociable. Idem pour les jetons de xiangqi. On repartira avec. La deuxième étape de l'envoi est marrante aussi : formulaires à remplir en triple exemplaire, photocopie des passeports, numéros de téléphones des destinataires à fournir obligatoirement etc. Bienvenu dans la maison qui rend fou ! (cf les 12 travaux d'Astérix) Bilan de l'opération : 1h d'efforts pour envoyer deux petits paquets.

On a fait la sortie obligatoire au Mausolée de l'empereur Qin, dit aussi, Armée de terre cuite, ou encore Terracotta Army. On trouve assez facilement le bus public qui nous y amène, c'est quand même à 2h de route, et nous sommes les seuls touristes occidentaux dans le bus. Il nous dépose sur un parking qui est encore à 10mn à pied des caisses. Sur le chemin, nous repoussons plusieurs vagues de guides touristiques anglophones à grands coups de "No, thank you". Mais ils sont tenaces, et tentent de nous convaincre de leur indispensabilité à grands coups de chiffres et de dates. Après les caisses, rebelote pour 10mn de marche et à nouveau quelques vagues de guides. Enfin, nous arrivons sur le site. Il a été découvert par des paysans en train de creuser un puits, forcément au milieu d'un champ. Un grand hall a été construit autour de la zone explorée, un grand rectangle autour duquel on peut se promener. Bien entendu, interdiction d'approcher les statues sans faire partie de l'équipe d'archéologues qui continue le travail d'exploration et de restauration.


\begin{figure}[h]
\centering
\includegraphics[height=6cm,width=9cm,keepaspectratio]{pc093635.jpg}
\caption*{Il y a encore un peu de boulot.}
\end{figure}


\begin{figure}[h]
\centering
\includegraphics[height=6cm,width=12cm,keepaspectratio]{pc093615-panorama.jpg}
\caption*{Mais quand même, c'est impressionnant !}
\end{figure}

C'est grand, mais on a vite fait le tour. Il y a beaucoup d'explications sur les techniques de restauration, mais moins sur comment le mec a pu se convaincre qu'une armée de terre cuite allait pouvoir l'aider une fois mort. Je pense qu'il faudrait enquêter du coté de la guilde des potiers, qui a sûrement réussi à corrompre les conseillers de l'empereur (suivez l'argent, comme on dit).
En conclusion, on est content d'avoir vu ça, c'est un site incroyable, toussa toussa, mais l'impression qu'il m'en reste, c'est surtout celle d'un empereur qui a probablement oublié de profiter de la vie à force de craindre la mort.


\begin{figure}[h]
\centering
\includegraphics[height=6cm,width=9cm,keepaspectratio]{pc083583.jpg}
\caption*{Le cerf-volant, ce n'est pas pour rigoler !}
\end{figure}

Finalement, la chose la plus intéressante qui nous est arrivée à Xi'an, c'est la rencontre avec Scott, un couchsurfer qui nous accueille deux jours chez lui. Il vit dans un nouvel ensemble d'immeubles comme il doit s'en construire un par semaine en Chine : 20 buildings d'une 60aines d'étages, la population d'Annonay dans un petit quartier. Il est spécialisé dans les énergies renouvelables et la transition énergétique mais gagne plus d'argent en faisant des traductions. Il nous apprend qu'au premier étage (donc le rez-de-chaussée pour les Français), il y a souvent des salles de jeux dédiées au majong, équipées de tables mécaniques qui mélangent les jetons et les placent sur la table automatiquement. Les chinois en sont fous, on en voit partout dans les rues, des jeunes, des vieux, des femmes, des hommes, et évidemment, ils jouent de l'argent.


\begin{figure}[h]
\centering
\includegraphics[height=6cm,width=9cm,keepaspectratio]{pc083611.jpg}
\caption*{Une partie de majong.}
\end{figure}

Il m'apprend à jouer à cet autre jeu qu'on voit partout dans les rues : le xiangqi, une sorte de jeu d'échec chinois. On découvre tout un tas de petits restaurants typiques dans les environs en sa compagnie. Quand on sait où chercher, c'est possible de manger très bon pour vraiment pas cher, mais il faut savoir, ou alors au moins lire le chinois (à ce propos, je recommande vivement l'application memrise qui m'a permis d'apprendre juste ce qu'il faut de caractères chinois pour savoir si on commandait un plat avec des nouilles ou du riz). Il raconte aussi la transition de la Chine vers le capitalisme telle qu'il l'a vécue alors qu'il était un enfant. Il vivait avec ses parents dans un appartement d'état d'une seule pièce avec cuisine et salle de bain commune. Toute l'économie était gérée par l’État : le travail, le logement, et ce qu'on avait le droit d'acheter, géré à base de coupons, depuis le sac de riz jusqu'à la télévision.
C'est avec ce témoignage qu'on commence à prendre la mesure du chemin parcouru par la Chine ces dernières années.


\begin{figure}[h]
\centering
\includegraphics[height=6cm,width=9cm,keepaspectratio]{pc123701.jpg}
\caption*{Ça ressemble à un épluche légume tout doux et sans lame. On l'a logiquement baptisé : Le Caresse-Carotte !}
\end{figure}



\chapter{Shanghai}
Nous arrivâmes donc à Shanghai. Et nous sommes accueilli comme des princes chez Charles, et en même temps, c'était en toute simplicité... bref, il est cool. C'est un ami d'un cousin qui vit en Chine depuis plus de 10 ans. C'est un bonheur de le laisser commander pour nous au restaurant ! Fini les interrogations sans fin : Est-ce qu'ils ont bien compris ? C'est quoi qu'on a commandé en vrai ? C'est du piment ou de la tomate ce truc là ? On va même se faire un restaurant japonais à volonté avec table chauffante (teppanyaki). Oui, à volonté. Sushi compris. Boisson comprise ! Pour 30\euro. A Shanghai, ce qui nous a plu, c'est que c'est un peu moins la Chine qu'ailleurs en Chine. Entendons nous bien : c'est la toujours la Chine hein ! Mais pour la première fois depuis qu'on est en Chine, on arrête de nous prendre en photo dans la rue, car on est plutôt commun parmi tous ces expats. On a même trouvé un restaurant savoyard ! Et des bières belges !!! Bref, on fait une pause quelques jours sur les nouilles sautées. Et ça nous fait comme une sorte de plaisir coupable de manger un burger et des frites au milieu de la Chine, comme si c'était honteux de ne pas profiter à fond de toutes les spécialités locales surprenantes... mais c'est bon la honte...


\begin{figure}[h]
\centering
\includegraphics[height=6cm,width=9cm,keepaspectratio]{pc133710.jpg}
\caption*{Shanghai, de nuit.}
\end{figure}

Pour être honnête, on n'a pas fait lourd à Shanghai en une semaine. On a pas mal glandé. On est allé voir un concert live de jazz. On s'est baladé dans la concession française. On est allé voir les lumières de la ville, et on a pris les mêmes photos du Bund que tout le monde. On sait bien que tout le monde a déjà pris cette photo, alors on cherche une autre possibilité, un angle auquel personne n'a pensé, et au bout d'un moment, faut se rendre à l'évidence, ce n'est pas pour rien que tout le monde a pris exactement cette photo et pas une autre.


\begin{figure}[h]
\centering
\includegraphics[height=6cm,width=9cm,keepaspectratio]{pc153758.jpg}
\caption*{Shanghai, le Bund, de nuit.}
\end{figure}

On a donc vu les trois grandes tours de Shanghai. Le fameux décapsuleur (World Financial Center), une autre grande tour tellement pleine de barres qu'Alain Robert a dit qu'il aurait pu la gravir d'une seule main (Jin Mao Tower), et la petite dernière, la Shanghai Tower, 632m, qui est en fait la 3ème plus grande tour du monde, et qui n'était malheureusement pas éclairée au moment où on est passés. Pour la petite histoire, non, le décapsuleur n'a pas un trou pour laisser passer les avions (hahaha), mais, d'après les architectes, pour laisser passer le vent... Genre. Le vent. Je ne saisis pas vraiment pourquoi ils ont besoin de se cacher derrières des pseudos raisons techniques plutôt que de juste avouer qu'ils voulaient faire un truc cool.


\begin{figure}[h]
\centering
\includegraphics[height=6cm,width=9cm,keepaspectratio]{pc143722.jpg}
\caption*{A gauche : le décapsuleur. En haut : la tour pleine de barre. En bas : le petite dernière.}
\end{figure}

En fait, une des raisons qui nous pousse à rester aussi longtemps à Shanghai, c'est le visa vietnamien. C'est une des seules ville en Chine où c'est possible de le faire faire. On essaie de trouver ce qu'il faut fournir comme papier, où est le formulaire à remplir... et rien. On se pointe à l'ambassade avec un passeport, du liquide et une photo d'identité, on revient 5 jours plus tard, et ça y est ! Ça change de la Russie ou de la Chine !


\begin{figure}[h]
\centering
\includegraphics[height=6cm,width=9cm,keepaspectratio]{pc153783.jpg}
\caption*{C'est pour moi qu'elle fait semblant de croire que les fleurs en plastique lumineuses ont une odeur...}
\end{figure}

On se perd un peu dans la ville, on arrive dans un quartier assez calme, propre, avec de jolis lampadaires, et d'un coup, paf, sans prévenir, côte à côte, les concessions de Maserati, puis, Mac Laren, et enfin Rolls Royce. Le genre de quartier où la bonne arrive en Audi... Et dans le genre moins luxueux, on a passé pas mal de temps à trouver le fameux marché du faux. Évidemment, ce n'est pas son nom officiel, ce serait vraiment trop provocateur. Mais tout le monde le connait sous cette appellation. C'est un grand centre commercial, plein de petites boutiques dans lesquelles on peut trouver des montres, des sacs à main, de l'électronique, des jouets, des bijoux. Et il faut négocier ! Bon, il est vrai qu'il faut négocier partout en Chine, mais là plus encore qu'ailleurs. C'est dur au début de négocier ! On ne sait pas quel est le vrai prix. La règle "diviser le prix de départ par 3" est connue des commerçants et donc ne marche pas ! Ils peuvent faire n'importe quoi entre fois deux et fois cent ! Alors on balance des prix un peu au hasard, ce qui nous semble correct pour nous, et on regarde ce qui se passe. Invariablement, le commerçant fait une tête comme si on venait de lui arracher un bras. L'affront le fait suffoquer. En tant que touristes faisant attention au maximum à rester respectueux des us des pays visités, le premier réflexe est de se dire qu'on a fait une bourde, et qu'il va penser que tous les Français sont vraiment des cons. Mais ce serait une erreur, il faut lutter contre ce réflexe, et se moquer gentiment de lui, rester ferme, et se barrer en dernier recours. Si le dernier prix offert laisse une marge au commerçant, il nous courra après. Pour l'anecdote, on est quand même tombé sur une vendeuse qui nous soutenait mordicus que ses boucles d'oreilles étaient serties de vraies perles... au marché du faux... à 5\euro la paire... mais bien sûr ! Elle n'en a pas démordu de toute la négociation, allant jusqu'à gratter une perle avec une lame pour faire de la poudre. On n'a toujours pas compris en quoi ça prouvait quoi que ce soit.


\begin{figure}[h]
\centering
\includegraphics[height=6cm,width=9cm,keepaspectratio]{pc153748.jpg}
\caption*{On trouve ça en plein centre de Shanghai, ça nous donnerait presque le mal du pays !}
\end{figure}


\begin{figure}[h]
\centering
\includegraphics[height=6cm,width=9cm,keepaspectratio]{pc143745.jpg}
\caption*{Toujours Shanghai. Encore de nuit.}
\end{figure}



\chapter{Huangshan, la montagne aux milles marches}
Nous arrivâmes donc à Huangshan. Huangshan est une des montagnes sacrées de Chine. Et visiblement, en Chine, sacré est synonyme de payant et blindé de touristes. Huangshan City est aussi l'autre nom de Tunxi, une ville à plus de 100km des montagnes, qu'il ne faut pas confondre avec Tangkou qui elle est vraiment la ville à coté des montagnes appelées Huangshan, mais Tangkou n'a pas de gare, et donc quand on veut visiter Huangshan, on prend un train pour Huangshan City, mais quand on arrive à la gare, on arrive à Tunxi, puis il faut prendre un bus pour Huangshan, mais on arrive à Tangkou, et enfin on prend un téléphérique pour les montagnes. Bref, il nous a fallu un peu de temps pour démêler tout ça, mais on y est arrivé ! Pour l’anecdote, Tunxi est une ville de plus d'un million d'habitants. Dans le Lonely Planet qui nous sert de guide, cette ville occupe 5 lignes  : elle n'existe que pour prendre le train, et ne mérite pas qu'on s'y arrête ne serait-ce qu'une heure. Je me demande comment le million d'habitants qui la composent le prendraient s'ils l’apprenaient un jour.

C'est la basse saison pour Huangshan. Seul un téléphérique sur deux est en fonctionnement, et on voit bien que la foule de touristes chinois qui nous entourent ne remplissent qu'une toute petite fraction des files d'attente à l'entrée du téléphérique. L'entrée est aussi à moitié prix. Il faut dire que la météo est loin d'être idéale. La couche de nuages est juste au dessus de nos têtes, une bruine froide et continue essaie désespérément de nous faire rentrer nous mettre au chaud sous la couette, mais on tient bon ! Après tout, peut-être que le téléphérique nous fera traverser la couche de nuages, et qu'au dessus, c'est grand ciel bleu ? Et en effet, on traverse rapidement la couche de nuages ! Mais... bon, il y en a une deuxième au dessus... On ne perd toujours pas espoir : peut-être qu'on va aussi la traverser ? On s'en approche, on rentre dedans, et... on est arrivés. Dans le nuage. Il pleut. Il fait froid. Il y a même de la neige et de la glace sur le chemin. Et c'est envahi de chinois.


\begin{figure}[h]
\centering
\includegraphics[height=6cm,width=9cm,keepaspectratio]{pc203811.jpg}
\caption*{Entre les deux couches de nuages.}
\end{figure}

D'ailleurs, on ne comprend pas : c'est très clairement la basse saison, et à vue de pif, il doit y avoir au moins 10 fois plus de gens en haute saison, mais dans ce cas, comment est-ce possible de tous les faire tenir ? Parce que là, c'est embouteillage permanent au sommet de la montagne, alors avec 10 fois plus de gens ? D'ailleurs, le sommet, c'est un peu plus qu'un sommet. Il y a tout un tas de petits sommets et de points de vue reliés entre eux par des chemins. Il doit y avoir au moins une dizaine d'hôtels et une cinquantaine de km de chemins qui n'ont rien à voir avec un sentier de montagne : tout est bétonné et empierré, avec des escaliers et des rambardes partout, d'où le titre de l'article. En réalité, le surnom de cette montagne c'est plutôt "les Monts Jaunes". C'est tellement accessible que les chinoises  se permettent d'y aller en robe et talons aiguille, c'est quand même mieux pour les photos.


\begin{figure}[h]
\centering
\includegraphics[height=6cm,width=9cm,keepaspectratio]{pc203820.jpg}
\caption*{Visibilité au top !}
\end{figure}

Il faut le savoir, les chinois ne font pas du tourisme pour voir des beaux paysages, ils font du tourisme pour ramener de jolies photos et se la péter auprès des amis et des collègues. Ça s'appelle "avoir la face", et c'est surtout une question de montrer qu'on a de la thune, donc ils aiment ce qui brille, et ce qui coûte cher, voire ce qui donne l'impression de couter cher. Un chinois peut tout à fait se prendre en photo devant un hôtel de luxe, histoire d'avoir la face quand il montrera les photos, pour ensuite aller dormir en dortoir. Tout ça pour dire qu'on est dans un des endroits les plus prisés des chinois pour prendre des photos pour montrer qu'ils ont la face. Et il pleut. Du coup, ils achètent tous un poncho jaune jetable, ainsi que le pantalon de pluie, les protège-chaussures et la canne assortis (tant pis pour la belle robe). Nous sommes pris dans une marée jaune qui veut absolument sa photo au bon endroit, quand bien même le paysage est inexistant car il y a une visibilité de 10m. Même ceux qui ont déjà des super vestes de montagne achètent aussi le poncho... On voit trois explications à ce comportement : soit on ne leur a pas expliqué le gore-tex et ils ont acheté la veste juste parce que c'était la plus chère, soit ils ne voudraient pas que les autres chinois pensent qu'ils sont trop pauvres pour s'acheter un poncho jetable, soit elle vient du fake market, et les coutures fondent quand il pleut.


\begin{figure}[h]
\centering
\includegraphics[height=6cm,width=9cm,keepaspectratio]{pc203818.jpg}
\caption*{Un porteur, et des chinois en ciré jaune !}
\end{figure}

On passe donc la journée à marcher doucement dans la pluie et les chinois, en voyant tout juste le bout de nos pieds, en essayant de ne pas trop penser au fait qu'on est en train de passer à côté des plus beaux paysages du monde. Et on part se coucher tôt sans avoir vu le fameux coucher de soleil, plein d'espoir dans la journée du lendemain, qui, selon certaines sources mais pas toutes, devrait être ensoleillée. A 6h du matin, le réveil sonne, et le dortoir se réveille. On a trente minutes pour rejoindre le sommet à coté de l’hôtel, où le lever de soleil promet d'être magnifique. Hélas, nous sommes toujours plongés dans les nuages.


\begin{figure}[h]
\centering
\includegraphics[height=6cm,width=9cm,keepaspectratio]{pc203822.jpg}
\caption*{Mi-figue, mi-raisin devant le lever de soleil.}
\end{figure}


\begin{figure}[h]
\centering
\includegraphics[height=6cm,width=9cm,keepaspectratio]{pc203823.jpg}
\caption*{Mi-raisin, mi-figue devant le lever de soleil.}
\end{figure}

Un peu dégoutés, on retourne se coucher sur les coups de 7h.

Et à 7h20, nos compagnons de dortoir nous re-réveillent pour dire que les nuages se sont dissipés, et que le ciel est grand bleu. Ils avaient raison : grand ciel bleu, mer de nuages en contrebas. Et vous savez le plus beau ? On était lundi, et tous les chinois étaient rentrés chez eux. On a eu la montagne pour nous.

Et maintenant, vous avez mérité quelques photos :


\begin{figure}[h]
\centering
\includegraphics[height=6cm,width=9cm,keepaspectratio]{pc213842.jpg}
\caption*{C'est beau !}
\end{figure}




\begin{figure}[h]
\centering
\includegraphics[height=9cm,width=12cm,keepaspectratio]{pc214178.jpg}
\caption*{Ces paysages ont servi d'inspiration pour les décors d'Avatar.}
\end{figure}


\begin{figure}[h]
\centering
\includegraphics[height=9cm,width=12cm,keepaspectratio]{pc2138381.jpg}
\caption*{C'est vraiment beau !}
\end{figure}


Dans le genre, on ne comprend pas les chinois, il y a le coup des porteurs. On a croisé des porteurs partout sur les chemins, avec tout et n'importe quoi : de la nourriture, de la boisson, des poubelles, des matériaux de construction, et même des pierres ! Ils ont l'épaule déformée par le poids des marchandises, qui doit dépasser 80kg pour certains. C'est sûr que vu les chemins, il n'y aurait guère que l'hélicoptère qui pourrait les remplacer, au prix de sérieuses nuisances sonores. Mais là où on ne comprend pas, c'est que le boulot des porteurs commence tout en bas de la montagne, juste à coté du départ du téléphérique. Ils préfèrent payer des gens une misère, à faire un boulot qui leur détruit la santé, plutôt que d'utiliser la machine toute neuve juste à coté...


\begin{figure}[h]
\centering
\includegraphics[height=9cm,width=12cm,keepaspectratio]{pc2138471.jpg}
\caption*{Non, ce n'est pas du polystyrène peint.}
\end{figure}


\begin{figure}[h]
\centering
\includegraphics[height=9cm,width=12cm,keepaspectratio]{pc2138501.jpg}
\caption*{Une petite pause.}
\end{figure}



\chapter{Hong Kong, Buildings et Dim Sum}


Nous arrivâmes donc à Shenzhen. Oui, je sais l'article s'intitule Hong-Kong, mais on n'entre pas dans Hong-Kong comme ça. On a d'abord passé une nuit à Shenzhen car notre train arrivait trop tard pour qu'on puisse passer la frontière le même jour. Le lendemain, on a passé la frontière, et ça nous change des frontières russes ou chinoises. Un passeport à montrer, pas besoin de visa, pas besoin de payer, et juste de l'autre coté de la frontière, le métro Hongkongais, un des métros les mieux conçus du monde, qui nous amène très rapidement au centre-ville.


\begin{figure}[h]
\centering
\includegraphics[height=6cm,width=9cm,keepaspectratio]{pc244404.jpg}
\caption*{Centre ville de Hong Kong}
\end{figure}

Nous arrivâmes donc à Hong-Kong. Par le métro. Qui est hyper bien fait. Le plus marquant je trouve, ce sont les correspondances : pas besoin de se perdre dans un labyrinthe de tunnels souterrains pour passer d'une ligne à l'autre. Non, la correspondance est toujours le quai en face,on n'a jamais plus de 10m à faire. Et ils ont aussi la fameuse carte octopuss. C'est une carte de paiement sans contact utilisable partout : tous les transports publics, les distributeurs automatiques de cartes sim (oui, ils ont ça aussi), les petits magasins, et certains restaurants. Bref, c'est tellement pratique que même pour 3 jours sur place, ça vaut le coup d'en prendre une chacun ! La taille des pièces de monnaie joue aussi en faveur de la carte. Si on ne fait pas attention à la petite monnaie, on risque vite d'en avoir 2kg dans les poches...


\begin{figure}[h]
\centering
\includegraphics[height=6cm,width=9cm,keepaspectratio]{pc244401.jpg}
\caption*{On voit un peu les décorations de Noël géantes sur certains immeubles.}
\end{figure}

L'ambiance est très différente de la Chine. On se fait nettement moins bousculer dans la rue, quand bien même la rue est bondée. Même tard dans le soir, les files d'attente aux arrêts de bus restent parfaitement alignées, personne ne resquille, tout le monde est calme, et quand le bus arrive, si il est plein, on ne bourre pas dedans comme un âne, mais on attend sagement le suivant.


\begin{wrapfigure}{l}{0.55\textwidth}
\centering
\includegraphics[width=0.5\textwidth]{pc234340.jpg}
\caption*{File d'attente à l'arrêt de bus.}
\end{wrapfigure}

On dort dans le dortoir le plus cher de tout notre voyage. Hong-Kong n'est pas vraiment la destination la plus abordable d'Asie, et ça ne s'arrange pas à Noël... Heureusement, la nourriture est plutôt peu chère comparée au logement. On découvre la spécialité du coin : les dim-sum. Et quand je dis \emph{la} spécialité, en fait, ce sont plutôt \emph{les} spécialités. Les dim-sum, on ne sait même pas vraiment à quoi ça fait référence, ce sont juste plein de petits plats, pour la plupart cuits à la vapeur, mais pas uniquement. Mais tout est extrêmement bon ! Il y a des petits trucs sucrés, salés, sucrés-salés, au poisson, à la viande, et même à la crème anglaise. La commande arrive petit à petit, dans le désordre : ce n'est pas la peine d'espérer que le dessert arrive à la fin. Vraiment aucun respect pour les bonnes manières !


\begin{figure}[h]
\centering
\includegraphics[height=6cm,width=9cm,keepaspectratio]{pc234327.jpg}
\caption*{Ces petits cochons mignons sont fourrés à la crème anglaise.}
\end{figure}

Pour continuer sur le thème de la nourriture,  on a mangé du \emph{congee} pour la veille de Noël. Mais qu'est-ce donc ? On n'en savait rien non plus, quand on est entrés dans le boui-boui à coté de notre hôtel.Pourquoi avons nous choisi ce restaurant ? C'est simple : il était plein de locaux en permanence.  J'avoue comme repas de Noël, on a déjà fait plus festif : c'est une sorte de bouillie de riz, mélangée avec ce qu'on veut : viande, poisson, herbes, épices. Ce n'est pas du tout appétissant, mais il ne faut pas s'arrêter à l'apparence ! C'est en fait plutôt bon, et très nourrissant.  Mais comparé aux photos du repas de Noël qu'on a reçu de la famille, comment dire...


\begin{figure}[h]
\centering
\includegraphics[height=6cm,width=9cm,keepaspectratio]{pc244414.jpg}
\caption*{Joyeux Noël !}
\end{figure}

Et puis, ce n'est pas comme s'ils fêtaient Noël là-bas. En voyant la ville, on peut penser que ça a de l'importance : il y a les mêmes décorations lumineuses dans les rues qu'en Europe, les devantures de magasins sont remplies de guirlandes et de sapins, et il y a même un show lumineux sur le thème de Noël, avec des lutins qui luttent pour apporter les cadeaux à temps (bon, cette partie, c'est peut-être simplement une allégorie de la Chine). Mais pour les Hong-kongais, c'est une journée comme les autres.


\begin{figure}[h]
\centering
\includegraphics[height=6cm,width=9cm,keepaspectratio]{pc2444111.jpg}
\caption*{Les lutins en action.}
\end{figure}

Dès le premier jour, nous rencontrons dans l'auberge de jeunesse un autre français qui habite à Hong-Kong depuis quelques mois. Il nous accompagne directement pour une visite guidée de la ville, avec trajet en ferry, volière géante, et legos star-wars. C'est une ville vraiment surprenante : une rangée de buildings coincés entre un bras de mer et une montagne.


\begin{figure}[h]
\centering
\includegraphics[height=6cm,width=9cm,keepaspectratio]{pc234242.jpg}
\caption*{Dans la volière, les oiseaux ne sont pas vraiment farouches.}
\end{figure}


\begin{figure}[h]
\centering
\includegraphics[height=6cm,width=9cm,keepaspectratio]{pc234217.jpg}
\caption*{Chantier d'extension sur la mer.}
\end{figure}

La ville est on ne peut plus active, les rues grouillent de gens à toute heure, les buildings poussent comme des petits pains, et pourtant la nature est toute proche. En une demie heure de transports publics, on se retrouve au pied du \emph{dragon back trail}, le sentier du dos du dragon, une magnifique rando qui suit la ligne de crête des montagnes de l'île. Et ce coup-ci, contrairement à ce qu'on trouve en Chine, c'est une vraie rando, un petit sentier avec des cailloux et de la boue. Pour la première fois depuis une éternité, on avait l'impression de retrouver de la nature sauvage ! On était tellement contents qu'on s'est mis à galoper sur le sentier en rigolant, complètement ivres d'oxygène.

Si on prend le bon ferry, on peut même visiter l'île de Lamma. Aucun rapport avec ce quadrupède cracheur, ni avec Serge. C'est une île sans voiture ! Il y a des sentiers, des villages de pêcheurs, des plages, et ... une énorme centrale électrique qui alimente tout Hong Kong, ce qui donne des paysages assez surprenants !


\begin{figure}[h]
\centering
\includegraphics[height=6cm,width=9cm,keepaspectratio]{pc244362.jpg}
\caption*{L'île de Lamma, avec la centrale au charbon.}
\end{figure}

Le temps était estival, 25\textdegree C pour le jour de Noël, ça ne nous était encore jamais arrivé. Et après le froid des semaines précédentes, ça faisait vraiment du bien. Au retour, si comme nous vous loupez le ferry, il n'y a plus qu'à attendre le prochain deux heures plus tard en admirant les fruits de mers encore vivants qui servent de cartes aux restaurants !


\begin{figure}[h]
\centering
\includegraphics[height=6cm,width=9cm,keepaspectratio]{pc244383.jpg}
\caption*{Bel appendice !}
\end{figure}


\begin{figure}[h]
\centering
\includegraphics[height=6cm,width=9cm,keepaspectratio]{pc244388.jpg}
\caption*{A voir son regard, on dirait qu'il sait qu'il va être mangé.}
\end{figure}



\chapter{Hainan, comme un parfum de tropiques}
Nous partîmes donc pour l'île de Hainan. Le point le plus au sud de la Chine, et, on l'espère, le plus chaud et ensoleillé de la Chine également ! C'est un peu la Corse de la Chine. Le trajet fut... comment dire... intéressant ! Il n'y avait plus de place en couchette dans le train, étant donné la proximité avec nouvel an, donc on a pris des places en \emph{hard seat} (un siège tout simple), pour un trajet de plus de 20h. Et ça a été long... Déjà, le dossier des sièges est parfaitement vertical, mais il y a quand même un petit appui-tête. Donc c'est impossible de dormir assis sans tomber en avant. Encore faudrait-il pouvoir s'endormir : les lumières sont allumées toute la nuit, et les chinois parlent, s'engueulent, mettent de la musique, regardent des films, toute la nuit durant. On a même vu deux chinois bourrés commencer à se taper dessus. Et la gestion des déchets est assez simple : tout fini par terre, même si une petite poubelle se trouve sur chaque tablette. Bref, on n'a pas beaucoup dormi. Et on est arrivé sur l'île. Dans le train ! Eh oui, le train embarque sur un ferry sans que les passagers aient à quitter leur compartiment. On voit la mer à travers les vitres du train puis les hublots du ferry, et on sent le train tanguer. Qui eut crut qu'il fût possible d'avoir le mal de mer en train ?


\begin{figure}[h]
\centering
\includegraphics[height=6cm,width=9cm,keepaspectratio]{pc294514.jpg}
\caption*{Marion dans le ciel.}
\end{figure}


Une fois arrivés sur l'île, on devait encore prendre un bus pour rejoindre le petit village un peu hors des sentiers touristiques qu'on avait ciblé : Wenchang. Une plage, un hôtel, et une plantation de cocotier, ça nous semblait bien comme programme. Par contre, après avoir un peu galéré, on finit par comprendre qu'il n'y a plus de bus, mais qu'il faut prendre le train. Eh oui, nos informations étaient un peu dépassées : le village était désormais accessible en train à grande vitesse.


\begin{figure}[h]
\centering
\includegraphics[height=6cm,width=9cm,keepaspectratio]{pc294491.jpg}
\caption*{Il y a quatre ans, seul l'hôtel depuis lequel je prend la photo existait, tous les autres bâtiments sont apparus depuis.}
\end{figure}

Bref, le village avait un petit peu changé, et les centres de vacances s'alignent maintenant les uns à coté des autres le long de la plage. On rencontre un couple germano-chinois qui vit sur cette île depuis quelques années et qui nous a raconté tous les changements depuis l'arrivée du train. La Chine, ça change très très vite ! Ils nous donnent aussi une petite leçon de cerf-volant, après nous avoir montré leur spectacle à quatre cerf-volants de toute beauté !


\begin{figure}[h]
\centering
\includegraphics[height=6cm,width=9cm,keepaspectratio]{pc294493.jpg}
\caption*{Il faut voir ça en mouvement...}
\end{figure}


\begin{figure}[h]
\centering
\includegraphics[height=6cm,width=9cm,keepaspectratio]{pc294501.jpg}
\caption*{Si je pratique tous les jours pendant un an, je pourrai ensuite attaquer le vol en formation !}
\end{figure}

Depuis le début de la Chine, on n'avait pas vu beaucoup de vie sauvage, mais on mettait ça sur le compte des grosses villes. Là, on était sur une île assez peu urbanisée, et en se baladant le long d'un massif de fleurs, ça nous a soudain frappé. Depuis un certain temps, on trouvait qu'il y avait un truc bizarre dans l'air, une ambiance inhabituelle, mais on a mis du temps à mettre le doigt dessus : il n'y a pas d'insectes dans les fleurs. Aucun bourdonnement, aucune activité, rien, comme si c'étaient des fleurs en plastique. On a cherché longtemps, et on a bien réussi à trouver une abeille par ci ou par là, souvent mourante d'ailleurs, mais sans plus. On savait déjà qu'à certains endroits de Chine, la pollinisation se faisait désormais à la main, mais constater l'absence des insectes par soi-même, ça fait bizarre.


\begin{figure}[h]
\centering
\includegraphics[height=6cm,width=9cm,keepaspectratio]{pc284467.jpg}
\caption*{Pas d'insectes, mais des \st{bernards l'hermites} \st{bernard-l'hermites} ... un bernard l'hermite et son copain !}
\end{figure}

Ici, nouvel an, tout comme Noël, c'est un truc de touristes, pas vraiment pour les chinois. Mais on arrive quand même à trouver un bar à touristes qui fait un peu la fête ce soir là. Et pour fêter ça comme il se doit, on se fait un burger ! Et c'est bon un burger après un mois de nouilles et de riz ! Le tout en compagnie d'un Camerounais et d'une Russe qui nous ont abreuvés de vin français tout au long de la soirée !


\begin{figure}[h]
\centering
\includegraphics[height=6cm,width=9cm,keepaspectratio]{pc284458.jpg}
\caption*{WAHOOOUUUUUU !}
\end{figure}

Pour fêter la nouvelle année, j'ai aussi décidé de mettre un peu d'ordre dans le bordel que certains audacieux pourraient appeler une chevelure et une barbe. Direction le village, il y aura bien un coiffeur compatissant. Un truc étonnant en Chine, c'est que les coiffeurs semblent être ouverts tout le temps. 


\begin{wrapfigure}{l}{0.55\textwidth}
\centering
\includegraphics[width=0.5\textwidth]{p1034546.jpg}
\caption*{Résultat des opérations ! (Un jour plus tard, dans une autre ville, pour ceux qui reconnaissent le paysage)}
\end{wrapfigure}

C'est le monde à l'envers : en fin de soirée il peut devenir très compliqué de manger un morceau ou boire un coup, mais pour se faire une nouvelle coupe, aucun souci ! On trouve donc ce qui semble être un salon de coiffure normal (oui, parce qu'il arrive aussi que le salon de coiffure soit juste une façade, et on se fait tailler d'autres trucs dedans...), et ils semblent comprendre que je veux me faire couper les cheveux et la barbe. Ils me font signe de suivre une jeune femme dans la salle du fond, elle me fait m'allonger sur un matelas, et c'est parti pour 30mn... de shampoing- massage ! Puis coupe de cheveux par un coiffeur obsessionnel compulsif à l'affut du moindre cheveu qui dépasse. Enfin il s'est intéressé à ma barbe : un petit coup de tondeuse pour commencer, et il a ensuite légèrement humidifié les 2mm de barbe restante et il a sorti une lame de rasoir neuve.

Ce fut un massacre.



Il ne m'a pas beaucoup coupé, mais j'en ai chié... Heureusement que la blouse me couvrait les mains, il n'a pas vu comment je m'agrippais aux accoudoirs à chaque fois qu'il faisait un mouvement. Je me suis mis à transpirer à cause de la douleur, ce qui a au moins eu le mérite de ramollir un peu les poils, mais ce fut quand même très long : 20mn pour me raser. Il galérait tellement qu'il a même sorti une deuxième lame neuve à mi-parcours. Heureusement pour moi et pour lui que, pour un français, je ne suis pas très barbu. Mais comparé au chinois moyen, ça a du lui faire bizarre !

Ça a aussi fait bizarre aux employés de l'auberge. A mon retour, ils ne m'ont pas reconnu et ils m'ont accueilli comme un nouveau client. J'ai joué le jeu avec un grand sourire, jusqu'au moment ils se sont dit que ce grand sourire n'était pas normal, et se sont mis à éclater de rire en réalisant leur méprise !


Sur cette île, on a vu que beaucoup de gens avaient les dents rouges. Mais ce n'est pas un problème d'hygiène dentaire : ils passent leurs journées à mâcher une sorte de petit fruit vert. Je crois que ça s'appelle noix de bétel en français, mais ça demande confirmation. On a voulu goûter, et on a essayé d'en acheter aux marchands dans les rues. Mais ils ont refusé en nous mimant la folie. En effet, c'est un psychotrope euphorisant, d'ailleurs interdit en France, et on dirait que les marchands ont voulu nous en protéger ! Et dire que tous les chauffeurs de taxis carburent à ça...


\begin{figure}[h]
\centering
\includegraphics[height=9cm,width=12cm,keepaspectratio]{coconut.jpg}
\caption*{A défaut de noix de betel, on se gave de noix de coco !}
\end{figure}


\begin{figure}[h]
\centering
\includegraphics[height=8cm,width=9cm,keepaspectratio]{pc274434.jpg}
\caption*{Notre dealer officiel.}
\end{figure}



\chapter{Yangshuo : enfin de la campagne  !}


Nous arrivâmes donc à Yangshuo, après avoir de nouveau pris le train qui prend le ferry. Mais ce coup-ci, fini le \emph{hard seat}, on voyage de nouveau en wagon lit ! Le train nous a d'abord amené à Guilin, une autre de ces villes gigantesques dont personne n'a entendu parlé, et de là, on a de nouveau pris un bus. A propos de ces bus, c'est toujours un peu compliqué de savoir quel bus on doit prendre. On essaie de privilégier les bus publics, mais les compagnies privées (qui sont parfois juste composées d'un bus avec un chauffeur et un rabatteur) font tout pour nous embrouiller. Les rabatteurs attendent à la sortie des gares ferroviaires et à l'entrée des gares routières, et nous parlent en continu pour surtout qu'on ait aucune chance de réfléchir et qu'on monte le plus vite possible dans leur bus. Si on les écoute, leur bus est toujours le seul disponible, et il va bientôt partir. Parfois, ils reprennent même les couleurs et les logos des bus publics. Bien entendu, tout est faux, le bus attend d'être plein pour partir, ce qui peut parfois prendre beaucoup de temps, les tarifs peuvent être beaucoup plus chers, et pour couronner le tout, il peut même arriver que la destination ne soit pas exactement la bonne.


\begin{figure}[h]
\centering
\includegraphics[height=6cm,width=9cm,keepaspectratio]{p1054700.jpg}
\caption*{Je n'avais pas de photo de bus...}
\end{figure}

Bref, on essaie de se concentrer, de les ignorer sans être trop impolis, mais parfois, c'est un peu dur car ils peuvent être très tenaces. Ils ne nous lâchent pas et on a très envie de les envoyer paître. C'est là que le temps passé à lire les guides et à parler aux autres voyageurs paye : on sait ce qui nous attend, on connait les arnaques avant d'y être confrontés, on connait les vrais prix. Si on était vraiment livrés à nous même, on se ferait arnaquer à tout bout de champ ! Tout ça pour dire qu'on arrive à Yangshuo avec le bon bus et à l'heure prévue.


\begin{figure}[h]
\centering
\includegraphics[height=6cm,width=9cm,keepaspectratio]{p1034540-panorama.jpg}
\caption*{On est monté au sommet pour trouver un mur, une porte, et une chinoise qui vendait un ticket pour avoir le droit de voir la vue. Je ne vous raconte pas l'état de Marion...}
\end{figure}

C'est une ville surprenante : selon le standard chinois, c'est minuscule et c'est aussi coincé entre les massifs karstiques. Les routes et les bâtiments se faufilent entre les pics calcaires et l'horizon est constellé de petites collines verdoyantes. Le coté touristique est toujours bien présent bien entendu : quand on longe la rivière connue pour ses paysages, les 500 premiers mètres sont tellement bordés de boutiques que la rivière est invisible. Mais si on a le courage de continuer un peu, on arrive dans une zone complètement différente. D'un seul coup, plus de bâtiments, plus de boutiques, plus de vendeurs : juste la rivière qui zigzague au milieu des rizières et des montagnes. En continuant à se balader au hasard, on a même finit pas se retrouver au milieu des champs de mandariniers. On avait l'impression de trouver pour la première fois en Chine de la vraie campagne !


\begin{figure}[h]
\centering
\includegraphics[height=6cm,width=9cm,keepaspectratio]{p1034561.jpg}
\caption*{Des mandarines !}
\end{figure}

Évidemment, qui dit massif de calcaire dit escalade ! C'est un des endroit les plus connus de Chine par les grimpeurs. Nous avons dormi dans une auberge de jeunesse spéciale grimpeurs : quand on arrive, les premières questions ne sont jamais "Vous venez d'où ?", ou bien "Vous allez où ?" comme c'est le cas d'habitude, mais directement : "Vous avez quel niveau ?". On s'est donc fait une petite journée de grimpe à Swiss Cheese, une falaise couverte de gros trous avec plein de voies faciles et une forêt de bambous au pied. Après 3 mois sans exercice, sur un mur mouillé et avec des chaussons empruntés, on n'a pas été méga performants, mais ça faisait bien plaisir quand même !


\begin{figure}[h]
\centering
\includegraphics[height=6cm,width=9cm,keepaspectratio]{p1044577.jpg}
\caption*{J'aurais volontiers pris une photo de Marion en train de grimper, mais elle était réticente à l'idée que je lâche la corde.}
\end{figure}

Évidemment, qui dit massif de calcaire dit grottes ! C'est une des activités classiques de Yangshuo. Tant qu'à faire, on a choisi une grotte connue pour ses bains de boue et ses bains d'eau chaudes. En France, quand on visite une grotte, souvent, le guide fait en même temps un cours de géologie, et nous explique la formation de la grotte, des stalactites, etc. Ici, pas du tout, la visite s'apparente à une observation de nuages par des maternelles : ici, vous pouvez voir un cochon bourré, là c'est un chameau, et enfin au fond à droite, c'est un Bouddha. On voit surtout des bouddhas et des animaux, et de temps en temps un choux-fleur, le tout éclairé par un mec qui a abusé du LSD.


\begin{figure}[h]
\centering
\includegraphics[height=6cm,width=9cm,keepaspectratio]{p10648172.jpg}
\caption*{Ça pique les yeux, non ?}
\end{figure}

Une fois passée la partie visite toute simple, on passe au bain de boue : une grande flaque de boue toute froide. On rentre prudemment dedans, on s'enfonce jusqu'à mi-cuisse, un peu inquiet à la fois de ne pas voir dans quoi on met les pieds (il y a surement des animaux bizarres et affamés au fond des flaques de boue), et aussi inquiet à l'idée de peut-être rester coincé dedans ! Tout ces doutes dissipés, on peut enfin se détendre, régresser de quelques années et se lancer enfin dans une bataille de boue de bon aloi ! Une fois bien recouverts de boue, et un peu refroidis, on peut attaquer le bain d'eau chaude, complètement artificiel, mais on s'en fout, c'est nécessaire. Tout le reste de la grotte est conçu pour nous faire dépenser encore un peu d'argent en plus du ticket : il y a des casiers payants à l'entrée, on peut acheter des petits marteaux pour taper sur des cloches dans la grotte, il y a trois magasins de souvenirs régulièrement espacés, et enfin, on peut se payer un nettoyage des pieds par des poissons avides de peaux mortes. Je trouve que ça fait beaucoup pour une seule grotte !


\begin{figure}[h]
\centering
\includegraphics[height=6cm,width=9cm,keepaspectratio]{p1064825.jpg}
\caption*{Jolie babiole en vente dans la grotte.}
\end{figure}

Les environs de Yangshuo sont aussi très agréables en vélo. En quelques minutes, on peut être au milieu de nulle part, quasiment seuls sur une petite route au milieu des rizières.


\begin{figure}[h]
\centering
\includegraphics[height=6cm,width=9cm,keepaspectratio]{p1054805.jpg}
\caption*{Demain, je ne sais pas où on va, mais ce sera à vélo !}
\end{figure}

C'est ainsi qu'on a aperçu deux magnifiques buffles dans une rizière au bord de la route, genre paysage de carte postale. Sans trop réfléchir, on s'arrête, on descend dans le champ pour prendre des photos. Et le plus gros des deux buffles s'est mis à s'approcher de nous doucement. En soufflant, la tête levée, probablement l'air un peu menaçant, mais c'est difficile à dire quand on n'en côtoie pas tous les jours. Le paysan un peu plus loin nous faisait signe et semblait rigoler et être content, mais quand le buffle a commencé à être vraiment proche de nous, on a quand même rejoint tant bien que mal nos vélos, le buffle sur nos talons (c'est boueux une rizière), et le paysan est finalement venu à notre secours en chassant son buffle à coups de bâtons...


\begin{figure}[h]
\centering
\includegraphics[height=6cm,width=9cm,keepaspectratio]{p1054692.jpg}
\caption*{Confiance ou pas confiance face à ce bestiau ?}
\end{figure}

La ville en elle même est une petite ville touristique. Il y a tout un quartier plein de restaurants allemand, où on peut boire les meilleures bières allemandes pour un prix supérieur à celui qu'on trouve en Allemagne. C'est difficile de se balader tranquillement, car les serveurs des restaurants peuvent être TRÈS insistants, certains nous ont tellement mis leur menu dans la figure qu'on a du les repousser physiquement à plusieurs reprises. On a du mal à croire qu'ils peuvent remplir leurs restaurants comme ça... Dans les restaurants, c'est aussi très intéressants. Alors qu'on mangeait un excellent porc caramélisé, on a vu passer un plat pour le moins original : un "pole dancing chicken", qu'on pourrait traduire par "poulet stripteaseur". Le plat comporte une barre verticale sur laquelle le poulet est empalé, le cou lascivement enroulé autour de la barre. De quoi donner du grain à moudre à une armée de psychanalystes !


\begin{figure}[h]
\centering
\includegraphics[height=9cm,width=12cm,keepaspectratio]{p1034544.jpg}
\caption*{Je n'ai pas de photo de poulet, et de toute façon, le karst, c'est plus joli !}
\end{figure}

Une autre tradition typiquement chinoise, c'est de commander beaucoup plus que ce qu'ils ne peuvent manger, surtout en public. C'est toujours cette histoire de face qui est en jeu, évidemment. C'est même parfois recommandé de toujours laisser un peu de nourriture dans son assiette à la fin du repas, particulièrement quand on est invité, car tout finir pourrait être interprété comme "il n'y en a pas assez", ce qui pourrait être vexant. Le gouvernement essaie de faire changer ces mentalités en instaurant une nouvelle habitude : le contrat de l'assiette vide. Moi, c'est ma maman qui m'a appris à vider mon assiette. Je me demande comment ça se serait passé si ça avait été le boulot d'un agent du gouvernement chinois. En parlant avec quelques Chinois, on se rend compte que c'est entrain de changer doucement mais il y a encore beaucoup de gaspillage : les plats au restaurant sont rarement finis, et on a même vu certains plats rester intacts sur les tables voisines...



\chapter{Lijiang et les gorges du saut du tigre}
Nous arrivâmes donc à Lijiang, dans le Yunnan. Nous sommes arrivés tard le soir à la gare, et après avoir attendu en vain un bus, on prend un taxi pour le centre-ville. Notre hôtel était déjà réservé comme d'habitude, et il se trouvait en plein dans la vieille ville. Après s'être un tout petit peu perdu dans le dédale de petites ruelles, on arrive à notre hôtel et on va directement se coucher.


\begin{figure}[h]
\centering
\includegraphics[height=6cm,width=9cm,keepaspectratio]{p1094828.jpg}
\caption*{Le centre-ville.}
\end{figure}

Le lendemain, on se balade un peu dans cette vieille ville, et on se rend compte que ce n'est qu'un grand centre commercial à ciel ouvert. C'est joli, certes, mais ça fait tellement faux ! Les mêmes boutiques se répètent tous les 50m : argenterie, babioles, djembé, vêtements, puis à nouveau argenterie, etc. De temps en temps un petit restaurant hors de prix, et puis voilà ! L'ambiance musicale est assurée par les magasins de djembés, qui passent tous les mêmes trois chansons, accompagnés à chaque fois par la vendeuse qui tape sur son instrument avec l'entrain d'un paresseux sous bêtabloquants.


\begin{figure}[h]
\centering
\includegraphics[height=6cm,width=9cm,keepaspectratio]{boutiqueslijiang.jpg}
\caption*{Ctrl+C, Ctrl+V}
\end{figure}

Et l'attraction du coin, (vous savez, là où il faut faire une photo absolument) c'est une roue à aubes : Même pas une vraie en plus, une roue à aubes de parc d'attractions, actionnée par un moteur électrique.


\begin{figure}[h]
\centering
\includegraphics[height=6cm,width=9cm,keepaspectratio]{p1094832.jpg}
\caption*{Marion fait la chinoise.}
\end{figure}

La cerise sur le pompon est atteinte quand, au moment de retourner vers l'hôtel, on nous demande les billets d'entrée. Eh ben oui, il faut payer pour entrer dans le centre-ville/centre-commercial. Et si on n'a pas payé le premier soir, c'est parce qu'on est arrivés tard par une petite ruelle. On ne se démonte pas, on fait demi-tour, et on a vite fait de trouver un passage non surveillé. Non mais, on ne va pas payer pour avoir le privilège de se balader dans un centre commercial en plastique !

On a quand même tenté une petite dégustation de thé dans une des boutiques. Ici, ils font le thé puer, qui se présente sous la forme d'un gros bloc. La cérémonie est très complexe, et nécessite une quantité impressionnante de vaisselle. Ils ont même une bouilloire et un robinet automatique : un bouton, et le robinet se place tout seul au dessus de la bouilloire, la rempli, et l'eau est chauffée à la température idéale. La tasse pour infuser, la petite théière pour servir, et les petites tasses sont ensuite rincées à l'eau chaude, puis le thé est rincé plusieurs fois, et l'eau de rinçage du thé est utilisée pour  rincer les tasse et la théière, et enfin on nous sert une toute petite tasse. On a vraiment l'impression de voir une petite fille jouer à la dinette. La première tasse bue, on nous en sert tout de suite une autre. On sent le gout du thé changer au fur et à mesure du nombre d'infusions. Boire le thé dans ces conditions est une opération à part entière : pas question de manger des petits gâteaux en même temps.


\begin{figure}[h]
\centering
\includegraphics[height=6cm,width=9cm,keepaspectratio]{p1094840.jpg}
\caption*{Dégustation de thé.}
\end{figure}

Notre hôtel aussi voulait nous vendre du rêve : International Youth Hostel qu'il s'appelle. On se dit qu'on va rencontrer des gens de partout, mais non, nous sommes les seuls touristes occidentaux, et à l'accueil, ils ne parlent pas anglais, même pas un mot, on doit communiquer à coups de traducteurs et de dessins. Ils arrivent même à nous envoyer à la mauvaise station de bus quand on veut partir pour les gorges du saut du tigre, notre étape suivante. Heureusement qu'il y avait plusieurs bus dans la journée, on a quand même pu arriver à la bonne station de bus assez tôt le matin.


\begin{figure}[h]
\centering
\includegraphics[height=6cm,width=9cm,keepaspectratio]{p1104846.jpg}
\caption*{La fin des gorges.}
\end{figure}



Nous arrivâmes donc dans les gorges du saut du tigre, après avoir traversé la ville de Lijiang en courant pour changer de station de bus. Les gorges du saut du tigre sont supposées être le canyon le plus profond du monde, mais c'est bizarre, car j'avais déjà visite le canyon le plus profond du monde au Pérou, mais bref... C'est de la montagne, c'est beau. Il y a deux chemins possibles pour visiter ce canyon : une route le long de la rivière accessible en bus, ou un vrai sentier de montagne qui passe plus haut. Devinez par où on est passés...


\begin{figure}[h]
\centering
\includegraphics[height=6cm,width=9cm,keepaspectratio]{p1104859.jpg}
\caption*{Vue de l'autre coté des gorges.}
\end{figure}



\begin{wrapfigure}{l}{0.55\textwidth}
\centering
\includegraphics[width=0.5\textwidth]{p1114984.jpg}
\caption*{Le bas des gorges, avec le chemin taillé dans la falaise.}
\end{wrapfigure}

Ça nous prend deux jours pour aller au fond des gorges, en passant par quelques petits villages dont les habitants ont vite compris que les guesthouses, c'est plus rentable que la culture du riz. On peut manger et dormir sur le sentier pour le même prix qu'en ville, ce qui surprend quand on est habitués aux tarifs des gîtes de montagne européens. Les paysages sont magnifiques, et on sympathise rapidement avec quelques autres marcheurs à force de les croiser sur le chemin et dans les guesthouses.
A la fin du chemin, on redescend le long de la rivière, et on atteint l'endroit où le tigre à sauté par dessus la rivière comme le dit la légende. Et ben ce sont de sacrés tigres qui vivaient là-bas !




\chapter{Yuanyang, là où Dieu pleure pour les photographes.}
Nous arrivâmes donc à Yuanyang. Là encore, il y a un piège : ça correspond à la fois à toute une zone, et à un village en particulier. Mais ce n'est pas tout : le village en particulier est en deux parties, la nouvelle ville et la vieille ville qui sont quand même à 30 km de distance. Et ce sont trente km de lacets dans la montagne. Encore une fois, il faut bien savoir où on veut aller, et nous, on voulait aller chez Jacky, au milieu des rizières. Le coin est connu pour avoir les plus belles rizières du monde. Jacky est retourné dans son village pour ouvrir cette guesthouse après avoir voyagé dans le monde entier comme photographe professionnel. Elle est magnifiquement décorée par ses photos des rizières. Il dit que Dieu pleure pour les photographes qui ne sont jamais venus à Yuanyang. On ne peut qu'acquiescer.


\begin{figure}[h]
\centering
\includegraphics[height=6cm,width=9cm,keepaspectratio]{p1135531.jpg}
\caption*{On commence en douceur.}
\end{figure}

J'ai aussi envie de vous parler un peu de l'envers du décors : Quand on voit ces magnifiques photos, on imagine souvent que c'est un endroit paisible, qu'on est seul au paradis, qu'il a fallut marcher longtemps, que ce fut épuisant de se lever si tôt pour avoir le temps de monter au sommet avant le lever du soleil. Ce n'est pas exactement comme ça ici : on peut acheter un ticket qui donne accès à plusieurs plateformes, certaines dédiées au lever, d'autres au coucher de soleil. Il faut venir tôt si on veut une bonne place pour poser son trépied.


\begin{figure}[h]
\centering
\includegraphics[height=6cm,width=9cm,keepaspectratio]{p1135614.jpg}
\caption*{Il y en a toujours un qui essaie de faire son intéressant...}
\end{figure}

Quand le lever commence, il doit y avoir 10 000 \euro de matos photo pour chaque mètre de rambarde. C'est à se demander quel est l'intérêt de prendre des photos si c'est pour avoir les mêmes que tout le monde. Mais on oublie vite tout ça devant la magnificence des paysages, et on se laisse absorber par le jeu de miroirs dans les rizières, les vagues dans les brumes, et les couleurs du lever de soleil. Voici une maigre tentative de rendre compte de l'ambiance :


\begin{figure}[h]
\centering
\includegraphics[height=6cm,width=9cm,keepaspectratio]{p1135514.jpg}
\caption*{Le détail.}
\end{figure}


\begin{figure}[h]
\centering
\includegraphics[height=6cm,width=9cm,keepaspectratio]{p1135515.jpg}
\caption*{La vue d'ensemble.}
\end{figure}


\begin{figure}[h]
\centering
\includegraphics[height=6cm,width=9cm,keepaspectratio]{p1135557.jpg}
\caption*{Ça marche aussi en noir et blanc.}
\end{figure}

Jacky, en plus d'être un très bon photographe, est un excellent cuisinier. Quand on mange chez lui, on a un seul choix à faire : avec ou sans viande, et ensuite, la valse des plats commence. Plus on est une grande tablée, mieux c'est, car il y a d'autant plus de plats différents à partager et à goûter. On a donc tôt fait de se faire plein de compagnons de repas. On ira jusqu'à en ramener depuis d'autres auberges. La plus grande surprise viendra d'un plat en apparence tout bête : des frites au poivre. Mais pas n'importe quel poivre ! Du poivre du Sichuan. J'avais déjà entendu parlé de ça, et je pensais que c'était du poivre normal qui poussait au Sichuan. Que nenni ! C'est une autre plante, dont l'intérêt est qu'elle anesthésie la bouche. La sensation est assez difficile à expliquer, ce n'est pas vraiment comme chez le dentiste, car on a encore quelques sensations en bouche, mais la sensation est plus diffuse, accompagnée de petits picotements et de l'impression étrange que cette langue n'a rien à faire ici.


\begin{figure}[h]
\centering
\includegraphics[height=5.5cm,width=9cm,keepaspectratio]{p1135541.jpg}
\caption*{Un futur ingrédient de la cuisine de Jacky.}
\end{figure}

On fait également un tour au marché, où on se frotte à la photographie de rue. On aimerait bien prendre des photos qui retranscrivent l'ambiance, mais on a l'impression d'être des voleurs si on ne demande pas l'accord d'abord. Mais si on demande l'accord, soit les gens disent non, soit ils demandent de l'argent, soit ils perdent totalement leur naturel. C'est un peu compliqué. On achète quand même des sauterelles, parce qu'on est curieux et que ça nous permet de faire des photos de la vendeuse. Mais au moment de gouter les sauterelles, on se rend compte qu'elles ne sont pas tout à fait mortes : dès qu'on en prend une dans la main, elle bouge un peu ! Ça nous a coupé l'appétit pour le coup.


\begin{figure}[h]
\centering
\includegraphics[height=5.5cm,width=9cm,keepaspectratio]{p1145820.jpg}
\caption*{En effet, c'est une perceuse.}
\end{figure}


\begin{figure}[h]
\centering
\includegraphics[height=5.5cm,width=9cm,keepaspectratio]{p1145828.jpg}
\caption*{Cuit ou cru, telle est la question.}
\end{figure}




\begin{figure}[h]
\centering
\includegraphics[height=5.5cm,width=9cm,keepaspectratio]{p1145846.jpg}
\caption*{Pour les moins de 20 ans : ceci est une télé !}
\end{figure}


\begin{figure}[h]
\centering
\includegraphics[height=5.5cm,width=9cm,keepaspectratio]{p1145835.jpg}
\caption*{Au marché, les hommes aussi font leur part !}
\end{figure}

On part ensuite pour le Vietnam. Nous faisons 200 km en six heures de bus. C'est lent. Et pour nous tenir compagnie dans le bus, une Israélienne et sont fils avec lesquels nous avions déjà sympathisé chez Jacky (les repas, souvenez-vous). Le trajet est lent, et on fait plusieurs pauses, dont une pour nettoyer le bus et refroidir les freins au jet d'eau, à mi-parcours d'une grande descente, c'est rassurant... Le seul intérêt de ce voyage : lors d'une pause, on achète des glaces un peu au hasard, et on tombe sur une glace... aux petit-pois !


\begin{figure}[h]
\centering
\includegraphics[height=5.5cm,width=9cm,keepaspectratio]{p1145901.jpg}
\caption*{Les rizières, c'est aussi ça.}
\end{figure}

\begin{figure}[h]
\centering
\includegraphics[height=6cm,width=9cm,keepaspectratio]{p1145904.jpg}
\caption*{Je vous jure, ce sont les vraies couleurs !}
\end{figure}

\begin{figure}[h]
\centering
\includegraphics[height=8cm,width=12cm,keepaspectratio]{p1135635.jpg}
\caption*{Une petite dernière pour la route !}
\end{figure}



\tableofcontents
\end{document}
